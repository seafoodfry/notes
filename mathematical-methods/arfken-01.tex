\section{Preliminaries}

%%%%%%%%%%%%%%%%%%%%%%%%%%%%%%%%%%%%%%%%%%%%%%%%%%%%%%
\subsection{Infinite Series}

A necessary condition for convergence of an Infinite series to a limit $S$ is that
$\lim_{n\rightarrow \infty} u_n = 0$, where $u_n$ is a term in the series.
However, the above condition is not sufficient to guarantee convergence.

The argument for this is presented in Abbott in Theorem 2.7.2, the Cauchy criterion for series,
and in Theorem 2.7.3.
\\~\\

\textbf{D'Alembert (Cauchy) Ratio Test}

While looking at the D'Alembert (Cauchy) ratio test, keep in mind that the geometric
series defines $r = u_{n+1} / u_n$.

Also, in Example 1.1.4, we have the $n \geq 2$ condition because if $n = 1$,
then we would have an undeterminate result.
\\~\\


\textbf{Cauchy or Maclaurin Integral Test}

The trick to making sense of the first set of inequalities goes as follows...

$s_i$, by the fact that $f(x)$ is monotonic decreasing, will overestimate the area under the curve.
So the difference between $s_i \geq \int_{1}^{i+1} f(x) dx$ and $s_i \leq a_1 + \int_{1}^{i} f(x) dx$
is that there is a "shift" between the two (the graph as shown in the book).

Where as $\int_{1}^{i+1} f(x) dx$ starts from 1 and always goes to $\{2, 3, 4, \ldots\}$,
$\int_{1}^{i} f(x) dx$ cancels out its first "term" since $\int_{1}^{1} f(x) dx = 0$.

So when we look at the first two partial sums in the context of the second inequality we get
\begin{enumerate}
    \item $s_1 = a_1             \leq a_1 + \int_{1}^{1} f(x) dx = a_1$. The claim that $a_1 \leq a_1$ is true.
    \item $s_2 = a_1 + a_2       \leq a_1 + \int_{1}^{2} f(x) dx$. Here the claim is that $f(2) \leq \int_{1}^{2} f(x) dx$.
    \item $s_3 = a_1 + a_2 + a_3 \leq a_1 + \int_{1}^{3} f(x) dx$. Here the claim is that $f(2) + f(3) \leq \int_{1}^{3} f(x) dx$.
\end{enumerate}