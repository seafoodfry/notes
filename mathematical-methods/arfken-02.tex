\section{determinants and Matrices}

\subsection{determinants}

\subsubsection{Homogeneous Linear Equations}

$$
a \cdot (b \times c) = (a\times b) \cdot c
$$

An $n\times n$ homogeneous system of linear equations has a unique non-trivial solution
if and only if its determinant is non-zero.
If this determinant is zero, then the system has either no nontrivial solutions
or an infinite number of solutions.

If the determinant is zero, then one of the vectors lies in the plane spaned by one of the other
two vectors - it is not independent which is equivalent to saying that is linear combination of the other(s).