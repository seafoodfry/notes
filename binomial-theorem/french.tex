\section{Recherches Sur la Serie $1 + \frac{m}{1}x + \frac{m(m-1)}{1\cdot2}x^2 + \frac{m(m-1)(m-2)}{1\cdot 2 \cdot 3}x^3 + \ldots$}


If we subject the reasoning which we generally use when it comes to infinite series
to a more exact examination, we will find that it is, all things considered, unsatisfactory,
and that consequently the number of theorems, concerning infinite series, which can be
considered as rigorously founded, is very limited.
We ordinarily apply the operations of analysis to infinite series in the same way as if the
series were finite, which does not seem to me to be permitted without particular
demonstration.
If for example we have to multiply two infinite series one by the other, we put
\begin{align*}
\left( u_0 + u_1 + u_2 + \ldots \right) \left( v_0 + v_1 + v_2 + \ldots \right) \\
= u_0 v_0 + (u_0 v_1 + u_1 v_0) + ( u_0 v_2 + u_1 v_1 + u_2 v_2 ) + \ldots + \\
    (u_0 v_n + u_1 v_{n-1} + u_2 v_{n-2} + \ldots + u_n v_0 ) + \ldots
\end{align*}