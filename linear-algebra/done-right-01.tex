\section{Vector Spaces}

\subsection{Exercises}

\textbf{1.A.1}

Find $c$ and $d$ such that
$$
\frac{1}{a + bi} = c + di
$$

The trick here is as follows
$$
\frac{1}{a + bi} \left( \frac{a - bi}{a - bi} \right)= \frac{a - bi}{a^2 + b^2}
$$
So $c = a / |z|^2$ and $d = -b / |z|^2$, where $z = a + bi$ and $|z| = \sqrt{a^2 + b^2}$.
\\~\\



\textbf{1.A.2}

Show that
$$
\frac{-1 \sqrt{3}i}{2}
$$
is a cube root of 1.

$$
(-1 \sqrt{3}i) (-1 \sqrt{3}i) = -2 -2\sqrt{3}i
$$
$$
(-2 -2\sqrt{3}i) (-1 \sqrt{3}i) = 2 + 6 = 8
$$

Ad since $\frac{1}{2^3} = 1/8$, then we get our proof.
\\~\\



\textbf{1.A.3}

To find 2 distinct square roots of $i$ it helps to look at Euler's identity first
$$
e^{i\pi} + 1 = 0
$$
or
$$
e^{i\pi} = -1
$$
And since $i^2 = -1$, then
$$
e^{i\pi} = -1 = i^2
$$
Hence
$$
e^{i\pi/2} = i
$$

So our firsrt square root is $e^{i\pi/4} = \sqrt{i}$.
If we go around once, then we get another square root
$$
e^{(i\pi/2 + 2\pi i)/2} = e^{i5\pi / 4}
$$

We can still simply this further.
$$
e^{i\pi/4} = \cos{\frac{\pi}{4}} + i\sin{\frac{\pi}{4}} = \frac{1}{\sqrt{2}} + i\frac{1}{\sqrt{2}}
= \frac{1 + i}{\sqrt{2}}
$$
and
$$
e^{i5\pi/4} = \cos{\frac{5\pi}{4}} + i\sin{\frac{5\pi}{4}} = \frac{-1}{\sqrt{2}} + i\frac{-1}{\sqrt{2}}
= - \frac{1 + i}{\sqrt{2}} 
$$

Another interesting read is:
Ed Scheinerman (2023) A Third Real Solution to $x=x^{-1}$, Not Really,
The American Mathematical Monthly, 130:6, 514-514, DOI: 10.1080/00029890.2023.2184161
\\~\\



\textbf{1.A.4}

Show that $\alpha + \beta = \beta + \alpha$ for all $\alpha , \beta \in \mathbb{C}$.
\\

Let's define $\alpha = a + bi$ and $\beta = c + di$.
$$
\alpha + \beta = (a + bi) + (c + di) = (a+c) + (b+d)i
$$
and
$$
\beta + \alpha = (c + di) + (a + bi) = (c+a) + (d+b)i
$$
And since summation is commutative for the reals, QED.
\\~\\



\textbf{1.A.5}

Show that $(\alpha + \beta) + \lambda = \alpha + (\beta + \lambda)$ where $\alpha , \beta , \lambda \in \mathbb{C}$.
\\

Same mechanics,
$$
(\alpha + \beta) = (a+c) + (b+d)i
$$
If we define $\lambda = x + yi$
then
$$
(\alpha + \beta) + \lambda = [(a+c) + (b+d)i] + (x + yi) = (a+b+x) + (b+d+y)i
$$
And hopefully you can see the rest of the argument and how we are save by the commutative property of
the reals.
\\~\\



\textbf{1.A.8}

Show that for every $\alpha \in \mathbb{C}$ with $\alpha \neq 0$, there exist a unique $\beta \in \mathbb{C}$
such that $\alpha \beta = 1$.
\\

The result from exercise 1.A.1 comes in handy here!
this is because we get an expression for a an inverse complex number in terms that make it easy
to carry out calculations the way we are used to.

In 1.A.1 we had $\beta = c + di = (c, d)$ equal to
$$
(a, b)\left( \frac{a}{a^2 + b^2}, \frac{- b}{a^2 + b^2} \right) = (1, 0) = 1
$$
\\~\\


\textbf{1.A.10}

Find $x\in \mathbb{R}^4$ such that
$$
(4, -3, 1, 7) + 2x = (5, 9, -6, 8)
$$

Let's try component by component,
\begin{itemize}
    \item $4 + 2x_0 = 5$ results in $x_0 = 1/2$.
    \item $-3 + 2x_1 = 9$ results in $x_1 = 12/2 = 6$.
    \item $1 + 2x_2 = -6$ results in $x_2 = -7/2$.
    \item $7 + 2x_3 = 8$ results in $x_3 = 1/2$.
\end{itemize}

So $x = (1/2, 6, -7/2, 1/2)$.
\\~\\



\textbf{1.A.11}

Explain why there does not exist a $\lambda \in \mathbb{C}$ such that
$$
\lambda \left( 2-3i, 5+4i, -6+7i \right) = \left( 12-5i, 7+22i, -32-9i \right)
$$

This one is funky but try note that $\lambda(2-3i) = (12-5i)$, and $\lambda(5+4i) = (7+22i)$.
Then multipy $\lambda(2-3i)(7+22i)$ and compare that against $(12-5i)(5+4i)$, and so on.
\\~\\



\textbf{1.B.1}

Prove that $-(-v) = v$ for every $v\in V$.

Using the additive inverse property
$$
(-v) + -(-v) = 0 \rightarrow (-v) + v = 0
$$
And by the Uniqueness of the additive inverse, $-(-v) = v$.
\\~\\



\textbf{1.B.3}

Suppose $v, w \in V$.
Explain why there exists a unique $x\in V$ such that
$$
v + 3x = w
$$

There are two parts to this question: existance and Uniqueness.
The existance part can be seen by using
$$
x = \frac{1}{3}(w - v)
$$
in our original expression.
That is,
$$
v + 3x = v + 3 \frac{1}{3}(w - v) = v + w - v = w
$$

Uniqueness can be seen by noting that if we had
$$
v + 3x^{\prime} = w
$$
Then
$$
x^{\prime} = \frac{1}{3}(w - v)
$$
And so
$$
x - x^{\prime} = 0 
$$
\\~\\


\textbf{1.B.4}

The empty set is not a vector space because it fails to satisfy the additive identity -
can't have the existence of an element $0\in V$ if the set is empty.
\\~\\


\textbf{1.B.5}

Show that the additive inverse condition - for every $v\in V$, $\exists w\in V$ such that $v+w = 0$ -
can be replaced by the condition
$$
0v = 0
$$
for all $v\in V$.
Where the $0$ on the left is $0\in\mathbb{F}$ and the $0$ on the right is the additive identity of
$v$.
\\

Normally, we would think of the additive inverse for $v + w = 0$ as $w = -v$, so
$$
0 = v + w = v + (-v) = 1v + (-1v) = 0v = 0
$$
\\~\\



\textbf{1.B.6}

Let $\infty$ and $-\infty$ denote two distinct objects, neither of which is in $\mathbb{R}$.
Now, say we defien addition and scalar multiplication in $\mathbb{R}\cup\{\infty\}\cup\{-\infty\}$
as we normally would.

Is $\mathbb{R}\cup\{\infty\}\cup\{-\infty\}$ a vector space over $\mathbb{R}$?
\\

No.
For example, additive inverses and additive identities would not be unique - nor any of the other
conditions required of a vector space.
\\~\\




%%%%%%%%%%%%%%%%%%%%%%%%%%%%%%%%%%%%%%%%%%%%%%%%%%%%%%%%%%%%%%%%%%%%%%
\subsection{Subspaces}

One topic covered in the book is proving that sets are subspaces.
Let's see a couple worked out cases before attempting example 1.35.

First, let's say we have
$$
U = \{ (x_1, x_2, x_3) \in \mathbb{F}^4 : x_1 + 2x_2 = 0 \}
$$
and we claim this set is a subspace of $\mathbb{F}^3$.

We can see that the additive identity is part of the subspace, $0\in U$, since $x_1 +2x_2 = 0$.

Now let's see if this subspace is closed under addition.
If we have $u = (u_1, u_2, u_3)$ and $w = (w_1, w_2, w_3)$, then
$$
u + v = (u_1, u_2, u_3) + (w_1, w_2, w_3) = (u_1+w_1, u_2+w_2, u_3+w_3)
$$

Given the defintion of our set, we should have the above $u + w = (u_1+w_1, u_2+w_2, u_3+w_3)$
met the requirement that $(u_1+w_1) + 2(u_2+w_2) = 0$.
To see this, start from that restraint on $u$ and $w$:
$$
u_1 + 2u_2 + w_1 + 2w_2 = 0 + 0 = (u_1 +w_1) + 2(u_2+w_2) 
$$

Now, we just need to show that our subspace is closed under scalar multiplication.
This process is similar.
We want to see that $au = (au_1, au_2, au_3)$ stil matches the constrain given so that
$au_1 + a2u_2 = 0$.
Which we see does satisfy our constraint since $a(u_1 + 2u_2) = au_1 = a2u_2 = 0$.
$\blacksquare$
\\~\\

Similarly, we can now take a look at example 1.35:

If $b\in \mathbb{F}$, then
$$
U = \{ (x_1, x_2, x_3, x_4) \in \mathbb{F}^4 : x_3 = 5x_4 + b \}
$$
is a subspace of $\mathbb{F}^4$ if and only if $b=0$.

From the constraint imposed on our subspace, we want to see
$$
u + w = (u_1+w_1, u_2+w_2, u_3+w_3, u_4+w_4) \rightarrow (u_3+w_3) = 5(u_4+w_4) + 2b
$$

If we add the individual constrants we see that we indeed match the above
$$
u_3 + w_3 = 5u_4 + b + 5w_4 + b \rightarrow (u_3+w_3) = 5(u_4+w_4) + 2b
$$

Similarly, we can see that this subspace is closed under scalar multiplication
because $au = (au_1, au_2, au_3, au_4)$ would meet the constrain given when if
$au_3 = a5u_4 + ab$.
Which is indeed the case because $a(u_3) = a(5u_4 + b)$ matches the previous expression.

The only requirement we haven't meet is to show that this subspace meets the additive identity requirement.
This is when $b=0$ becomes a must.
$\blacksquare$
\\~\\




\textbf{The set of continuous real-valued functions on the interval $[0,1]$ is a subspace of $\mathbb{R}^{[0,1]}$.}
\\


\textbf{Note:} in this example $\mathbb{R}$ actually stands for $\mathbb{R}^n$. What this means is
that you need to keep in mind that the inputs and outputs to functions are tuples, not single numbers.
\\

Now, let's revisit $\mathbb{R}^{[0,1]}$. People say $\mathbb{R}^{[0,1]}$ denotes the set of functions
from the set $[0,1]$ to $\mathbb{R}$ (set of real-valued functions on $[0,1]$).

Let's start with noting the following:
$$
\mathbb{R}^{[0,1]} = \{ f | f : [0,1] \rightarrow \mathbb{R}\}
$$

Our first stop is checking the additive identity, $0\in\mathbb{R}^{[0,1]}$?
We could take the name literally but as Axler pointed out, the first requirement is a way to checking
that the subspace is not empty and to show that $0\in \mathbb{R}^{[0,1]}$.

To start, $f(x) = 0, \forall x\in [0,1]$ is a continuous function.
So $0\in\mathbb{R}^{[0,1]}$.

Next, assume we have $f : [0,1]\rightarrow \mathbb{R}$ and $g : [0,1]\rightarrow \mathbb{R}$.
And we have $f(x) = a$ and $g(x) = b$, where $a,b\in\mathbb{R}$, then
$$
(f+g)(x) = f(x) + g(x) = a + b \in \mathbb{R}
$$
These cases are valid since the sum of continuous functions is itself a continuous function,
and constant functions are constinuous.

Finally, if $\lambda \in [0,1]$, then
$$
( \lambda f )(x) = \lambda f(x) = \lambda a \in \mathbb{R} \quad \blacksquare
$$

\textbf{Note:} this is also an entrypoint into an interesting argument for showing that
$[0,1]$ and $\mathbb{R}$ have the same cardinality.
\\~\\




\textbf{The set of differentiable real-valued functions on $\mathbb{R}$ is a subspace of $\mathbb{R}^R$.}
\\

\textbf{Note:} in this example $\mathbb{R}$ actually stands for $\mathbb{R}^n$. What this means is
that you need to keep in mind that the inputs and outputs to functions are tuples, not single numbers.
\\

$\mathbb{R}^R$ denotes the set of functions from $\mathbb{R}$ to $\mathbb{R}$:
$$
\mathbb{R}^R = \{ f | f:\mathbb{R} \rightarrow \mathbb{R} \}
$$

In this case, our [proposed] subspace is defined as follows:
$$
U = \{ f | f:\mathbb{R} \rightarrow \mathbb{R}, \text{ where } \forall c\in\mathbb{R}, \exists f^\prime (c) \}
$$

We resorted to the internet to gain some insights and
\href{https://math.stackexchange.com/questions/1592249/what-is-mathbb-r-mathbb-r-as-a-vector-space}{What is $R^R$ as a vector space?}.

The accepted answer mentions this great point that we can think of $\mathbb{F}^n$ as the
assignment of elements of the set $\{1, 2, \ldots , n\}$ to elements of $\mathbb{F}$
by $g \in \mathbb{F}^{ \{1,2,\ldots , n\} }$ or as $g : \mathbb{N} \rightarrow \mathbb{F}$.
The function $g$ behaves such that the index of an element of the n-tuple corredsponds to a member of
the n-tuple or, $g(1)=x_1$, $g(2)=x_2$, and so on correspond to $(x_1, x_2, \ldots , x_n)$.

This means that in our case, $f\in U$, these are functions that map a uncountably infinite number
of elements to an uncountably infinite long tuple.

The math exchange answer also gives us $g: \mathbb{R}\rightarrow \{0\}$ as the additive identity.
There, $\{0\}$ is an uncountably infinite tuple of all zeros, which is also differentiable everywhere
within our domain.
Then the rest of the argument is very similar to when we discussed $\mathbb{R}^{[0,1]}$.
\\~\\