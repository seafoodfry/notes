\section{Setting the Stage}

\subsection{Euclidean Spaces and Vectors}

\subsubsection{Exercises}

\textbf{1.1.2}

Given $\vec{x}, \vec{y} \in \mathbb{R}^n$,
\begin{align*}
|\vec{x} + \vec{y} |^2 &= (\vec{x} +\vec{y}) (\vec{x} +\vec{y}) \\
&= \vec{x}\cdot\vec{x} + \vec{y}\cdot\vec{y} + 2\vec{x}\cdot\vec{y} \\
&= |\vec{x}|^2 + |\vec{y}|^2 + 2\vec{x}\cdot\vec{y}
\end{align*}

Similarly,
\begin{align*}
|\vec{x} - \vec{y} |^2 &= (\vec{x} - \vec{y}) (\vec{x} - \vec{y}) \\
    &= \vec{x}\cdot\vec{x} + \vec{y}\cdot\vec{y} - 2\vec{x}\cdot\vec{y} \\
    &= |\vec{x}|^2 + |\vec{y}|^2 - 2\vec{x}\cdot\vec{y}
\end{align*}

Hence
$$
|\vec{x} + \vec{y} |^2 + |\vec{x} - \vec{y} |^2 = 2\left( |\vec{x}|^2 + |\vec{y}|^2 \right)
$$
\\~\\



\textbf{1.1.7}

Suppose $\vec{a}, \vec{b} \in \mathbb{R}^3$
\\

Show that if $\vec{a}\cdot\vec{c} = \vec{b}\cdot\vec{c}$ and
$\vec{a}\times\vec{c} = \vec{b}\times\vec{c}$ for some non-zero $\vec{c} \in\mathbb{R}^3$,
then $\vec{a} = \vec{b}$.
\\

We could try to simply stare at
$$
\vec{a}\cdot\vec{c} = |\vec{a}| |\vec{c}| \cos{\theta_1} 
= |\vec{b}| |\vec{c}| \cos{\theta_2} = \vec{b}\cdot\vec{c}
$$
Which tells us
$$
|\vec{a}| \cos{\theta_1} = |\vec{b}| \cos{\theta_2}
$$

Let's try something else,
$$
|a\times c|^2 = |a||c| - (a\cdot c)^2
= |b||c| - (b\cdot c)^2 = |b \times c|^2
$$
We now have
$$
|a||c| - (a\cdot c)^2 = |b||c| - (b\cdot c)^2
$$
or
$$
|a||c| = |b||c| \rightarrow |a| = |b|
$$

So we can go back to our first attempt and see that
$$
|a| \cos{\theta_1} = |b| \cos{\theta_2} \rightarrow \cos{\theta_1} = \cos{\theta_2}
$$
\\~\\



\textbf{1.1.8}

To see that $a\cdot(b\times c)$ is the determinant of the three vectors,
simply write out the determinant for $b\times c$ and note that the explicit version of it
is a "normal" vector.
Since the dot product is defined as $x\cdot y = x_1 y_1 + x_2 y_2 + \ldots + x_n y_n$, when
$x, y \in \mathbb{R}^n$.

Putting these two facts together we can see how $a\cdot(b\times c)$ can be computed via a single determinant
operation.
\\~\\



%%%%%%%%%%%%%%%%%%%%%%%%%%%%%%%%%%%%%%%%%%%%%%%%%%%%%%%%%%%%%%%%%%%%%%%%%
\subsection{Subsets of Euclidean Space}

\textbf{Proposition 1.4}

Remember that to be a boundary point of $S \subset \mathbb{R}^n$ \textbf{every}
ball centered at $\vec{x}$ must contain points in $S$ and in $S^c$.

If $\vec{x}$ is an interior point, then for some $\vec{x} \in S$,
there is a ball $B(r,\vec{x}) \subset S$;
but also, there is an $B(r, \vec{x}) \not\subset S^c$ for some $r>0$.
This is why you ought to be an interior point in $S$, in $S^c$, or be a boundary point
- because to be a boundary point every single $B(r, \vec{x})$ must be in $S$ AND in $S^c$.

This last statement is why it is also the case that if $S$ is closed then $S^c$ must be open:
because any points must be interior points to either one or be a boundary point.
So if $S$ has all of the boundary points, then its compliment must be left with none and thus only
have interior points and be an open set.
\\~\\







\subsubsection{Exercises}