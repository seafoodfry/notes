\section{Vectors}

\subsection{Parametric Equations in General}

\subsubsection{Cycloid}

The parametric equation for $\overrightarrow{AP}$ looks the way it does because the starting point
if at $\frac{3\pi}{2}$.

\subsubsection{Involute of a Circle}

The lenght of $\overrightarrow{BP}$ is $a\theta$ which is the length of the arc! ($s=\theta r$.)


%%%%%%%%%%%%%%%%%%%%%%%%%%%%%%%%%%%%%%%%%%%%%%%%%%%%%%%%%%%%%%%%%%%%%%%%%%
\subsection{The Dot Product}


\subsection{Vector Projections}

\textbf{Example 4}

The magnitude of the projection of $\vect{F}$ onto $\vect{a}$ is equivalent to $\vect{F} \sin(30)$.
That is,
$$
|\text{proj}_{\vect{a}} \vect{F}| = |\vect{F}| \sin 30
$$

Remember that to convert degrees to radians you must multiply degrees by $\pi / 180$.
This works out so because gravity is only acting along a single direction.


%%%%%%%%%%%%%%%%%%%%%%%%%%%%%%%%%%%%%%%%%%%%%%%%%%%%%%%%%%%%%%%%%%%%%%%%%%
\subsection{The Cross Product}


\subsection{Rotation of a Rigid Body}

$|\vect{r}(t)|$ and $\theta$ are constant, but $\vect{r}(t)$ changes direction as time changes.
That change in direction corresponds to a change to the vector $\vect{r}(t)$ and thats
why it corresponds to the arc swept btween $t$ and $t+ \Delta t$.
\\~\\

\textbf{Proposition 4.3}, its proof is similar to an exercise we did for Folland, see
\ref{folland:1.1.7}


%%%%%%%%%%%%%%%%%%%%%%%%%%%%%%%%%%%%%%%%%%%%%%%%%%%%%%%%%%%%%%%%%%%%%%%%%%
\subsection{Equations for Planes and Distance Problems}