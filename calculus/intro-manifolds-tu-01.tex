\section{Euclidean Spaces}


\subsection{Smooth Functions on Euclidean Space}

\subsubsection{$C^\infty$ Versus Analytic Functions}

Starting out with the simplest notation that anyone may have seen, we can write the partial drivates
as
$$
\frac{\partial^2 f}{\partial x \partial y}
$$

There is a theorem called \textbf{Schwarz's theorem (or Clairaut's theorem on equality of mixed partials)}
that states that for a function $f : U \subset \mathbb{R}^n \rightarrow \mathbb{R} $
(note that $U$ is an open set - $U \subset \mathbb{R}^n$ not $U \subseteq \mathbb{R}^n$)
if $p = (p^1, p^2, \ldots , p^n) \in \mathbb{R}^n$
and $p \in U$ (some neighboorhod of $p$ is contained in $U$)
and $f$ has \textbf{continuous second partial derivatives on that neighboorhod of $p$},
then for all $i$ and $j$ in $\{ 1, 2, \ldots , n \}$,

$$
\frac{\partial^2 f (p)}{\partial x^i \partial x^j}
= \frac{\partial^2 f (p)}{\partial x^j \partial x^i}
$$

\textbf{The partial derivatives of this function commute at that point.}
\\

We wanted to mention Schwarz's theorem on equality of mixed partial derivatives to present some of the context
we will be visiting over and over.
Now, let's carry on with a little bit more notation.
From our abstraction in the statement of Schwarz's theorem, we can see that mixed partial derivates have the following
form,

$$
\frac{\partial^{i+j+k} f}{\partial x^i \partial x^j \partial x^k}
$$

In Tu's notation the above then becomes
$$
\frac{\partial^j f}{\partial x^{i_1} \ldots \partial x^{i_j}}
$$
where $\partial x^{i_1}$ means that we are taking the partial derivative of degree $i$ with respect to the
coordinate $x^1$.
There is also the implicit requirement that the $j$ partial derivatives of degree $i$ will sum to $j$, so essentially
$i=1$.

The above notation cold very well be reworded into
$$
\frac{\partial^\alpha f}{\partial x^{\alpha_1}_{1} \ldots \partial x^{\alpha_j}_{j}}
$$
where $\alpha = \alpha_1 + \ldots + \alpha_j$, we also moved the superscript for the dimension to a subscript just
to simplify the latex - but keep in mind that \textbf{Tu uses the second superscript to note the dimension}.
\\~\\



Lastly, there is a very interesting result that is mentioned in this section:
not all smooth functions are analytic.
This may sound odd at first because in the example given $f(x) = 0$ for $x \leq 0$.
But keep in mind that in order to be analytic at a point $p \in U$, the function's Taylor series must equal
the actual function in a neighborhood around $p$.
So for some $o+\epsilon$, where $\epsilon >0$, the function does not equal the non-zero part of the function.

The wikipedia page for \href{https://en.wikipedia.org/wiki/Taylor_series}{Taylor series}
explains this quite well:
[$f(x)$] is infinitely differentiable at $x = 0$, and has all derivatives zero there.
Consequently, the Taylor series of $f(x)$ about $x = 0$ is identically zero.
However, $f(x)$ is not the zero function, so does not equal its Taylor series around the origin.
Thus, $f(x)$ is an example of a non-analytic smooth function.





%%%%%%%%%%%%%%%%%%%%%%%%%%%%%%%%%%%%%%%%%%%%%%%%%%%%%%%%%%%%%%%%%%%%
\subsubsection{Taylor's Theorem with Remainder}