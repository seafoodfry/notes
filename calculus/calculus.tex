\documentclass{article}
\usepackage{blindtext} % for table of contents and other niceseties.
\usepackage{amsmath}
\usepackage{amsfonts} % things such as \mathbb{R}, or \Z. See https://en.wikipedia.org/wiki/List_of_mathematical_symbols_by_subject
\usepackage{amstext} % for \text macro.
\usepackage{array} % for tables.
\usepackage{float} % here for H placement parameter
\usepackage{hyperref} % for hyperlinks.
\usepackage{listings} % for code.
\usepackage[makeroom]{cancel} % corssing out terms in math https://tex.stackexchange.com/questions/75525/how-to-write-crossed-out-math-in-latex.

% Make greek letters bold too.
\usepackage{bm}
\newcommand{\vect}[1]{\boldsymbol{\mathbf{#1}}}

% Make paragrphs show in the table of contents.
\setcounter{secnumdepth}{4}
\setcounter{tocdepth}{4}

\title{Notes on Calculus}

\numberwithin{equation}{section}

% There is a \pmod for the "mod" symbol, but not one for the "div" symbol.
% Thus we are making it up.
\makeatletter
\newcommand*{\bdiv}{%
  \nonscript\mskip-\medmuskip\mkern5mu%
  \mathbin{\operator@font div}\penalty900\mkern5mu%
  \nonscript\mskip-\medmuskip
}
\makeatother

\begin{document}

\maketitle
\tableofcontents

\section{Vectors}

\subsection{Parametric Equations in General}

\subsubsection{Cycloid}

The parametric equation for $\overrightarrow{AP}$ looks the way it does because the starting point
if at $\frac{3\pi}{2}$.

\subsubsection{Involute of a Circle}

The lenght of $\overrightarrow{BP}$ is $a\theta$ which is the length of the arc! ($s=\theta r$.)


%%%%%%%%%%%%%%%%%%%%%%%%%%%%%%%%%%%%%%%%%%%%%%%%%%%%%%%%%%%%%%%%%%%%%%%%%%
\subsection{The Dot Product}


\subsection{Vector Projections}

\textbf{Example 4}

The magnitude of the projection of $\vect{F}$ onto $\vect{a}$ is equivalent to $\vect{F} \sin(30)$.
That is,
$$
|\text{proj}_{\vect{a}} \vect{F}| = |\vect{F}| \sin 30
$$

Remember that to convert degrees to radians you must multiply degrees by $\pi / 180$.
This works out so because gravity is only acting along a single direction.


%%%%%%%%%%%%%%%%%%%%%%%%%%%%%%%%%%%%%%%%%%%%%%%%%%%%%%%%%%%%%%%%%%%%%%%%%%
\subsection{The Cross Product}


\subsection{Rotation of a Rigid Body}

$|\vect{r}(t)|$ and $\theta$ are constant, but $\vect{r}(t)$ changes direction as time changes.
That change in direction corresponds to a change to the vector $\vect{r}(t)$ and thats
why it corresponds to the arc swept btween $t$ and $t+ \Delta t$.
\\~\\

\textbf{Proposition 4.3}, its proof is similar to an exercise we did for Folland, see
\ref{folland:1.1.7}


%%%%%%%%%%%%%%%%%%%%%%%%%%%%%%%%%%%%%%%%%%%%%%%%%%%%%%%%%%%%%%%%%%%%%%%%%%
\subsection{Equations for Planes and Distance Problems}
\section{Setting the Stage}

\subsection{Euclidean Spaces and Vectors}

\subsubsection{Exercises}

\textbf{1.1.2}

Given $\vec{x}, \vec{y} \in \mathbb{R}^n$,
\begin{align*}
|\vec{x} + \vec{y} |^2 &= (\vec{x} +\vec{y}) (\vec{x} +\vec{y}) \\
&= \vec{x}\cdot\vec{x} + \vec{y}\cdot\vec{y} + 2\vec{x}\cdot\vec{y} \\
&= |\vec{x}|^2 + |\vec{y}|^2 + 2\vec{x}\cdot\vec{y}
\end{align*}

Similarly,
\begin{align*}
|\vec{x} - \vec{y} |^2 &= (\vec{x} - \vec{y}) (\vec{x} - \vec{y}) \\
    &= \vec{x}\cdot\vec{x} + \vec{y}\cdot\vec{y} - 2\vec{x}\cdot\vec{y} \\
    &= |\vec{x}|^2 + |\vec{y}|^2 - 2\vec{x}\cdot\vec{y}
\end{align*}

Hence
$$
|\vec{x} + \vec{y} |^2 + |\vec{x} - \vec{y} |^2 = 2\left( |\vec{x}|^2 + |\vec{y}|^2 \right)
$$
\\~\\



\textbf{1.1.7}

Suppose $\vec{a}, \vec{b} \in \mathbb{R}^3$
\\

Show that if $\vec{a}\cdot\vec{c} = \vec{b}\cdot\vec{c}$ and
$\vec{a}\times\vec{c} = \vec{b}\times\vec{c}$ for some non-zero $\vec{c} \in\mathbb{R}^3$,
then $\vec{a} = \vec{b}$.
\\

We could try to simply stare at
$$
\vec{a}\cdot\vec{c} = |\vec{a}| |\vec{c}| \cos{\theta_1} 
= |\vec{b}| |\vec{c}| \cos{\theta_2} = \vec{b}\cdot\vec{c}
$$
Which tells us
$$
|\vec{a}| \cos{\theta_1} = |\vec{b}| \cos{\theta_2}
$$

Let's try something else,
$$
|a\times c|^2 = |a||c| - (a\cdot c)^2
= |b||c| - (b\cdot c)^2 = |b \times c|^2
$$
We now have
$$
|a||c| - (a\cdot c)^2 = |b||c| - (b\cdot c)^2
$$
or
$$
|a||c| = |b||c| \rightarrow |a| = |b|
$$

So we can go back to our first attempt and see that
$$
|a| \cos{\theta_1} = |b| \cos{\theta_2} \rightarrow \cos{\theta_1} = \cos{\theta_2}
$$
\\~\\



\textbf{1.1.8}

To see that $a\cdot(b\times c)$ is the determinant of the three vectors,
simply write out the determinant for $b\times c$ and note that the explicit version of it
is a "normal" vector.
Since the dot product is defined as $x\cdot y = x_1 y_1 + x_2 y_2 + \ldots + x_n y_n$, when
$x, y \in \mathbb{R}^n$.

Putting these two facts together we can see how $a\cdot(b\times c)$ can be computed via a single determinant
operation.
\\~\\



%%%%%%%%%%%%%%%%%%%%%%%%%%%%%%%%%%%%%%%%%%%%%%%%%%%%%%%%%%%%%%%%%%%%%%%%%
\subsection{Subsets of Euclidean Space}

\textbf{Proposition 1.4}

Remember that to be a boundary point of $S \subset \mathbb{R}^n$ \textbf{every}
ball centered at $\vec{x}$ must contain points in $S$ and in $S^c$.

If $\vec{x}$ is an interior point, then for some $\vec{x} \in S$,
there is a ball $B(r,\vec{x}) \subset S$;
but also, there is an $B(r, \vec{x}) \not\subset S^c$ for some $r>0$.
This is why you ought to be an interior point in $S$, in $S^c$, or be a boundary point
- because to be a boundary point every single $B(r, \vec{x})$ must be in $S$ AND in $S^c$.

This last statement is why it is also the case that if $S$ is closed then $S^c$ must be open:
because any points must be interior points to either one or be a boundary point.
So if $S$ has all of the boundary points, then its compliment must be left with none and thus only
have interior points and be an open set.
\\~\\







\subsubsection{Exercises}
\section{Euclidean Spaces}


subsection{Smooth Functions on Euclidean Space}

\subsubsection{$C^\infty$ Versus Analytic Functions}

Starting out with the simplest notation that anyone may have seen, we can write the partial drivates
as
$$
\frac{\partial^2 f}{\partial x \partial y}
$$

There is a theorem called \textbf{Schwarz's theorem (or Clairaut's theorem on equality of mixed partials)}
that states that for a function $f : U \subset \mathbb{R}^n \rightarrow \mathbb{R} $
(note that $U$ is an open set - $U \subset \mathbb{R}^n$ not $U \subseteq \mathbb{R}^n$)
if $p = (p^1, p^2, \ldots , p^n) \in \mathbb{R}^n$
and $p \in U$ (some neighboorhod of $p$ is contained in $U$)
and $f$ has \textbf{continuous second partial derivatives on that neighboorhod of $p$},
then for all $i$ and $j$ in $\{ 1, 2, \ldots , n \}$,

$$
\frac{\partial^2 f (p)}{\partial x^i \partial x^j}
= \frac{\partial^2 f (p)}{\partial x^j \partial x^i}
$$

\textbf{The partial derivatives of this function commute at that point.}
\\

We wanted to mention Schwarz's theorem on equality of mixed partial derivatives to present some of the context
we will be visiting over and over.
Now, let's carry on with a little bit more notation.
From our abstraction in the statement of Schwarz's theorem, we can see that mixed partial derivates have the following
form,

$$
\frac{\partial^{i+j+k} f}{\partial x^i \partial x^j \partial x^k}
$$

In Tu's notation the above then becomes
$$
\frac{\partial^j f}{\partial x^{i_1} \ldots \partial x^{i_j}}
$$
where $\partial x^{i_1}$ means that we are taking the partial derivative of degree $i$ with respect to the
coordinate $x^1$.
There is also the implicit requirement that the $j$ partial derivatives of degree $i$ will sum to $j$, so essentially
$i=1$.

The above notation cold very well be reworded into
$$
\frac{\partial^\alpha f}{\partial x^{\alpha_1}_{1} \ldots \partial x^{\alpha_j}_{j}}
$$
where $\alpha = \alpha_1 + \ldots + \alpha_j$, we also moved the superscript for the dimension to a subscript just
to simplify the latex - but keep in mind that \textbf{Tu uses the second superscript to note the dimension}.
%%%%%%%%%%%%%%%%%%%%%%%%%%%%%%%%%%%%%%%%%%%%%%%%%%%%%%%%%%%%%%%%%%%%
\section{Morse Theory}

Critical points can be degenerate or non-degenerate.
A non-degenerate critical point has a Hessian that can be inverted - the Hessian of the function $f$ does not vanish
at the critical point.

One associated a number called the \textbf{index}, the number of independent directions in which
$f$ decreases from a critical point.
More precisely, the index of a non-degenerate critical point $p$ of $f$ is the dimension of the largest subspace
of the tangent space to $M$ at $p$, where $M$ is a "landscape surface" function and $f : M \rightarrow \mathbb{R}$,
on which the Hessian of $f$ is negavite definite.

In mathematics, a symmetric matrix $M$ with real entries is positive-definite if the real number $z^T M z$
is positive for every nonzero real column vector $z$.
More generally, a Hermitian matrix (that is, a complex matrix equal to its conjugate transpose) is positive-definite
if the real number $z^* M z$ is positive for every nonzero complex column vector $z$.

The \textbf{index} also corresponds to the number of negative eigenvalues of the Hessian matrix at the critical cpoint $p$.


%%%%%%%%%%%%%%%%%%%%%%%%%%%%%%%%%%%%%%%%%%%%%%%%%%%%%%%%%
\subsubsection{Morse's Lemma}

This lemma is fundamental because it tells us that near a non-degenerate critical point, the function looks
like a quadratic form, and this form is determined by the signature of the Hessian matrix
(i.e., the number of positive and negative eigenvalues).

For example, suppose we have $f(x,y) = x^2 + y^2 - 2x + 4y$, $f : \mathbb{R}^2 \rightarrow \mathbb{R}$.

First, we find the critical point $\nabla f(x,y) = 0$,
\begin{align*}
\nabla f &= \left< \partial_x f, \partial_y f \right> \\
&= \left< 2x -2, 2y + 4 \right>
\end{align*}

and we set $\left( 2x -2, 2y + 4 \right) = \left( 0, 0 \right)$.
So we have $2x - 2 = 0$ or $x = 1$.
And then $2y + 4 = 0$, or $y = -2$.
So $p = (1, -2)$.

Now, let's see if the critical point is non-degenerate.
For that, let's compute the Hessian.
Remember that the Hessian is given by,
$$
H_f
=
\begin{pmatrix} 
    \dfrac{\partial^2 f}{\partial x_{1}^{2}}         & \dfrac{\partial^2 f}{\partial x_1 \partial x_2} & \dots  & \dfrac{\partial^2 f}{\partial x_1 \partial x_n}  \\[2.2ex]
    \dfrac{\partial^2 f}{\partial x_2 \partial x_1}  & \dfrac{\partial^2 f}{\partial x_{2}^{2}}        & \dots  & \dfrac{\partial^2 f}{\partial x_2 \partial x_n}  \\[2.2ex]
    \vdots                                           & \vdots                                          & \ddots & \vdots                                           \\[2.2ex]
    \dfrac{\partial^2 f}{\partial x_n \partial x_1}  & \dfrac{\partial^2 f}{\partial x_n \partial x_2} & \dots  & \dfrac{\partial^2 f}{\partial x_{n}^{2}}         \\
\end{pmatrix}
$$

That is, the entry of the $i$-th row and the $j$-th column is
$$
\left(H_f\right)_{i,j} = \frac{\partial^2 f}{\partial x_i \partial x_j}
$$

So we want to compute
\begin{align*}
H_f
&=
\begin{pmatrix} 
    \dfrac{\partial^2 f}{\partial x^{2}}         & \dfrac{\partial^2 f}{\partial x \partial y}  \\[2.2ex]
    \dfrac{\partial^2 f}{\partial y \partial x}  & \dfrac{\partial^2 f}{\partial y^{2}}         \\
\end{pmatrix} \\
&= \begin{pmatrix} 
    \partial_x \partial_x f  & \partial_x \partial_y f  \\[2.2ex]
    \partial_y \partial_x f  & \partial_y \partial_y f  \\
\end{pmatrix} \\
&= \begin{pmatrix} 
    \partial_x (2x -2)  & \partial_x (2y + 4)  \\[2.2ex]
    \partial_y (2x -2)  & \partial_y (2y + 4)  \\
\end{pmatrix} \\
&= \begin{pmatrix} 
    2  & 0  \\
    0  & 2  \\
\end{pmatrix}
\end{align*}

Given that all the entries on the diagonal are positive and there are no off-diagonal terms, it suggests that 
$f$ has a parabolic behavior in both the $x$ and $y$ directions independently,
which typically indicates a local minimum at the critical point if the function is convex or a saddle point if it is not. 

Anyway, the Hessian is invertible and its eigenvalues are positive, so $p$ is a minimum.
\\

Now we can apply Morse' lemma!
Since both eigenvalues of our Hessian are positive, then index, $k$, is 0 (no negative eigenvalues),
Morse's lemma tells us that near $p$ we can find coordinates $(u, v)$ such that
$$
f(u, v) = f(p) + u^2 + v^2 = -5 + u^2 +v^2
$$

So near the critical point, $f$ looks like a \textbf{paraboloid} downshifted by 5.

$$
f(x) = f(p) - x_{1}^{2} - \ldots - x_{\gamma}^{2} + x_{\gamma +1}^{2} + \ldots + x_{n}^{2}
$$

$\gamma$ is the index of $f$ at $p$.
The index $\gamma$ essentially determines how many negative squared terms appear in the local form of the function.
Our index was 0 so we didn't have any.

In Morse's lemma, when it says there exists a chart $(x_1, x_2, \ldots , x_n)$
it means that near the critical point $p$, you can introduce a new coordinate system where:

\begin{itemize}
    \item Each $x_i$ for $i = 1, \ldots , \gamma$ corresponds to a direction in which the function decreases
        (related to the negative eigenvalues of the Hessian).
    \item Each $x_i$ for $i = \gamma +1, \ldots, n$ corresponds to a direction in which the function increases
        (related to the positive eigenvalues of the Hessian).
\end{itemize}

The coordinates $x_i (p) = 0$ imply that the critical point $p$ is at the origin of this new coordinate system.
The transformation to this new coordinate system is achieved by a linear change of variables that diagonalizes
the Hessian matrix (through a process similar to eigendecomposition).

So here we added $u^2 + v^2$ because our Hessian matrix had 2 positive eignevalues.

%%%%%%%%%%%%%%%%%%%%%%%%%%%%%%%%%%%%%%%%%%%%%%%%%%%%%%%%%
\subsubsection{Matrix Diagonalization and Eigendecomposition}



%%%%%%%%%%%%%%%%%%%%%%%%%%%%%%%%%%%%%%%%%%%%%%%%%%%%%%%%%
\subsubsection{Homotopy}

If $M$ is a "landscape surface" function and $f : M \rightarrow \mathbb{R}$,
and $M^a = f^{-1} (-\infty, a]$,
The topology of $M^{a}$ does not change except when $a$ passes the height of a critical point;
at this point, a $\gamma$-cell is attached to $M^{a}$, where $\gamma$ is the index of the point.
This does not address what happens when two critical points are at the same height,
which can be resolved by a slight perturbation of $f$.
In the case of a landscape or a manifold embedded in Euclidean space, this perturbation might simply be tilting slightly, rotating the coordinate system. 


\end{document}