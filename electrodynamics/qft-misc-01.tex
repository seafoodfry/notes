%%%%%%%%%%%%%%%%%%%%%%%%%%%%%%%%%%%%%%%%%%%%%%%%%%%%%%%%%%%%%%%%%%%%%%%%%%%%%%%%%
\section{Frobenius Method}

\subsection{A Generic Example}

The following example comes from
\href{https://en.wikipedia.org/wiki/Frobenius_method}{wikipedia: frobenius method}:

$$
z^2 f^{\prime\prime} - zf^\prime + (1-z)f = 0
$$

We assume the solution has the form,
$$
f = \sum_{k=0} A_k z^{k+r}
$$

Plugging the above series into the ODE and organizing like terms we land in
$$
\sum_k \left[ (k+r)(k+r-1) - (k+r) + (1-z) \right] A_k z^{k+r} = 0
$$
which can be further simplified to

$$
\sum_k \left[ (k+r)(k+r-1) - (k+r) + 1 \right] A_k z^{k+r}
    - \sum_k A_k z^{k+r+1} 
= 0
$$

The indicial polynomial can then be obtained by considering the lowest power of $z$, which is $z^r$,
$$
(r(r-1) + 1) A_0 z^r
$$

We now set $z=0$ and obtain
$$
r(r-1) + 1 = r^2 -r + 1 = (r-1)^2 = 0
$$

Meaning that $r=1$ (double root).

We now need to find the recurrence relation by plugging $f = \sum_{k=0} A_k z^{k+1}$ back into our original proble,
equate coefficients of like powers of $z$, and find a recurrence relation for $A_k$.

We now have,
$$
f^\prime = \sum_{k=0} A_k (k+1) z^k
$$
and
$$
f^{\prime\prime} = \sum_{k=0} A_k k(k+1) z^{k-1}
$$

And our ODE now becomes,
\begin{align*}
& z^2 \sum_{k} A_k k(k+1) z^{k-1}
    - z \sum_{k} A_k (k+1) z^k
    + (1-z) \sum_{k} A_k z^{k+1} \\
&= \sum_{k} A_k k(k+1) z^{k+1}
    - \sum_{k} A_k (k+1) z^{k+1}
    + \sum_{k} A_k z^{k+1}
    - \sum_{k} A_k z^{k+2} \\
&= \sum_k \left[ k(k+1) - (k+1) + 1 \right] A_k z^{k+1} - \sum_{k} A_k z^{k+2} \\
&= \sum_k k^2 A_k z^{k+1} - \sum_{k} A_k z^{k+2}
\end{align*}

To find a recurrence relation we need to align the sums.
So we can write
\begin{align*}
& \sum_{k=0} k^2 A_k z^{k+1} - \sum_{k=0} A_k z^{k+2} \\
&= \sum_{k=0} k^2 A_k z^{k+1} - \sum_{k=1} A_{k-1} z^{k+1}
\end{align*}

We could do a similar shift for the first sum but since the first term $k=0$ has a $k^2$ term then we can ignore it.
Thus,
$$
\sum_{k=1} \left( k^2 A_k - A_{k-1} \right) z^{k+1} = 0
$$

Since this sum must be zero for all $z$, the coefficients must satisfy,
$$
k^2 A_k - A_{k-1} = 0
$$
or
$$
A_k = \frac{ A_{k-1} }{k^2} 
$$

This recurrence realtionship now gives us all coefficients in terms of $A_0$.
The first couple coefficients are,
$$
A_1 = \frac{ A_0 }{1^2} = A_0
$$

$$
A_2 = \frac{ A_1 }{2^2} = \frac{ A_0 }{2^2}
$$

$$
A_3 = \frac{ A_2 }{3^2} = \frac{ A_0 }{2^2 3^2}
$$

This means that our general solution looks like
$$
f(z) = \sum_{k=0} A_k z^{k+1}
= A_0 \sum_{k=0} \frac{ z^{k+1} }{k!^2}
$$



%%%%%%%%%%%%%%%%%%%%%%%%%%%%%%%%%%%%%%%%%%%%%%%%%%%%%%%%%%%%%%%%%%%

\subsection{Legendre Equation}

Using the nomenclature of Arfken,
$$
\mathcal{L} y(x) =
-(1-x^2) y^{\prime\prime} + 2x y^\prime = \lambda y
$$

Which can also be expressed as
$$
(1-x^2) y^{\prime\prime} - 2x y^\prime + \lambda y = 0
$$

If we assume the solution is of the form
$$
y(x) =
\sum_{j=0} a_j x^{j+s}
$$

then the Legendre equation can be written as

\begin{align*}
& (1-x^2) \sum_{j} a_j (j+s)(j+s-1) x^{j+s-2}
    - 2x \sum_{j} a_j (j+s) x^{j+s-1}
    + \lambda \sum_{j} a_j x^{j+s} \\
&= \sum_{j} a_j (j+s)(j+s-1) x^{j+s-2}
    - \sum_{j} a_j (j+s)(j+s-1) x^{j+s}
    -2 \sum_{j} a_j (j+s) x^{j+s}
    + \lambda \sum_{j} x^{j+s} \\
&= \sum_{j} a_j (j+s)(j+s-1) x^{j+s-2}
    + \sum_j \left[ -(j+s)(j+s-1) -2(j+s) + \lambda \right] a_j x^{s+j}
\end{align*}

The indicial polynomial then comes from considering the lowest power of $x$, which is the first sum
$$
a_j (j+s)(j+s-1) x^{j+s-2}
$$
which for an arbitrary $x$ when $j=0$, it becomes
$$
s(s-1) = 0
$$
whose solutions are $s=0$ and $s=1$.

In this case, the solution will be a linear combination,

$$
y(x) = y_1 (x) + y_2 (x)
= \sum_j a_j x^{j} + \sum_j b_j x^{j+1}
$$

Let's start with the $s=0$ case then.
Plugging $y_1 (x)$ back into our ODE we have

\begin{align*}
& (1-x^2) \sum_j a_j j(j-1) x^{j-2}
    - 2x \sum_j a_j j x^{j-1}
    + \lambda \sum_j a_j x^j \\
&= \sum_j a_j j(j-1) x^{j-2}
    - \sum_j a_j j(j-1) x^{j}
    - 2 \sum_j a_j j x^{j}
    + \lambda \sum_j a_j x^j \\
&= \sum_j a_j j(j-1) x^{j-2}
    + \sum_j \left[ -j(j-1) -2j + \lambda \right] a_j x^j
\end{align*}

Now, to align the powers of $x^{j}$.
First, note that the first two terms in
$\sum_j a_j j(j-1) x^{j-2}$ are $0$.
So we might as well start at $j=2$
$$
\sum_{j=2} a_j j(j-1) x^{j-2}
$$

If we shift it back to $0$, we can align the powers of $x$.
$$
\sum_{j=2} a_j j(j-1) x^{j-2} =
\sum_{j=0} a_{j+2} (j+2)(j+1) x^{j}
$$

This gives us,
$$
\sum_{j=0} a_{j+2} (j+2)(j+1) x^{j}
+ \sum_j \left[ -j(j-1) -2j + \lambda \right] a_j x^j
= 0
$$

Now, we can equate coefficients,
$$
a_{j+2} (j+2)(j+1)
    + \left[ -j(j-1) -2j + \lambda \right] a_j = 0
$$
or
$$
a_{j+2} = \frac{ j(j-1) + 2j - \lambda }{ (j+1)(j+2) } a_j
= \frac{ j(j+1) - \lambda  }{ (j+1)(j+2) } a_j
$$
\\~\\

In the $s=1$ case, if we plug in $y_2$ into our ODE,

\begin{align*}
& (1-x^2) \sum_j a_j j(j+1) x^{j-1}
    - 2x \sum_j a_j (j+1) x^{j}
    + \lambda \sum_j a_j x^{j+1} \\
&= \sum_j a_j j(j+1) x^{j-1}
    - \sum_j a_j j(j+1) x^{j+1}
    - 2 \sum_j a_j (j+1) x^{j+1}
    + \lambda \sum_j a_j x^{j+1} \\
&= \sum_j a_j j(j+1) x^{j-1}
    + \sum_j \left[ -j(j+1) -2(j+1) + \lambda \right] a_j x^{j+1}
\end{align*}

Following the procedure we have been building: look for a series with redundant terms, and then try to shift it to match
the other powers of $x$.
Here we have
$$
\sum_{j=0} a_j j(j+1) x^{j-1}
$$
that can be written as
$$
\sum_{j=1} a_j j(j+1) x^{j-1}
$$
Since the term when $j=0$ equals to 0.
And we can conveniently shift this series back one to match the $x^j$ terms of the other sum:

$$
\sum_{j=1} a_j j(j+1) x^{j-1}
=
\sum_{j=0} a_{j+1} (j+1)(j+2) x^{j}
$$

At this point we have,

$$
\sum_{j=0} a_{j+1} (j+1)(j+2) x^{j}
    + \sum_j \left[ -j(j+1) -2(j+1) + \lambda \right] a_j x^{j+1}
$$

We need to do another shift and doing it on the first sum seems like the simplest approach.
If we do so, then we can arrive at

$$
\sum_{j=-1} a_{j+2} (j+2)(j+3) x^{j+1}
    + \sum_j \left[ -j(j+1) -2(j+1) + \lambda \right] a_j x^{j+1}
$$

But let's separate the first term of the first sum.
When $j=-1$, we get $a_1 (1)(2) x^0 = 2a_1$.
Thus,

$$
2 a_1 +
    \sum_{j=0} a_{j+2} (j+2)(j+3) x^{j+1}
    + \sum_j \left[ -j(j+1) -2(j+1) + \lambda \right] a_j x^{j+1} = 0
$$

And this equation must hold for all $x$, so the coefficients of each power of $x$ must be zero.
This means that $a_1 must be 0$.
And we are left with
$$
a_{j+2} (j+2)(j+3)
    + \left( -j(j+1) -2(j+1) + \lambda \right) a_j = 0
$$

Which simplifies to
$$
a_{j+2} = \frac{ j^2 + 3j + 2 - \lambda }{ (j+2)(j+3) } a_j
= \frac{ (j+1)(j+2) - \lambda }{ (j+2)(j+3) } a_j
$$

In order to obtain the eignevalues we need to go back to the beginning.
If you remember, this is how the problem was initially framed:
$$
\mathcal{L} y(x) =
-(1-x^2) y^{\prime\prime} + 2x y^\prime = \lambda y
$$

So whatever we can do to find a $\lambda$ will give us our eigenvalues.
And that is why Arfken and other resources look for the conditions that make both of the recursion relations result in
a series solution that converges.
In the first case we need a value of lambda that will cancel the numerator $j(j+1)$ and in the second we need a value that
will cancel the $(j+1)(j+2)$.
