\section{Introduction to Electrostatics}

\subsection{Gauss's Law}


%%%%%%%%%%%%%%%%%%%%%%%%%%%%%%%%%%%%%%%%%%%%%%%%%%%%%%%%%%%%%%%%%%%%%%%
\subsubsection{Surface elements $da$}

The following logic comes from Griffith's electrodynamics chapter 1.4.

In curvilinear coordinates, an infinitesimal displacement in the $\vect{\hat{r}}$ direction is
simply $dr$.
$$
dl_r = dr
$$

An infinitesimal element of length in the $\vect{\hat{\theta}}$ is
$$
dl_\theta = r d\theta
$$

And an infinitesimal element of length along the $\vect{\hat{\phi}}$ direction
is
$$
dl_\phi = r \sin\theta d\phi
$$

And thus, the general infinitesimal displacement becomes
$$
d\vect{l} =
dr \vect{\hat{r}}
+ r d\theta \vect{\hat{\theta}}
+ r \sin\theta d\phi \vect{\hat{\phi}}
$$

From there one can see how the infinitesimal volume element $d\tau$ is then
$$
d\tau = dl_r dl_\theta dl_\phi =
r^2 \sin\theta dr d\theta d\phi
$$

\textbf{However, the inifinitesimal area element depends on the orientation of the surface!}

For example, if we integrate over the surface of a sphere, where r stays contant but $\theta$
and $\phi$ vary,
$$
d\vect{a} = dl_\theta dl_\phi \vect{\hat{r}} = r^2 \sin\theta d\theta d\phi \vect{\hat{r}}
$$

Where as is the surface lies in the x-y plane, so that $\theta$ is contant, and let's say $\theta = \phi/2$,
while $r$ and $\phi$ vary,
$$
d\vect{a} = dl_r dl_\phi \vect{\hat{\theta}}
= r \sin\theta dr d\phi \vect{\hat{\theta}}
= r dr d\phi \vect{\hat{\theta}}
$$

\textbf{The area element on the sphere can be calculated from the cross products of other two elements,}

You should also, definetely, see
\href{https://en.wikipedia.org/wiki/Spherical_coordinate_system#Integration_and_differentiation_in_spherical_coordinates}{Integration and differentiation in spherical coordinates}.


%%%%%%%%%%%%%%%%%%%%%%%%%%%%%%%%%%%%%%%%%%%%%%%%%%%%%%%%%%%%

\subsubsection{Solid Angle}

The following is all wikipedia.

Whereas an angle in radians, projected onto a circle, gives a length of a circular arc
on the circumference, a solid angle in steradians, projected onto a sphere,
gives the area of a spherical cap on the surface.

The point from which the object is viewed is called the apex of the solid angle,
and the object is said to subtend its solid angle at that point.

In SI units, a solid angle is expressed in a dimensionless unit called a \textbf{steradian}.

One steradian corresponds to one unit of area on the unit sphere surounding the apex,
so an object that blocks all rays from the apex would cover a number of steradians equal
to the total surface area of the unit sphere, $4\pi$.


Just like a planar angle in radians is the ratio of the length of an arc to its radius, $\theta = s/r$,
a solid angle in steradians is the ratio of the area covered on a sphere by an object to the area given
by the square of the radius of said sphere. The formula is
$$
\Omega = \frac{A}{r^2}
$$
where A is the spherical surface area and r is the radius of the considered sphere.

Any area on a sphere which is equal in area to the square of its radius,
when observed from its center, subtends precisely one steradian.

The solid angle of a sphere measured from any point in its interior is $4\pi$ sr (steridians),
and the solid angle subtended at the center of a cube by one of its faces is one-sixth of that,
or $2\pi/3$ sr.

In spherical coordinates there is a formula for the differential,
$$
d\Omega = \sin \theta d\theta d\varphi 
$$
where $\theta$ is the colatitude (angle from the North Pole - z-axis range $[0,\pi]$)
and $\varphi$ is the longitude (range $[0,2\pi]$).

The solid angle for an arbitrary oriented surface S subtended at a point P is equal to
the solid angle of the projection of the surface S to the unit sphere with center P,
which can be calculated as the surface integral:
$$
\Omega =\iint _{S} \frac{\hat{r}\cdot\hat{n}}{r^2} \,dS
= \iint_{S} \sin \theta \,d\theta \,d\varphi
$$

where $\hat{r} = \vec{r}/r$ is the unit vector corresponding to $\vec {r}$,
the position vector of an infinitesimal area of surface $dS$ with respect to point P,
and where $\hat{n}$ represents the unit normal vector to $dS$.
Even if the projection on the unit sphere to the surface S is not isomorphic,
the multiple folds are correctly considered according to the surface orientation
described by the sign of the scalar product $\hat{r}\cdot\hat{n}$.

Thus one can approximate the solid angle subtended by a small facet having flat surface area
$dS$, orientation $\hat{n}$, and distance r from the viewer as:
$$
d\Omega = 4\pi \left( \frac{dS}{A} \right) (\hat{r}\cdot\hat{n})
$$
where the surface area of a sphere is $A = 4\pi r^2$.


%%%%%%%%%%%%%%%%%%%%%%%%%%%%%%%%%%%%%%%%%%%%%%%%%%%%%%%%%%%%
\subsubsection{Explanation of Jackson}

If the electric field makes an angle $\theta$ with the unit normal, then the projection of the
infinitesimal area element along the normal of the surface is $\cos\theta da$.
Using the solid angle formula, we then get $d\Omega = \cos\theta da / r^2$ or
$r^2 d\Omega = \cos\theta da$.



%%%%%%%%%%%%%%%%%%%%%%%%%%%%%%%%%%%%%%%%%%%%%%%%%%%%%%%%%%%%%%%%%%%%%%%%%%%%%%%%%%%%%%%%%%%%%%%%%%%%%%%%%%%%%%%%%%%%%%
\subsection{Exercises}

%%%%%%%%%%%%%%%%%%%%%%%%%%%%%%%%%%%%%%%%%%%%%%%%%%%%%%%%%%%%%%%
\textbf{1.2}
\\

The Dirac delta function in three dimensions can be taken as the improper limit as $\alpha \rightarrow 0$ of the
Gaussian function
$$
D\left(\alpha ; x, y, z \right) =
    \frac{1}{\left(\alpha \sqrt{2\pi}\right)^3}
    \exp{ \left[ -\frac{1}{2\alpha^2} \left( x^2 + y^2 + z^2 \right) \right] }
$$

Consider a general orthogonal coordinate system specified by the surfaces $u$ = constant, $v$ = constant,
$w$ = constant, with length elements $du/U$, $dv/V$, $dw/W$ in the three perpendicular directions.
Show that
$$
\delta\left( \mathbf{x} - \mathbf{x}^\prime \right) =
    \delta\left(u - u^\prime\right) \delta\left(v - v^\prime\right) \delta\left(w - w^\prime\right) \cdot UVW
$$

by considering the limit of the Gaussian above.
Note that as $\alpha \rightarrow 0$ only the infinitesimal length
element need be used for the distance between the points in the exponent.
\\

Browsing through the web, one will see the following defintions of the delta function,
\begin{align*}
\delta (x) &= \lim_{b\rightarrow 0} \frac{1}{|b|\sqrt{\pi}} \exp{ \left[ -\frac{1}{b^2} x^2 \right]} \\
    &= \lim_{b\rightarrow 0^+} \frac{1}{2\sqrt{\pi b}} \exp{ \left[ -\frac{1}{4b} x^2 \right]} \\
    &= \lim_{b\rightarrow 0} \frac{1}{b \sqrt{2\pi}} \exp{ \left[ -\frac{1}{2b^2} x^2 \right]}
\end{align*}

The thing to note here is how the arguments of the exponent make it into the constant that's infront of the exponent.
And the reason to even look at this now is because we found this amazingly interesting approach to this problem in
\href{https://www.wtamu.edu/~cbaird/courses.html}{West Texas CS Baird's} site:

We start with
$$
D\left(\alpha ; x, y, z \right) =
    \frac{1}{\left(\alpha \sqrt{2\pi}\right)^3}
    \exp{ \left[ -\frac{1}{2\alpha^2} \left( x^2 + y^2 + z^2 \right) \right] }
$$
and then we make a change of variables, a displacement if you will, by changing $x \rightarrow x - x^\prime$, etc.
This gives us
$$
D\left(\alpha ; x-x^\prime, y-y^\prime, z-z^\prime \right) =
    \frac{1}{\left(\alpha \sqrt{2\pi}\right)^3}
    \exp{ \left[ -\frac{1}{2\alpha^2} \left( (x-x^\prime)^2 + (y-y^\prime)^2 + (z-z^\prime)^2 \right) \right] }
$$

Now comes the limit taking.
First off is whether the limit is some actual number orr just infinity (we are skipping over the question of existence).
Since an exponent grows faster than a ever-increasing number to a power,
$\lim_{\alpha\rightarrow \infty} \alpha^3 \cdot e^{-\alpha} = 0$,
then we would expect our function $D$ to be bounded.
But, anyway, if we take the limit as $\alpha\rightarrow 0$, and if we look for a case where $D\not\rightarrow 0$,
then we would want $x-x^\prime$ to "balance out" $\alpha$ just enough so that we have some non-zero bounded value.

Let's assume the above is possible and rewirte $D$ as
$$
D\left(\alpha ; x-x^\prime, y-y^\prime, z-z^\prime \right) =
    \frac{1}{\left(\alpha \sqrt{2\pi}\right)^3}
    \exp{ \left[ -\frac{1}{2\alpha^2} \left( (dx)^2 + (dy)^2 + (dz)^2 \right) \right] }
$$

The infinitesimal displacements in the argument to the exponent look like an arc lenght, so let's rewrite it now as
$$
D\left(\alpha ; x-x^\prime, y-y^\prime, z-z^\prime \right) =
    \frac{1}{\left(\alpha \sqrt{2\pi}\right)^3}
    \exp{ \left[ -\frac{1}{2\alpha^2} \left( ds \right)^2 \right] }
$$

Since an arc length is a concept independent of the coordinate system, we can use our given general orthogonal coordinate system,
and rewrite our equation once more,
$$
D\left(\alpha ; u-u^\prime, v-v^\prime, w-w^\prime \right) =
    \frac{1}{\left(\alpha \sqrt{2\pi}\right)^3}
    \exp{ \left[ -\frac{1}{2\alpha^2} \left( 
        \left(\frac{du}{U}\right)^2 + \left(\frac{dv}{V}\right)^2 + \left(\frac{dw}{W}\right)^2 \right) 
    \right] }
$$

And if we reverse the process, by writing the differentials as differences,
$$
D\left(\alpha ; u-u^\prime, v-v^\prime, w-w^\prime \right) =
    \frac{1}{\left(\alpha \sqrt{2\pi}\right)^3}
    \exp{ \left[ -\frac{1}{2\alpha^2} \left(
        \left(\frac{u-u^\prime}{U}\right)^2 + \left(\frac{v-v^\prime}{V}\right)^2 + \left(\frac{w-w^\prime}{W}\right)^2 \right) 
    \right] }
$$

We can now look at the first couple definition of the delta function and make a final change of variahles.
For example, if we let $b_1 = \alpha U$, $b_2 = \alpha V$, and $b_3 = \alpha W$,
$$
D =
    \left[ \frac{
            \exp{ \left[ -\frac{
                \left(u-u^\prime\right)^2
            }{2b_{1}^{2}}  \right] }
        }{b_1 \sqrt{2\pi}}
        U
     \right]
     \left[ \frac{
            \exp{ \left[ -\frac{
                \left(v-v^\prime\right)^2
            }{2b_{2}^{2}}  \right] }
        }{b_2 \sqrt{2\pi}}
        V
     \right]
     \left[ \frac{
            \exp{ \left[ -\frac{
                \left(w-w^\prime\right)^2
            }{2b_{3}^{2}}  \right] }
        }{b_3 \sqrt{2\pi}}
        W
     \right]
$$

If we now take the limit of the $b_i$s, then we get our desired result.
\\~\\