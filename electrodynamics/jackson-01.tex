\chapter{Introduction to Electrostatics}


%%%%%%%%%%%%%%%%%%%%%%%%%%%%%%%%%%%%%%%%%%%%%%%%%%%%%%%%%%%%%%%%%%%%%%%
\section{Charge distributions}


%%%%%%%%%%%%%%%%%%%%%%%%%%%%%%%%%%%%%%%%%%%%%%%%%%%%%%%%%%%%%%%%%%%%%%%
\subsection{Surface elements $da$}

The following logic comes from \cite{griffiths-em} chapter 1.4.

Remember that in spherical coordinates we have the \textbf{polar angle} $\theta \in [0, \pi]$,
which is the angle from the $z$ coordinate to the vector $\vect{r}$.
And the \textbf{azimuthal angle} $\phi$ has the range $\phi \in [0, 2\pi)$, and is the angle from
the coordinate $x$ to the line vector $\vect{r}$.

In spherical coordinates, an infinitesimal displacement in the $\vect{\hat{r}}$ direction is
simply $dr$.
$$
dl_r = dr
$$

An infinitesimal element of length in the $\vect{\hat{\theta}}$ direction (along the arc from the poles)
is
$$
dl_\theta = r d\theta
$$

And an infinitesimal element of length along the $\vect{\hat{\phi}}$ direction (along the arc from the equator)
is
$$
dl_\phi = r \sin\theta d\phi
$$

$r \sin\theta$ is the projection of the line vector $\vect{r}$ onto the $x-y$ plane.
$r \cos\theta$ would be the hight (projection onto the $x$ axis).
\\

The general infinitesimal displacement becomes
$$
d\vect{l} =
dr \vect{\hat{r}}
+ r d\theta \vect{\hat{\theta}}
+ r \sin\theta d\phi \vect{\hat{\phi}}
$$

From there one can see how the infinitesimal volume element $d\tau$ is then
$$
d\tau = dl_r dl_\theta dl_\phi =
r^2 \sin\theta dr d\theta d\phi
$$

This is a straightforward of justifying the volume element in spherical coordinates.
The other way to arrive at it is to use the Jacobian to derive how a volume element morphs when seen
changing to spherical coordinates.
See \ref{subsection:jacobian}.
\\


\textbf{However, the inifinitesimal area element depends on the orientation of the surface!}

For example, if we integrate over the surface of a sphere, where r stays contant but $\theta$
and $\phi$ vary,
$$
d\vect{a} = dl_\theta dl_\phi \vect{\hat{r}} = r^2 \sin\theta d\theta d\phi \vect{\hat{r}}
$$

Where as is the surface lies in the x-y plane, so that $\theta$ is contant, and let's say $\theta = \phi/2$,
while $r$ and $\phi$ vary,
$$
d\vect{a} = dl_r dl_\phi \vect{\hat{\theta}}
= r \sin\theta dr d\phi \vect{\hat{\theta}}
= r dr d\phi \vect{\hat{\theta}}
$$

\textbf{The area element on the sphere can be calculated from the cross products of other two elements,}

You should also, definetely, see
\href{https://en.wikipedia.org/wiki/Spherical_coordinate_system#Integration_and_differentiation_in_spherical_coordinates}{Integration and differentiation in spherical coordinates}.
\\~\\


\textbf{Example 1}

One good example to get practice is \cite{griffiths-em} Example 1.13: find the volume of a sphere of radius $R$.

As we saw above, the optimal way to an answer is to integrate the volume element $d\tau$ in spherical coordinates.

\begin{align*}
V = \int d\tau &=
\int_{r=0}^{R} \int_{\theta=0}^{\pi} \int_{\phi=0}^{2\pi} dr \, d\theta \, d\phi r^2 \sin\theta \\
&=  \left( \int_{r=0}^{R} dr \,r^2 \right)
    \left( \int_{\theta=0}^{\pi} d\theta \, \sin\theta \right)
    \left( \int_{\phi=0}^{2\pi} d\phi \right) \\
&= \left( \frac{1}{3} R^3 \right) \left( -\cos\theta \Big|_{0}^{\pi} \right) \left( 2\pi \right) \\
&= \frac{4}{3} \pi R^3
\end{align*}



%%%%%%%%%%%%%%%%%%%%%%%%%%%%%%%%%%%%%%%%%%%%%%%%%%%%%%%%%%%%%%%%%%%%%%%
\subsection{Fundamental Theorems} \label{section:funamental-theorems}

This presentation is a quick summary of \cite{griffiths-em} Sections 1.3.2 to 1.3.5.

The fundamental theorem of calculus states

$$
\int_{a}^{b} dx \, \left(\frac{df}{dx}\right) = f(b) - f(a)
$$

The fundamental theorem for gradients is

$$
\int_{\vect{a}}^{\vect{b}} \left(\nabla T\right) \cdot d\vect{l}
= T(\vect{b}) - T(\vect{a})
$$

Here $T$ is a scalar-values function, $T : \mathbb{R}^3 \rightarrow \mathbb{R}$, and the application of the
$\nabla$ operator ($\nabla = \left< \partial_x, \partial_y, \partial_z \right>$) makes $\nabla T$ a vector-valued function,
$\nabla T : \mathbb{R}^3 \rightarrow \mathbb{R}^3$.
The gradient points in the direction of maximum increase of the function $T$ and it expreses the rate of change of $T$
along a given direction.

The interesting bit of our fundamental theorem for gradients is that $\int_{\vect{a}}^{\vect{b}} \left(\nabla T\right) \cdot d\vect{l}$
is path independent.
And by consequence, $\oint \left(\nabla T\right) \cdot d\vect{l} = 0$
(if you go up the stairs, measure how many meters above sea-level you are, then go back down, and come back up, then your
change is 0).
\\

The fundamental theorem for divergences (Gauss's of Green's theorem) states that

$$
\int_{\mathcal{V}} \left(\nabla\cdot \vect{v}\right) \, d\tau
= \oint_{\mathcal{S}} \vect{v}\cdot d\vect{a}
$$

For the two previous fundamental theorems we saw that the boundary of a line was two points, and now we see that the boundary
of a volume $\mathcal{V}$ is the closed region $\mathcal{S}$.
Accordingly, if we sum how much quantity $\vect{v}$ spreads out over an entire volume, this is equal to how much
is $\vect{v}$ is crossing through the closed enclosing surface.
\\

The fundamental theorem for curls (Stoke's theorem) states that

$$
\int_{\mathcal{S}} \left( \nabla\times\vect{v} \right) \cdot d\vect{a}
= \oint_{\mathcal{P}} \vect{v} \cdot d\vect{l}
$$

Now the boundary of a region $\mathcal{S}$ is the perimeter (a line) $\mathcal{P}$.
Here we see that the circulation of $\vect{v}$, $\oint_{\mathcal{P}} \vect{v} \cdot d\vect{l}$, over a perimeter,
is equal to the sum of the flux of the "twisting" of $\vect{v}$ over a surface.


%%%%%%%%%%%%%%%%%%%%%%%%%%%%%%%%%%%%%%%%%%%%%%%%%%%%%%%%%%%%%%%%%%%%%%%
\subsection{Work done by a particle}

\textbf{Taken from Kline, Chapter 7, Section 7, problem 3}

AM is a straight vertical line.
MO is a straight horizontal line. O is to the right of M.

A particle moving along AM is attracted to a fixed point O with a force that varies inversely
as the square of the distance from O.
Find the work done on the particle as it moves from A to a distance B (between A and M).

Suggestion: let $x$ be the distance from A to any point P on AB
and let $r$ be the variable distance from P to O.
The force $F$ attracting P to O is $k/r^2$.
But only the component of this force along AB serves to move the particle along AB.
This component is $(k/r^2) \cos OPM = (k/r^2) (c-x)/r$.
Then $dW/dx = k (c-x)/r^3$ and $r = \sqrt{p^2 + (c-x)^2}$.
\\

\begin{align*}
W &= \int_{A}^{B} \frac{dW}{dx} dx \\
&= \int_{A}^{B} \frac{k (c-x)}{r^3} \\
&= \int_{A}^{B} \frac{k (c-x)}{\left(p^2 + (c-x)^2\right)^{3/2}} dx \\
\end{align*}

The trick here is to note that if we chose $u = p^2 + (c-x)^2$, then $du = -2(c-x) dx$, so that
$$
\frac{k (c-x)}{\left(p^2 + (c-x)^2\right)^{3/2}} dx \rightarrow -\frac{k}{2}\frac{1}{u^{3/2}} du
$$

Since $\int u^{-3/2} du = -2 u^{-1/2} + C$.
So putting everything together,
\begin{align*}
W &= \int_{A}^{B} \frac{k (c-x)}{\left(p^2 + (c-x)^2\right)^{3/2}} dx \\
&= -\frac{k}{2} (-2) \frac{1}{u^{1/2}} \Big|_{A}^{B} \\
&= k \left( \frac{1}{\sqrt{p^2 + (c-B)^2}} - \frac{1}{\sqrt{p^2 + (c-A)^2}} \right) \\
&= \frac{k}{r_B} - \frac{k}{r_A}
\end{align*}

\textbf{Note:} the two interesting bits to note from this exercise are first, when we were looking for the component
of motion that contributed to the work being done, the vertix of the angle used was the point we were "on".
Second, note how our the u-substitution we used went, doing it the normal way may confuse you
and try to get you for somewhere where to use $u = p^2 + (c-x)^2$, which is not needed.
\\~\\


%%%%%%%%%%%%%%%%%%%%%%%%%%%%%%%%%%%%%%%%%%%%%%%%%%%%%%%%%%%%%%%%%%%%%%%
\section{Gauss's Law}



%%%%%%%%%%%%%%%%%%%%%%%%%%%%%%%%%%%%%%%%%%%%%%%%%%%%%%%%%%%%

\subsection{Solid Angle}

The following is all wikipedia.

Whereas an angle in radians, projected onto a circle, gives a length of a circular arc
on the circumference, a solid angle in steradians, projected onto a sphere,
gives the area of a spherical cap on the surface.

The point from which the object is viewed is called the apex of the solid angle,
and the object is said to subtend its solid angle at that point.

In SI units, a solid angle is expressed in a dimensionless unit called a \textbf{steradian}.

One steradian corresponds to one unit of area on the unit sphere surounding the apex,
so an object that blocks all rays from the apex would cover a number of steradians equal
to the total surface area of the unit sphere, $4\pi$.


Just like a planar angle in radians is the ratio of the length of an arc to its radius, $\theta = s/r$,
a solid angle in steradians is the ratio of the area covered on a sphere by an object to the area given
by the square of the radius of said sphere. The formula is
$$
\Omega = \frac{A}{r^2}
$$
where A is the spherical surface area and r is the radius of the considered sphere.

Any area on a sphere which is equal in area to the square of its radius,
when observed from its center, subtends precisely one steradian.

The solid angle of a sphere measured from any point in its interior is $4\pi$ sr (steridians),
and the solid angle subtended at the center of a cube by one of its faces is one-sixth of that,
or $2\pi/3$ sr.

In spherical coordinates there is a formula for the differential,
$$
d\Omega = \sin \theta d\theta d\varphi 
$$
where $\theta$ is the colatitude (angle from the North Pole - z-axis range $[0,\pi]$)
and $\varphi$ is the longitude (range $[0,2\pi]$).

The solid angle for an arbitrary oriented surface S subtended at a point P is equal to
the solid angle of the projection of the surface S to the unit sphere with center P,
which can be calculated as the surface integral:
$$
\Omega =\iint _{S} \frac{\hat{r}\cdot\hat{n}}{r^2} \,dS
= \iint_{S} \sin \theta \,d\theta \,d\varphi
$$

where $\hat{r} = \vec{r}/r$ is the unit vector corresponding to $\vec {r}$,
the position vector of an infinitesimal area of surface $dS$ with respect to point P,
and where $\hat{n}$ represents the unit normal vector to $dS$.
Even if the projection on the unit sphere to the surface S is not isomorphic,
the multiple folds are correctly considered according to the surface orientation
described by the sign of the scalar product $\hat{r}\cdot\hat{n}$.

Thus one can approximate the solid angle subtended by a small facet having flat surface area
$dS$, orientation $\hat{n}$, and distance r from the viewer as:
$$
d\Omega = 4\pi \left( \frac{dS}{A} \right) (\hat{r}\cdot\hat{n})
$$
where the surface area of a sphere is $A = 4\pi r^2$.


%%%%%%%%%%%%%%%%%%%%%%%%%%%%%%%%%%%%%%%%%%%%%%%%%%%%%%%%%%%%
\subsection{Explanation of Jackson}

If the electric field makes an angle $\theta$ with the unit normal, then the projection of the
infinitesimal area element along the normal of the surface is $\cos\theta da$.
Using the solid angle formula, we then get $d\Omega = \cos\theta da / r^2$ or
$r^2 d\Omega = \cos\theta da$.



%%%%%%%%%%%%%%%%%%%%%%%%%%%%%%%%%%%%%%%%%%%%%%%%%%%%%%%%%%%%%%%%%%%%%%%%%%%%%%%%%%%%%%%%%%%%%%%%%%%%%%%%%%%%%%%%%%%%%%
\section{Problems}

%%%%%%%%%%%%%%%%%%%%%%%%%%%%%%%%%%%%%%%%%%%%%%%%%%%%%%%%%%%%%%%
\subsection{Problem 1.2}
\label{jackson:problem-1.2}


The Dirac delta function in three dimensions can be taken as the improper limit as $\alpha \rightarrow 0$ of the
Gaussian function
$$
D\left(\alpha ; x, y, z \right) =
    \frac{1}{\left(\alpha \sqrt{2\pi}\right)^3}
    \exp{ \left[ -\frac{1}{2\alpha^2} \left( x^2 + y^2 + z^2 \right) \right] }
$$

Consider a general orthogonal coordinate system specified by the surfaces $u$ = constant, $v$ = constant,
$w$ = constant, with length elements $du/U$, $dv/V$, $dw/W$ in the three perpendicular directions.
Show that
$$
\delta\left( \mathbf{x} - \mathbf{x}^\prime \right) =
    \delta\left(u - u^\prime\right) \delta\left(v - v^\prime\right) \delta\left(w - w^\prime\right) \cdot UVW
$$

by considering the limit of the Gaussian above.
Note that as $\alpha \rightarrow 0$ only the infinitesimal length
element need be used for the distance between the points in the exponent.
\\

Browsing through the web, one will see the following defintions of the delta function,
\begin{align*}
\delta (x) &= \lim_{b\rightarrow 0} \frac{1}{|b|\sqrt{\pi}} \exp{ \left[ -\frac{1}{b^2} x^2 \right]} \\
    &= \lim_{b\rightarrow 0^+} \frac{1}{2\sqrt{\pi b}} \exp{ \left[ -\frac{1}{4b} x^2 \right]} \\
    &= \lim_{b\rightarrow 0} \frac{1}{b \sqrt{2\pi}} \exp{ \left[ -\frac{1}{2b^2} x^2 \right]}
\end{align*}

The thing to note here is how the arguments of the exponent make it into the constant that's infront of the exponent.
And the reason to even look at this now is because we found this amazingly interesting approach to this problem in
\href{https://www.wtamu.edu/~cbaird/courses.html}{West Texas CS Baird's} site:

We start with
$$
D\left(\alpha ; x, y, z \right) =
    \frac{1}{\left(\alpha \sqrt{2\pi}\right)^3}
    \exp{ \left[ -\frac{1}{2\alpha^2} \left( x^2 + y^2 + z^2 \right) \right] }
$$
and then we make a change of variables, a displacement if you will, by changing $x \rightarrow x - x^\prime$, etc.
This gives us
$$
D\left(\alpha ; x-x^\prime, y-y^\prime, z-z^\prime \right) =
    \frac{1}{\left(\alpha \sqrt{2\pi}\right)^3}
    \exp{ \left[ -\frac{1}{2\alpha^2} \left( (x-x^\prime)^2 + (y-y^\prime)^2 + (z-z^\prime)^2 \right) \right] }
$$

Now comes the limit taking.
First off is whether the limit is some actual number orr just infinity (we are skipping over the question of existence).
Since an exponent grows faster than a ever-increasing number to a power,
$\lim_{\alpha\rightarrow \infty} \alpha^3 \cdot e^{-\alpha} = 0$,
then we would expect our function $D$ to be bounded.
But, anyway, if we take the limit as $\alpha\rightarrow 0$, and if we look for a case where $D\not\rightarrow 0$,
then we would want $x-x^\prime$ to "balance out" $\alpha$ just enough so that we have some non-zero bounded value.

Let's assume the above is possible and rewirte $D$ as
$$
D\left(\alpha ; x-x^\prime, y-y^\prime, z-z^\prime \right) =
    \frac{1}{\left(\alpha \sqrt{2\pi}\right)^3}
    \exp{ \left[ -\frac{1}{2\alpha^2} \left( (dx)^2 + (dy)^2 + (dz)^2 \right) \right] }
$$

The infinitesimal displacements in the argument to the exponent look like an arc lenght, so let's rewrite it now as
$$
D\left(\alpha ; x-x^\prime, y-y^\prime, z-z^\prime \right) =
    \frac{1}{\left(\alpha \sqrt{2\pi}\right)^3}
    \exp{ \left[ -\frac{1}{2\alpha^2} \left( ds \right)^2 \right] }
$$

Since an arc length is a concept independent of the coordinate system, we can use our given general orthogonal coordinate system,
and rewrite our equation once more,
$$
D\left(\alpha ; u-u^\prime, v-v^\prime, w-w^\prime \right) =
    \frac{1}{\left(\alpha \sqrt{2\pi}\right)^3}
    \exp{ \left[ -\frac{1}{2\alpha^2} \left( 
        \left(\frac{du}{U}\right)^2 + \left(\frac{dv}{V}\right)^2 + \left(\frac{dw}{W}\right)^2 \right) 
    \right] }
$$

And if we reverse the process, by writing the differentials as differences,
$$
D\left(\alpha ; u-u^\prime, v-v^\prime, w-w^\prime \right) =
    \frac{1}{\left(\alpha \sqrt{2\pi}\right)^3}
    \exp{ \left[ -\frac{1}{2\alpha^2} \left(
        \left(\frac{u-u^\prime}{U}\right)^2 + \left(\frac{v-v^\prime}{V}\right)^2 + \left(\frac{w-w^\prime}{W}\right)^2 \right) 
    \right] }
$$

We can now look at the first couple definition of the delta function and make a final change of variahles.
For example, if we let $b_1 = \alpha U$, $b_2 = \alpha V$, and $b_3 = \alpha W$,
$$
D =
    \left[ \frac{
            \exp{ \left[ -\frac{
                \left(u-u^\prime\right)^2
            }{2b_{1}^{2}}  \right] }
        }{b_1 \sqrt{2\pi}}
        U
     \right]
     \left[ \frac{
            \exp{ \left[ -\frac{
                \left(v-v^\prime\right)^2
            }{2b_{2}^{2}}  \right] }
        }{b_2 \sqrt{2\pi}}
        V
     \right]
     \left[ \frac{
            \exp{ \left[ -\frac{
                \left(w-w^\prime\right)^2
            }{2b_{3}^{2}}  \right] }
        }{b_3 \sqrt{2\pi}}
        W
     \right]
$$

If we now take the limit of the $b_i$s, then we get our desired result.
\\~\\

Yet another way to look at this problem is as follows.
The general property of the delta function is that
$$
\int \int \int \delta(x) \delta(y) \delta(z) dx dy dz = 1
$$

If we use our limit,
$$
\int \int \int \lim_{\alpha\rightarrow 0} D(\alpha; x, y, z) dx dy dz = 1
$$

In our general orthogonal coordinate system the above would look as
$$
\int \int \int \lim_{\alpha\rightarrow 0} D(\alpha; x, y, z \rightarrow u, v, w) \frac{du}{U} \frac{dv}{V} \frac{dw}{W} = 1
$$

Note that we do not know how to transform $D$, so we could wrap up our ignorance by defining an intermediate variable as such
$$
F(u,v,w) = \lim_{\alpha\rightarrow 0} D(\alpha; x, y, z \rightarrow u, v, w) \frac{1}{UVW}
$$
In which case we have
$$
\int \int \int F(u,v,w) du dv dw = 1
$$

If we conveniently generalize the above by taking into account the possibility of a translation - because we are integrating over
the entire universe anyway - then we have,
$$
\int \int \int F(u-u^\prime, v-v^\prime, w-w^\prime) du dv dw = 1
$$

Which has the exact same form as the expression that we started with, except we are looking at some other set of integration variables.
So we could make the claim that
\begin{align*}
F(u-u^\prime, v-v^\prime, w-w^\prime) &= \delta(u-u^\prime) \delta(v-v^\prime) \delta(w-w^\prime) \\
&= \lim_{\alpha\rightarrow 0} D(\alpha; x, y, z \rightarrow u, v, w) \frac{1}{UVW}
\end{align*}


And the most interesting part of this problem is the application!

Le'ts think again about spherical coordinates!
The lenght elements along the three orthogonal coordinates are $dr$, $rd\theta$, and $r\sin\theta d\phi$.
So $U=1$, $V=1/r$, and $W=1/r\sin\theta$.
So a three-dimensional delta function is spherical coordinates becomes

$$
\delta\left(\mathbf{x}-\mathbf{x}^\prime\right) =
\delta(r-r^\prime) \delta(\theta-\theta^\prime) \delta(\phi-\phi^\prime) \frac{1}{r^2 \sin\theta}
$$

One interesting way of writing the above is by using $\delta(\cos\theta)$.
Since we have the composition of a function as an argument to our delta function, we can use the following formula
$$
\delta\left(f(x)\right) = \sum_i \frac{1}{\left|\frac{df}{dx} (x_i)\right|} \delta(x-x_i)
$$
where the sum extends over all simple roots.

So,
$$
\delta(\cos\theta - \cos\theta^\prime)
= \sum \frac{1}{\left|-\sin\theta^\prime \right|} \delta(\theta-\theta^\prime)
= \frac{\delta(\theta-\theta^\prime)}{\sin\theta^\prime}
$$

Thus,
$$
\delta\left(\mathbf{x}-\mathbf{x}^\prime\right) =
\delta(r-r^\prime) \delta(\cos\theta-\cos\theta^\prime) \delta(\phi-\phi^\prime) \frac{1}{r^2}
$$

Now, in cylindrical coordinates, the lenght elements are $dr$, $rd\theta$, and $dz$, so
$U=1$, $V=1/r$, and $W=1$.
And
$$
\delta\left(\mathbf{x}-\mathbf{x}^\prime\right) =
\delta(r-r^\prime) \delta(\theta-\theta^\prime) \delta(z-z^\prime) \frac{1}{r}
$$


%%%%%%%%%%%%%%%%%%%%%%%%%%%%%%%%%%%%%%%%%%%%%%%%%%%%%%%%%%%%%%%%%%%%%%%%%%%%%%%%
\subsection{Problem 1.3}

Using Dirac Delta functions in the appropiate coordinate, express the following charge distributions as three-dimensional
charge densities $\rho(\mathbf{x})$.

\begin{enumerate}[(a)]
\item In spherical coordinates, a charge $Q$ uniformly distributed over a spherical shell of radius $R$.

\item In cylindrical coordinates, a charge $\lambda$ per unit length uniformly distributed over a cylindrical
surface of radius $b$.

\item In cylindrical coordinates, a charge $Q$ spread uniformly over a flat circular disc of negligible thickness and radius $R$.

\item The same as part (c), but using spherical coordinates.
\end{enumerate}