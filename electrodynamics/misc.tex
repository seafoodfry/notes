\subsection{Tensor Product}

Given two vectors $v \in V$ and $w \in W$,
the tensor product $v \otimes w$ is defined as the element of $V \otimes W$ such that
$$
(v \otimes w)(h, g) = h(v) g(w)
$$
For all $h \in V^*$ and $g \in W^*$.

If we consider some $T \in V \otimes W$ such that
$$
T(h, g) = h_i g_j T(e^i, f^j) = h_i g_j T^{i j}
$$


We defined $T^{i j}$ as $T^{i j} = T(e^i, f^j)$, $T$ which is a (0,2) tensor,
meaning that it's a multilinear map that takes two dual vectors as inputs and returns a scalar.
The vectors it takes as inputs are from the dual spaces (or covectors) $V^*$ and $W^*$.
$\{e^i\}$ is a basis for $V^*$ and $\{f^j\}$ is a basis for $W^*$.
Both sets of dual basis vectors, which is how we know $T^{ij}$ is a type (0,2) tensor.



The next term in our expression is $e_i \otimes f_j$.
This tensor product takes two dual vectors as inputs and returns a scalar,
$$
(e_i \otimes f_j)(v^*, w^*) = v^*(e_i) w^*(f_j)
$$

The action of the tensor product on these dual vectors is given by the action of each dual vector on its
corresponding basis vector.

This is exactly the behavior of a (0,2) tensor - it takes two dual vectors as inputs and returns a scalar.
$v^*(e_i)$ is a scalar because - a linear functional acting on a vector. Similarly,
$w^*(f_j)$.
\\

If we expand the dual vectors $v^*$ and $w^*$ in terms of their respective dual bases:
$$
v^* = v_i e^i
$$
$$
w^* = w_j f^j
$$

Then,

\begin{align*}
(e_i \otimes f_j)(v^*, w^*) &= v^*(e_i) w^*(f_j) \\
&= v_k e^k(e_i) \, w_l f^l(f_j) \\
&= v_i w_j
\end{align*}

This follows the same approach as in the book when deriving the expression after 3.39.