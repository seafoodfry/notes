\chapter{Intro to QFT}

\section{Inivitation}

\subsection{Polarization Vectors}

In the invitation, we are presented with the polarization vector $\epsilon^\mu = (0, 1, i, 0)$.

To understand this notation let's go through some examples taken from
\href{https://scholar.harvard.edu/files/schwartz/files/lecture14-polarization.pdf}{Lecture 14: Polarization}.

We say that a plane wave is \textbf{linearly polarized} if there is no phase difference between $E_x$ and $E_y$:
$$
\vect{E}_0 = \left( E_x, E_y, 0 \right)
$$

Then a linearly polarized plane wave in the x direction looks like $\vect{E} = E_0 e^{i(kz-wt)} (1, 0, 0)$.
And a linearly polarized plane wave in the y direction looks like $\vect{E} = E_0 e^{i(kz-wt)} (0, 1, 0)$.

The thing to remember is that we are implicitly only looking at the real part, so a linearly polarized wave in the x
direction is actually $(E_0 \cos(kz-wt), 0, 0)$.
\\

\textbf{Circular polarization} is when the electric field components are one-quarter out of phase ($\pi /2$).
Then the field can be written as,

\begin{align*}
\vect{E}_0 &= \left( E_0, E_0 e^{i\pi/2}, 0 \right) \\
&= \left( E_0, iE_0, 0 \right) \\
&= E_0 \left( e^{i(kz-wt)}, ie^{i(kz-wt)}, 0 \right)
\end{align*}

And since we only care about the real parts,
$$
\vect{E}_0 = \left( cos(kz-wt), -sin(kz-wt), 0 \right)
$$

This is interesting because if you jump to the wikipedia page for "List of trigonometric identities"
and look for the "Shift by one quarter period" table, you'll see that
$$
\sin(\theta + \frac{\pi}{2}) = \cos(\theta)
$$
and
$$
\cos(\theta + \frac{\pi}{2}) = -\cos(\theta)
$$

(A shit by a quarter wavelength is essentially differentiation!)

But here is the catch and the connection with P\&S: we can also get the same result if we have
$$
\vect{E} = E_0 \left( \cos(wt-kz)\hat{x} + \sin(wt-kz)\hat{y} \right)
$$
We can also write is as,
$$
E_0 e^{i(wt - kz)} (0, 1, i, 0)
$$
and remembering that you only care about the real part.
If you do so, you'll end up with terms such as $\cos \hat{x} - \sin \hat{y}$ which just so happen to again be the
a $\sin$ and a $\cos$ (our original plane wave components) shifted by $\pi/2$.




\subsection{Cross Sections}

The next thing we want to document is how to solve the differential cross section (per unit solid angle).
The expression given was

$$
\frac{d\sigma}{d\Omega} = \frac{\alpha^2}{4E_{cm}^{2}} (1 + \cos^2 \theta)
$$

That when integrated gives the total cross section
$$
\sigma_{total} = \frac{4\pi\alpha^2}{3E_{cm}^{2}}
$$

The trick here is to identity $d\Omega = \sin\theta \, d\theta \, d\phi$.
So essentially the problem is to integrate

\begin{align*}
\int d\Omega \, \left( 1 + \cos\theta \right) &=
    \int_{\phi=0}^{2\pi} \int_{\theta=0}^{\pi} \left( 1 + \cos\theta \right) \sin\theta \, d\theta \, d\phi \\
&= \left( \int_{\phi=0}^{2\pi} \int_{\theta=0}^{\pi} \sin\theta \, d\theta \, d\phi \right)
    + \left( \int_{\phi=0}^{2\pi} \int_{\theta=0}^{\pi} \cos\theta \sin\theta \, d\theta \, d\phi \right) \\
&= \left( 4\pi \right) + \left( -\int u^2 du \right) \\
&= \left( 4\pi \right) + \left( -\frac{1}{3} \cos^3 \theta \Big|_{0}^{\pi} du \right) \\
&= \left( 4\pi \right) \left( 2\pi \frac{8}{3} \right) \\
&= \frac{16\pi}{3}
\end{align*}

Note that we did a $u$-substitution in the second integral: $u = \cos\theta$, so $du = -\sin\theta \, d\theta$.







%%%%%%%%%%%%%%%%%%%%%%%%%%%%%%%%%%%%%%%%%%%%%%%%%%%%%%%%%%%%%%%%%%%%%%%%%%%%%%%%%%%%

\section{The Klein-Gordon Field}

\subsection{Klein-Gordon Inconsistensies}

The Schrodinger equation canr readily be abtain treating the energy and momentum as operators.
In quantum mechanics $E = i\partial_t$ and $p = -i\nabla$.
Using the realtionship $E = \frac{p^2}{2}$, we get


$$
i\partial_t \ket{\psi} = -\frac{1}{2m} \nabla^2 \ket{\psi}
$$

Note that we are still in natural units, otherwise there would be an $\hbar$.

The Klein-Gordon equation comes if we instead use $E^2 - p^2 = m^2$,

$$
-\partial_{t}^2 \ket{\psi} + \nabla^2 \ket{\psi} = m^2 \ket{\psi}
$$

And remember that $\partial^2 = \partial_{t}^{2} - \nabla^2$, so there comes $(\partial^2 - m)\psi = 0$ equation.



%%%%%%%%%%%%%%%%%%%%%%%%%%%%%%%%%%%%%%%%%%%%%%%%%%%%%%%%%%%%%%%%%%%%%%%%%%%%%%%
\subsubsection{Klein-Gordon Solution}

One method to solve the equation is to use Fourier transforms.
So we can begin by expressing the wave function $\phi(x,t)$ as a Fourier integral,

$$
\psi(x, t) =
\int \frac{d^4k}{\left(2\pi\right)^4} \tilde{\psi}(k, \omega) e^{-i(k\cdot x - \omega t)}
$$

Here, $\tilde{\psi}(k, \omega)$ is the Fourier transform of $\psi(x, t)$.
One thing that's often missed is the rgument of the exponent: if you look back to examples where we talk about plane waves
(just go back to the previous section), a wave with positive momentum moving in positive direction $x$ has the argument $wt - xk$.
And the convention is to hae such a wave in the Fourier transform.

With this Fourier transofrm in place, we take the Fourier transform of our entire differential equation to reduce
the solution into an algebraic problem.

For example, applying the Fourier transform to the time derivative goes as follows:

\begin{align*}
\mathcal{F}\{ \partial_t \psi \} &=
    \int \frac{d^4k}{(2\pi)^4} \partial_t \psi e^{-i(\vect{k}\cdot\vect{x} - \omega t)}
\rightarrow
\left[
    \begin{alignedat}{2}
        u  &= e^{-i(\vect{k}\cdot\vect{x} - \omega t)}                \quad & v  &= \psi \\
        du &= i\omega e^{-i(\vect{k}\cdot\vect{x} - \omega t)} \, dt  \quad & dv &= \partial_t \psi \, dt
    \end{alignedat}\,
\right] \\
&= \cancelto{0}{ \psi(\vect{x}, t) e^{-i(\vect{k}\cdot\vect{x} - \omega t)} } \Big|_{t=-\infty}^{t=\infty}
    - \int \frac{d^4k}{(2\pi)^4} \psi(\vect{x}, t) \left(i\omega\right) e^{-i(\vect{k}\cdot\vect{x} - \omega t)} \\
&= -i\omega \tilde{\psi} (\vect{k}, \omega)
\end{align*}

We made use of the integration by parts technique ($\int u \, dv = uv - \int v \, du$) to transfer the time derivative
on $\psi$ to the exponent term.

As per the boudnary term, we have a couple words to say: physically, the wave function (or even field) should vanish at infinity.
Mathematically, in order for the Fourier transform to converge to a value the function that is being "transformed", $\psi$,
must be a \textbf{rapidly decreasing function} (a function of \href{https://en.wikipedia.org/wiki/Schwartz_space}{Schwartz space}).

Similarly then,

\begin{align*}
\mathcal{F}\{ \partial_{t}^2 \psi \} &=
    \int \frac{d^4k}{(2\pi)^4} \partial_{t}^{2} \psi e^{-i(\vect{k}\cdot\vect{x} - \omega t)}
\rightarrow
\left[
    \begin{alignedat}{2}
        u  &= e^{-i(\vect{k}\cdot\vect{x} - \omega t)}                \quad & v  &= \partial_t \psi \\
        du &= i\omega e^{-i(\vect{k}\cdot\vect{x} - \omega t)} \, dt  \quad & dv &= \partial_{t}^{2} \psi \, dt
    \end{alignedat}\,
\right] \\
&= \cancelto{0}{ \partial _t \psi(\vect{x}, t) e^{-i(\vect{k}\cdot\vect{x} - \omega t)} } \Big|_{t=-\infty}^{t=\infty}
    - \int \frac{d^4k}{(2\pi)^4} \partial_t \psi(\vect{x}, t) \left(i\omega\right) e^{-i(\vect{k}\cdot\vect{x} - \omega t)} \\
&= -i\omega \mathcal{F}\{ \partial_t \psi \} \\
&= - \omega^2 \tilde{\psi} (\vect{k}, \omega)
\end{align*}

The boundary term again goes to zero here.
We will skip the physical argument and just mention that a rapidly decreasin function also requires all of its derivatives
tend to zero.



For the spatial derivatives, each $\partial_i$ will bring down a factor of $-i k_i$, resulting in a factor of
$- k_{i}^2$ for each $\partial_{i}^{2}$.
And this is something that always triped me up: one may be inclined to write $-k^2$ in the Fourier transform
but that same one ought to remember that $k$ is actually $\vec{k}$ (a vector) and that is why we must
write $-|k|^2$, because we want the [L-2] norm.

But anyways, the Fourier transform of the Laplacian is,

\begin{align*}
\mathcal{F}\{ \nabla \psi \} &=
    \int \frac{d^4k}{(2\pi)^4} \nabla \psi e^{-i(\vect{k}\cdot\vect{x} - \omega t)}
\rightarrow
\left[
    \begin{alignedat}{2}
        u  &= e^{-i(\vect{k}\cdot\vect{x} - \omega t)}                          \quad & v  &= \psi \\
        du &= -i\vect{k} e^{-i(\vect{k}\cdot\vect{x} - \omega t)} \, d\vect{x}  \quad & dv &= \nabla \psi \, d\vect{x}
    \end{alignedat}\,
\right] \\
&= \cancelto{0}{ \psi(\vect{x}, t) e^{-i(\vect{k}\cdot\vect{x} - \omega t)} } \Big|_{\vect{x}=-\infty}^{\vect{x}=\infty}
    - \int \frac{d^4k}{(2\pi)^4} \psi(\vect{x}, t) \left(-i\vect{k}\right) e^{-i(\vect{k}\cdot\vect{x} - \omega t)} \\
&= -i\vect{k} \tilde{\psi} (\vect{k}, \omega)
\end{align*}


and

\begin{align*}
\mathcal{F}\{ \nabla^2 \psi \} &=
    \int \frac{d^4k}{(2\pi)^4} \nabla^2 \psi e^{-i(\vect{k}\cdot\vect{x} - \omega t)}
\rightarrow
\left[
    \begin{alignedat}{2}
        u  &= e^{-i(\vect{k}\cdot\vect{x} - \omega t)}                          \quad & v  &= \nabla \psi \\
        du &= -i\vect{k} e^{-i(\vect{k}\cdot\vect{x} - \omega t)} \, d\vect{x}  \quad & dv &= \nabla^2 \psi \, d\vect{x}
    \end{alignedat}\,
\right] \\
&= \cancelto{0}{ \nabla\psi(\vect{x}, t) e^{-i(\vect{k}\cdot\vect{x} - \omega t)} } \Big|_{\vect{x}=-\infty}^{\vect{x}=\infty}
    - \int \frac{d^4k}{(2\pi)^4} \nabla\psi(\vect{x}, t) \left(-i\vect{k}\right) e^{-i(\vect{k}\cdot\vect{x} - \omega t)} \\
&= -i\vect{k} \mathcal{F}\{ \nabla \psi \} \\
&= - |\vect{k}|^2 \tilde{\psi} (\vect{k}, \omega)
\end{align*}



And since the above must hold throughout all of space, that's where
$$
\left(-\omega^2 + |\vect{k}|^2 + m^2 \right) \tilde{\psi}(\vect{k}, \omega) = 0
$$
comes from!

Now, in order for us to not have a trivial solution, $\psi(\vect{x}, t) = 0$, it must be so that
$\left(-\omega^2 + |\vect{k}|^2 + m^2 \right) = 0$, and so our dispersion relation comes about
$$
\omega^2 = |\vect{k}|^2 + m^2
$$
So $\omega = \pm \sqrt{ |\vect{k}|^2 + m^2 }$.
So essentially any function will do as long as the dispersion relation (momentum conservation) is respected.

A thing that people do, specially since we are talking about Fourier transforms is to define

$$
\psi(\vect{k}, \omega) = A(\vec{k}) \delta\left( \omega - \sqrt{ |\vect{k}|^2 + m^2 }\right) +
    B(\vec{k}) \delta\left( \omega + \sqrt{ |\vect{k}|^2 + m^2 }\right)
$$

Note that $A(\vec{k})$ and $B(\vec{k})$ are arbitrary functions that only depend on $\vect{k}$ since the
Dirac delta functions specifies the value for $\omega$.

With that, we can finally write a solution for the differential equation we started with,

\begin{align*}
\psi(\vect{x}, t) &=
    \int \frac{d^4k}{(2\pi)^4} \left[
        A(\vec{k}) \delta\left( \omega - \sqrt{ |\vect{k}|^2 + m^2 }\right) e^{-i(\vect{k}\cdot\vect{x} -\omega t)} +
        B(\vec{k}) \delta\left( \omega + \sqrt{ |\vect{k}|^2 + m^2 }\right) e^{-i(\vect{k}\cdot\vect{x} -\omega t)}
    \right] \\
&= \int \frac{d^4k}{(2\pi)^4} \left[
    A(\vec{k}) e^{-i(\vect{k}\cdot\vect{x} - \sqrt{ |\vect{k}|^2 + m^2 } t)} +
    B(\vec{k}) e^{-i(\vect{k}\cdot\vect{x} + \sqrt{ |\vect{k}|^2 + m^2 } t)}
\right]
\end{align*}

Hopefully thiis result makes sense: we obtained a family of function as solution and only initial conditions or
boundary conditions will result in a specific sort of function.




%%%%%%%%%%%%%%%%%%%%%%%%%%%%%%%%%%%%%%%%%%%%%%%%%%%%%%%%%%%%%%%%%%%%%%%%%%%%%%%
\subsubsection{Klein-Gordon Negative Density}


The expression to compute a probability current is derived by using the analogous continuity equation
from fluid dynamics

$$
\frac{\partial\rho}{\partial t} + \nabla\cdot \vect{j} = 0
$$


Using the Schrodinger equation and using the fact that $\rho = |\psi|^2 = \psi^* \psi$ one
can arrive at the probability density
$$
\rho = 
-\frac{i}{2m} \left( \psi^* \partial_t \psi - \psi \partial_t \psi^* \right)
$$

For the Klein Gordon equation, the way to massage it and get an expression for $\partial_t \rho$
and for its probability current is to:
take Klein-Gordon multiply it by $\psi^*$,
take the complex conjugate multiply it by $\psi$,
subtract the two and rearange to get something like the continuity equation.

Following those steps we have,

$$
\left( \partial_{t}^{2} - \nabla^2 + m^2 \right) \psi = 0
$$
and
$$
\left( \partial_{t}^{2} - \nabla^2 + m^2 \right) \psi^* = 0
$$

Multiplying them with $\psi^*$ and $\psi$ respectively we get,
$$
\psi^* \partial_{t}^{2} \psi - \psi^* \nabla^2 \psi + \psi^* m^2 \psi = 0
$$
and
$$
\psi \partial_{t}^{2} \psi^* - \psi \nabla^2 \psi^* + \psi m^2 \psi^* = 0
$$

Subtracting the latter from the former,
\begin{align*}
& \psi^* \partial_{t}^{2} \psi - \psi^* \nabla^2 \psi + \psi^* m^2 \psi -
    \psi \partial_{t}^{2} \psi^* + \psi \nabla^2 \psi^* - \psi m^2 \psi^* \\
&= \left( \psi^* \partial_{t}^{2} \psi - \psi \partial_{t}^{2} \psi^* \right)
    - \left( \psi^* \nabla^2 \psi - \psi \nabla^2 \psi^* \right)
    + \left( \psi^* m^2 \psi - \psi m^2 \psi^* \right) \\
&= \left( \psi^* \partial_{t}^{2} \psi - \psi \partial_{t}^{2} \psi^* \right)
    - \left( \psi^* \nabla^2 \psi - \psi \nabla^2 \psi^* \right) \\
&= \partial_t \left( \psi^* \partial_{t} \psi - \psi \partial_{t} \psi^* \right)
    -\nabla \cdot \left( \psi^* \nabla \psi - \psi \nabla \psi^* \right)
\end{align*}

The time derivative is then equated to $\partial_t \rho$ and the spatial derivatives to the current.
though one interesting tangent to take here is concerning the missing $\frac{i}{2m}$ factor that
these equations have.

If you remember, the imaginery part of of a complex number can be isolated by taking the different of it with its
complex conjugate: $\im{z} = \frac{1}{2i}(z - \overline{z})$. (Recall that if $z = a +ib$, then $\im{z} = b$, not $ib$.)
It just so happens that we are doing the same sort of operation here, so we can throw an $i/2$ factor into our equation
to ensure we get a real quantity.

For the $1/m$ factor we have to do some dimensional analysis.
To get the probability we have to integrate $\int d^3x \, |\psi|^2$, which is dimensionless.
So the $|\psi|^2$ term needs to cancel out the integration over space.
Hence $[|\psi|^2] = [L]^{-3} = [M]^{3}$.
From there we can say that $[\psi] = [L]^{-3/2}$ and
$[\psi^* \partial_{t}^2 \psi] = [L]^{-3/2}[L]^{-2}[L]^{-3/2} = [L]^{-5} = [M]^{5}$.

This same term we just evaluated needs to match the dimensions of
the time derivative, $[L]^{-1}$,a probability density, $[L]^{-3}$.
It should be then that $[\partial_t \rho] = [L]^{-4} = [M^4]$.
But hey, what a coincidence that if we were to divide $\psi^* \partial_{t}^2 \psi$ by a mass
that we would get just the right dimensions for our probability current!
And so it goes that the expression we are used to seeing turns out to be excused as such in the Klein-Gordon case.
\\


Suppose we had a $\psi =  e^{-i(\vect{k}\cdot\vect{x} -\omega t)}$ as solution.

Then,
\begin{align*}
\rho &=
    -\frac{i}{2m} \left( 
        e^{i(\vect{k}\cdot\vect{x} -\omega t)} \partial_t e^{-i(\vect{k}\cdot\vect{x} -\omega t)} -
        e^{-i(\vect{k}\cdot\vect{x} -\omega t)} \partial_t e^{i(\vect{k}\cdot\vect{x} -\omega t)} 
    \right) \\
&= -\frac{i}{2m} \left(
    e^{i(\vect{k}\cdot\vect{x} -\omega t)} \left(i\omega\right) e^{-i(\vect{k}\cdot\vect{x} -\omega t)} -
    e^{-i(\vect{k}\cdot\vect{x} -\omega t)} \left(-i\omega\right) e^{i(\vect{k}\cdot\vect{x} -\omega t)}
    \right) \\
&= -\frac{i}{2m} \left(i\omega\right) \left( 2 \right) \\
&= \frac{\omega}{m}
\end{align*}

And since $\omega = \pm \sqrt{|\vect{k}|^ + m^2}$, then the density can be negative!



%%%%%%%%%%%%%%%%%%%%%%%%%%%%%%%%%%%%%%%%%%%%%%%%%%%%%%%%%%%%%%%%%%%%%%%%%%%%%%%
\subsubsection{A Propagator}

Describes the amplitude for a particle to propagate from one point to another.

With a source $j(\vect{x}, t)$, the Klein-Gordon equation becomes
$$
\left( \partial^2 + m^2 \right) \psi(\vect{x}, t) = j(\vect{x}, t)
$$

In momentum space the equation is
$$
\left( -\omega^2 + |\vect{k}|^2 + m^2 \right) \tilde{\psi}(\vect{k}, \omega) = \tilde{j}(\vect{\vect{k}, \omega})
$$

$$
\tilde{\psi}(\vect{k}, \omega) = \frac{ \tilde{j}(\vect{\vect{k}, \omega}) }{ \omega^2 - |\vect{k}|^2 - m^2 + i\epsilon }
$$

$i\epsilon$ Feynman's prescription.

The propagator $D_F(x-y)$ is the inverse transform of
$\frac{ 1 }{ \omega^2 - |\vect{k}|^2 - m^2 + i\epsilon }$

$$
D_F(x-y) = \int \frac{d^4k}{(2\pi)^4}
    \frac{ e^{-i(\vect{k}\cdot\vect{x} -\omega t)} }{ \omega^2 - |\vect{k}|^2 - m^2 + i\epsilon }
$$






%%%%%%%%%%%%%%%%%%%%%%%%%%%%%%%%%%%%%%%%%%%%%%%%%%%%%%%%%%%%%%%%%%%%%%%%%%%%%%%%%%%%%%%%%%%
\subsection{Causality Arguments: The Non-relativistic Case}

Consider the amplitude for a free particle to propagate from $\vect{x}_0$ to $\vect{x}$
$$
U(t) = \bra{\vect{x}} e^{-iHt} \ket{\vect{x}_0}
$$

In nonrelativistic quantum mechanics we have $E = \frac{1}{2m} p^2$, so
\begin{align}
U(t) &= \bra{\vect{x}} e^{-i(\vect{p}^2/2m)t} \ket{\vect{x}_0} \label{kg:nonrelativistic-01} \\
&= \int d^3p \, \bra{\vect{x}} e^{-i(\vect{p}^2/2m)t} \ket{\vect{p}} \braket{\vect{p} | \vect{x}_0} \label{kg:nonrelativistic-02} \\
&= \int \frac{d^3p}{(2\pi)^3} e^{-i(\vect{p}^2/2m)t} e^{i\vect{p}\cdot(\vect{x}-\vect{x}_0)} \label{kg:nonrelativistic-03} \\
&= \left( \frac{m}{2\pi i t} \right)^{3/2} e^{im (\vect{x}-\vect{x}_0)^2 /2t} \label{kg:nonrelativistic-04}
\end{align}


To go from \ref{kg:nonrelativistic-01} to \ref{kg:nonrelativistic-02},
we changed the basis to momentum eigenstates in order to apply the Hamiltonian operator and "extract" the exponential.
If we have the operator $\hat{p}$ act on a pure momentum eigenstate $\ket{p}$, then we get $\hat{p}\ket{p} = p\ket{p}$.
So changing basis helps us work with real quantities.

Here we are differing on convetnion from the book a bit.
The identity we used was $\mathbb{I} = \int dp\, \ket{p}\bra{p}$, whereas Pesking and Schroeder use the identity
$\mathbb{I} = \int \frac{d^dp}{(2\pi)^d} \ket{p}\bra{p}$.
The additional $1/(2\pi)^d$ factor picks up the normalization for factors such as $\braket{p | x_0}$.
But keep reading for details on those.
See the next section on formalism for more details (its a brief summary of stuff from quantum mechanics).


When you see the integrals over states in quantum mechanics, especially when involving time evolution and position states,
the calculation often utilizes the well-known expressions for the overlap of position and momentum eigenstates
$\braket{\vect{x} | \vect{p}}$ and $\braket{\vect{p} | \vect{x_0}}$.
These expressions are crucial in converting between position and momentum representations,
which allows us to analyze the system's dynamics.
In such cases, we have these useful identities - the next section will remind you where these come from:

$$
\braket{\vect{x} | \vect{p}} = \frac{1}{(2\pi)^{3/2}} e^{i\vect{p}\cdot\vect{x}}
$$
and 
$$
\braket{\vect{p} | \vect{x_0}} = \frac{1}{(2\pi)^{3/2}} e^{-i\vect{p}\cdot\vect{x_0}}
$$

We used these identities to go from \ref{kg:nonrelativistic-02} to \ref{kg:nonrelativistic-03},

\begin{align*}
& \int \frac{d^3p}{(2\pi)^3} e^{-i(\vect{p}^2/2m)t} \braket{\vect{x} | \vect{p}} \braket{\vect{p} | \vect{x}_0} \\
&= \int \frac{d^3p}{(2\pi)^3} e^{-i(\vect{p}^2/2m)t} \frac{1}{(2\pi)^3} e^{i\vect{p}\cdot(\vect{x}-\vect{x}_0)} 
\end{align*}

If these details don't ring a bell, read the following section.

Pesking and schroeder seem to use
$$
\braket{\vect{x} | \vect{p}} = e^{i\vect{p}\cdot\vect{x}}
$$
and 
$$
\braket{\vect{p} | \vect{x_0}} = e^{-i\vect{p}\cdot\vect{x_0}}
$$
Since the projection into momentum space carries the appropiate normalization factor.


As per the solution, see \ref{ps-kg:nonrelativistic-propagator}.
The trick is to complete the square for the arguments of the exponentials and extract a Gaussian integral.


%%%%%%%%%%%%%%%%%%%%%%%%%%%%%%%%%%%%%%%%%%%%%%%%%%%%%%%%%%%%%%%%%%%%%%%%%%%%%%%%%%%%%%%%%%%
\subsubsection{A Bit of Formalism}

The couple facts used in the above integral go as follows.

First, this comes from the section "outer products" in
\href{https://en.wikipedia.org/wiki/Bra%E2%80%93ket_notation}{Wikipedia: Bra-Ket Notation},
we have to keep in mind that $\ket{\psi}\bra{\psi}$ defines an \textbf{outer product}.
In a finite-dimensional vector space the outer product is defined as

$$
\ket{\phi}\bra{\psi}
=
\begin{pmatrix}
    \phi_1 \\
    \phi_2 \\
    \vdots \\
    \phi_N 
\end{pmatrix}
\begin{pmatrix}
    \psi_{1}^{*} & \psi_{2}^{*} & \ldots & \psi_{N}^{*}
\end{pmatrix}
=
\begin{pmatrix} 
    \phi_1 \psi_{1}^{*} & \phi_1 \psi_{2}^{*} & \dots  & \phi_1 \psi_{N}^{*} \\
    \phi_2 \psi_{1}^{*} & \phi_2 \psi_{2}^{*} & \dots  & \phi_2 \psi_{N}^{*} \\
    \vdots              & \vdots              & \ddots & \vdots              \\
    \phi_N \psi_{1}^{*} & \phi_N \psi_{2}^{*} & \dots  & \phi_N \psi_{N}^{*} \\
\end{pmatrix}
$$

One of the uses of the outer product is to construct \textbf{projection operators}. Given a ket $\ket{\psi}$
of norm 1, the orthogonal projection onto the subspace spanned by $\ket{\psi}$ is $\ket{\psi}\bra{\psi}$.


The "Unit operator" section on \href{https://en.wikipedia.org/wiki/Bra%E2%80%93ket_notation}{Wikipedia: Bra-Ket Notation},
also has this: if we have a complete orthonormal basis $\{ e_i | i\in\mathbb{N} \}$,
functional analysis tells us that any $\ket{\psi}$ can also be written as
\begin{equation}
\ket{\psi} =
\sum_{i\in\mathbb{N}} \braket{e_i | \psi} \ket{e_i} \label{qm-formalism:discrete-basis}
\end{equation}

This is how we "project" $\psi$ into a new basis.

It mentions that it can also be shown that
$$
\mathbb{I} = \sum_{i\in\mathbb{N}} \ket{e_i} \bra{\psi}
$$

There is also this result called
\href{https://en.wikipedia.org/wiki/Borel_functional_calculus#Resolution_of_the_identity}{resolution of the identity in Borel functional calculus}
that allows us to generalized this result to the continuous case,

$$
\mathbb{I} = \int dx \, \ket{x} \bra{x} = \int dp \, \ket{p} \bra{p}
$$


The analogous to \ref{qm-formalism:discrete-basis} in the continuous case is
$$
\ket{\psi} = \int dx\, \psi(x) \ket{x}
$$
where $\psi (x) = \braket{x | \psi}$ by analogy.
\\


The next bit of muscle memory to train is to remember that
$$
\delta^{4} (k) = \int \frac{d^4x}{(2\pi)^4} e^{ik\cdot x}
$$
and that
$$
\braket{x | x^\prime} = \delta(x - x^\prime)
$$

and
$$
\braket{p | q} = (2\pi)^d \delta^{(d)}(p - q)
$$

Along with the following bit of formalism comes from James Binney's book, section 2.3.2.

Position operator $\hat{x}$ acts in position space as $\hat{x} \psi = x\psi$.
The momentum operator in the position representation is $\hat{p} = -i\nabla$.

A position eigenstate $\ket{x}$ satisfies $\hat{x}\ket{x} = x\ket{x}$.
Similarly, momentum eigenstate $\ket{p}$ satisfies $\hat{p}\ket{p} = p\ket{p}$.

The state $\ket{p}$ in which a measurement of the momentum will yield the value $p$ has to be an eigenstate
of $\hat{p}$.
To find the wave function $u_p (x) = \braket{x | p}$ of this important state, $\ket{p}$,
we can use the following argument

$$
\hat{p} \phi(x) = -i\nabla \phi(x) = p\phi
$$

This results in a differential equation with a simple solution,
$$
-i\nabla u_p(x) = p u_p(x) \rightarrow u_p(x) = Ae^{ipx}
$$

Thus the wavefunction of a particle of well-defined momentum is a plane wave.
To normalize the wave function, we need to find $A$ which can be done as follows:

\begin{align*}
\delta(p - p^\prime) &= \braket{p | p^\prime} \\
&= \int dx \, \braket{p | x} \braket{x | p^\prime} \\
&= |A|^2 \int dx \, u_{p}^{*} (x) u_{p^\prime} (x) \\
&= |A|^2 \int dx \, e^{-ipx} e^{ip^\prime x} \\
&= |A|^2 \int dx \, e^{i(p - p^\prime) x} \\
&= 2\pi |A|^2 \delta(p - p^\prime)
\end{align*}

Here we used $\mathbb{I} = \int dx \, \ket{x}\bra{x}$ again.
Along with $\delta(x-x^\prime) = \int \frac{dk}{2\pi} e^{ik(x-x^\prime)}$.







%%%%%%%%%%%%%%%%%%%%%%%%%%%%%%%%%%%%%%%%%%%%%%%%%%%%%%%%%%%%%%%%%%%%%%%%%%%%%%%%%%%%%%%%%%%
\subsection{Causality Arguments: The Relativistic Case}

And now let's look at
\begin{align*}
U(t) &= \bra{\vect{x}} e^{-it\sqrt{\vect{p}^2 + m^2}} \ket{\vect{x}_0} \\
&= \int d^3p \, \bra{\vect{x}} e^{-it\sqrt{\vect{p}^2 + m^2}} \ket{\vect{p}} \braket{\vect{p} | \vect{x}_0} \\
&= \int d^3p \, e^{-it\sqrt{\vect{p}^2 + m^2}} \braket{\vect{x} | \vect{p}} \braket{\vect{p} | \vect{x}_0} \\
&= \int \frac{d^3p}{(2\pi)^3} \, e^{-it\sqrt{\vect{p}^2 + m^2}} e^{i\vect{p}\cdot\vect{x}} e^{-i\vect{p}\cdot\vect{x}_0} \\
&= \int \frac{d^3p}{(2\pi)^3} \, e^{-it\sqrt{\vect{p}^2 + m^2}} e^{i\vect{p}\cdot(\vect{x} - \vect{x}_0)} \\
\end{align*}

The result then happens to be
$$
\frac{1}{2\pi^2 |\vect{x}-\vect{x}_0|}
    \int_{0}^{\infty} dp \, p \sin(p|\vect{x}-\vect{x}_0|) e^{-it\sqrt{\vect{p}^2 + m^2}}
$$

The trick for this one is to use spherical variables to simplify it a bit.
$$
\int \frac{d^3p}{(2\pi)^3} \, e^{-it\sqrt{\vect{p}^2 + m^2}} e^{i\vect{p}\cdot(\vect{x} - \vect{x}_0)} 
$$
becomes
$$
\int \frac{dp}{(2\pi)^3} d\Omega \, p^2 e^{-it\sqrt{p^2 + m^2}} e^{ip\vect{r}\cos\theta} 
$$
where $r = |\vect{x} - \vect{x}_0|$ and $p = |\vect{p}|$.

From here we can do the angular integrals first,

\begin{align*}
\int d\Omega \, e^{ipr\cos\theta} &= 
    \int_{0}^{2\pi} d\phi \int_{0}^{\pi} d\theta \, \sin\theta e^{ipr\cos\theta} \\
&= 2\pi \int_{0}^{\pi} d\theta \, \sin\theta e^{ipr\cos\theta} \\
&= -2\pi \int_{1}^{-1} du \, e^{ipr u} \\
&= -2\pi \left( \frac{ e^{ipr u} }{ ipr } \right) \Big|_{u=1}^{u=-1} \\
&= -\frac{2\pi}{ipr} \left( e^{-ipr} - e^{ipr} \right)
\end{align*}

Next, we can make use of the formula $\im{z} = \frac{z - \overline{z}}{2i}$ and apply it to $z = e^{ix}$.
Or just do it manually: $e^{ix} = \cos + i\sin$, $e^{-ix} = \cos -i\sin$, so
$e^{-ix} - e^{ix} = -i\sin -i\sin = -2i \sin$.

\begin{align*}
\int d\Omega \, e^{ipr\cos\theta} &= 
    -\frac{2\pi}{ipr} \left( e^{-ipr} - e^{ipr} \right) \\
&= -\frac{2\pi}{ipr} \left(-2i \sin(pr) \right) \\
&= \frac{4\pi}{pr} \sin(pr)
\end{align*}

Pluging this back in,
We got
\begin{align}
\int \frac{dp}{(2\pi)^3} d\Omega \, |p|^2 e^{-it\sqrt{p^2 + m^2}} e^{ipr\cos\theta} &=
\int \frac{dp}{(2\pi)^3} p \frac{4\pi}{r} \sin(pr) e^{-it\sqrt{p^2 + m^2}} \\
&= \frac{1}{2\pi^2 |\vect{x}-\vect{x}_0|}
    \int_{0}^{\infty} dp \, p \sin(p|\vect{x}-\vect{x}_0|) e^{-it\sqrt{p^2 + m^2}} \label{kg-relativistic-simplified}
\end{align}

The reference to Gradsheteyn and Ryzhik to 3.914 lists the following: on page 491, item 6 has
\begin{equation}
\int_{0} dx\, x \sin(bx) e^{-\beta \sqrt{x^2 + \gamma^2}} =
    \frac{b \beta \gamma^2}{\beta^2 + b^2} K_2\left(\gamma \sqrt{b^2 + \beta^2}\right) \label{Gradsheteyn-kg-relativistic}
\end{equation}

ET I 175(35).
I refers to volume 1 of the reference, 175 is the page where it should be found, (35) refers to the number of the formula
in that work.

We found a PDF of ET and on said place we found ourselves in the Laplace transofrms section.
This section has two columns, one of them is $f(t)$ and the other one is $g(p) = \int_{0}^{\infty} dt\, f(t) d^{-pt}$.
Forumal 35 had on the first column
$$
t^\alpha L_{n}^{\alpha} (\lambda t) L_{m}^{\alpha} (k t)
$$
for $\re{\alpha} > -1$.

On the second column was the following:
\begin{align*}
\frac{\Gamma(m+n+\alpha + 1)}{m! n!}
\frac{(p-\lambda)^n (p-k)^m}{p^{m+n+\alpha +1}}
{}_{2}F_{1} \left[ -m, -n; -m-n-\alpha; \frac{p(p-\lambda -k)}{(p-\lambda)(p-k)} \right]
\end{align*}
When $\re{p} > 0$.
\\

Anyway, backing up a bit, $K_\nu (z)$ is a Bessel function for an imaginary argument.
Which is defined in 8.407 and in 8.43, on page 911.
Gradsheteyn and Ryzhik have the following definitions.
If $-\pi < \arg{z} \leq \frac{1}{2}\pi$, then
$$
K_\nu (z) = \frac{i\pi}{2} e^{i\pi\nu /2} H_{\nu}^{(1)} \left(z e^{\frac{1}{2} i\pi} \right)
$$

If $-\frac{1}{2}\pi < \arg{z} \leq \pi$, then
$$
K_\nu (z) = -\frac{i\pi}{2} e^{-i\pi\nu /2} H_{-\nu}^{(2)} \left(z e^{-\frac{1}{2} i\pi} \right)
$$

$H_{\nu}^{(1)} (z)$ and $H_{\nu}^{(2)} (z)$ are Bessel function of the third kind, or Hankel functions.
Which are defined in terms of the Bessel functions of the first kind, $J_{\nu}(z)$, and on Bessel functions
of the second kind, $Y_\nu (z)$ (also called Neumman functions and written as $N_\nu (z)$).

$$
H_{\nu}^{(1)} (z) = J_\nu (z) + i Y_\nu (z)
$$
$$
H_{\nu}^{(2)} (z) = J_\nu (z) - i Y_\nu (z)
$$

Comparing \ref{kg-relativistic-simplified} with \ref{Gradsheteyn-kg-relativistic},
\begin{itemize}
    \item $x \rightarrow p$
    \item $b \rightarrow |x-x_0|$
    \item $\beta \rightarrow it$
    \item $\gamma \rightarrow m$
\end{itemize}

So
$$
\frac{b \beta \gamma^2}{\beta^2 + b^2} K_2\left(\gamma \sqrt{b^2 + \beta^2}\right)
$$
becomes
$$
\frac{it m^2 |x-x_0| }{-t^2 + |x-x_0|^2} K_2\left(m \sqrt{ -t^2 + |x-x_0|^2}\right)
$$

Now let's look at the modified function $K_2$,
\begin{align*}
K_2\left(m \sqrt{ -t^2 + |x-x_0|^2}\right) &=
    \frac{i\pi}{2} e^{i\pi} H_{2}^{(1)} \left(m \sqrt{ -t^2 + |x-x_0|^2} e^{\frac{1}{2} i\pi} \right) \\
    &= - \frac{i\pi}{2} H_{2}^{(1)} \left(m \sqrt{ -t^2 + |x-x_0|^2} \right)
\end{align*}

Looking at Gradsheteyn and Ryzhik again,
$$
H_{2}^{(1)} \left(m \sqrt{ -t^2 + |x-x_0|^2} \right) =
    J_2 \left(m \sqrt{ -t^2 + |x-x_0|^2} \right) + i Y_2 \left(m \sqrt{ -t^2 + |x-x_0|^2} \right)
$$

And here we stop and appreciate why the method of stationary phase was mentioned and use by the authors.

%%%%%%%%%%%%%%%%%%%%%%%%%%%%%%%%%%%%%%%%%%%%%%%%%%%%%%%%%%%%%%%%%%%%%%%%%%%%%%%%%%%%%%%%%%%
\subsubsection{On The Way To The Method of Stationary Phase: Laplace's Method}

To understand this method well it helps to first read
\href{https://en.wikipedia.org/wiki/Laplace%27s_method}{Wikipedia: Laplace's Method}.

If anything we think these are the highlights from Laplace's method.

If $f(x)$ has a unique global maximum at $x_0$ and $M>0$ and if we deifne the following functions
$$
g(x) = Mf(x)
$$
and
$$
h(x) = e^{g(x)} = e^{Mf(x)}
$$

Then look at the ratios
$$
\frac{g(x_0)}{g(x)} = \frac{f(x_0)}{f(x)}
$$
and
$$
\frac{h(x_0)}{h(x)} = e^{M(f(x_0) - f(x))}
$$

As $M$ increases, the ratio for $h$ will grow exponentially, while the ratio for $g$ does not change.
Since $f(x) \leq f(x_0)$, then the ratio of $h$ will always look like $e^{+{x}M}$.
Said another way, in places where the value of $f(x)$ is a lot smaller than $f(x_0)$, the ratio of $h$ will look like
$e^{Mf(x_0)}$, in places where the value of $f(x)$ is close to $f(x_0)$, the ratio will be smaller.
\\

But in general the trick is to expand $f(x)$ around $x_0$ following Taylor's theorem
$$
f(x) \approx
    f(x_0) + f^{\prime} (x_0) (x-x_0) + \frac{2}{2} f^{\prime\prime} (x_0) (x-x_0)^2 + R
$$

Since $x_0$ is a global maxima, then $f^{\prime} (x_0) = 0$ and $f^{\prime\prime} (x_0) < 0$.
Hence,
$$
f(x) \approx
    f(x_0) - \frac{1}{2} |f^{\prime\prime} (x_0)| (x-x_0)^2 + R
$$

So we have this approximation,
$$
\int_{a}^{b} e^{Mf(x)} dx \approx
    e^{Mf(x_0)} \int_{a}^{b} e^{- \frac{1}{2} M |f^{\prime\prime} (x_0)| (x-x_0)^2} dx
$$

So we are approximating the function with a Gaussian.
Here the last part of the approximation is to extend the limits of itnegration so that we do indeed have a Gaussian integral,
which we assume we can ebcause the tails decay quickly.
Since
$$
\int dx\, e^{-a(x+b)^2} = \sqrt{ \frac{\pi}{a} }
$$

$$
\int_{a}^{b} e^{Mf(x)} dx \approx
    e^{Mf(x_0)} \sqrt{ \frac{2\pi}{M |f^{\prime\prime} (x_0)|} }
$$

The next stepping stone is
\href{https://en.wikipedia.org/wiki/Method_of_steepest_descent}{Wikipedia: the method of steepest descent}.
This method build upon Laplace's method and aplies to contour integrals in the complex plane.
So now we are looking at approximating integrals such as
$$
\int_{C} f(z) e^{\lambda g(z)} dz
$$


%%%%%%%%%%%%%%%%%%%%%%%%%%%%%%%%%%%%%%%%%%%%%%%%%%%%%%%%%%%%%%%%%%%%%%%%%%%%%%%%%%%%%%%%%%%
\subsubsection{A Bit of Formalism: Bilinear Maps}

$$
\left( x \times y \right)^i = \sum_{j,k=1}^{n=3} \epsilon_{ijk} x_j y_k = \epsilon_{ijk} x_j y_k
$$

Expanding the sum it looks like
\begin{align*}
\left( x \times y \right)^i =& \, \epsilon_{ijk} x_j y_k \\
=& \, \epsilon_{ij1} x_j y_1 + \epsilon_{ij2} x_j y_2 + \epsilon_{ij3} x_j y_3 \\
=& \, \epsilon_{i11} x_1 y_1 + \epsilon_{i12} x_1 y_2 + \epsilon_{i13} x_1 y_3 \\
    & \epsilon_{i21} x_2 y_1 + \epsilon_{i22} x_2 y_2 + \epsilon_{i23} x_2 y_3 \\
    & \epsilon_{i31} x_3 y_1 + \epsilon_{i32} x_3 y_2 + \epsilon_{i33} x_3 y_3 \\
=& \, 0 + \epsilon_{i12} x_1 y_2 + \epsilon_{i13} x_1 y_3 \\
    & \epsilon_{i21} x_2 y_1 + 0 + \epsilon_{i23} x_2 y_3 \\
    & \epsilon_{i31} x_3 y_1 + \epsilon_{i32} x_3 y_2 + 0 \\
\end{align*}

The thing to notice here is that the above operation was just to obtain the $i$-th component of the cross product
of $x \times y$.

This also leads us to the notation where a tensor of rank 3 is to be fed 3 vectors,

$$
\epsilon (e_i, e_j, e_k) = \epsilon_{ijk} = e_i \cdot (e_j \times e_k)
$$

Where, again, we see that the rank 3 tensor takes 3 vectors as input and produces a number.
(a linear map or a functional one might say.)

From the explicit computation of the cross product we can also see that we obtain this matrix-like
structure from computing each of the $i$-th components
$$
\begin{pmatrix}
\epsilon_{i11} & \epsilon_{i12} & \epsilon_{i13} \\
\epsilon_{i21} & \epsilon_{i22} & \epsilon_{i23} \\
\epsilon_{i31} & \epsilon_{i32} & \epsilon_{i33}
\end{pmatrix}
$$

when $i=1$,
$$
\begin{pmatrix}
\epsilon_{111} & \epsilon_{112} & \epsilon_{113} \\
\epsilon_{121} & \epsilon_{122} & \epsilon_{123} \\
\epsilon_{131} & \epsilon_{132} & \epsilon_{133}
\end{pmatrix}
=
\begin{pmatrix}
0 & 0 & 0 \\
0 & 0 & 1 \\
0 & -1 & 0
\end{pmatrix}
$$

and for $n=3$ there are two other such matrices.
And such is the Levi-Civita tensor of rank 3!

The other two "layers" of the tensor are

$$
\begin{pmatrix}
\epsilon_{211} & \epsilon_{212} & \epsilon_{213} \\
\epsilon_{221} & \epsilon_{222} & \epsilon_{223} \\
\epsilon_{231} & \epsilon_{232} & \epsilon_{233}
\end{pmatrix}
=
\begin{pmatrix}
0 & 0 & -1 \\
0 & 0 & 0 \\
1 & 0 & 0
\end{pmatrix}
$$

$$
\begin{pmatrix}
\epsilon_{311} & \epsilon_{312} & \epsilon_{313} \\
\epsilon_{321} & \epsilon_{322} & \epsilon_{323} \\
\epsilon_{331} & \epsilon_{332} & \epsilon_{333}
\end{pmatrix}
=
\begin{pmatrix}
0 & 1 & 0 \\
-1 & 0 & 0 \\
0 & 0 & 0
\end{pmatrix}
$$