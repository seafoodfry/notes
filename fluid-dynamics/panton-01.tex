\section{Continuum mechanics}

%%%%%%%%%%%%%%%%%%%%%%%%%%%%%%%%%%%%%%%%%%%%%%%%%%%%%%%%%%%%%%%%%%%%%%
\subsection{Exercises}

%%%%%%%%%%%%%%%%%%%%%%%
\textbf{1.1}
\\

(B) Consider an unsteady one-dimensional flow where the density and velocity depend on $x$ and $t$.
A Galilean transformation into a new set of variables $x^\prime$, $t^\prime$ is given by the equation $x = x^\prime + V t^\prime$,
$t = t^\prime$, where $V$ is a constant velocity.
For the moment let $f = f(x,t)$ stand for a function that we wish to express in the $x^{\prime}-t^{\prime}$ coordinate system.
By careful use of the chain rules of calculus, find the expressions for $\partial f / \partial t^\prime$ and $\partial f / \partial x^\prime$.
Next, consider the substantial derivatives of $\rho$ and $v$, which are
$$
\frac{\partial \rho^\prime}{\partial t^\prime} + v^\prime \frac{\partial \rho^\prime}{\partial x^\prime},
$$
$$
\frac{\partial v^\prime}{\partial t^\prime} + v^\prime \frac{\partial v^\prime}{\partial x^\prime},
$$

Show that the substantial derivatives above have exactly the same mathematical form when transformed into the $x-t$ coordinate system
(note that $\rho^\prime = \rho$ and $v^\prime = v - V$).
\\

Let's start with the chain rule to obtain the coordinate transformation for our partial derivatives.
Since $f = f(x,t) = f \left(x(x^\prime, t^\prime), t(t^\prime)\right)$, the new partial derivatives will be
$$
\frac{\partial f}{\partial t^\prime} = 
    \frac{\partial f}{\partial x}\frac{\partial x}{\partial t^\prime} +
    \frac{\partial f}{\partial t}\frac{\partial t}{\partial t^\prime}
$$

Since $t = t^\prime$, $\partial t / \partial t^\prime = 1$, and since $x = x^\prime + V t^\prime$,
$\partial x / \partial t^\prime = \partial_{t^\prime} \left( x^\prime + V t^\prime \right) = V$.
So
$$
\frac{\partial f}{\partial t} =
    \frac{\partial f}{\partial t} +
    V \frac{\partial f}{\partial x}
$$

Similarly,
$$
\frac{\partial f}{\partial x^\prime} = 
    \frac{\partial f}{\partial x} \cancelto{1}{ \frac{\partial x}{\partial x^\prime} } +
    \frac{\partial f}{\partial t} \cancelto{0}{ \frac{\partial t}{\partial x^\prime} }
= \frac{\partial f}{\partial x}
$$

The first simplification comes about because $\partial_{x^\prime} x = \partial_{x^\prime} \left( x^\prime + Vt^\prime\right)$.
The second because $t$ does not depend on $x^\prime$.


Now we can look at the substantial derivatives.
\begin{align*}
\frac{\partial \rho^\prime}{\partial t^\prime} + v^\prime \frac{\partial \rho^\prime}{\partial x^\prime}
    &\rightarrow \frac{\partial \rho}{\partial t^\prime} + (v-V) \frac{\partial \rho}{\partial x^\prime} \\
&= \frac{\partial \rho}{\partial t} + V \frac{\partial \rho}{\partial x}
    + (v-V) \frac{\partial \rho}{\partial x} \\
&= \frac{\partial \rho}{\partial t} + v \frac{\partial \rho}{\partial x}
\end{align*}



And,
\begin{align*}
\frac{\partial v^\prime}{\partial t^\prime} + v^\prime \frac{\partial v^\prime}{\partial x^\prime}
    &\rightarrow \frac{\partial (v-V)}{\partial t^\prime} + (v-V) \frac{\partial (v-V)}{\partial x^\prime} \\
&= \frac{\partial v}{\partial t} + V \frac{\partial v}{\partial x} +
    (v-V) \frac{\partial v}{\partial x} \\
&= \frac{\partial v}{\partial t} + v \frac{\partial v}{\partial x}
\end{align*}

So $\frac{D\rho}{Dt}$ and $\frac{Dv}{Dt}$ are both invariant under Galilean transformation.
\\~\\



%%%%%%%%%%%%%%%%%%%%%%%
\textbf{1.2}
\\

(A) A droplet of liquid is moving through a gas.
It evaporates uniformly, does not deform, and has no internal circulation.
A control region coinciding with the liquid is what type of region?
\\

Since there is no internal circulation, a material or volume region does not apply.
The droplet evaporates, so it could be a fixed region, but since the droplet is moving and the control region
coincides with the liquid then we must use an arbitrary region.
\\~\\


%%%%%%%%%%%%%%%%%%%%%%%
\textbf{1.3}
\\

(A) A droplet of liquid is moving through a gas.
It does not evaporate or deform, but it does have an internal (and surface) circulation.
Describe the velocity of a material region whose surface encloses the droplet.
\\

A material region has a local velocity composed of three parts:
1.) rising velocity, 2.) expansion velocity, and 3.) the velocity at the interface due to internal circulation.
If the droplet is not deforming, then there is no expansion velocity.
\\~\\

%%%%%%%%%%%%%%%%%%%%%%%
\textbf{1.4}
\\

(B) A material region was defined as one where the surface velocity $\mathbf{w}$ is everywhere equal to the
fluid velocity $\mathbf{v}$.
Such a region always contains the same fluid.
Can you define $\mathbf{w}$ in a less restrictive way and still have a region that always contains the same material?
\\

A suggestion we found across the internet was to have $\mathbf{w} = \left(\mathbf{w}\cdot\mathbf{n}\right)\mathbf{n} = 0$ be
the normal velocity of the fluid at the control region.
Fluid would be able to slide or circulate but the normal component of the velocity ebing zero would mean that all the fluid stays within
the control surface.
\\~\\


%%%%%%%%%%%%%%%%%%%%%%%
\textbf{1.5}
\\

(C) Prove that the avergae of the random molecular velocities $\mathbf{v}^{\prime}_{i}$ is zero, that is
$$
\lim_{L\rightarrow 0} \Sigma m_i \mathbf{v}^{\prime}_{i} = 0
$$

In terms of notation, $\mathbf{v}^{\prime}_{i}$ is the random molecular velocity of particle $i$.
This random molecular velocity can also be expressed as
$$
\mathbf{v}^{\prime}_{i} = \mathbf{v}_i - \mathbf{v}
$$

Where $\mathbf{v}_i$ is the molecular velocity of particle $i$ and $\mathbf{v}$ is the average fluid velocity.

In our case, the aerage velocity is the mass-average velocity
$$
\mathbf{v} = \lim_{L\rightarrow 0} \frac{\Sigma m_i \mathbf{v}_i}{\Sigma m_i}
$$