%%%%%%%%%%%%%%%%%%%%%%%%%%%%%%%%%%%%%%%%%%%%%%%%%%%%%%%%%%%%%%%%%%%%%%%%%%%%%%%%%%%%%%%%%%%%%%%
\section{Dynamic Properties of the Monte Carlo Method in Statistical Physics}

These are notes from the paper
"Dynamic Properties of the Monte Carlo Method in Statistical Physics"
by H. Muller-Krumbhaar and K Binder,
\href{https://link.springer.com/article/10.1007/BF01008440}{Journal of Statistical Physics, Vol 8, No. 1, 1973}.

References 5, 6, 27, and 28 mention processes for estimating statistical errors.

However, the processes mentioned in ref 28 disregard the knowledge that one can draw from
transition probabilities about the well-defined correlations between subsequent configurations.




%%%%%%%%%%%%%%%%%%%%%%%%%%%%%%%%%%%%%%%%%%%%%%%%%%%%%%%%%%%%%%%%%%%%%%%%%%%%%%%%%%%%%%%%%%%%%%%%%%%
\section{Central Limit Theorem}

A collection of random variables is independent and identically distributed (IID)
if each random variable has the same probability distribution as the others and
all are mutually independent.

Another way of thinking about an IID set of random variables is as a random sample.

If an arbitrarily large number of samples, each involving multiple observations (data points),
were separately used in order to compute one value of a statistic
(such as, for example, the sample mean or sample variance) for each sample,
then the sampling distribution is the probability distribution of the values that the statistic takes on.
\\

The central limit theorem establishes that, in many situations,
for independent and identically distributed random variables,
the sampling distribution of the standardized sample mean tends towards the standard normal distribution
even if the original variables themselves are not normally distributed.
\\

Normal distributions are often used to represent real-valued random variables
whose distributions are not known because of the central limit theorem.
One example of IID data could be measurement errors.




%%%%%%%%%%%%%%%%%%%%%%%%%%%%%%%%%%%%%%%%%%%%%%%%%%%%%%%%%%%%%%%%%%%%%%%%%%%%%%%%%%%%%%%%%%%%%%%%%%%
\section{Gaussians and the Error Function}

The error function was first formally introduced in
\href{https://books.google.com/books?id=8Po7AQAAMAAJ&pg=RA1-PA294#v=onepage&q&f=false}{On a class of definite integrals by J.W.L. Glaisher B.A. F.R.A.S. F.C.P.S.}.
(We found that link in the wikipedia page talking about the Error function
\href{https://en.wikipedia.org/wiki/Error_function}{Error function}, it was reference number 3.)

Some great references we recommend watching and working through are
\begin{enumerate}
\item \href{https://www.youtube.com/watch?v=jkytxdedxhU}{The impossible integral of e\^ (x\^ 2) \& the error function by blackpenredpen}
\item \href{https://www.youtube.com/watch?v=zorcLisjRUI}{Integral of e\^(x\^2) \& the Imaginary Error Function by blackpenredpen}
\end{enumerate}