\section{The Laplace Transform}

\begin{align*}
L[1] &= \int_{0}^{\infty} e^{-st} dt \\
&= -\frac{1}{s} e^{-st} \big|_{0}^{\infty} \\
&= \frac{1}{s}
\end{align*}



\begin{align*}
L[t] &= \int_{0}^{\infty} t e^{-st} dt \rightarrow
\left[
    \begin{alignedat}{2}
    u  &= t     \quad & v  &= -\frac{1}{s} e^{-st} \\
    du &= dt    \quad & dv &= e^{-st} dt 
    \end{alignedat}\,
\right] \\
&= -\frac{t}{s} e^{-st} \big|_{0}^{\infty} + \int_{0}^{\infty} \frac{1}{s} e^{-st} dt \\
&= \frac{1}{s^2}
\end{align*}


\begin{align*}
L[t^2] &= \int_{0}^{\infty} t^2 e^{-st} dt \rightarrow
\left[
    \begin{alignedat}{2}
    u  &= t^2      \quad & v  &= -\frac{1}{s} e^{-st} \\
    du &= 2t dt    \quad & dv &= e^{-st} dt 
    \end{alignedat}\,
\right] \\
&= -\frac{t^2}{s} e^{-st} \big|_{0}^{\infty} + \int_{0}^{\infty} \frac{2}{s} t e^{-st} dt \\
&= \frac{2}{s} \int_{0}^{\infty} t e^{-st} dt \\
&= \frac{2}{s^3}
\end{align*}

There is an interesting pattern showing here, let's try a more generic case now.

\begin{align*}
L[t^n] &= \int_{0}^{\infty} t^n e^{-st} dt \rightarrow
\left[
    \begin{alignedat}{2}
    u  &= t^n             \quad & v  &= -\frac{1}{s} e^{-st} \\
    du &= n t^{n-1} dt    \quad & dv &= e^{-st} dt 
    \end{alignedat}\,
\right] \\
&= \cancelto{0}{ -\frac{t^n}{s} e^{-st} \big|_{0}^{\infty} } + \frac{n}{s} \int_{0}^{\infty} t^{n-1} e^{-st} dt
\end{align*}

Note that we can tell that the "boundary term" will be zero for any $n$ as an exponential grows
faster than $t^n$.

However, we get an interesting recurrence relation:
$$
L[t^n] = \int_{0}^{\infty} t^n e^{-st} dt = \frac{n}{s} \int_{0}^{\infty} t^{n-1} e^{-st} dt
$$
By doing a couple terms we can even see how
$$
L[t^n] = \frac{n!}{s^{n+1}}
$$

%%%%%%%%%%%%%%%%%%%%%%%%%%%%%

Now, let's look into some other functions.

\begin{align*}
L[ e^{kt} ] &= \int_{0}^{\infty} e^{kt} e^{-st} dt \\
&= \int_{0}^{\infty} e^{-(s-k)t} dt
\end{align*}
Note that if $s < k$, the the integral blows up.
So let's assume $s>k$,
\begin{align*}
L[ e^{kt} ] &= \int_{0}^{\infty} e^{-(s-k)t} dt \\
&= \frac{1}{s-k}
\end{align*}


\begin{align*}
L[ \sin(kt) ] &= \int_{0}^{\infty} \sin(kt) e^{-st} dt \rightarrow
\left[
    \begin{alignedat}{2}
    u  &= \sin kt         \quad & v  &= -\frac{1}{s} e^{-st} \\
    du &= k \cos (kt) dt    \quad & dv &= e^{-st} dt 
    \end{alignedat}\,
\right] \\
&= \cancelto{0}{ -\frac{1}{s} \sin (kt) e^{-st} \big|_{0}^{\infty} } +
    \frac{k}{s} \int_{0}^{\infty} \cos (kt) e^{-st}
\end{align*}

We now need to calculate $L[ \cos(kt) ]$,
\begin{align*}
L[ \cos(kt) ] &= \int_{0}^{\infty} \cos(kt) e^{-st} dt \rightarrow
\left[
    \begin{alignedat}{2}
    u  &= \cos kt         \quad & v  &= -\frac{1}{s} e^{-st} \\
    du &= -k \sin (kt) dt    \quad & dv &= e^{-st} dt 
    \end{alignedat}\,
\right] \\
&= -\frac{1}{s} \cos (kt) e^{-st} \big|_{0}^{\infty}
    - \frac{k}{s} \int_{0}^{\infty} \sin (kt) e^{-st} \\
&= \frac{1}{s} - \frac{k}{s} \int_{0}^{\infty} \sin (kt) e^{-st}
\end{align*}

We get two interesting results here together.
First,
\begin{align*}
L[ \sin(kt) ] &= \int_{0}^{\infty} \sin(kt) e^{-st} dt \\
&= \frac{k}{s} \int_{0}^{\infty} \cos (kt) e^{-st} \\
&= \frac{k}{s} \left( \frac{1}{s} - \frac{k}{s} \int_{0}^{\infty} \sin (kt) e^{-st} \right) \\
&= \frac{k}{s^2} - \frac{k^2}{s^2} \int_{0}^{\infty} \sin (kt) e^{-st}
\end{align*}

Which can be re-arranged into,
\begin{align*}
\left( 1 + \frac{k^2}{s^2} \right) \int_{0}^{\infty} \sin(kt) e^{-st} dt 
&= \frac{s^2 + k^2}{s^2} \int_{0}^{\infty} \sin(kt) e^{-st} dt \\
&= \frac{k}{s^2}
\end{align*}

Meaning that,
\begin{align*}
L[ \sin(kt) ] &= \int_{0}^{\infty} \sin(kt) e^{-st} dt \\
&= \frac{k}{s^2 + k^2}
\end{align*}

And now that we have this result, we can go back to,
\begin{align*}
L[ \cos(kt) ] &= \int_{0}^{\infty} \cos(kt) e^{-st} dt \\
&= \frac{1}{s} - \frac{k}{s} \int_{0}^{\infty} \sin (kt) e^{-st} \\
&= \frac{1}{s} - \frac{k}{s} \frac{k}{s^2 + k^2} \\
&= \frac{1}{s}  - \frac{k^2}{s(s^2 + k^2)} \\
&= \frac{s^2 + k^2  - k^2}{s(s^2 + k^2)} \\
&= \frac{s}{s^2 + k^2}
\end{align*}