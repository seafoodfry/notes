\section{Numerical Analysis}

\subsection{Errors}

As NR explains: Arithmetic among numbers in floating-point representation is not exact, even if
the operands happen to be exactly represented.
For example, two floating numbers are added by first right-shifting
(dividing by two) the mantissa of the smaller (in magnitude) one and simultaneously
increasing its exponent until the two operands have the same exponent. Low-order
(least significant) bits of the smaller operand are lost by this shifting. If the two
operands differ too greatly in magnitude, then the smaller operand is effectively re-
placed by zero, since it is right-shifted to oblivion.
\\

The smallest (in magnitude) floating-point number that, when added to the
floating-point number 1.0, produces a floating-point result different from 1.0 is term-
ed the \textbf{machine accuracy}, $\epsilon_m$.

It is important to understand that $\epsilon_m$ is not the smallest floating-point number
that can be represented on a machine.
That number depends on how many bits there
are in the exponent, while $\epsilon_m$ depends on how many bits there are in the mantissa.


Pretty much any
arithmetic operation among floating numbers should be thought of as introducing an
additional fractional error of at least $\epsilon_m$.
This type of error is called roundoff error.

Roundoff errors accumulate with increasing amounts of calculation. If, in the
course of obtaining a calculated value, you perform $N$ such arithmetic operations,
you might be so lucky as to have a total roundoff error on the order of $\sqrt{N}\epsilon_m$, if
the roundoff errors come in randomly up or down. (The square root comes from
a random-walk.) However, this estimate can be very badly off the mark for two
reasons:

(1) It very frequently happens that the regularities of your calculation, or the
peculiarities of your computer, cause the roundoff errors to accumulate preferentially
in one direction. In this case the total will be of order $N\epsilon_m$.

(2) Some especially unfavorable occurrences can vastly increase the roundoff
error of single operations. Generally these can be traced to the subtraction of two
very nearly equal numbers, giving a result whose only significant bits are those (few)
low-order ones in which the operands differed.
\\~\\

...It is only a slight exaggeration to say that clever minimization of
truncation error is practically the entire content of the field of numerical analysis!
\\