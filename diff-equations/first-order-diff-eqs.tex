\section{First Order Differential Equations}

%%%%%%%%%%%%%%%%%%%%%%%%%%%%%%%%%%%%%%%%%%%%%%%%%%%%%%%%%%%%%%%%%
\subsection{Linear Equations}

%%%%%%%%%%%%%%%%%%%%%%%%%%%%%%%%%%%%%%%%%%%%%%%%%%%%%%%%%%%%%%%%%
\subsubsection{Method of Integrating Factors}


$$
p(t) \frac{dy}{dx} + q(t)y = g(t)
$$

$$
\frac{d}{dt} \left[ \mu(t)y \right] =
\mu(t) \frac{dy}{dt} + y \frac{d\mu(t)}{dt} \sim p(t) \frac{dy}{dx} + q(t)y
$$


%%%%%%%%%%%%%%%%%%%%%%%%%%%%%%%%%%%%%%%%%%%%%%%%%%%%%%%%%%%%%%%%%
\subsubsection{Separable Equations}

$$
M(x) + N(y) \frac{dy}{dx} = 0
$$

Can be written in \textbf{differential form} as
$$
M(x)dx + N(y)dy = 0
$$


\subsubsection{Notes}

Sometimes equations of the form
$$
\frac{dy}{dx} = f(x,y)
$$
have a constant solution $y = y_0$.

For example,
$$
\frac{dy}{dx} = \frac{(y-3) \cos{x}}{1+2y^2}
$$
Has a constant solution $y=3$.
\\





%%%%%%%%%%%%%%%%%%%%%%%%%%%%%%%%%%%%%%%%%%%%%%%%%%%%%%%%%%%%%%%%%
\subsection{Modeling with First Order Equations}


\subsubsection{Example 1: Mixing}
$$
\frac{d Q}{dt} + \frac{r}{100}Q = \frac{r}{4}
$$

Using the method of Integrating factors, we have
$$
\frac{d}{dt}\left[\mu(t)Q(t)\right] = \mu\frac{dQ}{dt} + Q \frac{d\mu}{dt}
= \mu \frac{d Q}{dt} + \mu \frac{r}{100}Q = \mu \frac{r}{4}
$$

Comparing
$$
Q \frac{d\mu}{dt} \sim \mu \frac{r}{100}Q
$$
We have that
$$
\frac{d\mu}{dt} = \frac{r}{100} \mu
$$

So the integrating factor must be
$$
\int \frac{1}{\mu} \frac{d\mu}{dt} dt = ln |\mu| 
= \int \frac{r}{100} = \frac{r}{100}t + C_0
$$

And so
$$
\mu(t) = e^{\frac{r}{100}t + C_0} = C_1e^{\frac{rt}{100}}
$$

Our original equation becomes

\begin{align*}
    \frac{d}{dt} \left[ C_1e^{\frac{rt}{100}} Q \right]  &=   
    C_1e^{\frac{rt}{100}} \frac{d Q}{dt} + C_1e^{\frac{rt}{100}} \frac{r}{100}Q = C_1e^{\frac{rt}{100}} \frac{r}{4}
\end{align*}

Now, we can finally integrate both sides,
\begin{align*}
    \int \frac{d}{dt} \left[ C_1e^{\frac{rt}{100}} Q \right] dt  &=  C_1e^{\frac{rt}{100}} Q     \\
    &= \int C_1e^{\frac{rt}{100}} \frac{r}{4} dt    \\
    &= \frac{r}{4} \frac{100}{r} C_1e^{\frac{rt}{100}} + C_2 \\
    &= 25 C_1e^{\frac{rt}{100}} + C_2 \\
\end{align*}

So our general solution is
$$
C_1e^{\frac{rt}{100}} Q = 25 C_1e^{\frac{rt}{100}} + C_2
$$
or
$$
Q = 25 + C e^{\frac{-rt}{100}}
$$

Since $Q(t=0) = Q_0$
$$
Q_0 = 25 + C \rightarrow C = Q_0 - 25
$$

And
\begin{align*}
    Q(t) &= 25 + (Q_0 - 25) e^{\frac{-rt}{100}} \\
    &= 25(1 - e^{\frac{-rt}{100}}) + Q_0 e^{\frac{-rt}{100}}
\end{align*}

When we want to solve for the time $T$ after which the salt level is within $2\%$ of $Q_L$ (the limiting ammount),
we do it as follows:
$$
25.5 = 25 + 25 e^{-rT/100} \rightarrow \frac{1}{2} = 25 e^{-rT/100}
$$
$$
= \frac{1}{50} = e^{-rT/100} \rightarrow \ln (1/50) = \frac{-rT}{100}
$$
$$
= - \frac{100}{r} \ln(1/50) = \frac{100}{r} \ln{50}
$$