\section{First Order Differential Equations}

%%%%%%%%%%%%%%%%%%%%%%%%%%%%%%%%%%%%%%%%%%%%%%%%%%%%%%%%%%%%%%%%%
\subsection{Linear Equations}

$$
\frac{dy}{dt} = f(y,t)
$$

If $f$ is a linear function on $y$, the we have a first order linear differential equation.

The simplest type first order linear equation is one in which the coefficients are constants.
For example,
$$
\frac{dy}{dt} = -ay + b
$$

The above can be generalized into
$$
\frac{dy}{dt} + p(t) y = g(t)
$$
Where the coefficients are now functions of the independent variable.
Furhtermore, the above can also be generalized as
$$
p(t) \frac{dy}{dx} + q(t)y = g(t)
$$

%%%%%%%%%%%%%%%%%%%%%%%%%%%%%%%%%%%%%%%%%%%%%%%%%%%%%%%%%%%%%%%%%
\subsubsection{Method of Integrating Factors}

Multiply the equation my the integrating factor and the equation is converted into one that can
be integrated using the product rule for derivates.


$$
\frac{d}{dt} \left[ \mu(t)y \right] =
\mu(t) \frac{dy}{dt} + y \frac{d\mu(t)}{dt} \sim p(t) \frac{dy}{dx} + q(t)y
$$

A common presentation for equations
that can readily be solved by the method of integrating factors,
$$
\frac{dy}{dt} + cy = f(t)
$$
Where $c$ is a constant.

Also, make sure to remember to do the comparsion of $y \frac{d\mu(t)}{dt}$ properly.
For example, using the last version we wrote, the integrating factor would come from the comparison of
$$
y \frac{d\mu(t)}{dt} \sim y c\mu(t) \rightarrow \frac{d\mu(t)}{dt} \sim c\mu(t)
$$

This integrating factor also looks like an exponential after differentiation.



%%%%%%%%%%%%%%%%%%%%%%%%%%%%%%%%%%%%%%%%%%%%%%%%%%%%%%%%%%%%%%%%%
\subsubsection{Separable Equations}

$$
M(x) + N(y) \frac{dy}{dx} = 0
$$

Can be written in \textbf{differential form} as
$$
M(x)dx + N(y)dy = 0
$$


\subsubsection{Notes}

Sometimes equations of the form
$$
\frac{dy}{dx} = f(x,y)
$$
have a constant solution $y = y_0$.

For example,
$$
\frac{dy}{dx} = \frac{(y-3) \cos{x}}{1+2y^2}
$$
Has a constant solution $y=3$.
\\





%%%%%%%%%%%%%%%%%%%%%%%%%%%%%%%%%%%%%%%%%%%%%%%%%%%%%%%%%%%%%%%%%
\subsection{Modeling with First Order Equations}


\subsubsection{Example 1: Mixing}
$$
\frac{d Q}{dt} + \frac{r}{100}Q = \frac{r}{4}
$$

Using the method of Integrating factors, we have
$$
\frac{d}{dt}\left[\mu(t)Q(t)\right] = \mu\frac{dQ}{dt} + Q \frac{d\mu}{dt}
= \mu \frac{d Q}{dt} + \mu \frac{r}{100}Q = \mu \frac{r}{4}
$$

Comparing
$$
Q \frac{d\mu}{dt} \sim \mu \frac{r}{100}Q
$$
We have that
$$
\frac{d\mu}{dt} = \frac{r}{100} \mu
$$

So the integrating factor must be
$$
\int \frac{1}{\mu} \frac{d\mu}{dt} dt = ln |\mu| 
= \int \frac{r}{100} = \frac{r}{100}t + C_0
$$

And so
$$
\mu(t) = e^{\frac{r}{100}t + C_0} = C_1e^{\frac{rt}{100}}
$$

Our original equation becomes

\begin{align*}
    \frac{d}{dt} \left[ C_1e^{\frac{rt}{100}} Q \right]  &=   
    C_1e^{\frac{rt}{100}} \frac{d Q}{dt} + C_1e^{\frac{rt}{100}} \frac{r}{100}Q = C_1e^{\frac{rt}{100}} \frac{r}{4}
\end{align*}

Now, we can finally integrate both sides,
\begin{align*}
    \int \frac{d}{dt} \left[ C_1e^{\frac{rt}{100}} Q \right] dt  &=  C_1e^{\frac{rt}{100}} Q     \\
    &= \int C_1e^{\frac{rt}{100}} \frac{r}{4} dt    \\
    &= \frac{r}{4} \frac{100}{r} C_1e^{\frac{rt}{100}} + C_2 \\
    &= 25 C_1e^{\frac{rt}{100}} + C_2 \\
\end{align*}

So our general solution is
$$
C_1e^{\frac{rt}{100}} Q = 25 C_1e^{\frac{rt}{100}} + C_2
$$
or
$$
Q = 25 + C e^{\frac{-rt}{100}}
$$

Since $Q(t=0) = Q_0$
$$
Q_0 = 25 + C \rightarrow C = Q_0 - 25
$$

And
\begin{align*}
    Q(t) &= 25 + (Q_0 - 25) e^{\frac{-rt}{100}} \\
    &= 25(1 - e^{\frac{-rt}{100}}) + Q_0 e^{\frac{-rt}{100}}
\end{align*}

When we want to solve for the time $T$ after which the salt level is within $2\%$ of $Q_L$ (the limiting ammount),
we do it as follows:
$$
25.5 = 25 + 25 e^{-rT/100} \rightarrow \frac{1}{2} = 25 e^{-rT/100}
$$
$$
= \frac{1}{50} = e^{-rT/100} \rightarrow \ln (1/50) = \frac{-rT}{100}
$$
$$
= - \frac{100}{r} \ln(1/50) = \frac{100}{r} \ln{50}
$$


\subsubsection{Example 3: Chemicals in a pond}

We will pick up from
$$
\frac{dt}{dt} + \frac{1}{2}q = 10 + 5\sin(2t)
$$
And we can see that we have a nice, simple, first order, linear equation, so we will proceed with
the method of integrating factors.
\begin{align*}
\frac{d}{dt} \left[ \mu(t) q(t) \right] &= \mu \frac{dq}{dt} + q\frac{d\mu}{dt} \\
&= \mu\frac{dt}{dt} + \frac{1}{2}\mu q = 10\mu + 5\mu\sin(2t)
\end{align*}
Means that the integrating factor will be
$$
q\frac{d\mu}{dt} \sim \frac{1}{2}\mu q \rightarrow
\frac{1}{\mu}\frac{d\mu}{dt} \sim \frac{1}{2}
$$
Or
$$
\int \frac{1}{\mu}\frac{d\mu}{dt} dt = \int \frac{1}{2}
$$
Which leads to $\mu(t) = e^{t/2}$.

So our equation becomes
\begin{align*}
\frac{d}{dt} \left[ e^{t/2} q(t) \right] &= e^{t/2} \frac{dt}{dt} + \frac{1}{2} e^{t/2} q = 10 e^{t/2} + 5 e^{t/2} \sin(2t)
\end{align*}

Hence,
\begin{align*}
e^{t/2} q(t) &= \int 10 e^{t/2} dt + \int 5 e^{t/2} \sin(2t) dt \\
&= 20 e^{t/2} + \int 5 e^{t/2} \sin(2t) dt
\end{align*}

Here we have an interesting integral so let's break it down.

\paragraph{An interesting integral}

In the previous expression we ended up with
$$
\int 5 e^{t/2} \sin(2t) dt
$$

The tip here is a chain of integrations by parts and $u$-substitutions.
First, let's recall the rule for integration by parts
$$
\int u dv = uv - \int v du
$$

Now, let's get to it.

\begin{align*}
    \int e^{t/2} \sin(2t) dt &= 
    \left[
      \begin{alignedat}{2}
      u  &= e^{t/2}               \quad & v  &= -\frac{1}{2}\cos(2t) \\
      du &= \frac{1}{2}e^{t/2}dt  \quad & dv &= \sin(2t) dt 
      \end{alignedat}\,
    \right] \\
    &=
    -\frac{1}{2}e^{t/2}\cos(2t) + \frac{1}{4} \left[ 
        \frac{1}{2}e^{t/2}\sin(2t) - \frac{1}{4} \int e^{t/2} \sin(2t) dt
    \right] \\
    &= -\frac{1}{2}e^{t/2}\cos(2t) + \frac{1}{2^3} e^{t/2}\sin(2t)
      - \frac{1}{2^4} \int e^{t/2}\sin(2t) dt
\end{align*}

Notice that we got our initial integral back, so now some algebra will lead us to
\begin{align*}
\left( \int e^{t/2}\sin(2t) dt \right) \left(1 + \frac{1}{2^4} \right) =
-\frac{1}{2}e^{t/2}\cos(2t) + \frac{1}{2^3} e^{t/2}\sin(2t)
\end{align*}

Which can be simplified to
\begin{align*}
\int e^{t/2}\sin(2t) dt   &=
-\frac{2^4}{2}\frac{1}{2^4+1} e^{t/2}\cos(2t) + \frac{2^4}{2^3}\frac{1}{2^4+1} e^{t/2}\sin(2t) \\
&= -\frac{2^3}{2^4+1}e^{t/2}\cos(2t) + \frac{2}{2^4+1} e^{t/2}\sin(2t)
\end{align*}
\\~\\

Now, we can put everything together!
\begin{align*}
e^{t/2} q(t) &= 20 e^{t/2} + \int 5 e^{t/2} \sin(2t) dt \\
&= 20 e^{t/2} + 5 \left[ -\frac{2^3}{2^4+1}e^{t/2}\cos(2t) + \frac{2}{2^4+1} e^{t/2}\sin(2t) \right] \\
&= 20 e^{t/2} -\frac{40}{17}e^{t/2}\cos(2t) + \frac{10}{17} e^{t/2}\sin(2t) + C
\end{align*}

Notice that we trhew in an integration coefficient at the end.
And our final answer is now
$$
q(t) =
20 -\frac{40}{17}\cos(2t) + \frac{10}{17} \sin(2t) + Ce^{-t/2}
$$