\section{Lattic Simulations of a $\phi^4$ scalar Theory}


This document is a guide to help anyone wanting to learn quantum field theory.

This document has come about because I spent a way too much time trying to learn QFT.
There are plenty of great books out there but they only made sense after I had spent a very considerable time
inmersed in mathematics, engineering, and other sciences.

The one thing that always caused me issues is that almost all textbooks or calss notes build the subject from the ground up
- the same way any other subject is taught in school.
You probably even thought "duh!" while reading that previous sentence.
But recent studies, see \cite{hidden-potential}, made me rethink this approach for QFT!
So instead of picking up from whatever may have been taught in your last QM or EM course, I will start with an actual
non-trivial research problem.
This will have you doing research type work right away.
Then we will work backwards and provide you with the background you need!

For a better argument as to why, definetely go and read \cite{hidden-potential}, but the TL;DR is: starting with a real
problem right away will help you start building the experience for what matters and what tools are applied when.
When we go from the ground up, there is the unspoken rule that you agically have to remember some key fact or formula
that someone mentioned years ago and magically know how to apply it to your problem now.
Here, we won't have that issue.

We will begin by building upon the work of \cite{david-lattice-qft}, which is one of the few works available that is more or
less self-contained but also provides some of the code used for its simulations.
A lot of the code snippets in \cite{david-lattice-qft} does not actually work as in, so we fixed it and made it
available in \cite{our-phi-4}.


%%%%%%%%%%%%%%%%%%%%%%%%%%%%%%%%%%%%%%%%%%%%%%%%%%%%%%%%%%%%%%%%%%%%%%%%%%%%%%%%%%%

\subsection{Metropolis Algorithm for the Ising Model}


%%%%%%%%%%%%%%%%%%%%%%%%%%%%%%%%%%%%%%%%%%%%%%%%%%%%%%%%%%%%%%%%%%%%%%%%%%%%%%%%%%%

\subsection{Metropolis Algorithm for a Scalar Field}

We will begin by building upon the work of \cite{david-lattice-qft}, which is one of the few works available that is more or
less self-contained and provides some of the code used to obtain its results.
A lot of the code snippets in \cite{david-lattice-qft} do not actually work as-is, so we fixed it and made it
available in \cite{our-phi-4}.

For the metropolis steps, we need to be able to compute the energy of our system in order to determine whether we
accept or reject a change.

As we saw in the Isisng model example, in order to meet the more rigurous condition of detailed balance we need to
probabilitically accept a change according to

$$
A(\mu \rightarrow \nu)
=
\begin{cases}
1           & \text{if } E_\nu \leq E_\mu \quad(\Delta E \leq 0) \\
e^{-\beta (E_\nu - E_\mu)} = e^{-\beta \Delta E}   & \text{if } E_\nu > E_\mu \quad(\Delta E > 0)
\end{cases}
$$

And this is where we will take a more research-based pedagogy and work backwards - as opposed to sending you off
to learn QFT, quantum mechanics, electrodynamics, etc, etc.

The $\phi^4$ theory in four-dimensional Euclidean space has the following expression for its action

$$
S_{E} = 
\int d^4 x_E \mathcal{L}_{E} =
\int d^4 x \left[ 
    \frac{1}{2} \left( \partial_{E_\mu} \phi \right)^2 + \frac{1}{2} \mu_0 \phi^2 + \frac{1}{4} \lambda_0 \phi^4
\right]
$$

And it turns out that in Euclidean space the action, or Euclidean action, has the same units as the energy,
so from here on out we will refer to the above as the energy.

Just to start out simple, we will work in 2D first.
And we will:

\begin{itemize}
\item Discretize $\phi$: the continuous function $\phi(x_\mu)$ will be represented by $\phi_n$ for $0 \leq n \leq N$,
    on an Lattice of zise $N$ and spacing $a$ between points.
\item Discretize the derivatives: for example, $\partial \phi / \partial x$ becomes
    $\frac{1}{a} \left( \phi_n(x+a,t) - \phi_n(x,t) \right)$
\item Discretize the integral: $\int dx f$ becomes $\sum^{N}_{i} f(n_i) a$
\end{itemize}

All to define the operations we need in a lattice.
Following these steps, and again, starting in 2D, our Euclidean action, the energy, first becomes

$$
S_{E}^{(2)} = 
\int dx dy
\left[ 
    \frac{1}{2} \left[ \left( \frac{\partial \phi}{\partial t} \right)^2 +  \left( \frac{\partial \phi}{\partial x} \right)^2 \right] 
    + \frac{1}{2} \mu_0 \phi^2 + \frac{1}{4} \lambda_0 \phi^4
\right]
$$

Note the kinetic term!


%%%%%%%%%%%%%%%%%%%%%%%%%%%%%%%%%%%%%%%%%%%%%%%%%%%%%%%%%%%%%%%%%%%%%%%%%%%%%%%%%%%

\subsubsection{Interlude: The Kinetic Term}

Note how we started with the kinetic term
$$
\frac{1}{2} \left( \partial_{E_\mu} \phi \right)^2
$$
and in 2D it became
$$
\frac{1}{2} \left[ 
    \left( \frac{\partial \phi}{\partial t} \right)^2 +  \left( \frac{\partial \phi}{\partial x} \right)^2
\right]
$$

The first thing to note is that we are in Euclidean space where the four-vector
$x_{E}^{2} = x_{0}^{2} + x_{1}^{2} + x_{2}^{2} + x_{3}^{2}$.






%%%%%%%%%%%%%%%%%%%%%%%%%%%%%%%%%%%%%%%%%%%%%%%%%%%%%%%%%%%%%%%%%%%%%%%%%%%%%%%%%%%%%%%%%%%%%%%%%%%%%%%%%
\section{Tensors}

Working with a spacetime with a metric signature $(+ - - -)$,
$$
\begin{pmatrix}
    1 & 0  & 0  & 0  \\
    0 & -1 & 0  & 0  \\
    0 & 0  & -1 & 0  \\
    0 & 0  & 0  & -1 \\
\end{pmatrix}
$$

$x^\mu = (x^0, \vect{x})$ and $x_\mu = g_{\mu\nu} x^\nu = (x^0, -\vect{0})$.
To see this explciitly, let's carry out the summations.

When $\mu = 0$, $x_0 = g_{00}x^0 + g_{01}x^1 + g_{02}x^2 + g_{03}x^3 = x^0 + 0 + 0 +0 = x^0$.
For $\mu = 1$, $x_1 = g_{10}x^0 + g_{11}x^1 + g_{12}x^2 + g_{13}x^3 = 0 + x^1 + 0 +0 = -x^0$.
Similarly, when $\mu = 2$, $x_2 = -x^2$, and when $\mu =3$, $x_3 = -x^3$.


%%%%%%%%%%%%%%%%%%%%%%%%%%%%%%%%%%%%%%%
\subsection{Inner Product}

Why does $p \cdot x = g_{\mu\nu} p^\mu x^\nu = g^{\mu\nu} p_\mu x_\nu = p_\mu x^\nu = p^\mu x_\nu = p^0 x^0 - \vect{p}\cdot\vect{x}$?

The metric tensor defines the geometry of spacetime, including the way distances and angles are measured.
The implied summation, einstein summation, effectively 'weights' the components according to the geometry of spacetime.

$g_{\mu\nu}$ lowers indices, while $g^{\mu\nu}$ raises them.
$\delta^{\mu}_{\nu}$ is a diagonal matrix with ones on the diagonal and zeros elsewhere.
It selects the $\mu$-th component when used in a summation, acting like an identity element.

$g^{\mu\alpha} g_{\alpha\nu} = \delta^{\mu}_{\nu}$

$$
g^{\mu\alpha} g_{\alpha\nu} = \sum_{\alpha = 0} g^{\mu\alpha} g_{\alpha\nu} =
g^{\mu 0} g_{0 \nu} + g^{\mu 1} g_{1\nu} + g^{\mu 2} g_{2\nu} + g^{\mu 3} g_{3\nu}
$$

This is essentially a matrix multiplication.
$$
\begin{pmatrix}
    1 & 0  & 0  & 0  \\
    0 & -1 & 0  & 0  \\
    0 & 0  & -1 & 0  \\
    0 & 0  & 0  & -1 \\
\end{pmatrix}
\cdot
\begin{pmatrix}
    1 & 0  & 0  & 0  \\
    0 & -1 & 0  & 0  \\
    0 & 0  & -1 & 0  \\
    0 & 0  & 0  & -1 \\
\end{pmatrix}
=
\begin{pmatrix}
    1 & 0 & 0 & 0 \\
    0 & 1 & 0 & 0 \\
    0 & 0 & 1 & 0 \\
    0 & 0 & 0 & 1 \\
\end{pmatrix}
$$

Going component by component, take the second row and second column, $\mu=1$ and $\nu=1$.
The element for that entry is given by
$$
g^{1\alpha}g_{\alpha 1} =
g^{1 0} g_{0 1} + g^{1 1} g_{1 1} + g^{1 2} g_{2 1} + g^{1 3} g_{3 1}
= 0 + (-1)(-1) + 0 + 0 = 1
$$