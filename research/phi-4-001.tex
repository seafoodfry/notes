\chapter{Lattic Simulations of a $\phi^4$ scalar Theory}


This document is a guide to help anyone wanting to learn quantum field theory.

This document has come about because I spent a way too much time trying to learn QFT.
There are plenty of great books out there but they only made sense after I had spent a very considerable time
inmersed in mathematics, engineering, and other sciences.

The one thing that always caused me issues is that almost all textbooks or calss notes build the subject from the ground up
- the same way any other subject is taught in school.
You probably even thought "duh!" while reading that previous sentence.
But recent studies, see \cite{hidden-potential}, made me rethink this approach for QFT!
So instead of picking up from whatever may have been taught in your last QM or EM course, I will start with an actual
non-trivial research problem.
This will have you doing research type work right away.
Then we will work backwards and provide you with the background you need!

For a better argument as to why, definetely go and read \cite{hidden-potential}, but the TL;DR is: starting with a real
problem right away will help you start building the experience for what matters and what tools are applied when.
When we go from the ground up, there is the unspoken rule that you agically have to remember some key fact or formula
that someone mentioned years ago and magically know how to apply it to your problem now.
Here, we won't have that issue.

We will begin by building upon the work of \cite{david-lattice-qft}, which is one of the few works available that is more or
less self-contained but also provides some of the code used for its simulations.
A lot of the code snippets in \cite{david-lattice-qft} does not actually work as in, so we fixed it and made it
available in \cite{our-phi-4}.


%%%%%%%%%%%%%%%%%%%%%%%%%%%%%%%%%%%%%%%%%%%%%%%%%%%%%%%%%%%%%%%%%%%%%%%%%%%%%%%%%%%

\section{Metropolis Algorithm for the Ising Model}

Single site is updated at a time and the probability of generating such as state is $1/N$.

The detailed condition for Markov Chains Monte Carlos is
$$
\frac{P(\mu \rightarrow \nu)}{P(\nu \rightarrow \mu)}
= \frac{g(\mu \rightarrow \nu) A(\mu \rightarrow \nu)}{g(\nu \rightarrow \mu) A(\mu \rightarrow \nu)}
= \frac{A(\mu \rightarrow \nu)}{A(\mu \rightarrow \nu)}
$$

Where $g$ is the probability of generating a given state and $A$ is the probability of accepting the transition.

Remember that the more stringent expression of detailed balance to guarantee static equilibirum is written as
$$
p_mu P(\mu\rightarrow\nu) = p_nu P(\nu\rightarrow\mu)
$$

and since the probabilities $p_mu$ and $p_nu$ are $e^{-\beta E_mu}/Z$ and $e^{-\beta E_nu}/Z$ respectively,

$$
\frac{P(\mu \rightarrow \nu)}{P(\nu \rightarrow \mu)}
= \frac{A(\mu \rightarrow \nu)}{A(\mu \rightarrow \nu)}
= \frac{p_\nu}{p_\mu}
= e^{-\beta (E_\nu - E_\mu)}
$$

Note that if $E_\nu \leq E_\mu$, then $e^{-\beta (E_\nu - E_\mu)}$, then the argument of the exponent will be zero
or positive, so $e^{+(\ldots)} \geq 1$.
And if $E_\nu > E_\mu$, then $e^{-\beta (E_\nu - E_\mu)}$ will looke like $e^{-(\ldots)} < 1$.
Which intuitevely makes sense, transitioning to a state $\nu$ from a state $\mu$ is favored when
the end state $\nu$ has a lower energy (rememeber that the exponents are proportional to the probability of the given state).

Thus, the Metropolis algorithm 
$$
A(\mu \rightarrow \nu)
=
\begin{cases}
    1                                                    & \text{if } E_\nu \leq E_\mu \quad(\Delta E \leq 0) \\
    e^{-\beta (E_\nu - E_\mu)} = e^{-\beta \Delta E} \   & \text{if } E_\nu > E_\mu \quad(\Delta E > 0)
\end{cases}
$$


%%%%%%%%%%%%%%%%%%%%%%%%%%%%%%%%%%%%%%%%%%%%%%%%%%%%%%%%%%%%%%%%%%%%%%%%%%%%%%%%%%%

\section{Metropolis Algorithm for a Scalar Field}

We will begin by building upon the work of \cite{david-lattice-qft}, which is one of the few works available that is more or
less self-contained and provides some of the code used to obtain its results.
A lot of the code snippets in \cite{david-lattice-qft} do not actually work as-is, so we fixed it and made it
available in \cite{our-phi-4}.

For the metropolis steps, we need to be able to compute the energy of our system in order to determine whether we
accept or reject a change.

As we saw in the Isisng model example, in order to meet the more rigurous condition of detailed balance we need to
probabilitically accept a change according to

$$
A(\mu \rightarrow \nu)
=
\begin{cases}
1           & \text{if } E_\nu \leq E_\mu \quad(\Delta E \leq 0) \\
e^{-\beta (E_\nu - E_\mu)} = e^{-\beta \Delta E}   & \text{if } E_\nu > E_\mu \quad(\Delta E > 0)
\end{cases}
$$

And this is where we will take a more research-based pedagogy and work backwards - as opposed to sending you off
to learn QFT, quantum mechanics, electrodynamics, etc, etc.

The $\phi^4$ theory in four-dimensional Euclidean space has the following expression for its action

$$
S_{E} = 
\int d^4 x_E \mathcal{L}_{E} =
\int d^4 x \left[ 
    \frac{1}{2} \left( \partial_{E_\mu} \phi \right)^2 + \frac{1}{2} \mu^{2} \phi^2 + \frac{1}{4!} \lambda \phi^4
\right]
$$

And it turns out that in Euclidean space the action, or Euclidean action, has the same units as the energy,
so from here on out we will refer to the above as the energy.

Just to start out simple, we will work in 2D first.
And we will:

\begin{itemize}
\item Discretize $\phi$: the continuous function $\phi(x_\mu)$ will be represented by $\phi_n$ for $0 \leq n \leq N$,
    on an Lattice of zise $N$ and spacing $a$ between points.
\item Discretize the derivatives: for example, $\partial \phi / \partial x$ becomes
    $\frac{1}{a} \left( \phi_n(x+a,t) - \phi_n(x,t) \right)$
\item Discretize the integral: $\int dx f$ becomes $\sum^{N}_{i} f(n_i) a$
\end{itemize}

All to define the operations we need in a lattice.
Following these steps, and again, starting in 2D, our Euclidean action, the energy, first becomes

$$
S_{E}^{(2)} = 
\int dx dy
\left[ 
    \frac{1}{2} \left[ \left( \frac{\partial \phi}{\partial t} \right)^2 +  \left( \frac{\partial \phi}{\partial x} \right)^2 \right] 
    + \frac{1}{2} \mu^{2} \phi^2 + \frac{1}{4!} \lambda \phi^4
\right]
$$

Note the kinetic term!
But before that, let's look into tensors, because I never saw a class where this material was clearly covered.
Most of my physics courses just magically skipped over this or gave you the results.
And the math courses I found dove right into the "analysis" (which is pretty damn important but it is not where one
ought to start.)





%%%%%%%%%%%%%%%%%%%%%%%%%%%%%%%%%%%%%%%%%%%%%%%%%%%%%%%%%%%%%%%%%%%%%%%%%%%%%%%%%%%%%%%%%%%%%%%%%%%%%%%%%
\section{Tensors}

Working with a spacetime metric signature $(+ - - -)$,
$$
\begin{pmatrix}
    1 & 0  & 0  & 0  \\
    0 & -1 & 0  & 0  \\
    0 & 0  & -1 & 0  \\
    0 & 0  & 0  & -1 \\
\end{pmatrix}
$$

$x^\mu = (x^0, \vect{x})$ and $x_\mu = g_{\mu\nu} x^\nu = (x^0, -\vect{x})$.
To see this explciitly, let's carry out the summations.

When $\mu = 0$, $x_0 = g_{00}x^0 + g_{01}x^1 + g_{02}x^2 + g_{03}x^3 = x^0 + 0 + 0 +0 = x^0$.
For $\mu = 1$, $x_1 = g_{10}x^0 + g_{11}x^1 + g_{12}x^2 + g_{13}x^3 = 0 + x^1 + 0 +0 = -x^0$.
Similarly, when $\mu = 2$, $x_2 = -x^2$, and when $\mu =3$, $x_3 = -x^3$.


%%%%%%%%%%%%%%%%%%%%%%%%%%%%%%%%%%%%%%%
\subsection{Inner Product}

Why does $p \cdot x = g_{\mu\nu} p^\mu x^\nu = g^{\mu\nu} p_\mu x_\nu = p_\mu x^\nu = p^\mu x_\nu = p^0 x^0 - \vect{p}\cdot\vect{x}$?

The metric tensor defines the geometry of spacetime, including the way distances and angles are measured.
The implied summation, einstein summation, effectively 'weights' the components according to the geometry of spacetime.

$g_{\mu\nu}$ lowers indices, while $g^{\mu\nu}$ raises them.
$\delta^{\mu}_{\nu}$ is a diagonal matrix with ones on the diagonal and zeros elsewhere.
It selects the $\mu$-th component when used in a summation, acting like an identity element.

$g^{\mu\alpha} g_{\alpha\nu} = \delta^{\mu}_{\nu}$

$$
g^{\mu\alpha} g_{\alpha\nu} = \sum_{\alpha = 0} g^{\mu\alpha} g_{\alpha\nu} =
g^{\mu 0} g_{0 \nu} + g^{\mu 1} g_{1\nu} + g^{\mu 2} g_{2\nu} + g^{\mu 3} g_{3\nu}
$$

This is essentially a matrix multiplication.
$$
\begin{pmatrix}
    1 & 0  & 0  & 0  \\
    0 & -1 & 0  & 0  \\
    0 & 0  & -1 & 0  \\
    0 & 0  & 0  & -1 \\
\end{pmatrix}
\cdot
\begin{pmatrix}
    1 & 0  & 0  & 0  \\
    0 & -1 & 0  & 0  \\
    0 & 0  & -1 & 0  \\
    0 & 0  & 0  & -1 \\
\end{pmatrix}
=
\begin{pmatrix}
    1 & 0 & 0 & 0 \\
    0 & 1 & 0 & 0 \\
    0 & 0 & 1 & 0 \\
    0 & 0 & 0 & 1 \\
\end{pmatrix}
$$

Going component by component, take the second row and second column, $\mu=1$ and $\nu=1$.
The element for that entry is given by
$$
g^{1\alpha}g_{\alpha 1} =
g^{1 0} g_{0 1} + g^{1 1} g_{1 1} + g^{1 2} g_{2 1} + g^{1 3} g_{3 1}
= 0 + (-1)(-1) + 0 + 0 = 1
$$


%%%%%%%%%%%%%%%%%%%%%%%%%%%%%%%%%%%%%%%
\subsection{Derivatives}

The examples here come from \cite{tensor-calc}.
The aim here is to see a few examples to get used to the quantities we are going to be seeing later on.
\\

Some useful identities before we carry on:
\begin{enumerate}
    \item
    $$
    \frac{\partial x_p}{\partial x_q} = \delta_{pq}
    $$
\end{enumerate}



\textbf{Example 1}

If $a_{ij}$ are contants, to compute the partial derivative
$$
\frac{\partial}{\partial x_k} a_{ij} x_i x_j
$$

We'll firs write out the sums explciitly.
\begin{align*}
a_{ij} x_i x_j &= \sum_{i,j} a_{ij} x_i x_j
\end{align*}

Someone once happened to noticed that the sum could be broken into parts depending on when $x_{\{i,j\}} = x_k$, where $x_k$
is the one we want to differentiate with respect to,
\begin{align*}
\sum_{i,j} a_{ij} x_i x_j &=
    \sum_{i \neq k, j \neq k} a_{ij} x_i x_j +
    \sum_{i, j \neq k} a_{ij} x_i x_j + 
    \sum_{i \neq k, j} a_{ij} x_i x_j +
    \sum_{i=k, j=k} a_{ij} x_i x_j \\
&= C + \sum_{j \neq k} \left( a_{kj} x_j \right) x_k +
    \sum_{i \neq k} \left( a_{ik} x_i \right) x_k +
    a_{kk} x_k x_k \\
    &= C + \sum_{j \neq k} \left( a_{kj} x_j \right) x_k +
    \sum_{i \neq k} \left( a_{ik} x_i \right) x_k +
    a_{kk} (x_k)^2 \\
\end{align*}

We wrote the first sum, where no $x$s are equal to $x_k$ as $C$, because as you may guess this is going to go away with
any derivative.

Back to our original task,
\begin{align*}
\frac{\partial}{\partial x_k} a_{ij} x_i x_j &=
    0 + \sum_{j \neq k} a_{kj} x_j +
    \sum_{i \neq k} a_{ik} x_i +
    2 a_{kk} x_k \\
&= \sum_{j} a_{kj} x_j + \sum_{i} a_{ik} x_i \\
&= a_{kj} x_j + a_{ik} x_i \\
&= a_{ki} x_i + a_{ik} x_i \\
&= (a_{ik} + a_{ki}) x_i
\end{align*}



\textbf{Example 2}

Another way to work out
$$
\frac{\partial}{\partial x_k} a_{ij} x_i x_j
$$
goes as follows.

\begin{align*}
\frac{\partial}{\partial x_k} a_{ij} x_i x_j &= a_{ij} \frac{\partial}{\partial x_k} x_i x_j \\
&= a_{ij} \left( \frac{\partial x_i}{\partial x_k} x_j + x_i \frac{\partial x_j}{\partial x_k} \right) \\
&= a_{ij} \left( \delta_{ik} x_j + x_i \delta_{jk} \right) \\
&= a_{kj} x_j + a_{ik} x_i \\
&= (a_{ik} + a_{ki}) x_i
\end{align*}


\textbf{Example 3}

If $a_{ij} = a_{ji}$ are constants, compute
$$
\frac{\partial^2}{\partial x_k \partial x_l} a_{ij} x_i x_j
$$

Let's start with our previous result,
\begin{align*}
\frac{\partial^2}{\partial x_k \partial x_l} a_{ij} x_i x_j &=
    \frac{\partial}{\partial x_k} \left[ \frac{\partial}{\partial x_l} a_{ij} x_i x_j \right] \\
&= \frac{\partial}{\partial x_k} (a_{il} + a_{li}) x_i \\
&= 2 a_{li} \frac{\partial x_i}{\partial x_k} \\
&= 2 a_{li} \delta_{ik} \\
&= 2 a_{lk}
\end{align*}



%%%%%%%%%%%%%%%%%%%%%%%%%%%%%%%%%%%%%%%%%%%%%%%%%%%%%%%%%%%%%%%%%%%%%%%%%%
\subsection{Changing Variables: the Jacobian}


Promise we are not doing the conventional thing where we go from scratch but there is one more pattern we have to covered
now in order for some other equations to make sense and this is the Jacobian.


There is a result from calculus that is presented a bit like this (the following presentation comes from
\cite{jacobian}):


If we define the Jacobian of a transformation $x = g(u,v)$, $y = h(u,v)$ as
\begin{align*}
\frac{ \partial (x,y) }{ \partial (u,v) } &= 
\begin{vmatrix}
\dfrac{ \partial x }{ \partial u } & \dfrac{ \partial x }{ \partial v } \\ \addlinespace
\dfrac{ \partial y }{ \partial u } & \dfrac{ \partial y }{ \partial v }
\end{vmatrix} \\
&= \frac{ \partial x }{ \partial u } \frac{ \partial y }{ \partial v } -
    \frac{ \partial x }{ \partial v }\frac{ \partial y }{ \partial u }
\end{align*}

Then if we want to integrate $f(x)$ over some surface $R$, this region will become some other surface $S$
following our transformation and the integral can be computed as follows:
$$
\int \int_{R} f(x,y) dA
= \int \int_{S} f\left( g(u,v), h(u,v) \right) \Biggl| \frac{ \partial (x,y) }{ \partial (u,v) } \Biggl| d\overline{A}
$$

For example, when we go from Euclidean to polar coordinates, the transformation we use is $x = r\cos\theta$
and $y = r\sin\theta$ and the Jacobian is $r$ so
$dA = \left| \frac{\partial (x, y)}{\partial (r, \theta)} \right| dx dy = r dr d\theta$:

\begin{align*}
\frac{\partial (x, y)}{\partial (r, \theta)} &= 
\begin{vmatrix}
    \dfrac{ \partial x }{ \partial r } & \dfrac{ \partial x }{ \partial \theta } \\ \addlinespace
    \dfrac{ \partial y }{ \partial r } & \dfrac{ \partial y }{ \partial \theta }
\end{vmatrix} \\
& = \begin{vmatrix}
    \cos\theta & -r\sin\theta \\ \addlinespace
    \sin\theta & r\cos\theta
\end{vmatrix} \\
&= r \cos^2 \theta + r \sin^2 \theta \\
&= r
\end{align*}



In 3-dimensions this extends to
$$
\int \int \int_{V} f(x,y,z) dV
= \int \int \int_{B} f\left( g(u,v,w), h(u,v,w), k(u,v,w) \right) \Biggl| \frac{ \partial (x,y,z) }{ \partial (u,v,w) } \Biggl| d\overline{V}
$$

For the case of the transformation of Euclidean coordinates to spherical coordinates, the transformation of variables is
$x = r \sin\phi \cos\theta $, $y = r \sin\phi \sin\theta$, and $z = r\cos\phi$.
In that case the Jacobian is $r^2 \sin\phi$ and thus
$dV = \left| \frac{\partial (x, y, z)}{\partial (r, \theta, \phi)} \right| dx dy dz = r^2 \sin\phi dr d\phi d\theta$.


Besides being very widely used, the reason we mentioned it is because this is also that is very well used in
\cite{gifted-qft} to introduce the notation we see so much in QFT.
Specifically, the Jacobian as we have been writing it, can also be seen in expressions such as

$$
\overline{a}^{\mu} = \left( \frac{\partial \overline{x}^\mu}{\partial x^\nu} \right) a^\nu
$$

$$
\frac{\partial \phi}{\partial \overline{x}^\mu} =
    \left( \frac{\partial x^\nu}{\partial \overline{x}^\mu} \right) \frac{\partial \phi}{\partial x^\nu}
$$

The determinant of the Jacobian matrix is used specifically when you are dealing with volume transformations
(like integrating over a volume or changing the measure of integration) because it scales the volume elements according
to how the coordinate transformation stretches or compresses the space.

When transforming vectors, however, you're not scaling a volume but rather reorienting or rescaling each component
of the vector according to how the coordinate axes themselves have changed.
This is why the transformation involves a sum over the product of vector components and the appropriate Jacobian elements:

\begin{itemize}
\item Contravariant vectors (standard vectors): The components of these vectors are transformed by multiplying with the
    Jacobian matrix $\partial \overline{x}^\mu / \partial x^\nu$. Here, each new vector component is a linear combination
    of old components weighted by how much each new coordinate axis changes with respect to each old axis.
    \begin{itemize}
        \item These objects, such as displacement vectors, inherently "follow" the coordinate axes. As the axes stretch
        or rotate, so do the components of the vector. 
    \end{itemize}
\item Covariant vectors (gradients and one-forms): These vectors transform using the inverse of the Jacobian matrix
    $\partial x^\nu / \partial \overline{x}^\mu$. This reflects how changes in the old coordinates map to the new
    coordinates, suitable for objects that naturally pair with vectors, like gradients.
    \begin{itemize}
        \item They measure rates of change along these coordinates.
        \item A larger coordinate stretch means a smaller gradient in that direction to maintain the same rate of change.
    \end{itemize}
\end{itemize}




%%%%%%%%%%%%%%%%%%%%%%%%%%%%%%%%%%%%%%%%%%%%%%%%%%%%%%%%%%%%%%%%%%%%%%%%%%%%%%%%%%%

\section{The Kinetic Term}

Note how we started with the kinetic term
$$
\frac{1}{2} \left( \partial_{E_\mu} \phi \right)^2
$$
and in 2D it became
$$
\frac{1}{2} \left[ 
    \left( \frac{\partial \phi}{\partial t} \right)^2 +  \left( \frac{\partial \phi}{\partial x} \right)^2
\right]
$$

The first thing to note is that we are in Euclidean space where the four-vector
$x_{E}^{2} = x_{0}^{2} + x_{1}^{2} + x_{2}^{2} + x_{3}^{2}$.
If we were back in Mikowski space with out metric signature $(+ - - -)$,
then $x^{2} = x_{0}^{2} - x_{1}^{2} - x_{2}^{2} - x_{3}^{2}$.

To see why the kinetic term looks the way it does, it helps to see it other ways.

For example, its definition is
$$
\frac{1}{2} (\partial_{\mu} \phi)^2 = \frac{1}{2} \sum_{\mu=0}^{3} (\partial^{\mu} \phi)^2
$$

and so in 4D it looks like

$$
\frac{1}{2} (\partial_{\mu} \phi)^2 =
\frac{1}{2} \left( 
    \left(\frac{\partial \phi}{\partial t}\right)^2 +
    \left(\frac{\partial \phi}{\partial x}\right)^2 +
    \left(\frac{\partial \phi}{\partial y}\right)^2 +
    \left(\frac{\partial \phi}{\partial z}\right)^2
\right)
$$

However, remember that this is all in Euclidean space.
In Mikowski space we have,
\begin{align*}
\partial^2 &= \partial^\mu \partial_\mu \\
&= \frac{\partial^2}{\partial t^2} - \frac{\partial^2}{\partial x^2} - \frac{\partial^2}{\partial y^2} - \frac{\partial^2}{\partial z^2} \\
&= \frac{\partial^2}{\partial t^2} - \vect{\nabla}^2
\end{align*}

And this is why it sometimes helps to see the metric included in the definition,

\begin{align*}
\eta^{\mu\nu} \partial_\mu \partial_\nu \phi &= \partial_\mu \partial^\nu \phi \\
&= \left( \partial_t \phi \right)^2 - \delta^{ij} \partial_i \partial_j \phi
\end{align*}

If $\eta^{\mu\nu}$ has the signature $(+ - - -)$.
\\~\\

Anyway, back to our problem of Discretization.
If we define a lattice, with a spacing $a$, then


$$
\lim_{a \rightarrow 0} \partial_x \phi = \frac{1}{a} \left( \phi(x+a, t) - \phi(x, t) \right)
$$

Dropping the $\lim$ notation - trusting in our approximation for the time being - we can then write

$$
\left( \partial_t \phi \right)^2 =
\frac{1}{a^2} \left( \phi^2 (x, t+a) + \phi^2 (x, t) - 2\phi(x, t+a) \phi(x, t) \right)
$$
and
$$
\left( \partial_x \phi \right)^2 =
\frac{1}{a^2} \left( \phi^2 (x+a, t) + \phi^2 (x, t) - 2\phi(x+a, t) \phi(x, t) \right)
$$


If we put this back inside of an integral we have,
\begin{align*}
\int dx \left( \partial_x \phi \right)^2 &\approx
    \int dx \frac{1}{a^2} \left( \phi^2 (x+a, t) + \phi^2 (x, t) - 2\phi(x+a, t) \phi(x, t) \right)
\end{align*}

And so our 2D integral becomes,
\begin{equation*}
\begin{split}
\int dt dx \left[ \left( \partial_t \phi \right)^2 + \left( \partial_x \phi \right)^2 \right] &\approx
\frac{1}{a^2} \int dt dx \left[ \phi^2 (x+a, t) + \phi^2 (x, t) - 2\phi(x+a, t) \phi(x, t) \right. \\
& \left. + \phi^2 (x, t+a) + \phi^2 (x, t) - 2\phi(x, t+a) \phi(x, t) \right]
\end{split}
\end{equation*}


The next step we can take in our road to discretization comes from remembering that your run of the mill
integral is defined via Riemann sums, such that
$$
\int dx f(x) = \lim_{|\Delta x| \rightarrow 0} \sum_{i=1}^{n} f(x_i) \Delta x_i
$$

In our case, we are doing a double integral so $dt \, dx \rightarrow a^2$.
Then we have,
\begin{equation*}
\begin{split}
\int dt dx \left[ \left( \partial_t \phi \right)^2 + \left( \partial_x \phi \right)^2 \right] &\approx
\sum_{x,t} \left[ \phi^2 (x+a, t) + \phi^2 (x, t) - 2\phi(x+a, t) \phi(x, t) \right. \\
    & \left. + \phi^2 (x, t+a) + \phi^2 (x, t) - 2\phi(x, t+a) \phi(x, t) \right]
\end{split}
\end{equation*}


And here lies yet another trick of the trade.
If you remember, we actually want to integrate over these partial derivatives.
And it just so happens that when using periodic boundary conditions (which we always use), a sum over all studies
of $\phi(x+a, t)$ is equal to a sum over all sites of $\phi(x, t)$, the only differnce is what the first and last terms are.

For example, imagine a lattice with just 3 sites, where the lattice spacing $a=1$.
If we drop the $t$ from the list of arguments, these sums are:

$$
\sum_{x=1}^{3} \phi(x+a) = \phi(2) + \phi(3) + \phi(1) 
$$
and
$$
\sum_{x=1}^{3} \phi(x) = \phi(1) + \phi(2) + \phi(3) 
$$

Since we are working with a double (finite) sum, $\sum_{x,t} = \sum_t \sum_x$, then we can apply the above result twice,
allowing us to simplify our integral.

\begin{equation}
\begin{aligned}
\int dt dx \left[ \left( \partial_t \phi \right)^2 + \left( \partial_x \phi \right)^2 \right] &\approx
\sum_{x,t} \left[ \phi^2 (x, t) + \phi^2 (x, t) - 2\phi(x+a, t) \phi(x, t) \right. \\
    & \left. + \phi^2 (x, t) + \phi^2 (x, t) - 2\phi(x, t+a) \phi(x, t) \right] \\
&= \sum_{x,t} \left[ 4\phi^2 (x, t) - 2\phi(x+a, t) \phi(x, t) - 2\phi(x, t+a) \phi(x, t) \right] \label{phi4:2d-kinetic-term}
\end{aligned}
\end{equation}

Here the sum in $x$ is done for each $x_i = an_x$, where $n_x = 0, \ldots, N-1$,
and the sum over $t$ is similarly defined with $t_i = an_t$, where $n_t = 0, \ldots , N-1$.

The next step now lies in thinking about us actually writing the code.
If we think how we would implement a lattice as an array and just wrap things around, we could identify every point in
our lattice with an index $n$ and "reindex" our coordinates.
To map our double sum into a single dimension we use the mapping $n = x + t\times N$.

This way, by introducing some other indices $i$ and $j$ such tha $n+i = t\times N + x + 1$ and
$n+j = t\times N + x + N = (t+1)\times N + x$.
we can simplify our notation by writing
$\phi(x+a, t) = \phi(a(n_x + 1), an_t) \rightarrow \phi(n+1)$ and
$\phi(x, t+a) = \phi(an_x, a(n_t + 1)) \rightarrow \phi(n+j)$.
Giving us

\begin{align*}
\int dt dx \left[ \left( \partial_t \phi \right)^2 + \left( \partial_x \phi \right)^2 \right] &\approx
    \sum_{n} \left[ 4\phi^2 (n) - 2\phi(n+i) \phi(n) - 2\phi(n+j) \phi(n) \right] \\
&= \sum_{n} \left[ 4\phi^2 (n) - 2\phi(n) \left( \phi(n+i) + \phi(n+j) \right) \right]
\end{align*}

If you look at the terms $- 2\phi(n) \left( \phi(n+i) + \phi(n+j) \right)$, these look like the nearest-neighbour
interactions that are seen in the Ising model Hamiltonina, $H = -\sum_{<ij>}s_i s_j$.
The other two products, $- 2\phi(n) \left( \phi(n-i) + \phi(n-j) \right)$, would again come from performing the sum with
periodic boundary conditions.
If we use this, we can now write,

\begin{align}
\int dt dx \left[ \left( \partial_t \phi \right)^2 + \left( \partial_x \phi \right)^2 \right] &\approx
- 2 \sum_{<ij>} \phi_i \phi_j + 4 \sum_n \phi^{2}_{n} \label{phi4:2d-kinetic-term-lattice}
\end{align}


%%%%%%%%%%%%%%%%%%%%%%%%%%%%%%%%%%%%%%%%%%
\subsection{The 4D Kinetic Term}

To get the 4-dimensional expression of the kinetic term we need to go back to \ref{phi4:2d-kinetic-term}, which now becomes

\begin{align*}
\int dt \, dx \, dy \, dz & \left[ 
    \left( \partial_t \phi \right)^2 + \left( \partial_x \phi \right)^2 +  \left( \partial_y \phi \right)^2 + \left( \partial_z \phi \right)^2
\right] \approx \\
\sum & \left[ \phi^2 + \phi^2 - 2\phi(t+a, \ldots) \phi \right. \\
    & + \phi^2 + \phi^2 - 2\phi(x+a, \ldots) \phi \\
    & + \phi^2 + \phi^2 - 2\phi(y+a, \ldots) \phi \\
    & \left. + \phi^2 + \phi^2 - 2\phi(z+a, \ldots) \phi \right] \\
&= \sum \left[ 8\phi^2 - 2\phi(t+a)\phi - 2\phi(x+a)\phi - 2\phi(y+a)\phi - 2\phi(z+a)\phi \right] \\
&= 8 \sum \phi^2 - 2 \sum \phi \left( \phi(t+a) + \phi(x+a) + \phi(y+a) + \phi(z+a) \right) \\
&= 8 \sum \phi^{2} - 2 \sum \phi \left( \phi(n+i) + \phi(n+j) + \phi(n+k) + \phi(n+l) \right)
\end{align*}

Which again, since the summation is over the entire lattice and we have periodic boundary conditions,
we end up with
\begin{align*}
\int dt \, dx \, dy \, dz \left[ 
    \left( \partial_t \phi \right)^2 + \left( \partial_x \phi \right)^2 +  \left( \partial_y \phi \right)^2 + \left( \partial_z \phi \right)^2
\right] \approx 
- 2 \sum_{<ij>} \phi_i \phi_j + 8 \sum_n \phi^{2}_{n}
\end{align*}

Keep in mind that the product of nearest neighbours is now for each $ij$ over each dimension, so it is 8 multiplications in total.





%%%%%%%%%%%%%%%%%%%%%%%%%%%%%%%%%%%%%%%%%%%%%%%%%%%%%%%%%%%%%%%%%%%%%%%%%%%%%%%%%%
\section{The Potential Term}

The next piece of wisdom we need to use is to make our phisical quantities dimensionless.
This helps during the simulation to sustain numerical stability.
Imagine if we were working with electrons and had to multiply everything by $9.1 \times 10^{-31}$ kg.
The electron Mass in MeV is just 0.5 but hopefully you get the point that
it could happen that we are working with a quantity that is just right for us to run into a numerical
underflow.

Let's now look at our potential.
Based on what we did for the kinetic term, we can see right away that

$$
\int dt \, dx \left( \frac{1}{2} \mu^{2} \phi^2 + \frac{1}{4!} \lambda \phi^4 \right)
\approx \sum a^2 \left( \frac{1}{2} \mu^{2} \phi^2 + \frac{1}{4!} \lambda \phi^4 \right)
$$


%%%%%%%%%%%%%%%%%%%%%%%%%%%%%%%%%%%%%%%%%%%%%%%%%%%%%%%%%%%%%%
\subsection{Dimensional Analysis}

Defining a unitless mass $\mu_0$ is straightforward: in natural units, mass has the units of inverse length.
If we square a mass term, then we need to multiply it by a square length term to have a unitless quantity,
and it just so happens that we get that term from the double integral.
So we can define the unitless quantity $\mu^{2} = a^2 \mu^{2}$.

The logic we just followed was a lucky coincidence that happens in 2D.
For example, one could naively think that the coupling constant $\lambda$ would have the units of energy, inverse length.
But since we have an $a^2$ term, that would make our new coupling constant have units of length.

To get the correct answer, in two and four dimensions we have to go back a bit.

To start, the action $S$ is dimensionless - $S = \int dt \, L$, the lagrangian has units of energy,
inverse lenght, so it cancels with $dt$.

In four dimensions, the action is $S = \int d^4x \, L$.
The integration "volume" has units of $[M]^{-4}$, so all terms in the lagrangian need to have units of $[M]^4$.
(In two dimensions all terms would need units of $[M]^2$.)

The first term in our laganragian in a spacetime derivative $\partial_\mu$.
If you remember your calculus days, derivatives are rates of change, i.e.,
$\dot{y}$ is the velocity which has dimensions of meters per second.
And thus $[\partial_mu] = [L^{-1}] = [M^1]$.

From here we can figure out the dimensions of the field $\phi$,
$$
[\left(\partial_\mu \phi\right)^2] = [M]^2 [\phi]^2 = [M]^4
$$
Meaning that $[\phi]^2 = [M]^2$, or $[\phi] = [M]$.

This result is often generalized by noting that
$$
[\left(\partial_\mu \phi\right)^2] = [M]^2 [\phi]^2 = [M]^d
$$
which makes $[\phi] = [M]^{(d-2)/2}$ in $d$ dimensions.
($[\phi] = 1$ (unitless) in two dimensions.)


Let's keep going we are on a roll!
The next term in our lagrangian is,

$$
\left[ \frac{1}{2} \mu^2 \phi^2 \right] = [M]^4
$$

So it happens that in four dimensions, $[\mu]^2 = [M]^2$, since $[\phi]^2 = [M]^2$.
(In two dimensions, $[\mu] = [M]$ as well since $\phi$ is dimensionless.)


Finally, our interaction term, the one that made us start this route,
$$
\left[ \frac{1}{4!} \lambda \phi^4 \right] = [\lambda] [M]^4 = [M]^4
$$
Therefore, $[\lambda] = 1$, so it is dimensionless in four dimensions.
But in two dimensions we have,

$$
\left[ \frac{1}{4!} \lambda \phi^4 \right] = [\lambda] [M]^0 = [M]^2
$$

Here, $[\lambda] = [M]^2 = [\mu^2]$.



%%%%%%%%%%%%%%%%%%%%%%%%%%%%%%%%%%%%%%%%%%%%%%%%%%%%%%%%%%%%%%
\subsection{Full Lagrangian in 2D}

The results we need to highlight from the previous section are that $[\lambda] = [M]^2 = [\mu^2]$.
So we can define the dimensionless quantities in 2D $\mu^{2}_{L} = a^2 \mu^2$ and $\lambda_L = a^2 \lambda$,
but we will drop the $L$ subscript since it gets old.

That way the potential term for our Lagrangian becomes

$$
\int dt \, dx \left( \frac{1}{2} \mu^{2} \phi^2 + \frac{1}{4!} \lambda \phi^4 \right)
\approx \sum \left( \frac{1}{2} \mu^{2} \phi^2 + \frac{1}{4!} \lambda \phi^4 \right)
$$

Putting it together with the kinetic term \ref{phi4:2d-kinetic-term-lattice} (and the factor of $1/2$ we didn't carry around)
we have

\begin{equation}
\begin{split}
\int dt \, dx & \left(
    \frac{1}{2} \left[ \left( \partial_t \phi \right)^2 + \left( \partial_x \phi \right)^2 \right] + 
    \frac{1}{2} \mu^{2} \phi^2 + \frac{1}{4!} \lambda \phi^4
\right) \\
& \approx - \sum_{<ij>} \phi_i \phi_j + 2 \sum_n \phi^{2}_{n}
    + \sum_n \left( \frac{1}{2} \mu^{2} \phi^{2}_{n} + \frac{1}{4!} \lambda \phi^{4}_{n} \right) \\
&= - \sum_{<ij>} \phi_i \phi_j 
+ \sum_n \left( 2 + \frac{\mu^{2}}{2} \right) \phi^{2}_{n} + \frac{\lambda}{4!} \phi^{4}_{n} \label{phi4:2d-lattice-lagrangian}
\end{split}
\end{equation}