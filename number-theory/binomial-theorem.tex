\section{Binomial Theorem}

\subsection{Proof of Binomial Theorem}

The following was taken from an exercise in chapter 1 of Complex Variables and Applications
from Brown and Churchill.

Use mathematical induction to verify the binomial formula.
More precisely, note that the formula is true when $n=1$.
Then, then assuming it is valid when $n=m$ where $m$ denotes any positive integer,
show that it must hold when $n=m+1$.

Suggestion: when $n=m+1$, write

\begin{align*}
(z_1 + z_2)^{m+1} &=
(z_1 + z_2)(z_1 + z_2)^m =
(z_1 + z_2) \sum^{m}_{k=0} \binom{m}{k} z_{1}^{k} z_{2}^{m-k} \\
&= \sum^{m}_{k=0} \binom{m}{k} z_{1}^{k} z_{2}^{m+1-k} + \sum^{m}_{k=0} \binom{m}{k} z_{1}^{k+1} z_{2}^{m-k}
\end{align*}

Reaplce $k$ by $k-1$ in the last sum.
To see how this would work take this example,
$$
\sum^{n-1}_{k=0} ar^k =
\sum^{n  }_{k=1} ar^{k-1}
$$
So
\begin{align*}
\sum^{m}_{k=0}   \binom{m}{k} z_{1}^{k+1} z_{2}^{m-k} &=
\sum^{m+1}_{k=1} \binom{m}{k-1} z_{1}^{k} z_{2}^{m-(k-1)} \\
&= \sum^{m+1}_{k=1} \binom{m}{k-1} z_{1}^{k} z_{2}^{m+1-k} \\
&= \sum^{m}_{k=1} \binom{m}{k-1} z_{1}^{k} z_{2}^{m+1-k} + z_{1}^{m+1}
\end{align*}

Note that in the last operation we explicitly did the very last summation to reduce the summation
back from $k$ to $m$.


Then we can take the sum we didn't shift as
$$
\sum^{m}_{k=0} \binom{m}{k} z_{1}^{k} z_{2}^{m+1-k} =
z_{2}^{m+1} + \sum^{m}_{k=1} \binom{m}{k} z_{1}^{k} z_{2}^{m+1-k}
$$

Putting these back together we get
$$
(z_1 + z_2)^{m+1} =
z_{2}^{m+1} + 
\sum^{m}_{k=1} \left[ \binom{m}{k} + \binom{m}{k-1} \right] z_{1}^{k} z_{2}^{m+1-k} + z_{1}^{m+1}
$$

%%%% coeff proof.
One more thing to note, is that the binomial coefficients met the following recurrence relation
$$
\binom{n+1}{k} = \binom{n}{k} + \binom{n}{k-1}
$$

Note that
$$
\binom{n}{k} = \frac{n!}{k! (n-k)!}
$$
and
$$
\binom{n}{k-1} = \frac{n!}{(k-1)! (n - k + 1)!} = \frac{n!}{(k-1)! (n - k + 1) (n-k)!}
$$

So
\begin{align*}
\binom{n}{k} + \binom{n}{k-1} &=
    n! \left[ \frac{1}{k (k-1)! (n-k)!} + \frac{1}{(k-1)! (n - k + 1) (n-k)!} \right] \\
&= n! \left[ \frac{n - k + 1}{k (k-1)! (n - k + 1) (n-k)!} + \frac{k}{k (k-1)! (n - k + 1) (n-k)!} \right] \\
&= n! \left[ \frac{n - k + 1 + k}{k (k-1)! (n - k + 1) (n-k)!} \right] \\
&= n! \left[ \frac{(n + 1) n!}{k (k-1)! (n - k + 1) (n-k)!} \right] \\
&= \frac{(n + 1)!}{k! (n - k + 1)!} \\
&= \binom{n+1}{k}
\end{align*}

Using this result, we can rewrite our previous sum as
\begin{align*}
(z_1 + z_2)^{m+1} &= 
z_{2}^{m+1} + 
\sum^{m}_{k=1} \left[ \binom{m}{k} + \binom{m}{k-1} \right] z_{1}^{k} z_{2}^{m+1-k} + z_{1}^{m+1} \\
&= z_{1}^{m+1} + z_{2}^{m+1} + \sum^{m}_{k=1} \binom{m+1}{k} z_{1}^{k} z_{2}^{m+1-k}
\end{align*}

Now the magic is in seeing that the 2 stragglers are the "endpoint" terms of a binomial expansion:
think how $(x+y)^2 = x^2 + 2xy + y^2$, the first and last term are raised to the $n$-th power
of the binomial expansion and have a coefficient of 1 (and this pattern is seen in all such expansions).
This means we can start the sum at $k=0$ by including $z_{1}^{m+1}$ and end the sum at $m+1$
by addinf the $z_{2}^{m+1}$ term, thus
\begin{align*}
(z_1 + z_2)^{m+1} &=
\sum^{m+1}_{k=0} \binom{m+1}{k} z_{1}^{k} z_{2}^{m+1-k}
\end{align*}