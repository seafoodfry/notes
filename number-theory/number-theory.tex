\documentclass{article}
\usepackage{blindtext} % for table of contents and other niceseties.
\usepackage{amsmath}
\usepackage{amsfonts} % things such as \mathbb{R}, or \Z. See https://en.wikipedia.org/wiki/List_of_mathematical_symbols_by_subject
\usepackage{amstext} % for \text macro.
\usepackage{array} % for tables.
\usepackage{float} % here for H placement parameter
\usepackage{hyperref} % for hyperlinks.
\usepackage{listings} % for code.
\usepackage[makeroom]{cancel} % corssing out terms in math https://tex.stackexchange.com/questions/75525/how-to-write-crossed-out-math-in-latex

% Make paragrphs show in the table of contents.
\setcounter{secnumdepth}{4}
\setcounter{tocdepth}{4}

\title{Notes on Number Theory}

\numberwithin{equation}{section}

% There is a \pmod for the "mod" symbol, but not one for the "div" symbol.
% Thus we are making it up.
\makeatletter
\newcommand*{\bdiv}{%
  \nonscript\mskip-\medmuskip\mkern5mu%
  \mathbin{\operator@font div}\penalty900\mkern5mu%
  \nonscript\mskip-\medmuskip
}
\makeatother

\begin{document}

\maketitle
\tableofcontents

\section{Logic}


\begin{table}[H]
    \centering
    \begin{tabular}{| l | l |}
        \hline
        Original Statement & $P \rightarrow Q$  \\ \hline
        Contrapositive & $\neg Q \rightarrow \neg P$  \\ \hline
        Converse & $Q \rightarrow P$  \\ \hline
        Inverse & $\neg P \rightarrow \neg Q$  \\ \hline
    \end{tabular}
    \caption{The contrapositive is equivalent to the original statement; the Converse to the inverse.}
\end{table}
\section{Binomial Theorem}

\subsection{Proof of Binomial Theorem}

The following was taken from an exercise in chapter 1 of Complex Variables and Applications
from Brown and Churchill.

Use mathematical induction to verify the binomial formula.
More precisely, note that the formula is true when $n=1$.
Then, then assuming it is valid when $n=m$ where $m$ denotes any positive integer,
show that it must hold when $n=m+1$.

Suggestion: when $n=m+1$, write

\begin{align*}
(z_1 + z_2)^{m+1} &=
(z_1 + z_2)(z_1 + z_2)^m =
(z_1 + z_2) \sum^{m}_{k=0} \binom{m}{k} z_{1}^{k} z_{2}^{m-k} \\
&= \sum^{m}_{k=0} \binom{m}{k} z_{1}^{k} z_{2}^{m+1-k} + \sum^{m}_{k=0} \binom{m}{k} z_{1}^{k+1} z_{2}^{m-k}
\end{align*}

Reaplce $k$ by $k-1$ in the last sum.
To see how this would work take this example,
$$
\sum^{n-1}_{k=0} ar^k =
\sum^{n  }_{k=1} ar^{k-1}
$$
So
\begin{align*}
\sum^{m}_{k=0}   \binom{m}{k} z_{1}^{k+1} z_{2}^{m-k} &=
\sum^{m+1}_{k=1} \binom{m}{k-1} z_{1}^{k} z_{2}^{m-(k-1)} \\
&= \sum^{m+1}_{k=1} \binom{m}{k-1} z_{1}^{k} z_{2}^{m+1-k} \\
&= \sum^{m}_{k=1} \binom{m}{k-1} z_{1}^{k} z_{2}^{m+1-k} + z_{1}^{m+1}
\end{align*}

Note that in the last operation we explicitly did the very last summation to reduce the summation
back from $k$ to $m$.


Then we can take the sum we didn't shift as
$$
\sum^{m}_{k=0} \binom{m}{k} z_{1}^{k} z_{2}^{m+1-k} =
z_{2}^{m+1} + \sum^{m}_{k=1} \binom{m}{k} z_{1}^{k} z_{2}^{m+1-k}
$$

Putting these back together we get
$$
(z_1 + z_2)^{m+1} =
z_{2}^{m+1} + 
\sum^{m}_{k=1} \left[ \binom{m}{k} + \binom{m}{k-1} \right] z_{1}^{k} z_{2}^{m+1-k} + z_{1}^{m+1}
$$

%%%% coeff proof.
One more thing to note, is that the binomial coefficients met the following recurrence relation
$$
\binom{n+1}{k} = \binom{n}{k} + \binom{n}{k-1}
$$

Note that
$$
\binom{n}{k} = \frac{n!}{k! (n-k)!}
$$
and
$$
\binom{n}{k-1} = \frac{n!}{(k-1)! (n - k + 1)!} = \frac{n!}{(k-1)! (n - k + 1) (n-k)!}
$$

So
\begin{align*}
\binom{n}{k} + \binom{n}{k-1} &=
    n! \left[ \frac{1}{k (k-1)! (n-k)!} + \frac{1}{(k-1)! (n - k + 1) (n-k)!} \right] \\
&= n! \left[ \frac{n - k + 1}{k (k-1)! (n - k + 1) (n-k)!} + \frac{k}{k (k-1)! (n - k + 1) (n-k)!} \right] \\
&= n! \left[ \frac{n - k + 1 + k}{k (k-1)! (n - k + 1) (n-k)!} \right] \\
&= n! \left[ \frac{(n + 1) n!}{k (k-1)! (n - k + 1) (n-k)!} \right] \\
&= \frac{(n + 1)!}{k! (n - k + 1)!} \\
&= \binom{n+1}{k}
\end{align*}

Using this result, we can rewrite our previous sum as
\begin{align*}
(z_1 + z_2)^{m+1} &= 
z_{2}^{m+1} + 
\sum^{m}_{k=1} \left[ \binom{m}{k} + \binom{m}{k-1} \right] z_{1}^{k} z_{2}^{m+1-k} + z_{1}^{m+1} \\
&= z_{1}^{m+1} + z_{2}^{m+1} + \sum^{m}_{k=1} \binom{m+1}{k} z_{1}^{k} z_{2}^{m+1-k}
\end{align*}

Now the magic is in seeing that the 2 stragglers are the "endpoint" terms of a binomial expansion:
think how $(x+y)^2 = x^2 + 2xy + y^2$, the first and last term are raised to the $n$-th power
of the binomial expansion and have a coefficient of 1 (and this pattern is seen in all such expansions).
This means we can start the sum at $k=0$ by including $z_{1}^{m+1}$ and end the sum at $m+1$
by addinf the $z_{2}^{m+1}$ term, thus
\begin{align*}
(z_1 + z_2)^{m+1} &=
\sum^{m+1}_{k=0} \binom{m+1}{k} z_{1}^{k} z_{2}^{m+1-k}
\end{align*}
\section{Modular Arithmetic}



%%%%%%%%%%%%%%%%%%%%%%%%%%%%%%%%%%%%%%%%%%%%%%%%%%%%%%%%%%%%%%%%%%%%%%%
\subsection{Divisibility}

Rosen's "Discrete Mathematics and its Applications"'s chapter 4 along with Gallian's "Contemporary
Abstract Algebra" chapter 0 make great references for this material.
\\

An $a \neq 0 \in \mathbb{Z}$ is called a \textbf{divisor}
of a $b \in \mathbb{Z}$
if there is a $c\in\mathbb{Z}$, such that $b = ac$.
We write $a|b$, "a divides b". We also commonly say that "b is a multiple of a".

Note that this working definition means that $a|b$ is an integer.
So for example, $3\not| 7$ since $7 / 3 \notin \mathbb{Z}$ but $3|12$ since $12/3 \in \mathbb{Z}$.
\\

If $n$ and $d$ are positive integers, how many positive integers not exceeding $n$ are divisible by $d$?

In order to be divisible by $d$, an integer must be of the form $dk$, for some positive integer $k$.
So the integers divisible by $d$ and not greater than $n$ are the integers with $k$ such that
$0 \leq dk < n$
or $0 < k < n/d$.
Thus, the number of integers divisible by $d$, not exceeding $n$, is $\lfloor n / d \rfloor$.
\\~\\


\textbf{Theorem describing the transitive properties of division:}

\begin{equation}
\text{ If $a|b$ and $a|c$, then $a|(b+c)$}
\end{equation}

To prove this use the fact that $a|b$ means that $b = as$, $a|c$ means that $c=at$,
and $b+c = a(s+t)$.
Hence $a|(b+c)$.

\begin{equation}
\text{If $a|b$, then $a|bc$, for $c \in \mathbb{Z}$}
\end{equation}

To prove it use the fact that $a|b$ means $b = as$, so $b*c = as * c$.

\begin{equation}
\text{If $a|b$, and $b|c$, then $a|c$}
\end{equation}

To prove it use $b = as$, $c = bt$. Then $c = bt = ast$ and hence $a|c$.
\\~\\

\textbf{Corollary: If a, b, and c are integers, where $a \neq 0$, such that $a|b$ and $a|c$, then $a|mb + nc$ whenever $m, n \in \mathbb{Z}$.}

Use if $a|b$ and $a|c$, then $a|(b+c)$ and if $a|b$, then $a|bc$, for $c \in \mathbb{Z}$, to prove it.
\\~\\



\subsubsection{Division Algorithm}

\begin{itemize}
\item If $a = dq + r$ where $0 \leq r < d$ and $d>0$
\item $q = a \bdiv d = \lfloor a/d \rfloor$
\item $r = a \pmod d = a - dq$
\end{itemize}

For example, when 101 is divided by 11, $11|101$
$$
101 = 11 \dot 9 + 2
$$

When -11 is divided by 3, $3|-11$
$$
-11 = 3 \dot -4 + 1
$$



\subsubsection{Congruences}

If a and b are congruent modulo m ($a,b \in \mathbb{Z}, m >0$), $a \equiv b \pmod m$, then m divides $a-b$.
Another way of seeing it is that a and b have the same remainder when divided by m.

If m divides $a-b$, then $a-b = mc$ for some integer c.

If both a and b have the same remainders when divided by m, then:
$r = a - mq$ 
$r = b - mp$
$0 = a - mq - b + mp$ or $a - b = mq - mp = m(q-p) = mc$, where $c = q-p$.

The above also means that $a = b + mk$, for some integer k.

Equivalently, \textbf{$a \equiv b \pmod m$ implies that $a \pmod m = a \pmod m$}.
\\~\\

\textbf{Theorem about multiplications and additions in congruences:}

If $a \equiv b \pmod m$ and $c \equiv d \pmod m$, then
\begin{equation}
    a + c \equiv b + d \pmod m
\end{equation}

and
\begin{equation}
    ac \equiv bd \pmod m
\end{equation}

To prove these, you can use something like the following reasoning:

$a - b = mp$ and $c - d = mq$
$a + c - (b + d) = m(p+q)$

Since $c = d + mq$
$$
ac = (b + mp)(d + mq) = bd + bmq + dmp + mmpq = bd + mc
$$

\textbf{Corollary detailing more forms of addition and multiplication}

\begin{equation}
    (a+b) \bmod m = [(a \bmod m) + (b \bmod m)] \bmod m
\end{equation}

To show this,

$a = mk + r = mk + (a \bmod m)$ hence
$a \equiv (a \bmod m) \pmod m$ and so $b \equiv (b \bmod m) \pmod m$
So $a + b \equiv [(a \bmod m) + (b \bmod m)] \pmod m$

Because $a \equiv b \pmod m$ implies $a \bmod m = b \bmod m$, the above can be written as
$(a+b) \bmod = [(a \bmod m) + (b \bmod m)] \pmod m$.

\begin{equation}
    ab \bmod m = [(a \bmod m)(b \bmod m)] \bmod m
\end{equation}

Following a similar logic as in the above proof, we can obtain the former equation by using $ab \equiv [(a \bmod m)(b \bmod m)] \bmod m$.

\section{Abstract Algebra}

\subsection{Preliminaries}

\subsubsection{Division Algorithm}

UPC example:
Correct code is $a_1 a_2 a_3 a_4 a_5$, incorrect code is $a_2 a_1 a_3 a_4 a_5$.
So correct check digit is $(3a_1 + a_2 + 3a_3 + a_4 + 3a_5) \bmod 10$. 
Incorrect check digit is $(3a_2 + a_1 + 3a_3 + a_4 + 3a_5) \bmod 10$. 

If $x \bmod 10$ and $y \bmod 10$ are equal, then $x \equiv y \pmod 10$, which implies that $x - y = 10k$.

Error won’t be caught is $(3a_1 + a_2 + 3a_3 + a_4 + 3a_5) - (3a_2 + a_1 + 3a_3 + a_4 + 3a_5)$ is a multiple of 10. 
The above simplifies to $[3a_1 \bmod 10 + a_1 \bmod 10 + \dots - 3a_2 \bmod 10 - a_1 \bmod 10 - \dots] mod 10$.
Which can be simplified to $3a_1 \bmod 10 + a_1 \bmod 10 - 3a_2 \bmod 10 - a_1 \bmod 10] \bmod 10$.
Or $(3a_1 + a_2 - 3a_2 - a_1) \bmod 10 = 0$. 
Which means $(2a_1 - 2a_2) \bmod 10 = 0$.
No error caught if $a_1 - a_2$ is a multiple of 10/2 = 5 same as writing $|a_1 - a_2| = 5$.
\\~\\



\textbf{Euclid’s lemma}

If p is a prime, and if p does not divide another integer a, then it means that $a \neq pu$ (no common factor).
And since a prime only has 1 and itself as divisors (factors), then the only other possibility is 1.
Hence p not dividing $a \geq \gcd(p, a) = 1$.
if $p | ab$: $ab = pc$, for some integer c.
Thus, $b = abs + ptp = pcs + ptb$.
\\~\\




\end{document}