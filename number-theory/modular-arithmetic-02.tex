\section{Abstract Algebra}

\subsection{Preliminaries}

\subsubsection{Division Algorithm}

UPC example:
Correct code is $a_1 a_2 a_3 a_4 a_5$, incorrect code is $a_2 a_1 a_3 a_4 a_5$.
So correct check digit is $(3a_1 + a_2 + 3a_3 + a_4 + 3a_5) \bmod 10$. 
Incorrect check digit is $(3a_2 + a_1 + 3a_3 + a_4 + 3a_5) \bmod 10$. 

If $x \bmod 10$ and $y \bmod 10$ are equal, then $x \equiv y \pmod 10$, which implies that $x - y = 10k$.

Error won’t be caught is $(3a_1 + a_2 + 3a_3 + a_4 + 3a_5) - (3a_2 + a_1 + 3a_3 + a_4 + 3a_5)$ is a multiple of 10. 
The above simplifies to $[3a_1 \bmod 10 + a_1 \bmod 10 + \dots - 3a_2 \bmod 10 - a_1 \bmod 10 - \dots] mod 10$.
Which can be simplified to $3a_1 \bmod 10 + a_1 \bmod 10 - 3a_2 \bmod 10 - a_1 \bmod 10] \bmod 10$.
Or $(3a_1 + a_2 - 3a_2 - a_1) \bmod 10 = 0$. 
Which means $(2a_1 - 2a_2) \bmod 10 = 0$.
No error caught if $a_1 - a_2$ is a multiple of 10/2 = 5 same as writing $|a_1 - a_2| = 5$.
\\~\\

\textbf{GCD is a linear combination}

Since $S = { am + bn : am + bn > 0}$.
Well ordering axiom says there must exist a d s.t. $d = as + bt$.
Claim is that d is also $\gcd(a ,b)$ meaning that $a = dq + r$ where $0 \leq r < d$.
If $r = 0$: then r is not in S, and we have no member in S smaller than d.
If $r > 0$: then any linear combination that was equal to r would have r in S and because $0 \leq r < d$,
it would be smaller than d, leading to a contradiction.
\\~\\

\textbf{Euclid’s lemma}

If p is a prime, and if p does not divide another integer a, then it means that $a \neq pu$ (no common factor).
And since a prime only has 1 and itself as divisors (factors), then the only other possibility is 1.
Hence p not dividing $a \geq \gcd(p, a) = 1$.
if $p | ab$: $ab = pc$, for some integer c.
Thus, $b = abs + ptp = pcs + ptb$.
\\~\\

