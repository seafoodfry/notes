\section{Preliminaries}

%%%%%%%%%%%%%%%%%%%%%%%%%%%%%%%%%%%%%%%%%%%%%%%%%%%%%%%
\subsection{Sets}

When you take the union of two disjoint intervals, the resulting set will no longer be
a single interval because it won't have a continuous range of values. Instead, it will be
a set that consists of two separate intervals (the original disjoint intervals) or
possibly more, depending on how many disjoint intervals you are combining.

Remember that the efinition of an open itnerval is as follows
$$
\forall x,y\in I, \forall z\in\mathbb{R} : \{ x < z < y \rightarrow z\in I\}
$$

\textbf{Hence, only unions of non-disjoint sets form an itnerval}.