\documentclass{article}
\usepackage{blindtext} % for table of contents and other niceseties.
\usepackage{amsmath}
\usepackage{amsfonts} % things such as \mathbb{R}, or \Z. See https://en.wikipedia.org/wiki/List_of_mathematical_symbols_by_subject
\usepackage{amstext} % for \text macro.
\usepackage{array} % for tables.
\usepackage{float} % here for H placement parameter
\usepackage{hyperref} % for hyperlinks.
\usepackage{listings} % for code.
\usepackage[makeroom]{cancel} % corssing out terms in math https://tex.stackexchange.com/questions/75525/how-to-write-crossed-out-math-in-latex

% Make greek letters bold too.
\usepackage{bm}
\newcommand{\vect}[1]{\boldsymbol{\mathbf{#1}}}

% Make paragrphs show in the table of contents.
\setcounter{secnumdepth}{4}
\setcounter{tocdepth}{4}

\title{Notes on Functional Analysis and PDEs}

\numberwithin{equation}{section}

% There is a \pmod for the "mod" symbol, but not one for the "div" symbol.
% Thus we are making it up.
\makeatletter
\newcommand*{\bdiv}{%
  \nonscript\mskip-\medmuskip\mkern5mu%
  \mathbin{\operator@font div}\penalty900\mkern5mu%
  \nonscript\mskip-\medmuskip
}
\makeatother

\begin{document}

\maketitle
\tableofcontents

\section{The Exponential Function}
\section{Introduction to Inner Product Spaces}
\section{The Hahn-Banach Theorems. Introduction to the Theory of Conjugate Convex Functions}


\subsection{The Analytic Form of the Hahn-Banach Theorem: Extension of Linear Functions}

%%%%%%%%%%%%%%%%%%%%%%%%%%%%%%%%%%%%%%%%%%%%
\subsubsection{Maximal Elements and Duality}

There is a small detour that we believe is an interesting conversation before diving into the
Zorn lemma. (We are going to be copy-pasting a lot from wikipedia.)

First, an \textbf{isomorphism} is defined as a "structure-preserving mapping between two structures
of the same type that can be reversed by an inverse mapping."

This matters because we want to discuss partially ordered sets (posets) and their dual or oppositve poset.
As per the definition,
a dual order $P^{op}$ is defined to be the same set, but with the inverse order,
i.e. $x \leq y$ holds in $P^{op}$ if and only if $y \leq x$ holds in $P$.
In a broader sense, two partially ordered sets are also said to be duals if they
are dually isomorphic, i.e. if one poset is order isomorphic to the dual of the other.

Whenever two posets are order isomorphic, they can be considered to be "essentially the same"
in the sense that either of the orders can be obtained from the other just by renaming of elements.

And again, we just wanted to have the above fresh in buffer because we actually want to talk about
\textbf{manimal elements} and when exactly these are not the same as an upper bound.

The wikipedia page has a great example using "containement" to define an order but the thing to
keep in mind is that in a totally ordered set - what we are used to using - the maximum element
mathces an upper bound but in a poset, there is only the idea of order when comparing any two elements.
In a poset, you can only ever say that two elements are ordered in a very specific context, and that order
doesn't extend to other elements.
For example, if $a \leq b$, and $b \leq c$, in a poset, it does not mean that $a \leq c$.

The minimal element is dually defined.

\section{Metric Spaces}
\section{The Hahn-Banach Theorems. Introduction to the Theory of Conjugate Convex Functions}


\subsection{The Analytic Form of the Hahn-Banach Theorem: Extension of Linear Functions}

%%%%%%%%%%%%%%%%%%%%%%%%%%%%%%%%%%%%%%%%%%%%
\subsubsection{Maximal Elements and Duality}

There is a small detour that we believe is an interesting conversation before diving into the
Zorn lemma. (We are going to be copy-pasting a lot from wikipedia.)

First, an \textbf{isomorphism} is defined as a "structure-preserving mapping between two structures
of the same type that can be reversed by an inverse mapping."

This matters because we want to discuss partially ordered sets (posets) and their dual or oppositve poset.
As per the definition,
a dual order $P^{op}$ is defined to be the same set, but with the inverse order,
i.e. $x \leq y$ holds in $P^{op}$ if and only if $y \leq x$ holds in $P$.
In a broader sense, two partially ordered sets are also said to be duals if they
are dually isomorphic, i.e. if one poset is order isomorphic to the dual of the other.

Whenever two posets are order isomorphic, they can be considered to be "essentially the same"
in the sense that either of the orders can be obtained from the other just by renaming of elements.

And again, we just wanted to have the above fresh in buffer because we actually want to talk about
\textbf{manimal elements} and when exactly these are not the same as an upper bound.

The wikipedia page has a great example using "containement" to define an order but the thing to
keep in mind is that in a totally ordered set - what we are used to using - the maximum element
mathces an upper bound but in a poset, there is only the idea of order when comparing any two elements.
In a poset, you can only ever say that two elements are ordered in a very specific context, and that order
doesn't extend to other elements.
For example, if $a \leq b$, and $b \leq c$, in a poset, it does not mean that $a \leq c$.

The minimal element is dually defined.


\end{document}