\documentclass{article}
\usepackage{blindtext} % for table of contents and other niceseties.
\usepackage{amsmath}
\usepackage{amsfonts} % things such as \mathbb{R}, or \Z. See https://en.wikipedia.org/wiki/List_of_mathematical_symbols_by_subject
\usepackage{amstext} % for \text macro.
\usepackage{array} % for tables.
\usepackage{float} % here for H placement parameter
\usepackage{hyperref} % for hyperlinks.
\usepackage{listings} % for code.
\usepackage[makeroom]{cancel} % corssing out terms in math https://tex.stackexchange.com/questions/75525/how-to-write-crossed-out-math-in-latex

% Make greek letters bold too.
\usepackage{bm}
\newcommand{\vect}[1]{\boldsymbol{\mathbf{#1}}}

% Make paragrphs show in the table of contents.
\setcounter{secnumdepth}{4}
\setcounter{tocdepth}{4}

\title{Notes on Functional Analysis and PDEs}

\numberwithin{equation}{section}

% There is a \pmod for the "mod" symbol, but not one for the "div" symbol.
% Thus we are making it up.
\makeatletter
\newcommand*{\bdiv}{%
  \nonscript\mskip-\medmuskip\mkern5mu%
  \mathbin{\operator@font div}\penalty900\mkern5mu%
  \nonscript\mskip-\medmuskip
}
\makeatother

\begin{document}

\maketitle
\tableofcontents

\section{Growth Functions}

For a given function $g(n)$,
$$
\Theta \left( g(n) \right) =
\{ f(n) : \exists c_1, c_2 > 0 \text{ such that }
    0 \leq c_1 g(n) \leq f(n) \leq c_2 g(n) \quad \forall n \geq n_0 \in \mathbb{N}
\}
$$
In the above defintion $c_1$ and $c_2$ are constants.
So $f(n) = \Theta\left(g(n)\right)$ means that $f(n)$ is bound by $g(n)$ to within a constant factor
- bound above and below.

This also means that $0 \leq c_1 \leq \frac{f(n)}{g(n)} \leq c_2$ for sufficiently large $n$.
We say that \textbf{$g(n)$ is an asymptotically tight bound for $f(n)$}.
\\

$\Theta$ expresses an asymptotic bound from above and from below.
For an \textbf{asymptotic upper bound} we have
$$
O\left(g(n)\right) =
\{
    f(n) : \exists c > 0 \text{ such that }
    0 \leq f(n) \leq c g(n) \quad \forall n \geq n_0 \in \mathbb{N}
\}
$$

The above can also be seen as $0 \leq \frac{f(n)}{g(n)} \leq c$.
\\

The \textbf{asymtptotic lower bound} is similarly defined as
$$
\Omega = 
\{
    f(n) : \exists c > 0 \text{ such that }
    0 \leq c g(n) \leq f(n) \quad \forall n \geq n_0 \in \mathbb{N}
\}
$$

or the set of functions that meet the following inequality $0 \leq c \leq \frac{f(n)}{g(n)}$.

\textbf{Note} that we can remove the "tightness" of the upper and lower bounds by converting the
last inequality for both definitions into a strict inequality (swap the last "$\leq$" for a "$<$").

With this language we can define limits and define an order.
For example,
\begin{itemize}
    \item $f(n) = \Theta (g(n))$ is like $a = b$
    \item $f(n) = O(g(n))$ is like $a \leq b$
    \item $f(n) = \Omega (g(n))$ is like $a \geq b$
    \item $f(n) = o(g(n))$ is like $a < b$. This denotes an upper bound that is not asymptotically tight.
\end{itemize}

Another thing that comes in handy is to remember the rates fo growth of polynomials and exponentials
$$
\lim_{n\rightarrow\infty} \frac{n^b}{a^n} = 0
$$
This is equivalent to saying $n^b = o(a^n)$.
\\~\\

Titchmarsh uses the following notation:
$f(x) = O\{\phi(x)\}$ means $|f(x)| < A\phi(x)$ if $x$ is sufficiently close to some limit.
In particular $O(1)$ means a bounded function (really think about it).

And $f(x) = o\{\phi(x)\}$ means $f(x) / \phi(x) \rightarrow 0$ as $x$ tends to a gven limit.
So in this way Titchmarsh notation matches the conventional mathematical notation.

\section{The Exponential Function}

There is a handy thing to note for the proof of part (a) of the first theorem.

If you look closely to
$$
e^z = \sum_{k=0}^{\infty} \frac{z^n}{n!}
$$
Then we ought to wonder why $e^0 = 1$ since the first term in the series would be $0^0$.

Looking around you may think to use l'hopital (Bernoulli's) rule and do something like:
if $y = x^x$
$$
\lim_{x\rightarrow 0} y = \lim e^{\ln y} = e^{\lim \ln y} = e^{\lim \ln x^x} = e^{\lim x \ln x}
$$
Remember that $e^z$ is continuous, so we can just pass the limit through it.
Then
$$
\lim_{x\rightarrow 0} x \ln x = \lim \frac{\ln x}{1/x}
$$
One application of Bernoulli's rule later, we have
$$
\lim_{x\rightarrow 0} x \ln x = \lim \frac{\ln x}{1/x} = \lim \frac{1/x}{-1/x^2} = \lim -x = 0
$$

So
$$
\lim_{x\rightarrow 0} y = \lim e^{\ln y} = e^{\lim x \ln x} = e^0
$$

So going that route leads us to a circular argument.
Instead, it helps to unfold the series and see that
$$
e^z = \sum_{n=0} \frac{z^n}{n!} = 1 + z + \frac{z^2}{2} + \ldots
$$
\section{Introduction to Inner Product Spaces}
\section{The Hahn-Banach Theorems. Introduction to the Theory of Conjugate Convex Functions}


\subsection{The Analytic Form of the Hahn-Banach Theorem: Extension of Linear Functions}

%%%%%%%%%%%%%%%%%%%%%%%%%%%%%%%%%%%%%%%%%%%%
\subsubsection{Maximal Elements and Duality}

There is a small detour that we believe is an interesting conversation before diving into the
Zorn lemma. (We are going to be copy-pasting a lot from wikipedia.)

First, an \textbf{isomorphism} is defined as a "structure-preserving mapping between two structures
of the same type that can be reversed by an inverse mapping."

This matters because we want to discuss partially ordered sets (posets) and their dual or oppositve poset.
As per the definition,
a dual order $P^{op}$ is defined to be the same set, but with the inverse order,
i.e. $x \leq y$ holds in $P^{op}$ if and only if $y \leq x$ holds in $P$.
In a broader sense, two partially ordered sets are also said to be duals if they
are dually isomorphic, i.e. if one poset is order isomorphic to the dual of the other.

Whenever two posets are order isomorphic, they can be considered to be "essentially the same"
in the sense that either of the orders can be obtained from the other just by renaming of elements.

And again, we just wanted to have the above fresh in buffer because we actually want to talk about
\textbf{manimal elements} and when exactly these are not the same as an upper bound.

The wikipedia page has a great example using "containement" to define an order but the thing to
keep in mind is that in a totally ordered set - what we are used to using - the maximum element
mathces an upper bound but in a poset, there is only the idea of order when comparing any two elements.
In a poset, you can only ever say that two elements are ordered in a very specific context, and that order
doesn't extend to other elements.
For example, if $a \leq b$, and $b \leq c$, in a poset, it does not mean that $a \leq c$.

The minimal element is dually defined.

\include{kreyzig-01}
\section{The Hahn-Banach Theorems. Introduction to the Theory of Conjugate Convex Functions}


\subsection{The Analytic Form of the Hahn-Banach Theorem: Extension of Linear Functions}

%%%%%%%%%%%%%%%%%%%%%%%%%%%%%%%%%%%%%%%%%%%%
\subsubsection{Maximal Elements and Duality}

There is a small detour that we believe is an interesting conversation before diving into the
Zorn lemma. (We are going to be copy-pasting a lot from wikipedia.)

First, an \textbf{isomorphism} is defined as a "structure-preserving mapping between two structures
of the same type that can be reversed by an inverse mapping."

This matters because we want to discuss partially ordered sets (posets) and their dual or oppositve poset.
As per the definition,
a dual order $P^{op}$ is defined to be the same set, but with the inverse order,
i.e. $x \leq y$ holds in $P^{op}$ if and only if $y \leq x$ holds in $P$.
In a broader sense, two partially ordered sets are also said to be duals if they
are dually isomorphic, i.e. if one poset is order isomorphic to the dual of the other.

Whenever two posets are order isomorphic, they can be considered to be "essentially the same"
in the sense that either of the orders can be obtained from the other just by renaming of elements.

And again, we just wanted to have the above fresh in buffer because we actually want to talk about
\textbf{manimal elements} and when exactly these are not the same as an upper bound.

The wikipedia page has a great example using "containement" to define an order but the thing to
keep in mind is that in a totally ordered set - what we are used to using - the maximum element
mathces an upper bound but in a poset, there is only the idea of order when comparing any two elements.
In a poset, you can only ever say that two elements are ordered in a very specific context, and that order
doesn't extend to other elements.
For example, if $a \leq b$, and $b \leq c$, in a poset, it does not mean that $a \leq c$.

The minimal element is dually defined.


\end{document}