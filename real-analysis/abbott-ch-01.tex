\section{The real numbers}

%%%%%%%%%%%%%%%%%%%%%%%%%%%%%%%%%%%%%%%%%%%%%%%%%%%%%%%%%%%%%%%%%%%%%%%
\subsection{Background}
More formally stated, a field is any set where addition and multiplication are well-defined operations
that are commutative, associative, and obey the familiar distributive property $a(b + c) = ab + ac$.
There must be an additive identity, and every element must have an additive inverse.
Finally, there must be a multiplicative identity, and multiplicative inverses must exist for
all nonzero elements of the field.



%%%%%%%%%%%%%%%%%%%%%%%%%%%%%%%%%%%%%%%%%%%%%%%%%%%%%%%%%%%%%%%%%%%%%%%
\subsection{Preliminaries}

\subsubsection{$x_{n+1} = \frac{1}{2}x_n + 1$}

The series is non-decreasing
$$
x_n \leq x_{n+1}
$$

And it just so happens that $x_n = 2 - \frac{1}{ n^{n-1} } = \frac{2^{n+1} - 1}{2}$
Write it all out to see the pattern.

$$
\lim\limits_{n \to \infty} \frac{2^{n+1} - 1}{2} = 2
$$

$\frac{2^{n+1} - 1}{2}$ grows as $\frac{2^{n+1}}{2}$.





\subsubsection{Triangle Inequality}

\begin{equation}
|x| =
  \begin{cases}
    x   & \quad x \geq 0\\
    -x  & \quad x < 0
  \end{cases}
\end{equation}

\begin{equation}
    |ab| = |a| |c|
\end{equation}

\begin{equation}
    |a+b| \leq |a| + |c|
\end{equation}

If $a,b \geq 0$, then $|a+b| = a+b = |a| + |b|$.
If $a,b < 0$, then $|a+b| = |a| + |b|$ (i.e., $|-1 + -2| = |-1| + |-2| = 3|$).
If $a<0, b>0$, then $|-1+2| = |1|=1$ and $|-1| + |2| = 3$, so $|a+b| < |a| + |b|$. Similarly if $a>0$ and $b<0$.

\begin{equation}
    |a-b| = |b-a|
\end{equation}
think about it as the distance between two points.

$|a-b| = |(a-c) + (c-b)| <= |a-c| + |c-b|$ using the triangle inequality. 
So, $|a-b| \leq |a-c| + |c-b|$ for any $c \in \mathbb{R}$.
So the above expression says something like "distance from a-to-b is equal to or less than the distance from a-to-c
plus the distance from c-to-b".

Proofs for triangle and reverse triangle inequalities are in \url{https://en.wikipedia.org/wiki/Triangle_inequality}.
Some takeaways are that:

$|b| \leq a$ can also be expressed as $-a \leq b \leq a$.

$-|x| \leq x \leq |x|$ so $-(|x|+|y|) \leq x+y \leq |x|+|y|$.

$|x| = |(x-y) + y| \leq |x-y| + |y|$ or $|x| - |y| \leq |x-y|$.

$-|x-y| \leq |x| - |y| \leq |x-y|$ also means $||x|-|y|| <= |x-y|$.

Additionally $|x| - |y| \leq |x+y| \leq |x| + |y|$.






\subsubsection{Exercises}

\textbf{1.1.1}

Prove that $\sqrt{3}$, $\sqrt{6}$, $\sqrt{4}$ are irrational.

\href{https://math.stackexchange.com/questions/930486/prove-that-the-square-root-of-3-is-irrational}{stackexchange: prove that the square root of 3 is irrational}:
A supposed equation $m^2 = 3n^2$ is a \textbf{direct contradiction to the Fundamental Theorem of Arithmetic,
because when the left-hand side is expressed as the product of primes, there are evenly many 3s there,
while there are oddly many on the right.}
\\~\\




\textbf{1.2.7}

$$
A = \{x \in \mathbb{R} : 0 \leq x \leq 2\}
$$
$$
B = \{x \in \mathbb{R} : 1 \leq x \leq 4\}
$$
so $A \cap B = \{x \in R : 1 <= x <= 2 \}$. 

$$
f(A) = \{ x \in \mathbb{R} : 0 \leq x \leq 4 \}
$$
$$
f(B) = \{ x \in \mathbb{R} : 1 \leq x \leq 16 \}
$$
so
$$
f(A) \cap f(B) = \{ x \in \mathbb{R} : 1 \leq x \leq 4 \}
$$
and $f(A \cap B) = \{ x \in \mathbb{R} : 1 \leq x \leq 4 \}$.

So in this case $f(A) \cap f(B) = f(A \cap B)$.

$$
A \cup B = \{x \in \mathbb{R} : 0 \leq x \leq 4\}
$$
so 
$$
f(A \cup B) = \{x \in \mathbb{R} : 0 \leq x \leq 16\}
$$
$$
f(A) \cup f(B) = \{x \in \mathbb{R} : 0 \leq x \leq 16\}
$$

So $f(A) \cup f(B) = f(A \cup B)$ as well.

Counterer example of $f(A \cap B) = f(A) \cap f(B)$ is if $A = \{x \in \mathbb{R} : -6 \leq x \leq 3\}$ And
$B = \{x \in \mathbb{R} : 3 \leq x \leq 6\}$.
There A and B is $\emptyset$, so $f(A \cup B)$ is $\emptyset$.
While f(A), f(B), and $f(A) \cup (B)$ are $\{x \in \mathbb{R} : 9 \leq x \leq 36\}$.

To show that for an arbitrary function $g: \mathbb{R} \rightarrow \mathbb{R}$, the statement
$g(A \cap B) \subseteq g(A) \cap g(B)$ holds for all sets $A, B \subseteq \mathbb{R}$,
we need to prove that every element in $g(A \cap B)$ is also an element of both $g(A)$ and $g(B)$.
Let's proceed with the proof:

Let x be an arbitrary element in $g(A \cap B)$. This means that there exists an element $y \in A \cap B$ such that $g(y) = x$.
Since $y \in A \cap B$, y is both in set A and set B. Therefore, $g(y) \in g(A)$ and $g(y) \in g(B)$.
Since $g(y) = x$, we can conclude that $x \in g(A)$ and $x \in g(B)$, which implies that $x \in g(A) \cap g(B)$.
Since x was an arbitrary element in $g(A \cap B)$, we have shown that every element in $g(A \cap B)$ is also an element of $g(A) \cap g(B)$.

Thus, we have proved that $g(A \cap B) \subseteq g(A) \cap g(B)$ for all sets $A, B \subseteq \mathbb{R}$.

Following a similar line of thinking for $g(A \cup B) and g(A) \cup g(B)$.
Let x be some element in $g(A \cup B)$, then there must be a y in either A or B such that $g(y) = x$.
Therefore, $g(y) \in g(A) \cup g(B)$.
\\~\\



\textbf{1.2.10}

$y_1 = 1$
$$
y_{n+1} = \frac{3}{4} y_n + 1 = \frac{3y_n + 4}{4}
$$

$n=1: y_1 < 4$

Now we want to show that if we have $y_n < 4$, then so will $y_n+1 < 4$
We are starting from the hypothesis that $y_n < 4$, then

$$
\frac{3}{4} y_n + 1 < \frac{3}{4} 4 + 1 = 4
$$

Which means $y_{n+1} < 4$.

$y_1 = 1$, $y_2 = \frac{7}{4}$, $y_3 = \frac{37}{16}$.

$n=1: y_1 < y_2$

Induction hypothesis $y_n < y_n+1$
$$
\frac{3}{4}  y_n + 1 < \frac{3}{4}  y_{n+1} + 1 \rightarrow y_{n+1} < y_{n+2}
$$
\\~\\



\textbf{1.2.11}

\textbf{If a set A contains n elements, prove that the number of different subsets of A is equal to 2n.}
\textbf{(Keep in mind that the empty set is considered to be a subset of every set.)}

Every element in A can be in or not, thus m multiplications of 2.
\\~\\


\textbf{1.2.12}

Trivial case is: $(A_1)^c  = A_1^c$.

Base case: $(A_1 \cup A2_)^c = A_1^c \cap A_2^c$ by de morgan’s theorem.

$(A_1 \cup A_2 \cup A_3)^c = (B_1 \cup A_3)^c = B_1^c \cap A_3^c = A_1^c \cap A_2^c \cap A3^c$.
Distributive property plus $B := A_1 \cup A2$. 
without loss of generality $B = A_1 \cup A_2 \cup \dots A_k$.
\\~\\



%%%%%%%%%%%%%%%%%%%%%%%%%%%%%%%%%%%%%%%%%%%%%%%%%%%%%%%%%%%%%%%%%%%%%%%%%%%%%%%
\subsection{The axiom of completeness}

\textbf{1.3.1}

Let $Z_5 = \{0, 1, 2, 3, 4\}$.
Additiona and multiplication modulo 5 would then be as,

$$
7+9 = 16 \bmod 5 = 1
$$
And
$$
7\times 9 = 63 \bmod 5 = 3
$$

Additive inverse: in order for $z+y = 0$ in $Z_5$, then the modulo addition of z and y must be a multiple of 5, in this case.
One way to define the additive inverse of z is to have it be $m-z$, where m is the modulo we are using.
That way $z + y = z + (m-z) \bmod m = m \mod m = 0$.
\\

Multiplicative inverse: if $z \neq 0$ in $Z_5$, $\exists x$ such that $zx = 1$.
So whatever product we end up getting must be $nm + 1$ for some integer n (always have a remainder of 1).
Yet another way of thinking about this is that $zx \equiv 1 \bmod m$.

There is an interesting pattern that shows up here...
If $z=1$, then $zx = x \bmod 5 = 1$ has multiple solutions but two of them are $x=1$ or $x=6$ ($x=z$ or $x=z+m$).
Similarly, if $z=2$, then 2 possible solutions would be $x=3$ or $x=8$ ($x=z+1$ or $x=z+m+1$).
If $z=3$, then 2 possible solutions are $x=2$ or $x=7$ ($x=z-1$ or $x=z+m-1$).
If then $z=4$, then 2 possible solutions are $x=4$ or $x=9$ ($x=z$ or $x=z+m$ - again!!!).
Since we have found the multiplicative inverse for each non-zero element in $Z_5$, we can conclude
that for any $z \neq 0$ in $Z_5$, there exists an element $x$ such that $zx = 1$.

\textbf{Multiplicative inverses exist only when $z$ and $m$ are relatively prime}.
For example if $z=5$, and we are in $Z_{10}$ then there is no number such that $zx \equiv 1 \bmod 10$.

Keeping in mind that a \textbf{congruence class} is an equivalence relation on an algebraic structure (e.g., a ring)
that is compatible with the structure in the sense that algebraic operations done with equivalent elements will yield equivalent elements.
More humanly put, a congruence class is the set of all integers that have the same remainder as a when divided by n.
Now in a ring $Z_n$, the units (i.e. the elements which have a multiplicative inverse) are the
congruence classes of the elements $m$ which are coprime to $z$, because for such an element, we have a
Bezout's relation, $um + vz = 1$ or $um = 1 - vz$, which means the class of $u$ is the inverse of that of $m$.
We obtained the previous argument from
\href{https://math.stackexchange.com/questions/2650336/for-z-5-0-1-4-and-z-in-z-n-prove-that-there-exists-a-multiplicative-i}{stackexchange: prove that there exists a multiplicative inverse}.
\\~\\


\textbf{1.3.2}

A real number A real number $l$ is the greatest lower bound for a set $A \subseteq \mathbb{R}$ if it meets the following
two criteria:

\begin{enumerate}
    \item $l$ is a lower bound of $A$,
    \item if $m$ is any lower bound for A, then $l \geq m$.
\end{enumerate}

Lemma: assume $l \in \mathbb{R}$ is a lower bound for a set $A \subseteq \mathbb{R}$.
Then $l = \inf{A}$ if and only if, $\forall \epsilon > 0$, $\exists a \in A$ satisfying $l + \epsilon > a$.

Given that $l$ is a lower bound, $l$ is the greatest lower bound iff any number greater than $l$ is not a lower bound.

In the forward direction we want to prove that: if $l$ is the greatest upper bound, then $l+\epsilon > a$.
In the forward direction we want to prove that: if $l$ is a lower bound satisfying $l+\epsilon > a$, then $l$ is also the
greatest upper bound.

For the former case, because $l + \epsilon > l$, then by definition $l+\epsilon$ is not a lower bound. Thus there must
$\exists a \in A$ for which $l+\epsilon > a$.
Note that we used the defintion to prove a point, we are not questioning the definition (whether $l$ is the greatest upper bound).

For the latter direction, assume $l$ is a lower bound such that $\forall \epsilon >0$, $l+\epsilon$ is no longer a lower bound
for A.
If this is so, a number slightly greater than $l$ (for any degree of slightly) is no longer a lower bound, then by Definition
of the greatest lower bound, $l$ can be the greatest lower bound.
\\~\\


\phantomsection
\label{abbott:1.3.3}

\textbf{1.3.3}


If A is bounded below, and we define $B = \{ b \in \mathbb{R}: \text{b is a lower bound of A} \}$.
Show that $\sup B = \inf A$.

By the axiom of completeness we can start by knowing that any non-empty set of real numbers that is bounded above
has a least upper bound.
So as long as A is not an empty set, then B will not be empty either.
In the case of the set B, any upper bound wil be equal or greater than any $b \in B$.
And the least upper bound will be equal to or smaller than any other bound we can find (smaller than or equal to
any $a \in A$).

At the same time, since B contains lower bounds of A, we already know that there is a $sup B$ that exist and is smaller
than any $a \in A$ (other upper bounds of B) while also being greater than or equal to any other bounds of A (members of B).
That is, $\sup B$ is a lower bound of A and is greather than or equal to any other bounds of A.
Hence, by defintion of the greatest lower bound $\sup B = \inf A$.
\\

\textbf{This exercise points to the interesting case when if $a$ is an upper bound for A,
and if $a \in A$, then it must be that $a = \sup A$}.
\\~\\


\textbf{1.3.4}

Assume that A and B are nonempty, bounded above, and satisfy $B \subseteq A$. Show $\sup B \leq \sup A$.

If B and A are equal then their least upper bound will be equal.
On the other hand, if A has elements that B doesn't, then those elements not in B can be smaller or greater than those in B.
If the extra elements are smaller than $b \in B$, then the least upper bounds of the two sets will not change.
But if A contains elements that are greater than B, then $\sup A > \sup B$.
\\~\\


\textbf{1.3.8}

If $\sup A < \sup B$, then show that $\exists b \in B$ that is an upper bound for A.

Since B has a least upper bound, then there must be $b \in B$ such that $b \leq \sup B$.
Similarly in A, there must exist $a \in A$ such that $a \leq \sup A$.
Since we know that $\sup A < \sup B$, then we have $a \leq \sup A < \sup B$ ($\sup B$ is an upper bound for A).

If $b$ is an upper bound of A, then that would mean that $b \geq a$ for any $a \in A$.
If $b$ was not an upper bound, then $b < a$ for all $b$.
But if this was the case, then $\sup B$, which is an upper bound of B ($\sup B \geq b$) would be smaller than some
$a$, leading us to a contradiction as $\sup A$ would then be greater than $\sup B$.

We can also see this as a case mention in problem 1.3.4 above.
\\~\\



%%%%%%%%%%%%%%%%%%%%%%%%%%%%%%%%%%%%%%%%%%%%%%%%%%%%%%%%%%%%%%%%%%%%%%%%%%%%%%%%%%%%%
\subsection{Consequences of completeness}


\textbf{1.4.1}

\textbf{Density of Q in R:} For every $a,b \in \mathbb{R}$ where $a < b$ and $a<0$, $\exists r \in \mathbb{Q}$
satisfying $a < r < b$.

Since we want to prove the case of $a<0$, we want to see if there is a rational number $r$ such that $a < r < 0$.
There rest of the proof in theorem 1.4.3 then applies as is.
\\~\\


\subsubsection{Existence of square roots}

$$
T = \{ t\in\mathbb{R} : t^2 < 2 \}
$$

Then we work out the expression
$$
\left( \alpha + \frac{1}{n} \right) < \alpha^2 + \frac{2\alpha + 1}{n}
$$

That bit $(2\alpha + 1)/n$ is what we need to fit between $\alpha^2$ and 2 while keeping $\alpha^2 <2$.
Since we want to fill that space, we come up with
$$
\frac{2\alpha + 1}{n} < 2 - \alpha^2
$$

Which then simplifies to what Abbott uses.


$$
\left( \alpha + \frac{1}{n_0} \right)^2 < \alpha^2 + (2 - \alpha^2) = 2
$$

contradicts the fact that $\alpha$ is an upper bound because $(\alpha + \frac{1}{n})^2$ is also a member of
$T$, and is larger than $\alpha$ alone.

For the case of $\alpha^2 > 2$, where we want to prove that this contradicts the fact that $\alpha$ is the least upper bound,
then it must be the case in which $\alpha$ is an upper bound for $\{ t \in \mathbb{R} : t^2 < 2 \}$ but it must not be
the smallest upper bound.
That is, $\alpha > b$, where $b$ is some other bound of $T$ ($b=2$ for example).

To show that the above would lead us to the said contradiction we can now take

$$
\left( \alpha - \frac{1}{n} \right)^2 =
    \alpha^2 - \frac{2\alpha}{n} + \frac{1}{n^2} >
    \alpha^2 - \frac{2\alpha}{n}
$$

If we chose an $n_0$ such that 
$$
\alpha^2 - \frac{2\alpha}{n_0} > 2
$$

That is, we look for an $n_0$ that can give us a number smaller than $\alpha$ but that it is an upper bound.
The above implies that $- \frac{2\alpha}{n_0} > 2 - \alpha^2$, and consecuently

$$
\left( \alpha - \frac{1}{n_0} \right)^2 >
    \alpha^2 + (2 - \alpha^2) = 2
$$

So we see that $\alpha - \frac{1}{n_0}$ is an upper bound and it is smaller than $\alpha$,
so $\alpha^2 > 2$ also be the case.


\subsubsection{$N \sim R$}

We known that for some real number $x_{n_0} \notin I$, so that
$$
x_{n_0} \notin \bigcap^{\infty}_{n=1} I_n
$$

Since we are assuming that $\mathbb{R} = \{ x_1, x_2, ... \}$ contains all the real numbers,
then $n_0$ can be any of the real numbers.
So for any given $n_0$, $I_{n_0}$ will not contain it, nor any $I_{n_0+1}$, or any of its subsets.
Going in the other direction, $x_{n_0}$ will be in $I_{n_0 - 1}$, but
$I_{n_0 - 1}$ will not contain $x_{n_0-1}$.
Thus, there is no one number that is contained by all sets $I_n$, Hence

$$
\bigcap^{\infty}_{n=1} I_n = \emptyset
$$

Which contradicts the nested-interval property.

When proving the nested interval property we showed that at the very elast the supremum of each interval $I_n$
would be in every $I_n$, which lead us to the discovery that their intersection would be a non-empty set.
However, if we don't have the supremum of each set present for each $I_n$, then by our experience with the axiom
of completeness, we would think that there are gaps in our list of numbers.

The "logical" issue in the above argument is that the real numbers are "enumerable", which doesn't show up in the original
proof of the nested interval theorem because hte axiom of completeness creates a sort of "continuity" among the numbers
(the filling in between the gaps).


%%%%%%%%%%%%%%%%%%%%%%%%%%%%%%%%%%%%%%%%%%%%%%%%%%%%%%%%%
\subsubsection{Exercises}

\textbf{1.4.1}

\textbf{For every two real numbers $a$ and $b$ with $a<b$, there exists a rational number satisfying $a < r < b$.}

As requirements we have that $a<b$ and this time we want to look at the case were $a<0$,
so $b$ can be either negative, zero, or positive, it just has to be greater than $a$.
However the rest of the proof can proceed as normal if we don't require $m \in \mathbb{N}$ but
instead generalize $m \in \mathbb{Z}$.
See
\href{https://math.stackexchange.com/questions/48537/how-does-this-proof-of-density-of-mathbbq-in-mathbbr-require-a-geq-0}{How does this proof of density of Q in R require $a\geq 0$?}.

The original proof uses the Archimedean Property and the Archimedean Principle for positive
numbers to find a positive integer $n$ such that $\frac{1}{n} < b - a$.
Then, it uses the Density Property of the set of natural numbers to find a natural number $m$
such that $m > n$ and $an < m$.
\\~\\



\textbf{1.4.2}

If we have $a = \frac{m}{n}$ and $b = \frac{p}{q}$, then

$$
a + b = \frac{m}{n} + \frac{p}{q}
    = \frac{mq + np}{nq}
$$
Which is a rational number since $m, n, p, q \in \mathbb{Z}$.

Similarly,
$$
ab = \frac{m}{n} \frac{p}{q}
    = \frac{mp}{nq}
$$

However, if we multiply (or add), let's say $a$, by $i \in \mathbb{I}$, we want to show that this would result in an irrational number.
To prove this we will use a proof by contradition.

In the case of a sum, let's begin by assuming that if we add a rational number and an irrational numbert
that the result is a rational number too.
Then,
$$
\frac{m}{n} + i = \frac{x}{y}
$$

This would mean that we could isolate $i$
$$
i = \frac{x}{y} - \frac{m}{n}
$$
Which as we showed above, would mean that $i$ is a rational number.

In the case of a multiplication,
$$
i \frac{m}{n} = \frac{x}{y}
$$

$i$ could be isolated,
$$
i = \frac{x}{y} \frac{n}{m}
$$

Which again, would lead to a contradiction since the product of two rational numbers is also a rational number
and the above says that we magically converted $i$ into a rational number.

One would then think that the irratinoal numbers are also closed under addition or multiplication, but this is not the case.
Consider $1 - \sqrt{2} \in \mathbb{I}$ (as we just saw) and $\sqrt{2}$.
$$
( 1 - \sqrt{2} ) + \sqrt{2} = 1 \in \mathbb{Z}
$$

Similarly,
$$
\sqrt{2}\sqrt{2} = 2
$$
\\~\\



\textbf{1.4.5}

\textbf{Given any two real numbers $a < b$, there exists an irrational number $t$ satisfying $a < t < b$.}
To prove this, we can use the results from the previous exercise by applying the theorem where we talked about the
density of the rational numbers in the real numbers to $a - \sqrt{2}$ and $b - \sqrt{2}$
(which are real and irrational numbers.)

This case is easy to see since we know that between any two real numbers there exists a rational number
$$
a < \frac{m}{n} < b
$$

And we also know that we can convert a rational number into an irrational by adding or multiplying by an irrational number.
So, if we shift the case we used to prove the density of the rational numbers by subtracting an irrational number
$$
a - \sqrt{2} < \frac{m}{n} - \sqrt{2} < b - \sqrt{2}
$$

We can arrive to the result we wanted to prove.
\\~\\



\textbf{1.4.4}

Use the Archimedean property of $\mathbb{R}$ prove that $\inf \{ \frac{1}{n} : n \in \mathbb{N} \} = 0$.

By the Archimedean property we know that for any $y>0 \in \mathbb{R}$, $\exists n \in \mathbb{N}$ such that
$y > \frac{1}{n}$.

We also know that for a number to be an infinimum then it has to be a lower bound and it has to be greater than any
otther lower bounds.

In the case of the set $A = \{ \frac{1}{n} : n \in \mathbb{N} \}$, we can see that 0 is indeed a lower bound.
But let's say that we have another lower bound $b$ that is greater than 0 (assume the greatest lower bound is $b$).

Since $0 \notin A$ and $b$ is a lower bound, then $b > 0$ but $b \leq a$, for any $a \in A$.
However, the Archimedean property tells us that for any $b$ we could have, there will be a $\frac{1}{n}$ that is smaller than it.
Thus disproving the fact that a $b>0$ could be a lower bound.
\\~\\



\textbf{1.4.5}

Prove that $\bigcap^{\infty}_{n=1} (0, 1/n) = \emptyset$. 
Notice that this demonstrates that the intervals in the Nested Interval Property
must be closed for the con- clusion of the theorem to hold.

The proof we used for the nested property theorem, theorem 1.4.1.
Following a similar schema, we can see that the set of right-hand endpoints $\{ 1/n : n\in\mathbb{N} \}$ which we have
already seen that its greatest lower bound is 0.
We have also saw in exercise 1.3.3 that the supremum of the left-hand endpoints must equal the infinimum of the right-hand
endpoints, which is zero.
So the approach we followed in theorem 1.4.1 doesn't apply here since 0 is not even part of any of the intervals $I_n$.
\\~\\





%%%%%%%%%%%%%%%%%%%%%%%%%%%%%%%%%%%%%%%%%%%%%%%%%%%%%%%%%%%%%%%%%%%%%%%%%%%%%%%%%%%%%
\subsection{Cardinality}


\textbf{1.5.1}

\textbf{Theorem 1.5.7: if $A \subseteq B$ and $B$ is countable, then either $A$ is countable, finite, or empty.}

Recall that a set $B$ is countable if $\mathbb{N} \sim B$.
Which means that the sets $\mathbb{N}$ and $B$ have the same cardinality as there exist some $f : N \rightarrow B$
that is 1-to-1 and onto.

Let $n_1 = \min \{ n \in \mathbb{N} : f(n) \in A \}$.
As a start to a definition of $g: \mathbb{N} \rightarrow A$, set $g(1) = f(n_1)$.

We know that $f$ is 1-to-1, so every natural number gets mappted to a different member of $B$, that is, $n_1 \neq n_2$, then
$f(n_1) \neq f(n_2)$.
We also know that for every $B$, and clearly for every $A$ (since $A \subseteq B$), there exists an $n \in \mathbb{N}$
such that $a = f(n)$ for every $a \in A$.
Thus, we only need to prove that $g$ is a 1-to-1 function.

Now back to our definition of $g$, if we now look at $A - \{f(n_1)\}$, and we define
$n_2 = \min \{ n \in \mathbb{N} : f(n) \in A - \{f(n_1)\} \}$, or more generally,
$n_n = \min \{ n \in \mathbb{N} : f(n) \in A, n \notin \{n_1, n_2, ..., n_{k-1}\} \}$,
then we have $g(k) = f(n_k)$, which we know is different from all other values of $g$ since $f$ is 1-to-1.
\\~\\



\textbf{1.5.3}

\textbf{If $A_1, A_2, \ldots, A_m$ are each coutable sets, then the union $A_1 \cup A_2 \cup \ldots \cup A_m$ is countable.}

Replace $A_2$ with $B_2 = A_2 - A_1 = \{ x \in A_2 : x \notin A_1 \}$.
$A_1 \cup B_2  = A_1 \cup A_2$ while keeping the two sets disjoint.

If we followed the example 1.4.8, then we could see how "laying out" the members of $A_1 \cup A_2 = A_1 \cup B_2$ could be
mapped to by the natural numbers, $\{ a_1, a_2, \ldots, b_1, b_2, \ldots \}$.
For the induction step, we can follow logic similar to when we prove the generalization of the de morgan laws in 1.2.12.
\\

\textbf{If $A_n$ is a countable set for each $b \in \mathbb{N}$, then $\bigcup^{\infty}_{n=1} A_n$ is countable.}

The same induction logic we used above could get us into trouble here because induction shows something
for all $n \in \mathbb{N}$, not for infinity - and we have already seen that there are various types of infinities.
If we followed the same logic we could end up with a problem similar as to when we claimed that the reals are
enumerable.
\\

If we were to arrange natural numbers in a two-dimensional array we end up with disjoint sets $B$ such that
$\cup^{\infty}_{n=1} B_n = \mathbb{N}$.
(Note how this can be a construction similar to the first instance in this problem where $B_1 = A_1$,
$B_2 = A_2-A_1 = A_2 - B_1$, $B_3 = A_3 - B_2$, and so on.)

We also now (and can see) that every $B$ has an $f_n$ has is 1-to-1 and onto.

So in order to prove that there exists a mapping that maps the natural numbers into our tw-dimensional array of disjoint sets,
it suffices to see that our way of ordering the sets is itself a mapping such that
$f : \mathbb{N} \rightarrow \cup^{\infty}_{n=1} B_n$.
The mapping could be expresed as $f (x_{m,n} \in \mathbb{N}) = (m, n) \in \cup B_n$.

And since we organized the two-dimensional array in the way we have, then the we have
$\mathbb{N} \subseteq \mathbb{N} \times \mathbb{N}$ and the mapping is bijective
($f (x_{m,n} \in \mathbb{N})$ is a specific mapping that creates order pairs that such that all order pairs are 1-to-1 and onto).

See
\href{https://math.stackexchange.com/questions/4403048/simplification-in-proof-of-countable-union-of-countable-sets-is-countable}{Simplification in Proof of Countable Union of Countable Sets is Countable}
for a similar presentation.
\\~\\



\textbf{1.5.4}

Is $(a,b) \sim \mathbb{R}$, for any open interval $(a,b)$?

Let's take example 1.5.4 as inpiration.
This example says that $f = x / (x^2 -1)$ takes the interval $(-1, 1)$ onto $\mathbb{R}$, showing that
$(-1,1) \sim \mathbb{R}$.

The books mentions some calculus so let's see what we can see.
First we have,
$$
\lim_{x \rightarrow -1^+ } \frac{x}{x^2 - 1} \rightarrow
    \frac{-1}{\text{some very small positive number}} 
    \rightarrow -\infty
$$

Similarly,
$$
\lim_{x \rightarrow 1^- } \frac{x}{x^2 - 1} \rightarrow
    \frac{1}{\text{some very small positive number}} 
    \rightarrow \infty
$$

Its derivative is
$$
f' = \frac{1}{(x^2 - 1)^2} ( (x2 - 1) - x(2x) ) =
    \frac{-x^2 - 1}{(x^2 - 1)^2} =
    - \frac{x^2 + 1}{(x^2 - 1)^2}
$$

For the derivative, no matter whether $x$ is negative or positive, its value will always be zero.
Which is a no-graphing way of triple checking that the original function won't "fluctuate" and we can indeed map all of
the domain of $f$ to unique values in $\mathbb{R}$.

Now in order to see whether $f$ is onto, we could see if we can build an inverse of f and see if it has any "holes".
But that's rather complicated in our case, so let's try something else.
We could use the intermediate value theorem but we don't yet know what "continuous" or any of the other requirements mean.
However, we have seen that within $(-1,1)$ the function goes from $-\infty$ to $\infty$, so we can imagine that any value
within the image will be mapped to by somewhere in the domain.

To shift the function we have been using into the interval $(a,b)$, let's define the midpoint between $b$ and $a$ as
$m = (b-a)/2$ and let's define the transformed function $f$ as
$$
g(x) = f\left(\frac{(x - m)}{m} \right)
$$

With the above transformation we are "stretching" the function by dividing $x$ my the mid-way length between $b$ and $x$
and we are also shifting it by the same distance.
Since the changes are all constant factors, all other properties we have been relying on remain the same.
\\~\\



\textbf{1.5.5}

Is $A \sim A$ for every set $A$?

If so, it'd mean that there is a function $f: A \rightarrow A$ that is 1-to-1 and onto.
Which means that $f(a_1) = f(a_2)$ iff $a_1 = a_2$.
However, $f(a_1) = a_1$ in our case.
Given this, it is easy to see that such a mapping will also be onto.
\\

If $A\sim B$, does that mean that $B\sim A$?

Well, for starters, we now know that there is an $f: A\rightarrow B$ that is 1-to-1 and onto.
So $b_1 = b_2 = f(a_1) = f(a_2)$ iff $a_1 = a_2$.
Also, for every $b_i = f(a_i)$ there is a unique $a_i$.
Putting these two properties together we know that for every element in $B$ there is a uniqie element
that can be mapped to it from $A$.
If we take this in the opposite direction, $B\rightarrow A$, we only need to prove that every every element we are mapping to
is unique, which is given to us by the mapping being 1-to-1.
\\

If $A\sim B$ and $B\sim C$, does that mean that $A\sim C$?

There exists an $f: A\rightarrow B$ that is 1-to-1 and onto.
There also exists a $g: B \rightarrow C$ that is 1-to-1 and onto.

So if we start with $f(a_i) = b_i$, then $g \circ f (a_i) = g(b_i) = c_i$.
Since $g(b_1) = g(b_2)$ iff $b_1 = b_2$, and $b_1 = f(a_1)$ and $b_2 = f(b_2)$, and $f(a_1) = f(a_2)$ iff $a_1 = a_2$,
then by extension, $g(b_1) = g(b_2)$ iff $a_1 = a_2$.

Similarly, since for every $c$ there exist a unique $b$, and for every $b$, there is a unique $a$.
Then we can map every $c$ to a unique $a$.
\\~\\



\textbf{1.5.9}

\textbf{A real number $x\in \mathbb{R}$ is algebraic if there exists integers $a_0, a_1,\dots ,a_n \in \mathbb{Z}$,
not all zero, such that}

\begin{equation}
    a_0 + a_1 x + a_2 x^2 + \ldots + a_n x^n = 0
\end{equation}

Said another way, a real number is algebraic if it is the root of a polynomial with integer coefficients.
Real numbers that are not algebraic are called transcendental numbers.

$\sqrt{2}$ is an algebraic number since $2x^2 - x^4 = 0$ can be a way to express $x^2 = 2$, $x = \sqrt{2}$.
\\

Fix $n \in \mathbb{N}$, and let $A_n$ be the algebraic numbers obtained as roots of polynomials with integer coefficients
that have degree $n$.
Using the fact that every polynomial has a finite number of roots, show that $A_n$ is countable.

For example, $A_2 = \{ x\in\mathbb{R} : \text{$x$ is a root of } a_0 + a_1 x + a_2 x^2 = 0 \}$.
Here problem 1.5.3 (theorem 1.5.8) will come in handy.
Essentially, if every polynomial of degree $n$ has a finite set of roots, then a piecewise application of
$f: \mathbb{N} \rightarrow A_n$ is doable.
If we also organize all $A_n$ to have only non-zero coefficients, then we have a collection of countable and disjointsets.
By the application of theorem 1.5.8, the union of all $A_n$ is also countable.
\\

Now, argue that the set of all algebraic numbers is countable.
What may  we conclude about the set of transcendental numbers?

As we saw in theorem 1.5.6 $\mathbb{R}$ is not countable.
And we just argued that the set of algebraic numbers is countable.
Roughly speaking, this could have us belief that the set of transcendental numbers is uncountable.
\\~\\



\textbf{1.5.11}

\textbf{Schröder-Bernstein theorem}
Assume there exists a 1-to-1 function $f: X \rightarrow Y$ and another 1-to-1 function $g: Y \rightarrow X$.
Follow the steps to show that there exists a 1-to-1, onto function $h: X \rightarrow Y$ and hence $X \sim Y$.

The strategy is to partition X and Y into components
$$
X = A \cup A'
$$
and
$$
Y = B \cup B'
$$

with $A \cap A' = \emptyset$ and $B \cap B' = \emptyset$, in such a way that $f$ maps A onto B,
and $g$ maps $B'$ onto $A'$.
\\

(a) Explain how achieving this would lead to a proof that $X \sim Y$.

If we assume that X and Y don't have the same cardinality, then it would be possible to map elements of Y
that are not reachable from X, and vice versa.
So a way that the partitioning shceme could help us show that there is a 1-1 and onto mapping $h$ would be
if it helped us see that all of the partitions of $X$ and $Y$ are reachable from some element in either of the
partitions on the other side - we can get to any element in $B$ or $B'$ from $A$ or $A'$.

The 1-1 and onto function $h(x)$ will probably look something like:
\begin{equation}
    h(x) =
      \begin{cases}
        f(x)       & \quad x \in A\\
        g^{-1}(x)  & \quad x \in A'
      \end{cases}
\end{equation}

Some good links to dive into similar arguments:
\begin{enumerate}
    \item \href{https://math.stackexchange.com/questions/3982170/schr%C3%B6der-bernstein-theorem-proof-help}{Schröder-Bernstein Theorem proof help}
    \item \href{https://math.stackexchange.com/questions/225576/intuition-behind-cantor-bernstein-schr%C3%B6der}{Intuition behind Cantor-Bernstein-Schröder}
    \item \href{https://math.stackexchange.com/questions/3296321/does-this-proof-of-the-schr%C3%B6der-bernstein-theorem-work}{Does this proof of the Schröder–Bernstein theorem work?}
    \item \href{https://math.stackexchange.com/questions/925203/how-to-prove-this-version-of-the-cantor-schroder-bernstein-theorem}{How to prove this version of the Cantor-Schroder-Bernstein theorem?}
    \item \href{https://math.stackexchange.com/questions/1726578/understanding-a-proof-of-schr%C3%B6der-bernstein-theorem}{Understanding a proof of Schröder-Bernstein theorem}
\end{enumerate}


(b) Set $A_1 = X - g(Y) = \{x \in X: x \notin g(Y)\}$ (what happens if $A_1 = \emptyset$?)
and inductively define a sequence of sets by letting $A_{n+1} = g(f(A_n))$.
Show that $\{A_n: n \in \mathbb{N} \}$ is a pairwise disjoint collection of subsets of X,
while $\{ f(A_n): n \in \mathbb{N} \}$ is a similar collection in Y.

$A_1 = X - g(Y)$ can be seen as "all of X, except the parts that can be reached from Y with $g$".
We then have $A_1 \subseteq X$.
If $A_1 = \emptyset$, then all of $g(Y)$ can be mapped into some $X$, so $g$ is 1-1 and onto,
and we have found our answer.
But let's carry out assuming that $A_1 \neq \emptyset$.

$A_1 \subseteq X$.
So $f(A_1) \subseteq Y$.
And consequently, $A_2 = g(f(A_1)) \subseteq X$ as well.
Since $A_1$ contained all $x \in \mathbb{R}$ such that $x \notin g(Y)$,
and $A_2$ was obtained by applying $g$ on $f(A_1)$, then we have $A_1 \cap A_2 = \emptyset$.

We now need to look at how $A_n$ and $A_{n+1}$ relate to one another.
$A_n = g(f(A_{n-1}))$, and $A_{n+1} = g(f(A_n)) = g(f(g(f(A_{n-1}))))$.
If we assume that $A_n \cap A_{n+1} \neq \emptyset$, then it means that there is
some $x\in X$ such that $g(f(A_n)) = x = g(f(A_{n+1}))$.
And since both $f$ and $g$ are 1-1 mappings, this would mean that $A_n = A_{n+1}$.
Thus, for $A_n \neq A_{n+1}$, we need that $A_n \cap A_{n+1} = \emptyset$, for all $n\in\mathbb{N}$.
\\



(c) Let $A =  \cup^{\infty}_{n=1} A_n$ and $B =  \cup^{\infty}_{n=1} f(A_n)$.
Show that $f: A \rightarrow B$.

As we just saw, we have that all $A_n$ are pairwise disjoint subsets of $X$, and by similar logic we can see that
$\{ f(A_n) : n\in\mathbb{N} \}$ are also pairwise disjoints subsets of $Y$.
If we revisit our partitioning scheme, we could define
$$
A = \cup^{\infty}_{n=1} A_n
$$
and
$$
B = \cup^{\infty}_{n=1} f(A_n)
$$

If we were now to apply $f$ on our partitions we would have something like the following,
$$
f(X) = f\left(
    A \cup A'
\right)
$$
But since we know that $A \cap A' = \emptyset$, then we have
$$
f(X) = f\left(   A \cup A' \right) =
    f(A) \cup f(A')
$$

If we look only at what happens with $A$ we then have,
$$
f(A) = f\left( \cup^{\infty}_{n=1} A_n \right) =
    \cup^{\infty}_{n=1} f(A_n) =
    f(B)
$$
The second step because we saw in the previous step that the series of $A_n$ is pairwise disjoint, the last step because
that was our definition of $B$.
\\

(d) Let $A' = X-A$ and $B' = Y-B$. Show $g: B' \rightarrow A'$.

If we follow a similar line of argument as we did above, we could make an argument that
$g: B\rightarrow A$.
So now, all we have to figure out is how $A'$ and $B'$ behave under our mappings.

We know that $f: X\rightarrow Y = A\cup A' \rightarrow B\cup B'$
and $g: Y \rightarrow X = B\cup B' \rightarrow A\cup A'$.
Since the sets $A$ and $A'$ are disjoint, as well as $B$ and $B'$, combining that with our previous answers,
we can see that $g$ also maps $B' \rightarrow A'$.

In part b we covered $A$ by looking at the infinite number of disjoint subsets that could make it up and we saw how
they all get mapped to $B$, and viceversa for $B \rightarrow A$.
Now that we have defined $A' = X-A$ and $B' = X-B$, we have covered all potential cases, so now we can
make a better argument for the existence of a 1-1 and onto mapping that shows $X \sim Y$.
\\~\\





%%%%%%%%%%%%%%%%%%%%%%%%%%%%%%%%%%%%%%%%%%%%%%%%%%%%%%%%%%%%%%%%%%%%%%%
\subsection{Cantor's Theorem}

When we pick $a$, $f(a) = \{ \{\emptyset\}, \{b\}, \{c\}, \{b,c\} \}$.
If we then look at $f(a)$, $f(b)$, and $f(c)$ (look at the mappings for all elements of $A$),
then we see that the set $B = \{ a, b, c \}$ is never a possibility for any of them - $B$ is
not in the range of the function used.
(A counter example to $f$ being onto, already.)

Since we are looking for a mapping that is onto, we would assume that
$B = f(a')$ for some $a' \in A$.

If we say that $a' \in B$, then $f(a') = B$ is not possible with the current mapping we are using,
$f(a)$ does not contain $a$.

If we look at the case were $a' \notin B$, then $f(a')$ would contain $a'$, which is again inconsistent
with our mapping.


\subsubsection{Exercises}

\textbf{1.6.4}

This result is used in Chapter 3 when Abbott makes an argument for Cantor sets being uncountable.
