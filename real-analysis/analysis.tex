\documentclass{article}
\usepackage{blindtext} % for table of contents and other niceseties.
\usepackage{amsmath}
\usepackage{amsfonts} % things such as \mathbb{R}, or \Z. See https://en.wikipedia.org/wiki/List_of_mathematical_symbols_by_subject
\usepackage{amstext} % for \text macro.
\usepackage{array} % for tables.
\usepackage{float} % here for H placement parameter
\usepackage{hyperref} % for hyperlinks.
\usepackage{listings} % for code.
\usepackage[makeroom]{cancel} % crossing out terms in math https://tex.stackexchange.com/questions/75525/how-to-write-crossed-out-math-in-latex
\usepackage[shortlabels]{enumitem} % enumerate with letters.


% Make greek letters bold too.
\usepackage{bm}
\newcommand{\vect}[1]{\boldsymbol{\mathbf{#1}}}

% Make paragrphs show in the table of contents.
\setcounter{secnumdepth}{4}
\setcounter{tocdepth}{4}

\title{Notes on Real Analysis}

\numberwithin{equation}{section}

% There is a \pmod for the "mod" symbol, but not one for the "div" symbol.
% Thus we are making it up.
\makeatletter
\newcommand*{\bdiv}{%
  \nonscript\mskip-\medmuskip\mkern5mu%
  \mathbin{\operator@font div}\penalty900\mkern5mu%
  \nonscript\mskip-\medmuskip
}
\makeatother

\begin{document}

\maketitle
\tableofcontents

\section{The real numbers}

%%%%%%%%%%%%%%%%%%%%%%%%%%%%%%%%%%%%%%%%%%%%%%%%%%%%%%%%%%%%%%%%%%%%%%%
\subsection{Background}
More formally stated, a field is any set where addition and multiplication are well-defined operations
that are commutative, associative, and obey the familiar distributive property $a(b + c) = ab + ac$.
There must be an additive identity, and every element must have an additive inverse.
Finally, there must be a multiplicative identity, and multiplicative inverses must exist for
all nonzero elements of the field.



%%%%%%%%%%%%%%%%%%%%%%%%%%%%%%%%%%%%%%%%%%%%%%%%%%%%%%%%%%%%%%%%%%%%%%%
\subsection{Preliminaries}

\subsubsection{$x_{n+1} = \frac{1}{2}x_n + 1$}

The series is non-decreasing
$$
x_n \leq x_{n+1}
$$

And it just so happens that $x_n = 2 - \frac{1}{ n^{n-1} } = \frac{2^{n+1} - 1}{2}$
Write it all out to see the pattern.

$$
\lim\limits_{n \to \infty} \frac{2^{n+1} - 1}{2} = 2
$$

$\frac{2^{n+1} - 1}{2}$ grows as $\frac{2^{n+1}}{2}$.





\subsubsection{Triangle Inequality}

\begin{equation}
|x| =
  \begin{cases}
    x   & \quad x \geq 0\\
    -x  & \quad x < 0
  \end{cases}
\end{equation}

\begin{equation}
    |ab| = |a| |c|
\end{equation}

\begin{equation}
    |a+b| \leq |a| + |c|
\end{equation}

If $a,b \geq 0$, then $|a+b| = a+b = |a| + |b|$.
If $a,b < 0$, then $|a+b| = |a| + |b|$ (i.e., $|-1 + -2| = |-1| + |-2| = 3|$).
If $a<0, b>0$, then $|-1+2| = |1|=1$ and $|-1| + |2| = 3$, so $|a+b| < |a| + |b|$. Similarly if $a>0$ and $b<0$.

\begin{equation}
    |a-b| = |b-a|
\end{equation}
think about it as the distance between two points.

$|a-b| = |(a-c) + (c-b)| <= |a-c| + |c-b|$ using the triangle inequality. 
So, $|a-b| \leq |a-c| + |c-b|$ for any $c \in \mathbb{R}$.
So the above expression says something like "distance from a-to-b is equal to or less than the distance from a-to-c
plus the distance from c-to-b".

Proofs for triangle and reverse triangle inequalities are in \url{https://en.wikipedia.org/wiki/Triangle_inequality}.
Some takeaways are that:

$|b| \leq a$ can also be expressed as $-a \leq b \leq a$.

$-|x| \leq x \leq |x|$ so $-(|x|+|y|) \leq x+y \leq |x|+|y|$.

$|x| = |(x-y) + y| \leq |x-y| + |y|$ or $|x| - |y| \leq |x-y|$.

$-|x-y| \leq |x| - |y| \leq |x-y|$ also means $||x|-|y|| <= |x-y|$.

Additionally $|x| - |y| \leq |x+y| \leq |x| + |y|$.






\subsubsection{Exercises}

\textbf{1.1.1}

Prove that $\sqrt{3}$, $\sqrt{6}$, $\sqrt{4}$ are irrational.

\href{https://math.stackexchange.com/questions/930486/prove-that-the-square-root-of-3-is-irrational}{stackexchange: prove that the square root of 3 is irrational}:
A supposed equation $m^2 = 3n^2$ is a \textbf{direct contradiction to the Fundamental Theorem of Arithmetic,
because when the left-hand side is expressed as the product of primes, there are evenly many 3s there,
while there are oddly many on the right.}
\\~\\




\textbf{1.2.7}

$$
A = \{x \in \mathbb{R} : 0 \leq x \leq 2\}
$$
$$
B = \{x \in \mathbb{R} : 1 \leq x \leq 4\}
$$
so $A \cap B = \{x \in R : 1 <= x <= 2 \}$. 

$$
f(A) = \{ x \in \mathbb{R} : 0 \leq x \leq 4 \}
$$
$$
f(B) = \{ x \in \mathbb{R} : 1 \leq x \leq 16 \}
$$
so
$$
f(A) \cap f(B) = \{ x \in \mathbb{R} : 1 \leq x \leq 4 \}
$$
and $f(A \cap B) = \{ x \in \mathbb{R} : 1 \leq x \leq 4 \}$.

So in this case $f(A) \cap f(B) = f(A \cap B)$.

$$
A \cup B = \{x \in \mathbb{R} : 0 \leq x \leq 4\}
$$
so 
$$
f(A \cup B) = \{x \in \mathbb{R} : 0 \leq x \leq 16\}
$$
$$
f(A) \cup f(B) = \{x \in \mathbb{R} : 0 \leq x \leq 16\}
$$

So $f(A) \cup f(B) = f(A \cup B)$ as well.

Counterer example of $f(A \cap B) = f(A) \cap f(B)$ is if $A = \{x \in \mathbb{R} : -6 \leq x \leq 3\}$ And
$B = \{x \in \mathbb{R} : 3 \leq x \leq 6\}$.
There A and B is $\emptyset$, so $f(A \cup B)$ is $\emptyset$.
While f(A), f(B), and $f(A) \cup (B)$ are $\{x \in \mathbb{R} : 9 \leq x \leq 36\}$.

To show that for an arbitrary function $g: \mathbb{R} \rightarrow \mathbb{R}$, the statement
$g(A \cap B) \subseteq g(A) \cap g(B)$ holds for all sets $A, B \subseteq \mathbb{R}$,
we need to prove that every element in $g(A \cap B)$ is also an element of both $g(A)$ and $g(B)$.
Let's proceed with the proof:

Let x be an arbitrary element in $g(A \cap B)$. This means that there exists an element $y \in A \cap B$ such that $g(y) = x$.
Since $y \in A \cap B$, y is both in set A and set B. Therefore, $g(y) \in g(A)$ and $g(y) \in g(B)$.
Since $g(y) = x$, we can conclude that $x \in g(A)$ and $x \in g(B)$, which implies that $x \in g(A) \cap g(B)$.
Since x was an arbitrary element in $g(A \cap B)$, we have shown that every element in $g(A \cap B)$ is also an element of $g(A) \cap g(B)$.

Thus, we have proved that $g(A \cap B) \subseteq g(A) \cap g(B)$ for all sets $A, B \subseteq \mathbb{R}$.

Following a similar line of thinking for $g(A \cup B) and g(A) \cup g(B)$.
Let x be some element in $g(A \cup B)$, then there must be a y in either A or B such that $g(y) = x$.
Therefore, $g(y) \in g(A) \cup g(B)$.
\\~\\



\textbf{1.2.10}

$y_1 = 1$
$$
y_{n+1} = \frac{3}{4} y_n + 1 = \frac{3y_n + 4}{4}
$$

$n=1: y_1 < 4$

Now we want to show that if we have $y_n < 4$, then so will $y_n+1 < 4$
We are starting from the hypothesis that $y_n < 4$, then

$$
\frac{3}{4} y_n + 1 < \frac{3}{4} 4 + 1 = 4
$$

Which means $y_{n+1} < 4$.

$y_1 = 1$, $y_2 = \frac{7}{4}$, $y_3 = \frac{37}{16}$.

$n=1: y_1 < y_2$

Induction hypothesis $y_n < y_n+1$
$$
\frac{3}{4}  y_n + 1 < \frac{3}{4}  y_{n+1} + 1 \rightarrow y_{n+1} < y_{n+2}
$$
\\~\\



\textbf{1.2.11}

\textbf{If a set A contains n elements, prove that the number of different subsets of A is equal to 2n.}
\textbf{(Keep in mind that the empty set is considered to be a subset of every set.)}

Every element in A can be in or not, thus m multiplications of 2.
\\~\\


\textbf{1.2.12}

Trivial case is: $(A_1)^c  = A_1^c$.

Base case: $(A_1 \cup A2_)^c = A_1^c \cap A_2^c$ by de morgan’s theorem.

$(A_1 \cup A_2 \cup A_3)^c = (B_1 \cup A_3)^c = B_1^c \cap A_3^c = A_1^c \cap A_2^c \cap A3^c$.
Distributive property plus $B := A_1 \cup A2$. 
without loss of generality $B = A_1 \cup A_2 \cup \dots A_k$.
\\~\\



%%%%%%%%%%%%%%%%%%%%%%%%%%%%%%%%%%%%%%%%%%%%%%%%%%%%%%%%%%%%%%%%%%%%%%%%%%%%%%%
\subsection{The axiom of completeness}

\textbf{1.3.1}

Let $Z_5 = \{0, 1, 2, 3, 4\}$.
Additiona and multiplication modulo 5 would then be as,

$$
7+9 = 16 \bmod 5 = 1
$$
And
$$
7\times 9 = 63 \bmod 5 = 3
$$

Additive inverse: in order for $z+y = 0$ in $Z_5$, then the modulo addition of z and y must be a multiple of 5, in this case.
One way to define the additive inverse of z is to have it be $m-z$, where m is the modulo we are using.
That way $z + y = z + (m-z) \bmod m = m \mod m = 0$.
\\

Multiplicative inverse: if $z \neq 0$ in $Z_5$, $\exists x$ such that $zx = 1$.
So whatever product we end up getting must be $nm + 1$ for some integer n (always have a remainder of 1).
Yet another way of thinking about this is that $zx \equiv 1 \bmod m$.

There is an interesting pattern that shows up here...
If $z=1$, then $zx = x \bmod 5 = 1$ has multiple solutions but two of them are $x=1$ or $x=6$ ($x=z$ or $x=z+m$).
Similarly, if $z=2$, then 2 possible solutions would be $x=3$ or $x=8$ ($x=z+1$ or $x=z+m+1$).
If $z=3$, then 2 possible solutions are $x=2$ or $x=7$ ($x=z-1$ or $x=z+m-1$).
If then $z=4$, then 2 possible solutions are $x=4$ or $x=9$ ($x=z$ or $x=z+m$ - again!!!).
Since we have found the multiplicative inverse for each non-zero element in $Z_5$, we can conclude
that for any $z \neq 0$ in $Z_5$, there exists an element $x$ such that $zx = 1$.

\textbf{Multiplicative inverses exist only when $z$ and $m$ are relatively prime}.
For example if $z=5$, and we are in $Z_{10}$ then there is no number such that $zx \equiv 1 \bmod 10$.

Keeping in mind that a \textbf{congruence class} is an equivalence relation on an algebraic structure (e.g., a ring)
that is compatible with the structure in the sense that algebraic operations done with equivalent elements will yield equivalent elements.
More humanly put, a congruence class is the set of all integers that have the same remainder as a when divided by n.
Now in a ring $Z_n$, the units (i.e. the elements which have a multiplicative inverse) are the
congruence classes of the elements $m$ which are coprime to $z$, because for such an element, we have a
Bezout's relation, $um + vz = 1$ or $um = 1 - vz$, which means the class of $u$ is the inverse of that of $m$.
We obtained the previous argument from
\href{https://math.stackexchange.com/questions/2650336/for-z-5-0-1-4-and-z-in-z-n-prove-that-there-exists-a-multiplicative-i}{stackexchange: prove that there exists a multiplicative inverse}.
\\~\\


\textbf{1.3.2}

A real number A real number $l$ is the greatest lower bound for a set $A \subseteq \mathbb{R}$ if it meets the following
two criteria:

\begin{enumerate}
    \item $l$ is a lower bound of $A$,
    \item if $m$ is any lower bound for A, then $l \geq m$.
\end{enumerate}

Lemma: assume $l \in \mathbb{R}$ is a lower bound for a set $A \subseteq \mathbb{R}$.
Then $l = \inf{A}$ if and only if, $\forall \epsilon > 0$, $\exists a \in A$ satisfying $l + \epsilon > a$.

Given that $l$ is a lower bound, $l$ is the greatest lower bound iff any number greater than $l$ is not a lower bound.

In the forward direction we want to prove that: if $l$ is the greatest upper bound, then $l+\epsilon > a$.
In the forward direction we want to prove that: if $l$ is a lower bound satisfying $l+\epsilon > a$, then $l$ is also the
greatest upper bound.

For the former case, because $l + \epsilon > l$, then by definition $l+\epsilon$ is not a lower bound. Thus there must
$\exists a \in A$ for which $l+\epsilon > a$.
Note that we used the defintion to prove a point, we are not questioning the definition (whether $l$ is the greatest upper bound).

For the latter direction, assume $l$ is a lower bound such that $\forall \epsilon >0$, $l+\epsilon$ is no longer a lower bound
for A.
If this is so, a number slightly greater than $l$ (for any degree of slightly) is no longer a lower bound, then by Definition
of the greatest lower bound, $l$ can be the greatest lower bound.
\\~\\


\phantomsection
\label{abbott:1.3.3}

\textbf{1.3.3}


If A is bounded below, and we define $B = \{ b \in \mathbb{R}: \text{b is a lower bound of A} \}$.
Show that $\sup B = \inf A$.

By the axiom of completeness we can start by knowing that any non-empty set of real numbers that is bounded above
has a least upper bound.
So as long as A is not an empty set, then B will not be empty either.
In the case of the set B, any upper bound wil be equal or greater than any $b \in B$.
And the least upper bound will be equal to or smaller than any other bound we can find (smaller than or equal to
any $a \in A$).

At the same time, since B contains lower bounds of A, we already know that there is a $sup B$ that exist and is smaller
than any $a \in A$ (other upper bounds of B) while also being greater than or equal to any other bounds of A (members of B).
That is, $\sup B$ is a lower bound of A and is greather than or equal to any other bounds of A.
Hence, by defintion of the greatest lower bound $\sup B = \inf A$.
\\

\textbf{This exercise points to the interesting case when if $a$ is an upper bound for A,
and if $a \in A$, then it must be that $a = \sup A$}.
\\~\\


\textbf{1.3.4}

Assume that A and B are nonempty, bounded above, and satisfy $B \subseteq A$. Show $\sup B \leq \sup A$.

If B and A are equal then their least upper bound will be equal.
On the other hand, if A has elements that B doesn't, then those elements not in B can be smaller or greater than those in B.
If the extra elements are smaller than $b \in B$, then the least upper bounds of the two sets will not change.
But if A contains elements that are greater than B, then $\sup A > \sup B$.
\\~\\


\textbf{1.3.8}

If $\sup A < \sup B$, then show that $\exists b \in B$ that is an upper bound for A.

Since B has a least upper bound, then there must be $b \in B$ such that $b \leq \sup B$.
Similarly in A, there must exist $a \in A$ such that $a \leq \sup A$.
Since we know that $\sup A < \sup B$, then we have $a \leq \sup A < \sup B$ ($\sup B$ is an upper bound for A).

If $b$ is an upper bound of A, then that would mean that $b \geq a$ for any $a \in A$.
If $b$ was not an upper bound, then $b < a$ for all $b$.
But if this was the case, then $\sup B$, which is an upper bound of B ($\sup B \geq b$) would be smaller than some
$a$, leading us to a contradiction as $\sup A$ would then be greater than $\sup B$.

We can also see this as a case mention in problem 1.3.4 above.
\\~\\



%%%%%%%%%%%%%%%%%%%%%%%%%%%%%%%%%%%%%%%%%%%%%%%%%%%%%%%%%%%%%%%%%%%%%%%%%%%%%%%%%%%%%
\subsection{Consequences of completeness}


\textbf{1.4.1}

\textbf{Density of Q in R:} For every $a,b \in \mathbb{R}$ where $a < b$ and $a<0$, $\exists r \in \mathbb{Q}$
satisfying $a < r < b$.

Since we want to prove the case of $a<0$, we want to see if there is a rational number $r$ such that $a < r < 0$.
There rest of the proof in theorem 1.4.3 then applies as is.
\\~\\


\subsubsection{Existence of square roots}

$$
T = \{ t\in\mathbb{R} : t^2 < 2 \}
$$

Then we work out the expression
$$
\left( \alpha + \frac{1}{n} \right) < \alpha^2 + \frac{2\alpha + 1}{n}
$$

That bit $(2\alpha + 1)/n$ is what we need to fit between $\alpha^2$ and 2 while keeping $\alpha^2 <2$.
Since we want to fill that space, we come up with
$$
\frac{2\alpha + 1}{n} < 2 - \alpha^2
$$

Which then simplifies to what Abbott uses.


$$
\left( \alpha + \frac{1}{n_0} \right)^2 < \alpha^2 + (2 - \alpha^2) = 2
$$

contradicts the fact that $\alpha$ is an upper bound because $(\alpha + \frac{1}{n})^2$ is also a member of
$T$, and is larger than $\alpha$ alone.

For the case of $\alpha^2 > 2$, where we want to prove that this contradicts the fact that $\alpha$ is the least upper bound,
then it must be the case in which $\alpha$ is an upper bound for $\{ t \in \mathbb{R} : t^2 < 2 \}$ but it must not be
the smallest upper bound.
That is, $\alpha > b$, where $b$ is some other bound of $T$ ($b=2$ for example).

To show that the above would lead us to the said contradiction we can now take

$$
\left( \alpha - \frac{1}{n} \right)^2 =
    \alpha^2 - \frac{2\alpha}{n} + \frac{1}{n^2} >
    \alpha^2 - \frac{2\alpha}{n}
$$

If we chose an $n_0$ such that 
$$
\alpha^2 - \frac{2\alpha}{n_0} > 2
$$

That is, we look for an $n_0$ that can give us a number smaller than $\alpha$ but that it is an upper bound.
The above implies that $- \frac{2\alpha}{n_0} > 2 - \alpha^2$, and consecuently

$$
\left( \alpha - \frac{1}{n_0} \right)^2 >
    \alpha^2 + (2 - \alpha^2) = 2
$$

So we see that $\alpha - \frac{1}{n_0}$ is an upper bound and it is smaller than $\alpha$,
so $\alpha^2 > 2$ also be the case.


\subsubsection{$N \sim R$}

We known that for some real number $x_{n_0} \notin I$, so that
$$
x_{n_0} \notin \bigcap^{\infty}_{n=1} I_n
$$

Since we are assuming that $\mathbb{R} = \{ x_1, x_2, ... \}$ contains all the real numbers,
then $n_0$ can be any of the real numbers.
So for any given $n_0$, $I_{n_0}$ will not contain it, nor any $I_{n_0+1}$, or any of its subsets.
Going in the other direction, $x_{n_0}$ will be in $I_{n_0 - 1}$, but
$I_{n_0 - 1}$ will not contain $x_{n_0-1}$.
Thus, there is no one number that is contained by all sets $I_n$, Hence

$$
\bigcap^{\infty}_{n=1} I_n = \emptyset
$$

Which contradicts the nested-interval property.

When proving the nested interval property we showed that at the very elast the supremum of each interval $I_n$
would be in every $I_n$, which lead us to the discovery that their intersection would be a non-empty set.
However, if we don't have the supremum of each set present for each $I_n$, then by our experience with the axiom
of completeness, we would think that there are gaps in our list of numbers.

The "logical" issue in the above argument is that the real numbers are "enumerable", which doesn't show up in the original
proof of the nested interval theorem because hte axiom of completeness creates a sort of "continuity" among the numbers
(the filling in between the gaps).


%%%%%%%%%%%%%%%%%%%%%%%%%%%%%%%%%%%%%%%%%%%%%%%%%%%%%%%%%
\subsubsection{Exercises}

\textbf{1.4.1}

\textbf{For every two real numbers $a$ and $b$ with $a<b$, there exists a rational number satisfying $a < r < b$.}

As requirements we have that $a<b$ and this time we want to look at the case were $a<0$,
so $b$ can be either negative, zero, or positive, it just has to be greater than $a$.
However the rest of the proof can proceed as normal if we don't require $m \in \mathbb{N}$ but
instead generalize $m \in \mathbb{Z}$.
See
\href{https://math.stackexchange.com/questions/48537/how-does-this-proof-of-density-of-mathbbq-in-mathbbr-require-a-geq-0}{How does this proof of density of Q in R require $a\geq 0$?}.

The original proof uses the Archimedean Property and the Archimedean Principle for positive
numbers to find a positive integer $n$ such that $\frac{1}{n} < b - a$.
Then, it uses the Density Property of the set of natural numbers to find a natural number $m$
such that $m > n$ and $an < m$.
\\~\\



\textbf{1.4.2}

If we have $a = \frac{m}{n}$ and $b = \frac{p}{q}$, then

$$
a + b = \frac{m}{n} + \frac{p}{q}
    = \frac{mq + np}{nq}
$$
Which is a rational number since $m, n, p, q \in \mathbb{Z}$.

Similarly,
$$
ab = \frac{m}{n} \frac{p}{q}
    = \frac{mp}{nq}
$$

However, if we multiply (or add), let's say $a$, by $i \in \mathbb{I}$, we want to show that this would result in an irrational number.
To prove this we will use a proof by contradition.

In the case of a sum, let's begin by assuming that if we add a rational number and an irrational numbert
that the result is a rational number too.
Then,
$$
\frac{m}{n} + i = \frac{x}{y}
$$

This would mean that we could isolate $i$
$$
i = \frac{x}{y} - \frac{m}{n}
$$
Which as we showed above, would mean that $i$ is a rational number.

In the case of a multiplication,
$$
i \frac{m}{n} = \frac{x}{y}
$$

$i$ could be isolated,
$$
i = \frac{x}{y} \frac{n}{m}
$$

Which again, would lead to a contradiction since the product of two rational numbers is also a rational number
and the above says that we magically converted $i$ into a rational number.

One would then think that the irratinoal numbers are also closed under addition or multiplication, but this is not the case.
Consider $1 - \sqrt{2} \in \mathbb{I}$ (as we just saw) and $\sqrt{2}$.
$$
( 1 - \sqrt{2} ) + \sqrt{2} = 1 \in \mathbb{Z}
$$

Similarly,
$$
\sqrt{2}\sqrt{2} = 2
$$
\\~\\



\textbf{1.4.5}

\textbf{Given any two real numbers $a < b$, there exists an irrational number $t$ satisfying $a < t < b$.}
To prove this, we can use the results from the previous exercise by applying the theorem where we talked about the
density of the rational numbers in the real numbers to $a - \sqrt{2}$ and $b - \sqrt{2}$
(which are real and irrational numbers.)

This case is easy to see since we know that between any two real numbers there exists a rational number
$$
a < \frac{m}{n} < b
$$

And we also know that we can convert a rational number into an irrational by adding or multiplying by an irrational number.
So, if we shift the case we used to prove the density of the rational numbers by subtracting an irrational number
$$
a - \sqrt{2} < \frac{m}{n} - \sqrt{2} < b - \sqrt{2}
$$

We can arrive to the result we wanted to prove.
\\~\\



\textbf{1.4.4}

Use the Archimedean property of $\mathbb{R}$ prove that $\inf \{ \frac{1}{n} : n \in \mathbb{N} \} = 0$.

By the Archimedean property we know that for any $y>0 \in \mathbb{R}$, $\exists n \in \mathbb{N}$ such that
$y > \frac{1}{n}$.

We also know that for a number to be an infinimum then it has to be a lower bound and it has to be greater than any
otther lower bounds.

In the case of the set $A = \{ \frac{1}{n} : n \in \mathbb{N} \}$, we can see that 0 is indeed a lower bound.
But let's say that we have another lower bound $b$ that is greater than 0 (assume the greatest lower bound is $b$).

Since $0 \notin A$ and $b$ is a lower bound, then $b > 0$ but $b \leq a$, for any $a \in A$.
However, the Archimedean property tells us that for any $b$ we could have, there will be a $\frac{1}{n}$ that is smaller than it.
Thus disproving the fact that a $b>0$ could be a lower bound.
\\~\\



\textbf{1.4.5}

Prove that $\bigcap^{\infty}_{n=1} (0, 1/n) = \emptyset$. 
Notice that this demonstrates that the intervals in the Nested Interval Property
must be closed for the con- clusion of the theorem to hold.

The proof we used for the nested property theorem, theorem 1.4.1.
Following a similar schema, we can see that the set of right-hand endpoints $\{ 1/n : n\in\mathbb{N} \}$ which we have
already seen that its greatest lower bound is 0.
We have also saw in exercise 1.3.3 that the supremum of the left-hand endpoints must equal the infinimum of the right-hand
endpoints, which is zero.
So the approach we followed in theorem 1.4.1 doesn't apply here since 0 is not even part of any of the intervals $I_n$.
\\~\\





%%%%%%%%%%%%%%%%%%%%%%%%%%%%%%%%%%%%%%%%%%%%%%%%%%%%%%%%%%%%%%%%%%%%%%%%%%%%%%%%%%%%%
\subsection{Cardinality}


\textbf{1.5.1}

\textbf{Theorem 1.5.7: if $A \subseteq B$ and $B$ is countable, then either $A$ is countable, finite, or empty.}

Recall that a set $B$ is countable if $\mathbb{N} \sim B$.
Which means that the sets $\mathbb{N}$ and $B$ have the same cardinality as there exist some $f : N \rightarrow B$
that is 1-to-1 and onto.

Let $n_1 = \min \{ n \in \mathbb{N} : f(n) \in A \}$.
As a start to a definition of $g: \mathbb{N} \rightarrow A$, set $g(1) = f(n_1)$.

We know that $f$ is 1-to-1, so every natural number gets mappted to a different member of $B$, that is, $n_1 \neq n_2$, then
$f(n_1) \neq f(n_2)$.
We also know that for every $B$, and clearly for every $A$ (since $A \subseteq B$), there exists an $n \in \mathbb{N}$
such that $a = f(n)$ for every $a \in A$.
Thus, we only need to prove that $g$ is a 1-to-1 function.

Now back to our definition of $g$, if we now look at $A - \{f(n_1)\}$, and we define
$n_2 = \min \{ n \in \mathbb{N} : f(n) \in A - \{f(n_1)\} \}$, or more generally,
$n_n = \min \{ n \in \mathbb{N} : f(n) \in A, n \notin \{n_1, n_2, ..., n_{k-1}\} \}$,
then we have $g(k) = f(n_k)$, which we know is different from all other values of $g$ since $f$ is 1-to-1.
\\~\\



\textbf{1.5.3}

\textbf{If $A_1, A_2, \ldots, A_m$ are each coutable sets, then the union $A_1 \cup A_2 \cup \ldots \cup A_m$ is countable.}

Replace $A_2$ with $B_2 = A_2 - A_1 = \{ x \in A_2 : x \notin A_1 \}$.
$A_1 \cup B_2  = A_1 \cup A_2$ while keeping the two sets disjoint.

If we followed the example 1.4.8, then we could see how "laying out" the members of $A_1 \cup A_2 = A_1 \cup B_2$ could be
mapped to by the natural numbers, $\{ a_1, a_2, \ldots, b_1, b_2, \ldots \}$.
For the induction step, we can follow logic similar to when we prove the generalization of the de morgan laws in 1.2.12.
\\

\textbf{If $A_n$ is a countable set for each $b \in \mathbb{N}$, then $\bigcup^{\infty}_{n=1} A_n$ is countable.}

The same induction logic we used above could get us into trouble here because induction shows something
for all $n \in \mathbb{N}$, not for infinity - and we have already seen that there are various types of infinities.
If we followed the same logic we could end up with a problem similar as to when we claimed that the reals are
enumerable.
\\

If we were to arrange natural numbers in a two-dimensional array we end up with disjoint sets $B$ such that
$\cup^{\infty}_{n=1} B_n = \mathbb{N}$.
(Note how this can be a construction similar to the first instance in this problem where $B_1 = A_1$,
$B_2 = A_2-A_1 = A_2 - B_1$, $B_3 = A_3 - B_2$, and so on.)

We also now (and can see) that every $B$ has an $f_n$ has is 1-to-1 and onto.

So in order to prove that there exists a mapping that maps the natural numbers into our tw-dimensional array of disjoint sets,
it suffices to see that our way of ordering the sets is itself a mapping such that
$f : \mathbb{N} \rightarrow \cup^{\infty}_{n=1} B_n$.
The mapping could be expresed as $f (x_{m,n} \in \mathbb{N}) = (m, n) \in \cup B_n$.

And since we organized the two-dimensional array in the way we have, then the we have
$\mathbb{N} \subseteq \mathbb{N} \times \mathbb{N}$ and the mapping is bijective
($f (x_{m,n} \in \mathbb{N})$ is a specific mapping that creates order pairs that such that all order pairs are 1-to-1 and onto).

See
\href{https://math.stackexchange.com/questions/4403048/simplification-in-proof-of-countable-union-of-countable-sets-is-countable}{Simplification in Proof of Countable Union of Countable Sets is Countable}
for a similar presentation.
\\~\\



\textbf{1.5.4}

Is $(a,b) \sim \mathbb{R}$, for any open interval $(a,b)$?

Let's take example 1.5.4 as inpiration.
This example says that $f = x / (x^2 -1)$ takes the interval $(-1, 1)$ onto $\mathbb{R}$, showing that
$(-1,1) \sim \mathbb{R}$.

The books mentions some calculus so let's see what we can see.
First we have,
$$
\lim_{x \rightarrow -1^+ } \frac{x}{x^2 - 1} \rightarrow
    \frac{-1}{\text{some very small positive number}} 
    \rightarrow -\infty
$$

Similarly,
$$
\lim_{x \rightarrow 1^- } \frac{x}{x^2 - 1} \rightarrow
    \frac{1}{\text{some very small positive number}} 
    \rightarrow \infty
$$

Its derivative is
$$
f' = \frac{1}{(x^2 - 1)^2} ( (x2 - 1) - x(2x) ) =
    \frac{-x^2 - 1}{(x^2 - 1)^2} =
    - \frac{x^2 + 1}{(x^2 - 1)^2}
$$

For the derivative, no matter whether $x$ is negative or positive, its value will always be zero.
Which is a no-graphing way of triple checking that the original function won't "fluctuate" and we can indeed map all of
the domain of $f$ to unique values in $\mathbb{R}$.

Now in order to see whether $f$ is onto, we could see if we can build an inverse of f and see if it has any "holes".
But that's rather complicated in our case, so let's try something else.
We could use the intermediate value theorem but we don't yet know what "continuous" or any of the other requirements mean.
However, we have seen that within $(-1,1)$ the function goes from $-\infty$ to $\infty$, so we can imagine that any value
within the image will be mapped to by somewhere in the domain.

To shift the function we have been using into the interval $(a,b)$, let's define the midpoint between $b$ and $a$ as
$m = (b-a)/2$ and let's define the transformed function $f$ as
$$
g(x) = f\left(\frac{(x - m)}{m} \right)
$$

With the above transformation we are "stretching" the function by dividing $x$ my the mid-way length between $b$ and $x$
and we are also shifting it by the same distance.
Since the changes are all constant factors, all other properties we have been relying on remain the same.
\\~\\



\textbf{1.5.5}

Is $A \sim A$ for every set $A$?

If so, it'd mean that there is a function $f: A \rightarrow A$ that is 1-to-1 and onto.
Which means that $f(a_1) = f(a_2)$ iff $a_1 = a_2$.
However, $f(a_1) = a_1$ in our case.
Given this, it is easy to see that such a mapping will also be onto.
\\

If $A\sim B$, does that mean that $B\sim A$?

Well, for starters, we now know that there is an $f: A\rightarrow B$ that is 1-to-1 and onto.
So $b_1 = b_2 = f(a_1) = f(a_2)$ iff $a_1 = a_2$.
Also, for every $b_i = f(a_i)$ there is a unique $a_i$.
Putting these two properties together we know that for every element in $B$ there is a uniqie element
that can be mapped to it from $A$.
If we take this in the opposite direction, $B\rightarrow A$, we only need to prove that every every element we are mapping to
is unique, which is given to us by the mapping being 1-to-1.
\\

If $A\sim B$ and $B\sim C$, does that mean that $A\sim C$?

There exists an $f: A\rightarrow B$ that is 1-to-1 and onto.
There also exists a $g: B \rightarrow C$ that is 1-to-1 and onto.

So if we start with $f(a_i) = b_i$, then $g \circ f (a_i) = g(b_i) = c_i$.
Since $g(b_1) = g(b_2)$ iff $b_1 = b_2$, and $b_1 = f(a_1)$ and $b_2 = f(b_2)$, and $f(a_1) = f(a_2)$ iff $a_1 = a_2$,
then by extension, $g(b_1) = g(b_2)$ iff $a_1 = a_2$.

Similarly, since for every $c$ there exist a unique $b$, and for every $b$, there is a unique $a$.
Then we can map every $c$ to a unique $a$.
\\~\\



\textbf{1.5.9}

\textbf{A real number $x\in \mathbb{R}$ is algebraic if there exists integers $a_0, a_1,\dots ,a_n \in \mathbb{Z}$,
not all zero, such that}

\begin{equation}
    a_0 + a_1 x + a_2 x^2 + \ldots + a_n x^n = 0
\end{equation}

Said another way, a real number is algebraic if it is the root of a polynomial with integer coefficients.
Real numbers that are not algebraic are called transcendental numbers.

$\sqrt{2}$ is an algebraic number since $2x^2 - x^4 = 0$ can be a way to express $x^2 = 2$, $x = \sqrt{2}$.
\\

Fix $n \in \mathbb{N}$, and let $A_n$ be the algebraic numbers obtained as roots of polynomials with integer coefficients
that have degree $n$.
Using the fact that every polynomial has a finite number of roots, show that $A_n$ is countable.

For example, $A_2 = \{ x\in\mathbb{R} : \text{$x$ is a root of } a_0 + a_1 x + a_2 x^2 = 0 \}$.
Here problem 1.5.3 (theorem 1.5.8) will come in handy.
Essentially, if every polynomial of degree $n$ has a finite set of roots, then a piecewise application of
$f: \mathbb{N} \rightarrow A_n$ is doable.
If we also organize all $A_n$ to have only non-zero coefficients, then we have a collection of countable and disjointsets.
By the application of theorem 1.5.8, the union of all $A_n$ is also countable.
\\

Now, argue that the set of all algebraic numbers is countable.
What may  we conclude about the set of transcendental numbers?

As we saw in theorem 1.5.6 $\mathbb{R}$ is not countable.
And we just argued that the set of algebraic numbers is countable.
Roughly speaking, this could have us belief that the set of transcendental numbers is uncountable.
\\~\\



\textbf{1.5.11}

\textbf{Schröder-Bernstein theorem}
Assume there exists a 1-to-1 function $f: X \rightarrow Y$ and another 1-to-1 function $g: Y \rightarrow X$.
Follow the steps to show that there exists a 1-to-1, onto function $h: X \rightarrow Y$ and hence $X \sim Y$.

The strategy is to partition X and Y into components
$$
X = A \cup A'
$$
and
$$
Y = B \cup B'
$$

with $A \cap A' = \emptyset$ and $B \cap B' = \emptyset$, in such a way that $f$ maps A onto B,
and $g$ maps $B'$ onto $A'$.
\\

(a) Explain how achieving this would lead to a proof that $X \sim Y$.

If we assume that X and Y don't have the same cardinality, then it would be possible to map elements of Y
that are not reachable from X, and vice versa.
So a way that the partitioning shceme could help us show that there is a 1-1 and onto mapping $h$ would be
if it helped us see that all of the partitions of $X$ and $Y$ are reachable from some element in either of the
partitions on the other side - we can get to any element in $B$ or $B'$ from $A$ or $A'$.

The 1-1 and onto function $h(x)$ will probably look something like:
\begin{equation}
    h(x) =
      \begin{cases}
        f(x)       & \quad x \in A\\
        g^{-1}(x)  & \quad x \in A'
      \end{cases}
\end{equation}

Some good links to dive into similar arguments:
\begin{enumerate}
    \item \href{https://math.stackexchange.com/questions/3982170/schr%C3%B6der-bernstein-theorem-proof-help}{Schröder-Bernstein Theorem proof help}
    \item \href{https://math.stackexchange.com/questions/225576/intuition-behind-cantor-bernstein-schr%C3%B6der}{Intuition behind Cantor-Bernstein-Schröder}
    \item \href{https://math.stackexchange.com/questions/3296321/does-this-proof-of-the-schr%C3%B6der-bernstein-theorem-work}{Does this proof of the Schröder–Bernstein theorem work?}
    \item \href{https://math.stackexchange.com/questions/925203/how-to-prove-this-version-of-the-cantor-schroder-bernstein-theorem}{How to prove this version of the Cantor-Schroder-Bernstein theorem?}
    \item \href{https://math.stackexchange.com/questions/1726578/understanding-a-proof-of-schr%C3%B6der-bernstein-theorem}{Understanding a proof of Schröder-Bernstein theorem}
\end{enumerate}


(b) Set $A_1 = X - g(Y) = \{x \in X: x \notin g(Y)\}$ (what happens if $A_1 = \emptyset$?)
and inductively define a sequence of sets by letting $A_{n+1} = g(f(A_n))$.
Show that $\{A_n: n \in \mathbb{N} \}$ is a pairwise disjoint collection of subsets of X,
while $\{ f(A_n): n \in \mathbb{N} \}$ is a similar collection in Y.

$A_1 = X - g(Y)$ can be seen as "all of X, except the parts that can be reached from Y with $g$".
We then have $A_1 \subseteq X$.
If $A_1 = \emptyset$, then all of $g(Y)$ can be mapped into some $X$, so $g$ is 1-1 and onto,
and we have found our answer.
But let's carry out assuming that $A_1 \neq \emptyset$.

$A_1 \subseteq X$.
So $f(A_1) \subseteq Y$.
And consequently, $A_2 = g(f(A_1)) \subseteq X$ as well.
Since $A_1$ contained all $x \in \mathbb{R}$ such that $x \notin g(Y)$,
and $A_2$ was obtained by applying $g$ on $f(A_1)$, then we have $A_1 \cap A_2 = \emptyset$.

We now need to look at how $A_n$ and $A_{n+1}$ relate to one another.
$A_n = g(f(A_{n-1}))$, and $A_{n+1} = g(f(A_n)) = g(f(g(f(A_{n-1}))))$.
If we assume that $A_n \cap A_{n+1} \neq \emptyset$, then it means that there is
some $x\in X$ such that $g(f(A_n)) = x = g(f(A_{n+1}))$.
And since both $f$ and $g$ are 1-1 mappings, this would mean that $A_n = A_{n+1}$.
Thus, for $A_n \neq A_{n+1}$, we need that $A_n \cap A_{n+1} = \emptyset$, for all $n\in\mathbb{N}$.
\\



(c) Let $A =  \cup^{\infty}_{n=1} A_n$ and $B =  \cup^{\infty}_{n=1} f(A_n)$.
Show that $f: A \rightarrow B$.

As we just saw, we have that all $A_n$ are pairwise disjoint subsets of $X$, and by similar logic we can see that
$\{ f(A_n) : n\in\mathbb{N} \}$ are also pairwise disjoints subsets of $Y$.
If we revisit our partitioning scheme, we could define
$$
A = \cup^{\infty}_{n=1} A_n
$$
and
$$
B = \cup^{\infty}_{n=1} f(A_n)
$$

If we were now to apply $f$ on our partitions we would have something like the following,
$$
f(X) = f\left(
    A \cup A'
\right)
$$
But since we know that $A \cap A' = \emptyset$, then we have
$$
f(X) = f\left(   A \cup A' \right) =
    f(A) \cup f(A')
$$

If we look only at what happens with $A$ we then have,
$$
f(A) = f\left( \cup^{\infty}_{n=1} A_n \right) =
    \cup^{\infty}_{n=1} f(A_n) =
    f(B)
$$
The second step because we saw in the previous step that the series of $A_n$ is pairwise disjoint, the last step because
that was our definition of $B$.
\\

(d) Let $A' = X-A$ and $B' = Y-B$. Show $g: B' \rightarrow A'$.

If we follow a similar line of argument as we did above, we could make an argument that
$g: B\rightarrow A$.
So now, all we have to figure out is how $A'$ and $B'$ behave under our mappings.

We know that $f: X\rightarrow Y = A\cup A' \rightarrow B\cup B'$
and $g: Y \rightarrow X = B\cup B' \rightarrow A\cup A'$.
Since the sets $A$ and $A'$ are disjoint, as well as $B$ and $B'$, combining that with our previous answers,
we can see that $g$ also maps $B' \rightarrow A'$.

In part b we covered $A$ by looking at the infinite number of disjoint subsets that could make it up and we saw how
they all get mapped to $B$, and viceversa for $B \rightarrow A$.
Now that we have defined $A' = X-A$ and $B' = X-B$, we have covered all potential cases, so now we can
make a better argument for the existence of a 1-1 and onto mapping that shows $X \sim Y$.
\\~\\





%%%%%%%%%%%%%%%%%%%%%%%%%%%%%%%%%%%%%%%%%%%%%%%%%%%%%%%%%%%%%%%%%%%%%%%
\subsection{Cantor's Theorem}

When we pick $a$, $f(a) = \{ \{\emptyset\}, \{b\}, \{c\}, \{b,c\} \}$.
If we then look at $f(a)$, $f(b)$, and $f(c)$ (look at the mappings for all elements of $A$),
then we see that the set $B = \{ a, b, c \}$ is never a possibility for any of them - $B$ is
not in the range of the function used.
(A counter example to $f$ being onto, already.)

Since we are looking for a mapping that is onto, we would assume that
$B = f(a')$ for some $a' \in A$.

If we say that $a' \in B$, then $f(a') = B$ is not possible with the current mapping we are using,
$f(a)$ does not contain $a$.

If we look at the case were $a' \notin B$, then $f(a')$ would contain $a'$, which is again inconsistent
with our mapping.

%%%%%%%%%%%%%%%%%%%%%%%%%%%%%%%%%%%%%%%%%%%%%%%%%%%%%%%%%%%%%%%%%%%%%%%
\section{Rearrangements of infinite series}


%%%%%%%%%%%%%%%%%%%%%%%%%%%%%%%%%%%%%%%%%%%%%%%%%%%%%%%%%%%%%%%%%%%%%%%
\subsection{Limit of a sequence}


%%%%%%%%%%%%%%%%%%%%%%%%%%%%%%%%%%%

\subsubsection{Exercises}

\textbf{2.2.2}

$$
\lim \frac{2n+1}{5n+4} = \frac{2}{5}
$$

We want to show that,
$$
\left| \frac{2n+1}{5n+4} - \frac{2}{5} \right| < \epsilon
$$
For any $\epsilon > 0 \in \mathbb{R}$ as long as $n \geq N \in \mathbb{N}$.
To find $N$,

$$
\left| \frac{2n+1}{5n+4} - \frac{2}{5} \right| =
\left| \frac{5(2n+1) - 2(5n+4)}{5(5n+4)}  \right| =
\left| \frac{-3}{5(5n+4)} \right| =
\frac{3}{25n+20} < \epsilon
$$

We can do the last simplification because $N$ is some positive number.
Simplifying the above we get,
$$
3 < \epsilon (25N + 20)
$$
or
$$
\frac{3}{25\epsilon} - \frac{20}{25} = \frac{3}{25\epsilon} - \frac{4}{5} < N
$$

Note that we could obtain a simpler expression if we note that
$$
\frac{3}{5(5n+4)} < \frac{1}{n} < \epsilon
$$

So $N > 1/\epsilon$, which follows the same pattern as our previous answer.
The first answer gives us a more accurate convergence rate.
\\

$$
\lim \frac{2n^2}{n^3 + 3} = 0
$$

Following similar logic, we are looking at
$$
\left| \frac{2n^2}{n^3 + 3} \right| < \epsilon
$$

Plain algebra doesn't really help us isolate $n$, so again look for patterns.
$$
\frac{2n^2}{n^3 + 3} < \frac{2n^2}{n^3} = \frac{2}{n} < \epsilon
$$
\\

Now for,
$$
\lim \frac{\sin (n^2)}{n^{1/3}} = 0
$$

$\sin$ will never be greater than one, so we can proceed as follows.
$$
\frac{\sin (n^2)}{n^{1/3}} < \frac{1}{n^{1/3}} < \epsilon
$$

So $N > 1 / \epsilon^3$.
\\~\\



\textbf{2.2.6}

\textbf{Uniqueness of limites:} if the limit of a sequence that converges were not unique, then we
should be able to find any other number that meets our requirements - trying
to prove that there are multiple limits for a generic sequence is hard, same as trying to then
generalize this result to any sequence.
Intead our approach will be as follows.

If the sequence $a_n$ converges to $a$, then let's assume that there it also converges to $b$.
If we see that $a = b$, then we have our proof, else, we have a great publication.
So we know that

$$
| a_n - a | < \epsilon
$$
and
$$
| a_n - b | < \epsilon
$$

For some arbitrary epsilon.
Given the definition of converge, the above requirements must hold for some $N \in \mathbb{N}$, but
that $N$ does not need to be the same for the two expressions above.
So let's follow along using some $N = \max (N_1, N_2)$, where $N_1$ and $N_2$ are the numbers that would
make the above expressions hold for any $\epsilon$.

We then have,

$$
| a_N - a | < \epsilon
$$
and
$$
| a_N - b | < \epsilon
$$

Now, we are assuming that $a \neq b$, meaning that $| a - b | > 0$.
If we use the triangle inequality,
$$
| a - b | = | a - a_N + a_N - b | = | (a - a_N) + (a_N - b) |
$$
$$
\leq | a - a_N | + | a_N - b |
$$

Since $|x-y| = |y-x|$, we then have
$$
| a - b | \leq | a - a_N | + | a_N - b | = | a_N - a | + | a_N - b |
$$
$$
< \epsilon + \epsilon = \epsilon'
$$

Interestingly enough, we already saw a theorem of this sort back in theorem 1.2.6, which stated that
two real numbers $a$ and $b$ are equal iff $\forall \epsilon > 0$ it follows that $|a-b|<\epsilon$.
This is exactly what we are trying to get at.
But for the sake of argument, we could also see that we reach a contradiction if we continue.
Let's assume $\epsilon = 1/10$, then,

$$
|a-b| < 2\epsilon = \frac{2}{10}|a-b|
$$
And a number cannot be less than (a fraction of) itself.
\\~\\





%%%%%%%%%%%%%%%%%%%%%%%%%%%%%%%%%%%%%%%%%%%%%%%%%%%%%%%%%%%%%%%%%%%%%%%
\subsection{The algebraic and order limit theorems}

why does $|b_n - b| < |b|/2$  implies $|b_n| > |b|/2$ ?

We have been seeing that for a lot of convergence proofs we end up using the triangle inequality.
And this is a good tool to try out here!
Let's try this out firsrt,

$$
|b_n| = | b_n - b + b | \leq |b_n - b| + |b| < \frac{|b|}{2} + |b| = \frac{3|b|}{2}
$$

So we get a sense that we have to multiply $|b|$ by a number greater than 1 in order to go above
$|b_n|$.
But its a weak argument.
Let's try it the other way now!

$$
|b| = |b - b_n + b_n| \leq |b - b_n| + |b_n| = |b_n -b| + |b_n| < \frac{|b|}{2} + |b_n|
$$
Which can be simplified to
$$
\frac{|b|}{2} < |b_n|
$$

Which is the bit we did need.
\\~\\

In the \textbf{order limit theorem} proof we also see the statement: $|a_N - a| < |a|$ implies
that $a_N < 0$.
One way to see this is by noting that
$$
|a_N - a| < |a| \rightarrow -|a| < a_N - a < |a|
$$

If $a<0$, then $-|a| = a < 0$ and $|a| = -a > 0$, so
$$
-|a| < a_N - a < |a| \rightarrow a < a_N - a < -a \rightarrow 2a < a_N < 0
$$

The above tells us that in this case we would land with $a_N < 0$.
\\~\\


\subsubsection{Exercises}

\textbf{2.3.1}

If $x_n \rightarrow 0$, then we know that $|x_n| < \epsilon$ when $n \geq N$.
Now let's look at $|\sqrt{x_n}| = \sqrt{x_n}$.

A useful thing to note here is that if $x_n \geq 0$, then the order limit theorem says
that $x\geq 0$.

Coming back to $|\sqrt x_n| = \sqrt{x_n}$ we can also see that $|x_n| = x_n < \epsilon$ in this case,
so $ = \sqrt{x_n} =\sqrt |x_n|$. Note that we can now make the
value under the square root as small as we want
$$
|\sqrt x_n | < \sqrt |x_n| < \epsilon
$$
\\~\\

If $x_n \rightarrow x$, then
$$
| \sqrt x_n - \sqrt x | = | \sqrt x_n - \sqrt x | \left| \frac{\sqrt x_n + \sqrt x}{\sqrt x_n + \sqrt x} \right|
$$
$$
= \frac{|x_n - x|}{\sqrt x_n + \sqrt x}
$$

Since $\sqrt x_n + \sqrt x \geq \sqrt x$, then we could further simplify the above expression to
$$
\frac{|x_n - x|}{\sqrt{x_n} + \sqrt{x}} \leq \frac{|x_n - x|}{\sqrt{x}}
$$

Because a smaller denominator, results in a larger overall quantity.
We now know that the numerator matches the expression that we know convrges, so we can now chose
a $\epsilon' = \epsilon \sqrt{x}$, so that

$$
\frac{|x_n - x|}{\sqrt{x}} < \frac{\epsilon \sqrt{x}}{\sqrt{x}} = \epsilon'
$$
\\~\\


\textbf{2.3.2}

We know that $x_n \rightarrow 2$, so if we look at $ \frac{2x_n -1}{3} $, we can apply
the same operations to $|x_n \rightarrow 2|$,
$$
\left| \frac{2x_n -1}{3} - \frac{2(2) -1}{3} \right| = \left| \frac{2x_n -1}{3} - \frac{3}{3} \right|
$$
$$
= \frac{1}{3} |(2x_n - 1) - 3| = \frac{1}{3} | 2x_n - 4 | = \frac{2}{3} |x_n - 2| < \frac{2}{3} \epsilon
$$
\\~\\

Let's now look at $1/x_n$.
$$
\left| \frac{1}{x_n} - \frac{1}{2} \right| = \left| \frac{2 - x_n}{2x_n} \right| = 
\frac{| x_n - 2 |}{|2x_n|} = \frac{| x_n - 2 |}{2|x_n|}
$$

If we look at the denominator, we'll see a $|x_n|$ which we know is bounded, $|x_n| < M$, because
$x_n$ is a convergent series.
However, an upper bound in the denominator, gives us a lower bound on the overall expression, which
is not what we want.
We instead need to look for an inequality of the form $x_n \geq \delta > 0$.

Let's look at,
$$
|2| = |2 - x_n + n_x| \leq |x_n - 2| + |x_n| < \epsilon + |x_n|
$$
Which means $|x_n| > |2| - \epsilon$.
Similarly,
$$
|x_n| = |x_n - 2 + 2| \leq |x_n - 2| + |2| < \epsilon + |2|
$$
Which means that $|x_n| < \epsilon + |2|$.
Putting these two expressions together we get $|2| - \epsilon < |x_n| < |2| + \epsilon$.
The left side of it gives us a lower bound which can be greater than 0 if we pick a good value for $\epsilon$.

Back to our original problem,
$$
\frac{| x_n - 2 |}{2|x_n|} < \frac{| x_n - 2 |}{2(|2| - \epsilon)} = \frac{| x_n - 2 |}{2}
$$
If we chose $\epsilon = 1$.
\\~\\



\textbf{2.3.3}

\textbf{Squeeze theorem:} if $x_n \leq y_n \leq z_n$ for all $n\in\mathbb{N}$, and if
$\lim x_n = \lim z_n = l$, then $\lim y_n = l$ as well.

Let's assume that $\lim x_n = x$, $\lim y_n = y$, and $\lim z_n = z$ for now.
By the order limit theorem we have know that $x \leq y \leq z$.
Since $x = z = l$, then $y=l$.
\\~\\


\textbf{2.3.5}

If $x_n$ converges, then it means that $|z_n - l| < \epsilon$ $\forall \epsilon >0 \in \mathbb{R}$
when $n \geq N \in \mathbb{N}$.
So when $n \geq N$, all terms in $x_n$ must be "$\epsilon$ close to whatever happens to be $l$."

This implies that all terms after $n>N$ must be "close to $l$"
$$
| x_n - l | < \epsilon
$$
and
$$
| y_n - l | < \epsilon
$$

Thus, if for some reason $x_n$ and $y_n$ diverged or converged to different values, then the above conditions
would not be met, and thus $z_n$ would not converge.
\\~\\



\textbf{2.3.6}

Consider
$$
b_n = n - \sqrt{n^2 + 2n}
$$

Find $\lim b_n$ given $1/n \rightarrow 0$; the fact that when $x \geq 0$ if $x_n \rightarrow x$, then
$\sqrt x_n \rightarrow \sqrt x$; and the algebraic limit theorem.

$$
n - \sqrt{n^2 + 2n}
= n - \sqrt{n^2 +2n} \left( \frac{n + \sqrt{n^2 + 2n}}{n + \sqrt{n^2 + 2n}} \right) 
= \frac{n^2 - n^2 - 2n }{n + \sqrt{n^2 + 2n}}
$$
$$
= \frac{-2n}{n + \sqrt{n^2 + 2n}} 
= \frac{-2}{1 + \sqrt{1 + 2/n}}
$$

Now we can use the info given clearly,
$$
\lim b_n = \lim \left( \frac{-2}{1 + \sqrt{1 + 2/n}} \right)
= \frac{-2}{1 + \sqrt{1 + \lim{2/n} }}
= \frac{-2}{1+ \sqrt{1}}
= -1
$$
\\~\\


\textbf{2.3.11}

\textbf{Cesaro mean:} if $x_n \rightarrow x$, then does
$$
y_n = \frac{x_1 + x_2 + \ldots + x_n}{n} \rightarrow x
$$
?

If we think about $\epsilon$-neighborhoods, then we can envision that at some point all members of the
series hover extremely close around a given value, so $n$ elements of the same value divided by $n$
would be "the value".
In a more math way,

$$
| y_n - x | 
= \left| \frac{x_1 + x_2 + \ldots + x_n}{n} - x \right|
= \left| \frac{|x_1 -x| + |x_2-x| + \ldots + |x_n-x|}{n} \right|
$$

Since $x_n$ converges, we know that for any $n \geq N$, $|x_n - x| < \epsilon$.
So
$$
\left| \frac{|x_1 -x| + |x_2-x| + \ldots + |x_n-x|}{n} \right|
< \left| \frac{|x_1 -x| + |x_2-x| + \ldots + \epsilon + \ldots + \epsilon + \ldots}{n} \right|
$$

If we define $M = \max{|x_1 -x|, |x_2-x|, \ldots , |x_{n-1} - x|}$, then
$$
\left| \frac{|x_1 -x| + |x_2-x| + \ldots + |x_n-x|}{n} \right|
< \left| \frac{M + M + \ldots + \epsilon + \ldots + \epsilon + \ldots}{n} \right|
$$

If we chose an $n$ that's far out, we the above expression grows as $M/n$, which will tend to
zero ($\lim \frac{1}{n} = 0$).
\\~\\



\textbf{2.3.13}

\textbf{Iterated limits:} given a doubly indexed array $a_{mn}$ where $m, n \in \mathbb{N}$,
what should $\lim_{m,n \rightarrow \infty} a_{mn}$ represent?

Let $a_{mn} = m / (m+n)$ and compute the iterated limits
$$
\lim_{m\rightarrow\infty} \left( \lim_{n\rightarrow\infty} a_{mn} \right)
$$
and
$$
\lim_{n\rightarrow\infty} \left( \lim_{m\rightarrow\infty} a_{mn} \right)
$$

Define $\lim_{m,n\rightarrow\infty} a_{mn} = a$ to mean that for all $\epsilon > 0$
there exists an $N \in \mathbb{N}$ such that if both $m,n \geq N$, then $|a_{mn} - a|<\epsilon$.
\\~\\



%%%%%%%%%%%%%%%%%%%%%%%%%%%%%%%%%%%%
\subsection{The monotone convergence theorem and a first look at infinite series}

\subsubsection{Exercises}

\textbf{2.4.1} For thiese sorts of problems it helps to always plugin a couple numbers to see the patterns.
The sequence we are working with is
$$
x_{n+1} = \frac{1}{4 - x_n}
$$
where $x_1 = 3$.

Let's start by finding the first couple terms in the sequence,
$$
x_1 = 3
$$
$$
x_2 = \frac{1}{4 - 3} = \frac{1}{1} = 1
$$
$$
x_3 = \frac{1}{4 - 1} = \frac{1}{3} = 1/3
$$
$$
x_4 = \frac{1}{4 - 1/3} = \frac{1}{\frac{12}{3} - \frac{1}{3}} = \frac{1}{11/3} = 3/11
$$
$$
x_5 = \frac{1}{4 - 3/11} = \frac{1}{\frac{44}{11} - \frac{3}{11}} = \frac{1}{41/11} = 11/41
$$
$$
x_6 = \frac{1}{4 - 11/41} = \frac{1}{\frac{164}{41} - \frac{11}{41}} = \frac{1}{153/41} = 41/153
$$

We could even obtain the rest of the numbers in this sequence, given our initial conditions with
\begin{lstlisting}[language=Python]
def new_term(numerator, denominator):
    new_denominator = 4*denominator - numerator
    return denominator, new_denominator
\end{lstlisting}

So it seems like we have a case of the $x_n > x_{n+1}$, or a decreasing function.
Induction is demanded here and now since we already have the base case of $x_1 > x_2$.
For the rest of the numbers, let's go step by step...

If we assume that $x_n \geq x_{n+1}$, then $4 - x_n \leq 4 - x_{n+1}$, since we are subtracting a bigger number
on the left side.
However, smaller denominators make for greater rational numbers, so $\frac{1}{4-x_n} \geq \frac{1}{4 - x_{n+1}}$.
This expression holds up becuause of our initial conditions (what if $x_1 = 4.5$?).

Intuitvely, we can also follow the pattern we saw.
After $x_2$, the next couple terms in the sequence were numbers smaller than 1.
If the new terms get smaller, then their subscessors will get close to $1/4$, or so we think!

But the monotone convergence theorem can help us out here.
Since the sequence is decreasing and 3 can serve as our bound, then this sequence should converge.
\\

\textbf{This is cool so put attention...}
based on everything we have seen thus far $\lim x_n = \lim x_{n+1}$.
So if we take the limit of each side of the recurisve equation
to explicitly compute $\lim x_n$, then we get,

$$
\lim x_n = x = \lim x_{n+1}
$$
So
$$
\lim x_n = x
$$
and
$$
\lim x_n = \frac{1}{4- \lim x_n} = \frac{1}{4-x}
$$
Which leads us to
$$
x = \frac{1}{4-x} \rightarrow 4x - x^2 = 1
$$
Or $x^2 - 4x + 1 = 0$.
We can readily use the quadratic equation now to find our solution,
$$
x = \frac{ -(-4) \pm \sqrt{ 16 - 4 } }{2} = \frac{4 \pm \sqrt{12}}{2} = 2 \pm \sqrt{3}
$$
If $x = 2 + \sqrt{3}$ then $x>3$ and this contradicts everything that we have seen thus far about the sequence
being decreasing.
So $x = 2 - \sqrt{3}$ must be it (and this matches the answer we get from usign the code above)!
\\~\\


\textbf{2.4.2}
This time let's work with the recurisevely defined sequence
$$
y_{n+1} = 3 - y_n
$$
When $y_1 = 1$.

Plugin in a couple numbers,
$$
y_1 = 1
$$
$$
y_2 = 3 - y_1 = 3-1 = 2
$$
$$
y_3 = 3 - y_2 = 3 - 2 = 1
$$
$$
y_4 = 3 - y_3 = 3 - 1 = 2
$$
$$
y_5 = 3 - y_4 = 3 - 2 = 1
$$
$$
y_6 = 3 - y_5 = 3 - 1 = 2
$$
$$
y_7 = 3 - y_6 = 3 - 2 = 1
$$
So this time the series is not monotone but alternating.
Hence, we have no reason to believe that his converges, if anything it looks like it diverges (without doing any more work).
Because of this, it would not be sensible to follow the same procedure as in \textbf{2.4.1}
and apply the limit on both sides of the recurisve definition of this sequence, because
the limits do not exist.
\\

Now, if instead we were working with
$$
y_{n+1} = 3 - 1/y_n
$$
and $y_1 = 1$, then
$$
y_1 = 1
$$
$$
y_2 = 3 - 1/1 = 2
$$
$$
y_3 = 3 - 1/2 = 5/2
$$
$$
y_4 = 3 - 2/5 = \frac{15 - 2}{5} = 13/5
$$
$$
y_5 = 3 - 5/13 = \frac{39 - 5}{13} = 34/13
$$
$$
y_6 = 3 - 13/34 = \frac{102 - 13}{34} = 89/34 \approx 2.6176471
$$

We can see that $y_1 \leq y_2$, so maybe $y_n \leq y_{n+1}$?

As in problem \textbf{2.4.1}, we have the hypothesis that $y_n \leq y_{n+1}$,
That implies that $1/y_n \geq 1/y_{n+1}$, since 1 divided by a large denominator results in a small number.
Which in turns implies that $3 - 1/y_n \leq 3 - 1/y_{n+1}$, since 3 minus $1/y_{n+1}$ is 3 minus a smaller number.
Thus $y_{n+1} \leq y_{n+2}$.
We see that the series is monotonically increasing, but is it bounded (and thus convergent)?

What we do know is that $y_n > 0$, then $y_{n+1} = 3 - 1/y_n \leq 3$.
So 3 could be an upper bound as long as we are sure thant $y_n$ stays positive.

The way for $y_n < 0$ to happen would be for $3 - 1/y_n < 0$ or $3 < 1/y_n$, or $y_n < 1/3$.
Since we know that the sequence is increasing, as long as start with a number above $1/3$ we should be okay
and we should be bounded by 3.

Now we actually have some justification for attempting to calculate the limits of our
recursive formula.
But the algebra is not simple, so we wont.
\\~\\



\textbf{2.4.3}
does
$$
\sqrt{2}, \sqrt{2 + \sqrt{2}}, \sqrt{2 + \sqrt{2 + \sqrt{2}}}, \ldots
$$
converge?

Let's try following some of the similar steps as in the previous problems.

We can define the sequence as $x_1 = \sqrt{2}$ and
$$
x_{n+1} = \sqrt{2 + x_n}
$$

We can see that $x_1 = \sqrt{2} \leq x_2 = \sqrt{2 + \sqrt{2}}$, so it seems like the sequence would
be increasing as $n$ increases.
If $x_{n+1} \geq x_{n}$, then
$$
2 + x_{n+1} \geq 2 + x_n \rightarrow \sqrt{2 + x_{n+1}} \geq \sqrt{2 + x_n}
\rightarrow x_{n+2} \geq x_{n+1}
$$

(Remember problem 2.3.1? It comes in handy a lot!)
So the sequence does seem to be an increasing one.
Now let's see if we can argue that it is bounded.

Right now we know that $0 < x_1 = \sqrt{2} < 2$.
So what if we said that $x_n < 2$?
Well we could aso say that $2 + x_n < 4$, or that $\sqrt{2 + x_n} = x_{n+1} < 2$.
Which makes us believe that 2 is a bound for this sequence.

Given these two findings, we can assume that $\lim x_n$ exists and that $\lim x_n = \lim x_{n+1}$.
So if we apply the limit to both sides of our re-defined sequence we get
$$
\lim x_{n+1} = x = \lim \sqrt{2 + x_n} = \sqrt{2 + \lim x_n} = \sqrt{2 + x}
$$
$$
\rightarrow x = \sqrt{2 + x}
$$
A bit of algebra leads us to
$$
x = \sqrt{2 + x} \rightarrow x^2  = 2 + x \rightarrow
x^2 - x - 2 = 0
$$
And a little bit of the quadratic equation would land us in
$$
\frac{-(-1) \pm \sqrt{1 - 4 (-2)}}{2} = \frac{1 \pm \sqrt{1 + 8}}{2}
= \frac{1 \pm 3}{2}
$$
$x$ is either 2 or -1.
For our case, $x=2$ must be the one.
\\~\\

Now, what about the sequence
$$
\sqrt 2, \sqrt{2 \sqrt 2}, \sqrt{2 \sqrt{2 \sqrt 2}}, \ldots
$$

Let's go through this as we have been practicing...
$x_1 = \sqrt 2$, and
$$
x_{n+1} = \sqrt{2x_n}
$$

Since $\sqrt 2 > 1$, $x1 \leq x_2$, so again we are increasing.
Now let's say $x_{n+1} \geq x_n$, in which case
$$
x_{n+1} \geq x_n \rightarrow \sqrt{2x_{n+1}} \geq \sqrt{2x_n}
\rightarrow x_{n+2} \geq x_{n+1}
$$

Similarly, $x_1 = \sqrt 2 < 2$, so if $x_n < 2$, then $\sqrt{2x_n} =x_{n+1} < \sqrt{4} = 2$,
so again $x_n$ is bounded by 2.

Now that we got our excuse to apply the limits, we have
$$
x = \sqrt{2x} \rightarrow x^2 = 2x
$$
or $x = \pm 2$, and again we chose the positive value.
\\~\\



\phantomsection
\label{abbott:2.4.4}

\textbf{2.4.4}

\textbf{Show that the monotone convergence theorem can also be used to prove the archimedean property
without making use of the axiom of completeness.}

The archiemedean property states that: given any number $x\in\mathbb{R}$, there exists an
$n\in\mathbb{N}$ satisfying $n>x$.
And that given any real number $y>0$, there exists an $n$ satisfying $1/n < y$.

The trick to seeing this is to find two sequences that are convergent, because if they are convergent,
then they must have an upper bound, and thus the archiemedean property flows.

One way is to find some sequence we know and see how it the archimedean principle makes itself visible.
For example, we know that $\lim 1/n = 0$, so $|1/n - 0| = 1/n < \epsilon$, and if we set
$\epsilon$ to some real number $y$, then we have second statement of the archiemedean principle.

Another way, is to follow the exact same line of reasoning used in the book when proving it.
In the book we went with a proof by contradaction - we assumed that the natural numbers were bounded.
Since the natural numbers are monotonically increasing and we are assuming that they are bounded,
then the sequence of natural numbers must converge to some number.
Which is hard to argue.

Mathematically, we are saying that $\lim n = N$, so $\lim n+1 = N+1$, meaning that $N = N+1$.

There is a similar argument made here
\href{https://math.stackexchange.com/questions/90127/tfae-completeness-axiom-and-monotone-convergence-theorem}{TFAE: Completeness Axiom and Monotone Convergence Theorem}.
\\~\\


\textbf{Use the monotone convergence theorem to supply a proof for the nested interval property.}

The nested interval property tells us that for each $n\in\mathbb{N}$, if we are given a close interval
$I_n = [a_n, b_n] = \{ x\in\mathbb{R}: a_n \leq x \leq b_n \}$,
and if $I_{n+1} \subseteq I_n$, then
$\bigcap^{\infty}_{n=1} I_n \neq \emptyset$.

For this part, we can also follow the logic of the book.
The sequence of $a_n$ is increasing and bounded by any $b_n$, so it must converge to some value.
Similar argument can be made for the sequence $b_n$ but this one being bounded from below and decreasing.
Since $\lim a_n = a$ and $\lim b_n = b$ and $a_n \leq b_n$ for all $n$, then the order limit theorem
tells us that $a \leq b$.
From here can follow the argument in the book by looking at $a_n \leq a$ for a particular $I_n$.
In this same instance, $a<b_n$, so $a\in I_n$ for all $n$.
\\~\\



\textbf{2.4.5}
Let $x_1 = 2$ and
$$
x_{n+1} = \frac{1}{2}\left( x_n + \frac{2}{x_n} \right)
$$

To show that $\forall n$, $x^{2}_{n} \geq 2$, we can use induction.
$x^{2}_{1} = 4 \geq 2$.
To see the rest of the argument,
$$
x^{2}_{n+1} = \frac{1}{4}x^{2}_{n} + \frac{1}{x^{2}_{n}} + 1
$$
Hence, if $x^{2}_{n} \geq 2$, then $x^{2}_{n+1} \geq 2$.
\\

Now that we have an argument made for $x^{2}_{n} \geq 2$ for all $n$, let's see if we can prove that
$x_n - x_{n+1} \geq 0$.

One way of seeing this is by exploring whether the sequence is decreasing.
We know that $x_1 = 2 > x_2 = 3/2$.
Now let's poke around and see what we can find about $x_n$ in general.

$$
\frac{1}{2}\left(x_n + \frac{2}{x_n}\right) = \frac{1}{2}\left(\frac{x^{2}_{n} +2}{x^{2}_{n}}\right) 
\leq \frac{1}{2} (x^{2}_{n} + 2) \geq 1
$$
So its a tad odd to carry on with an induction argument given the expression we just made.

So let's try another way,
$$
x_n - x_{n+1} = x_n - \frac{1}{2}x_n - \frac{1}{x_n} = \frac{1}{2}x_n - \frac{1}{x_n}
$$

If $x_n - x_{n+1} \geq 0$, then
$$
\frac{1}{2}x_n - \frac{1}{x_n} \geq 0
\rightarrow \frac{1}{2}x_n \geq \frac{1}{x_n}
\rightarrow x^{2}_{n} \geq 2
$$
Which matches what we had above.

If instead we had assumed that $x_n - x_{n+1} < 0$, then we would have ended up with $x^{2}_{n} < 2$.
\\

Since we have showned that the sequence is decreasing and bounded, then we know that it converges, and we
can apply the limit to both sides of our definition.
This will land us with $\lim x_n = \pm 2$, but only $+2$ is a valid limit for us.
\\~\\



\textbf{2.4.6}

\textbf{Explain why the geometric mean is always less than the arithmetic mean:} why is
$\sqrt{xy} \leq (x+y)/2$, for any two positive real numbers $x$ and $y$.

To see why, let's square both sides
$$
\sqrt{xy} \leq \frac{x+y}{2} \rightarrow xy \leq \frac{1}{4} (x^2 + y^2 + 2xy)
\rightarrow 4xy \leq x^2 + y^2 + 2xy
$$
The last expression can be re-arranged to
$$
0 \leq x^2 + y^2 - 2xy = (x - y)^2
$$

Which helps us see that only in the case in which $x=y$, will the geometric and the arithmetic
means be equal, otherwise, the arithmetic mean will always be greater.
\\

If we have $0 \leq x_1 \leq y_1$ and define
$$
x_{n+1} = \sqrt{x_n y_n}
$$
and
$$
y_{n+1} = \frac{x_n + y_n}{2}
$$

and we want to see whether $\lim x_n$ and $\lim y_n$ esit, then the simplest thing is to investigate
whether the sequences are monotone and bounded.

We already saw that the geometric mean is always less than the arithmetic mean, so $x_n \leq y_n$
for all $n\in\mathbb{N}$.
Thus we can carry our initial condition as a general case and assume that $x_n \leq y_n$ throughout.

If we now look at $x_{n+1}$ we see that if it was that if $x_{n+1} = \sqrt{x_n x_n}$.
However, $y_n$ is greater than or equal to $x_n$ so $x_{n+1} \geq x_n$.

As for $y_n$, it helps if we rewrite it as $y_{n+1} = \frac{1}{2}x_n + \frac{1}{2}y_n$.
Again, $y_{n+1} = y_n$ if $x_n = y_n$, but since $x_n$ is equal to or smaller, then $y_{n+1} \leq y_{n}$.

So we see that $x_n$ is increasing, while $y_n$ is decreasing.
We also see that if $x_n = y_n$ we have some sort of equilibrium where $x_{n+1} = x_n$ and
$y_{n+1} = y_n$.
Which shows us that $\lim x_n = \lim y_n$ when that is the case.
\\~\\



\textbf{2.4.7}

\textbf{Limit superior} let $(a_n)$ be a bounded sequence.

Prove that the sequence defined by $y_n = \sup\{a_k : k\geq n\}$ converges.

Let's see if the monotone convergence theorem can help us out here too.
The sequence $y_n$ essentially looks at all the elements of the sequence $a_n$ that come after
a given $k$.
So for a given $y_n$ we have a corresponding least-upper bound from the subsequence $\{a_k : k\geq n\}$.
$y_1$ will then be the supremum of the entire sequence $a_n$.
$y_2$ will be the supremum if $a_n$ did not have its first element $a_1$, and so on and so forth.
Since $y_1$ is the supremum for the entire sequence, all other spremums from sub-sequences will be less
than or equal to it.
Similar argument can be made for $y_2$, any of the following supremums will be smaller than or equal to it.
So we can already see that the sequence is decreasing and bounded by $y_1 = \sup a_n = \alpha$.
\\

The \textbf{limit superior} of $a_n$ is defined as follows:
$$
\lim \sup a_n = \lim y_n
$$
Where $y_n$ is the sequence we defined above, and we also made an argument for its existance.

Now what about $\lim \inf a_n$?
Given our definition of $y_n$, we could define $z_n = \inf\{a_k : k\geq n\}$.
In such an instance, $z_1$ would be the infinimum of the entire sequence.
Similarly, $z_2$ would be the infinimum of $a_n$ without its first element, so it could be the same
(if the sequence is decreasing), it could be of greater value (if the sequence is increasing),
or it could be equal (if the first two values are similar).

Since our definition starts with the entire sequence and then gets rid of the elements at the begining,
we will not be finding any infinimum value that is smaller than any of the $z_n$ that have been computed before.
Thus we can see that $z_n$ is increasing and bounded from below by $z_1 = = \inf a_n = \beta$.

However, $a_n$ is itself bounded, so this definition, although it results in an increasing sequence
and the infinimum is a lower bound, it also converges and thus $\lim \inf a_n$ should exist.
\\

From experience, definition, whatever you want to call it, $\inf a_n \leq \sup a_n$.
And it would make sense to think that $\lim \inf a_n \leq \lim \sup a_n$.
And thus the oreder limit theorem makes this make sense.
\\

Now, if if $\lim \inf a_n = \lim \sup a_n$, thenthe squeeze theorem can get us $\lim a_n$ as it would
be the same value as the limit superior and the limit inferior.
And if we know that $\lim a_n$ exists, then Intuitvely we can imagine that after some point, all elements
of $a_n$ never leave some $\epsilon$-neighborhood (all values get arbitrarily close to the value they converge to).
In such an intansce, the supremum and infinimum would be equal to one another and the order limit theorem
would again come in handy to see that all three limits must equal each other.
\\~\\



\textbf{2.4.8}

Find an explicit formula for the sequence of partial sums and determine if the following series converge:

$$
\sum^{\infty}_{n=1} \frac{1}{2^n}
$$

Writing out the terms, we get
$$
\sum^{\infty}_{n=1} \frac{1}{2^n} = 1/2 + 1/4 + 1/8 + \ldots
$$

This is a geometric series where the common term is $1/2$.
The explicit formula for geometric series is
$$
\sum^{\infty}_{n=0} r^n = \frac{1}{1-r}
$$
But since we are starting at $n=1$, we need to subtract the first term.
Thus
$$
\sum^{\infty}_{n=1} \frac{1}{2^n}
= -1 + \sum^{\infty}_{n=1} r^n
= -1 + \frac{1}{1-\frac{1}{2}} = 1
$$
\\

In the case of
$$
\sum^{\infty}_{n=1} \frac{1}{n(n+1)}
$$
This is the wikipedia example of a telescoping series.

For this series,
$$
\sum^{\infty}_{n=1} \frac{1}{n(n+1)} = \lim_{N\rightarrow\infty} \left(1 - \frac{1}{N+1}\right) = 1
$$

And finally,
$$
\sum^{\infty}_{n=1} \log\left(\frac{n+1}{n}\right)
$$

Is a telescoping series of sorts since $\log \frac{n+1}{n} = \log (n+1) - \log n$.
So we have
$$
\sum^{\infty}_{n=1} \log\left(\frac{n+1}{n}\right)
= (\log 2 - \log 1) + (log 3 - \log 2) + (\log 4 - \log 3) + (\log 5 - \log 4) + (\log 6 - \log 5) + \ldots
$$
$$
= -\log 1 + \lim \log N
$$
Which doesn't seem like it converges, since $\log{x}$ is an increasing function.
\\~\\



\textbf{2.4.9}

Show that if $\sum^{\infty}_{n=0} 2^n b_{2^n}$ diverges, then so does $\sum^{\infty}_{n=1} b_n$.

Abbott is trully a great teacher.
The beauty of this proof starts with the fact that we are proving the contrapositive of the
cauchy condensation test.
The following argument is completely taken from
\href{https://math.stackexchange.com/questions/1736699/proof-of-cauchy-condensation-test-using-contrapositive}{Proof of Cauchy Condensation Test using contrapositive}.
(I thought about it, and thought about it, and kept on thinking about it and never got anywhere.)

Since the series $b_n$ is decreasing, we were able to define
$$
s_{2^{k+1}-1} =
b_1 + (b_2 + b_3) + (b_4 + b_5 + b_6 + b_7) + \ldots + (b_{2^k} + \ldots + b_{2^{k+1}-1})
$$
$$
\leq b_1 + (b_2 + b_2) + (b_4 + b_4 + b_4 + b_4) + \ldots + (b_{2^k} + \ldots + b_{2^{k}})
$$
$$
= b_1 + 2(b_2) + 2^2(b_4) + \ldots + 2^k (b_{2^k})
= t_k
$$

One of the tricks used above is that $b_n \geq b_{n+1}$.
If we went the other route, if replaced some of the terms in our sequence with the smallest terms
within their bracketed groups then we could obtain a sequence that is smaller.

$$
s_{2^{k+1}-1} =
b_1 + (b_2 + b_3) + (b_4 + b_5 + b_6 + b_7) + \ldots + (b_{2^k} + \ldots + b_{2^{k+1}-1})
$$
$$
\geq b_1 + (b_3 + b_3) + (b_7 + b_7 + b_7 + b_7) + \ldots + (b_{2^{k+1}-1} + \ldots + b_{2^{k+1}-1})
$$

And we really recommend that you write out the terms and see the patterns but take a look at the end.
$s_{2^{k+1}-1}$ ends in $(b_{2^k} + \ldots + b_{2^{k+1}-1})$, it goes all the way to the index term in $s$
and the bracketing starts at the previous $2^{k}$.

Then, we can see how
$$
s_{2^k} = b_1 + b_2 + b_3 + \ldots
= b_1 + (b_2 + b_3) + \ldots + (b_{2^{k-1}+1} + \ldots + b_{2^k})
$$

We should also note that the factor infront of the last bracketing is the $2^n$ that happens to be closest.
Hence,
$$
s_{2^k}
= b_1 + (b_2 + b_3) + \ldots + (b_{2^{k-1}+1} + \ldots + b_{2^k})
$$
$$
\geq b_1 + 2 (b_3) + 2^2 (b_7) + \ldots + 2^{k-1} (b_{2^{k}})
$$

and since sequences are infinite, we could change the offset as such
$$
s_{2^k}
= b_1 + b_2 + (b_3 + b_4) + (b_5 + b_6 + b_7 + b_8) + \ldots + (b_{2^{k-1}+1} + \ldots + b_{2^k})
$$
$$
\geq b_1 + b_2 + 2 (b_4) + 2^2 (b_8) + \ldots + 2^{k-1} (b_{2^{k}})
$$
$$
= b_1 + t_{k-1}
$$

And so we get our proof.
\\~\\

However, a peek at wikipedia reveals another very great trick.
We know that
$$
t_k = b_1 + 2b_2 + 4b_4 + \ldots
$$

This could be bracketed also as
$$
t_k = (b_1 + b_2) + (b_2 + b_4 + b_4 + b_4) + (b_4 + b_8 + \ldots) + \ldots
$$
And since the series is decreasing,
$$
t_k = (b_1 + b_2) + (b_2 + b_4 + b_4 + b_4) + (b_4 + b_8 + \ldots) + \ldots
$$
$$
\leq (b_1 + b_1) + (b_2 + b_2 + b_3 + b_3) + (b_4 + b_4 + \ldots) + \ldots
$$
$$
= 2\sum^{\infty}_{n=1} b_n
$$

Hence the cauchy condensation test can be expanded to
$$
\sum^{\infty}_{n=1} b_n \leq \sum^{\infty}_{n=0} 2^n b_n \leq 2\sum^{\infty}_{n=1} b_n
$$
\\~\\



\phantomsection
\label{abbott:2.4.10}

\textbf{2.4.10}

The infinite product
$$
\prod^{\infty}_{n=1} b_n = b_1 b_2 b_3 \ldots
$$
can be understood in terms of its sequence of partial products (like infinite series and partial sums)
$$
p_m = \prod^{m}_{n=1} b_n = b_1 b_2 \ldots b_m
$$

We will focus here on the special class of infinite products that looks as such
$$
\prod^{\infty}_{n=1} (1 + a_n) = (1+a_1)(1+a_2)(1 +a_3)\ldots
$$
Where $a_n \geq 0$.
\\

If $a_n = 1/n$ we get this interesting looking partial product
$$
p_m = \left(1+1\right) \left(1+\frac{1}{2}\right) \left(1+\frac{1}{3}\right) \left(1+\frac{1}{4}\right) \left(1+\frac{1}{2}\right) \ldots \left(1+\frac{1}{m}\right)  
$$
$$
= \left(\frac{2}{1}\right) \left(\frac{3}{2}\right) \left(\frac{4}{3}\right) \left(\frac{5}{4}\right) \ldots \left(\frac{m+1}{m}\right) 
$$
$$
= \frac{m+1}{1} = m+1
$$
However, since $m \rightarrow \infty$ in the infinite product, we could think that this case diverges.
\\

Now, when $a_n = 1/n^2$, we get
$$
p_m =
\left(1+ \frac{1}{1} \right) \left(1+ \frac{1}{2^2} \right) \left(1+ \frac{1}{3^2} \right) \left(1+ \frac{1}{4^2} \right) \left(1+ \frac{1}{5^2} \right) \ldots \left(1+ \frac{1}{m^2} \right)
$$
$$
= \left(\frac{2}{1} \right) \left(\frac{2^2 +1}{2^2} \right) \left(\frac{3^2 +1}{3^2} \right) \left(\frac{4^2 +1}{4^2} \right) \left(\frac{5^2 +1}{5^2} \right) \ldots \left(\frac{m^2 +1}{m^2} \right)
$$
Here the first term is 2 but as $m$ grows, the terms begin to get closer and closer to 1.
\\


Can we show that the sequence of partial products converges if and only if
$\sum^{\infty}_{n=1} a_n$ converges?

Abbott gave us the trick to this, if we use the identity $1+x \leq e^x$, then we can rewrite the
partial product as follows
$$
p_m = (1+a_1) (1+a_2) + \ldots (1+a_m)
\leq e^{a_1} e^{a_1} \ldots e^{a_m}
= e^{s_m}
$$

Hence, if $\sum^{\infty}_{n=1} a_n \rightarrow a$, then $e^{s_m}$ is a number, and thus we get our
proof - the series is increasing and bounded.

To prove the other route (if the infinite product converges then the series converges),
it sufices to show that $p_m \geq s_m$. (Expand the product and you will see terms correspodning to
the partial sum plus other additional terms.)
\\~\\



%%%%%%%%%%%%%%%%%%%%%%%%%%%%%%%%%%%%%%%%%%%%%%%%%%%%%%%%%%%%%%%%%%%%%%%%%%%%%%%%%%%%%%%%%
\subsection{subsequences and the Bolzano-Weierstrass theorem}


\subsubsection{Exercises}

\textbf{2.5.3}

Assume $a_1 + a_2 + 1_3 + \ldots$ converges to a limit L
(i.e., the sequence of partial sums $(s_n) \rightarrow L$).
Show that any \textbf{regrouping} of the terms
$$
\left(a_1 + a_2 + \ldots + a_{n_1}\right) +
\left(a_{n_1 +1} + a_{n_1 +2} + \ldots + a_{n_2}\right) +
\left(a_{n_2 +1} + a_{n_2 +2} + \ldots + a_{n_3}\right) +
\ldots
$$
leads to a series that converges to L.
\\

A sequence of partial partial sums $s_n \rightarrow L$.
A regrouping of terms ends up creating a subsequence of $s_n$.
And since $s_n$ converges, the any of its subsequences converge as well.
\\~\\


\phantomsection
\label{abbott:2.5.4}

\textbf{2.5.4}

Assume the nested property is true and use it to provide a proof for the axiom of completeness.
\\

Some brief reminders:
\begin{itemize}
    \item Axiom of completeness: every non-empty set of real numbers that is bounded above has a least upper bound.
    \item nexted property: assume we are given a close interval $I_n$ and that $I_{n+1} \subseteq I_n$, then $\cap I_n \neq \emptyset$.
\end{itemize}

When proving the nested interval property, we laid out all $I_n = [a_n, b_n]$ on the real number line.
Since all intervals were nested, then we had an order for the set of $A = \{a_n : \in\mathbb{N}\}$
and $B = \{b_n : n\in\mathbb{N}\}$, where $a_1$ and $b_1$ were at the values fartest apart.
Then we saw that all $b\in B$ served as upper bounds, and thus a suppremum $x$ must exist such that
$a_n \leq x \leq b_n$.
And since this inequality held for all intervals (as they were all nested and equal to or smaller than their predecessor)
then we had a proof that there was an element that was part of all the intervals.
\\

When proving the Bolzano-Weierstrass theorem, we also built nested intervals with the same properties
as the ones we saw in the nested interval theorem.
So if we assume that the nested interval theorem is true, then right away we can see that
from the bisecting of our Bolzano-Weierstrass prove that there must be at least one element that is
part of all the intervals.

If we start with the nested interval property then we also get an upper bound for our sequence ($b_n$).

The only thing left to get the axiom of completeness from the use of the nested interval property on the
Bolzano-Weierstrass theorem is to see that as we continue forming subintervals, each of of length
$M(1/2)^{k-1}$, then as $k\rightarrow\infty$, the length will tend to zero (example 2.5.3).
And since the length of the intervals tends to zero then the element $x$ that we found above, can indeed
be the least upper bound, since we can make the lenght of the interval as small as we want ($\epsilon$)
such that $x-\epsilon$ is no longer an upper bound, $x-\epsilon \leq a_n$.
\\

The reason we are to assume that $1/2^{k-1}$ is a convergent sequence is because the lenght of
the subinterval $I_k$ is supposed to converge to 0, and using the normal $\epsilon$ convention we want
to prove that something like $1/n < \epsilon$ is true for any $\epsilon > 0$.
$\epsilon$ is a real number and so we are implicitly making use of the Archimedean property, which
we originally proved using the Axion of Completeness, which we are trying to prove here.
\\~\\


\textbf{2.5.6}

Show that $\lim b^{1/n}$ exists for all $b\geq 0$ and find the value of the limit.
\\

The series $b^{1/n}$ is decreasing, since $\frac{1}{n} > \frac{1}{b+1}$.
Since it is monotonically decreasing and abounded by $b$, then it must converge.
Go again, if $\lim b^{1/n} = l$, then $0 \leq l \leq b$.

Since $b^{1/n} \rightarrow l$, then so must $b^{1/(n+1)} \rightarrow$.
If we take the limit of both and equate then, we get
$$
b^{1/n} = b^{1/(n+1)} \rightarrow b = b^{\frac{n}{n+1}} = b^{1 + 1/n} = bb^{1/n}
$$
Hence
$$
b^{1/n} = 1
$$

If $b=0$, $\lim b^{1/n} = 0$. Recall that one problem where we proved that $\lim a_n = 0$,
then $\lim \sqrt{a_n} = 0$.
\\~\\



\phantomsection
\label{abbott:2.5.7}

\textbf{2.5.7}

When $|b| < 1$, then $b^n$ is decreasing and abounded by 1, hence convergent.
We follow similar steps as in the previous problem, then $b^n = b^{n+1} = l$.
And the only numbers that stay the same no matter how many powers you raise them to are
0 and 1. But since $|b| < 1$, the $|b^n| < 1$, so $\lim b^n = 0$.
\\~\\



\textbf{2.5.8}

Another way we could prove the Bolzano-Weierstrass theorem is by finding monotone subsequences
in a bounded sequence.
That way the monotone convergence theorem can be used to prove convergence.

If we can find a series of peak terms within our sequence, then we can build a monotone subsequence
from the sequence of peak terms.
Note that the peak points would define a decreasing subsequence.

However, if there is a finite number (or zero) peak points, then instead of peak points, we can look
for "nadir points" and instead define a monotonically increasing subsequence.
\\~\\



\textbf{2.5.9}

\textbf{Direct proof of Bolzano-Weierstrass theorem using the Axiom of Completeness}

Let $(a_n)$ be a bounded sequence, and define the set
$$
S = \{ x\in\mathbb{R} : x < a_n \text{ for inifinetely many terms in } a_n \}
$$

Show that there exists a subsequence $(a_{n_k})$ converging to $s = \sup S$.
\\

Since $(a_n)$ is bounded, then S will also be bounded.
Thus there must exist $s = \sup S$.
If we chose a subsequence $(a_{n_k})$ that is monotonically increasing, whose values go up to the bound
of $(a_n)$, the by the monotone convergence theorem, this subsequence will converge.
This subsequence should be such that $|a_{n_k} - s| < \epsilon$ for any $\epsilon > 0$
and any $k > N$.
Which is possible because the parent sequence converges and we are restricting ourselves to a mnonotone
increasing sequence.
\\~\\



%%%%%%%%%%%%%%%%%%%%%%%%%%%%%%%%%%%%%%%%%%%%%%%%%%%%%%%%%%%%%%%%%%%%%%
\subsection{The Cauchy Criterion}

In \textbf{lemma 2.6.3, cauchy sequences are bounded}, there is a bit of logic that was used back in
\textbf{theorem 2.3.2 every convergent sequence is bounded},
and also when de discussed the triangle inequality.

When we say that $|x_n - x| < \epsilon$, this can also be seen as $\epsilon < x_n - x < \epsilon$.
Thus, $|x_n - x_m| < 1$, could be seen as $-1 < x_n - x_m < 1$ or $-(x_m + 1) < x_n < x_m + 1$.


\subsubsection{Exercises}

\textbf{2.6.1}

\textbf{Every convergent sequence is a Cauchy sequence}

If we have a sequence $(x_n)$ such that $\lim x_n = x$, then
$$
|x_n - x| < \epsilon
$$
Whenever $n \geq N \in \mathbb{N}$.

Thus, if we look at a cauchy sequence, with $n, m \geq N$,
$$
|x_n - x_m| = |x_n - x + x - x_m| \leq |x_n - x| + |x - x_m| = |x_n - x| + |x_m - x| < \epsilon_1 + \epsilon_2
$$

We could have gone the other way around and started with,
$$
|x_n - x| \leq |x_n - x_m| + |x_m - x| < |x_n - x_m| + \epsilon
$$

And from there we would be back to the first use of the triangle inequality.

\textbf{Note:} it could be easy to make an argument about making $x_m = x$ and then throwing in another
epsilon. However, we would be fooling ourselves with the fake simplicity (why didn't we use that same
argument instead of using the triangle inequality above?).
To see why the fake simplicity is not correct, see the rest of contents in Abbott section 2.6.
Where Abbott instead went and used the Bolzano-Weierstrass theorem.

Also, it is possible that the limit is not actually a memeber of the sequence, think of $\lim 1/n$.
\\~\\



\textbf{2.6.3}

If $(x_n)$ and $(y_n)$ are cauchy sequences, then one easy way to prove that $(x_n + y_n)$ is
to use the cauchy criterion.
The cauchy criterion states that $(x_n)$ and $(y_n)$ must be convergent,
and the algebraic limit theorem then implies $(x_n + y_n)$ is convergent and hence cauchy.

Remember that the cauchy criterion says that a sequence converges if and only if it is a cauchy sequence.
\\

Now try giving a direct argument that $(x_n + y_n)$ is a cauchy sequence that does not use the
cauchy criterion or the algebraic limit theorem.
\\

We know that
$$
| x_n - x | < \epsilon \Longleftrightarrow |x_n - x_m| < \epsilon
$$
for all $\epsilon > 0$ when $n\geq N_1$.
We also know that
$$
| y_n - x | < \epsilon \Longleftrightarrow |y_n - y_m| < \epsilon
$$
for all $\epsilon > 0$ when $n\geq N_2$.

To make a direct argument, we need to prove that
$$
|(x_n + y_n) - (x_m + y_m)| < \epsilon
$$
for all $\epsilon >0$ when $n \geq \max{N_1, N_2}$.

To do so, note that
$$
|(x_n + y_n) - (x_m + y_m)| 
= |(x_n - x_m) + (y_n - y_m)|
$$
$$
\leq |x_n - x_m| + |y_n - y_m|
< \epsilon + \epsilon
$$
\\

Give a direct argument for $(x_n y_n)$ now.
\\

We now want to show that
$$
|(x_n y_n) - (x_m y_m)| < \epsilon
$$

Let's use a the same trick Abbott employed back when we were proving the algebraic
limit theorem.

\begin{align*}
    |(x_n y_n) - (x_m y_m)| &= |x_n y_n - x_m y_n + x_m y_n - x_m y_n|      \\
    &\leq |x_n y_n - x_m y_n| + |x_m y_n - x_m y_m|     \\
    &= |y_n| |x_n - x_m| + |x_m| |y_n - y_m|    \\
    &< M_2 \frac{\epsilon}{M_2} + M_1 \frac{\epsilon}{M_1} \\
\end{align*}

In the last step we used the fact that convergent sequences are bounded and defined
$M_1$ as the upper bound for $x_n$ and $M_2$ as the upper bound for $y_n$.
\\~\\



\textbf{2.6.7}

Exercises \ref{abbott:2.4.4} (use MCT to prove AoC) and
\ref{abbott:2.5.4} (use NIP, assume AP, use BW methodology to prove AoC) establish the equivalence of the axiom of
completeness and the monotone convergence theorem.
They also show the nested interval property is equivalent to these two when the archimedean property
is pressumed to be true.
\\

Before we carry on, we want to note Rudin's statement of the \textbf{Archimedean property (and the density
of the rational numbers within the reals)}:
If $x,y\in\mathbb{R}$ and $x>0$, then there is a $n>0 \in \mathbb{N}$ such that
$$
nx > y
$$
Abbott states this as $n > x$ for any $x\in\mathbb{R}$.
\\

If $x,y\in\mathbb{R}$ and $x<y$, then there is a $p\in\mathbb{Q}$ such that
$$
x < p < y
$$

Abbot states this as $1/n < y$ for any $y>0 \in\mathbb{R}$ and $n\in\mathbb{N}$.
\\



\textbf{Assume the Bolzano-Weierstrass theorem to be true and use it to construct a proof of the
monotone convergence theorem} without making any appeal to the archimedean property.
This shows that BW, AoC, and MCT are equivalent.
\\

By the BW theorem, we know that a bounded sequence has a convergent subsequence.
The subsequence generated by BW meets the condition that the elements we select
must be so $n_k > n_{k-1} > \ldots > n_1$ so that $a_{n_k}\in I_k$.
Which means that we are not going around picking elements as we want, we pick elements in the order that
they show up in the original sequence.

Ultimately, we know that if BW applies then $(a_n)$ is bounded and that we can build a subsequence
$(a_{n_k}) \rightarrow x$.

For the MCT, the sequence must also be bounded and all elements must be monotonically increasing or decreasing.
Let's pick the case in which the elements are monotonically increasing and our sequence is bounded above.

So
$$
|a_{n_k} - x| < \epsilon
$$
When $k \geq N$, and $a_{n_{k+1}} \geq a_{n_k}$.

Thus,
$$
- (x+\epsilon) < a_{n_k} < x+\epsilon
$$
or
$$
- (x+\epsilon) < a_{n_k} \leq a_{n_{k+1}} < x+\epsilon
$$

In this last expression, the added term works because the sequence is monotonically increasing,
but we know that for any $k\geq N$, $|a_{n_k} - x| < \epsilon$ must hold.
This last expression can also be rewritten as
$$
|a_{n_{k+1}} - x| < \epsilon
$$
And thus our prove.
\\~\\



\textbf{Use the Cauchy criterion to prove BW}, and find the point in the argument where the
Archimedean property is implicitly required.
This establishes the final link in the equivalence of the five characterizations of completeness.
\\

The argument for the BW theorem that Abbott shows us is rather generic.
One thing we can do is to note when it makes use of the NIP.
Originally this was to find what the limit for the subsequence could be.
If we instead assumed the Cauchy criterion, then we'd know that the subsequence is bounded
and that after some point, the elements of the sequence get close to each other.

Then the implicit use of thw Archimedean property comes from the assumption that $M(1/2)^{k-1}$
converges.
($M(1/2)^{k-1}$ is a rational number that we want to make smaller tthan some arbitrary real $\epsilon$.)
\\~\\

How do we know it is impossible to prove the AoC starting from the Archimedean property?
\\

The Archimedean property tells us that we can always find an $n\in\mathbb{N}$
that is greater than some real number.
However, it tells us about upper bounds, it doesn't tell us about the least-upper bounds
(or greatest upper bounds).
So the Archimedean property leaves us with possible holes in our number line.

Also, the first part of the Archimedean property, $n>x$, is true if $x\in\mathbb{Q}$.
So it doesn't give us any clues about us needing the reals.
\\~\\



%%%%%%%%%%%%%%%%%%%%%%%%%%%%%%%%%%%%%%%%%%%%%%%%%%%%%%%%%%%%%%%%%%%%%%%%%%%%%%%%%%%%%
\subsection{Properties of infinite Series}

\textbf{Cauchy Criterion for Series}

Since $n > m \geq N$,
$$
s_m = a_1 + a_2 + \ldots + a_N + \ldots + a_m
$$
and
$$
s_n = a_1 + a_2 + \ldots + a_N + \ldots + a_m + \ldots + a_n
$$

So
$$
|s_n - s_m| = 
\left| (\cancel{a_1} + \cancel{a_2} + \ldots + \cancel{a_N} + \ldots + \cancel{a_m} + \ldots + a_n) - 
(\cancel{a_1} + \cancel{a_2} + \ldots + \cancel{a_N} + \ldots + \cancel{a_m})\right|
$$
$$
= \left| a_{m+1} + a_{m+2} + \ldots + a_n \right|
$$



\subsubsection{Exercises}


\textbf{2.7.1}

The \textbf{alternating series tests} goes as follows:
let $(a_n)$ be a sequence satisfying
\begin{enumerate}
    \item $a_1 \geq a_2 \geq \ldots a_n \geq a_{n+1} \geq \ldots$
    \item $(a_n) \rightarrow 0$
\end{enumerate}

Then the alternating series $\sum^{\infty}_{n=1} (-1)^{n+1} a_n$ converges.

Proving the alternating series tests ammounts to showing that the sequence of partial sums
$$
s_n = a_1 - a_2 + a_3 - a_4 + \ldots \pm a_n
$$
converges.

\textbf{Different characterizations of completeness lead to different proofs.}
\\

%%%%%%%%%%%%%%%
\textbf{Prove it by showing that $(s_n)$ is a Cauchy sequence.}
\\

We have
$$
s_n = a_1 - a_2 + a_3 - a_4 + \ldots \pm a_n
$$
So if we chose an $n > m \geq N$
$$
| s_n - s_m | = |(a_1 - a_2 + a_3 - a_4 + \ldots \pm a_n) - (a_1 - a_2 + a_3 - a_4 + \ldots \pm a_m)|
$$
$$
= | \pm a_{m+1} \mp a_{m+2} \pm \ldots \pm a_n |
$$

In the special case in which $n = m + 1$, we get
$$
| s_n - s_m | = | a_n | < \epsilon
$$

The appended inequality is possible because $(a_n) \rightarrow 0$, so we can meet any $\epsilon$
as long as we move $N$ farther out.

From there, we can see how we could chose the terms so that
$$
| s_n - s_m | = |a_{m+1} - a_{m+2}| + \ldots + |a_{n-1} - a_n|
$$
\\~\\


%%%%%%%%%%%%%
\textbf{Prove it using the NIP}
\\

Let's play around with the sample alternating sequence
$$
10 -9 +8 -7 +6 -5 +4 -3 +2 -1 +0
$$

\begin{center}
\begin{tabular}{ |c|c| }
    \hline
    $s_1$ & 10       \\ 
    \hline
    $s_2$ & 10-9 = 1 \\  
    \hline
    $s_3$ & 1+8 = 9  \\
    \hline
    $s_4$ & 9-7 = 2  \\
    \hline
    $s_5$ & 2+6 = 8  \\
    \hline
    $s_6$ & 8-5 = 3  \\
    \hline
    $s_7$ & 3+4 = 7  \\
    \hline
    $s_8$ & 7-3 = 4  \\
    \hline
    $s_9$ & 4+2 = 6  \\
    \hline
    $s_{10}$ & 6-1 = 5 \\
    \hline
\end{tabular}
\end{center}

Using this as example, we can see how we could build intervals $I_n$ such that $I_{n+1} \subseteq I_n$
by chosing $I_n = [s_{2n}, s_{2n+1}]$.
That way we can guarantee that there will be an $x\in\cap I_n$.
Also, since the length of $I_n$ will become smaller and smaller, we can then have
$|s_{2n} - x| < \epsilon$ or $|s_{2n+1} - x| < \epsilon$.
\\~\\


\textbf{Consider the sequences $(s_{2n})$ and $(s_{2n+1})$, and show that the MCT leads to a third proof.}
\\

Another good way to llok at the alternating series is as follows:
$$
s_n = (a_1 - a_2) + (a_3 - a_4) + \ldots \pm a_n
$$

Each group consists of terms that are great than or equal to zero, give our preconditions.
Each grouping resulting in the sum of some non-negative number that get's smaller and smaller.

We can also read it as:
$$
s_n = a_1 - (a_2 - a_3) - (a_4 - a_5) - \ldots
$$

In this case, $s_1$ is an upper bound and we continuously subtract smaller and smaller numbers.

Comparing these two findings, we can see that the sequence is monotonically increasing and that its
upper bound is $s_1$, so MCT says it must converge.
\\

Now, let's turn to $s_{2n}$, which can equal $s_2, s_4, s_6$ etc.,
and $s_{2n+1}$, which can equal $s_3, s_5, s_7$, etc.

If you write out a couple terms you will see how Abbott was trying to help us see what we describe above
and it offers us a slightly different argument.

$s_{2n+1}$, when we count from $n=0$, and if we assume that $a_n \geq 0$ for all n,
then we have a bounded and monotonically decreasing sequence - what we mentioned above.
We can also see that $s_{2n+1} = s_{2n} + a_{2n+1}$, or $s_{2n} = s_{2n+1} - a_{2n+1}$.
So we can do this,
$$
|s_{2n} - l| \leq |s_{2n} - s_{2n+1}| + |s_{2n+1} - l|
= |-a_{2n+1}| + |s_{2n+1} - l|
$$
The first term on the right we can make it smaller by choosing greater by going farther to the right,
the second term is just a restatement that $s_{2n+1}$ converges.
\\~\\




\textbf{2.7.2}

Does $\sum^{\infty}_{n=1} \frac{1}{2^n + n}$ converge?
\\

This series look a tad like
\begin{align*}
\sum_{n=1} \frac{1}{2^n + n} &= \frac{1}{2+1} + \frac{1}{2^2 +2} + \frac{1}{2^3 + 3} + \frac{1}{2^4 +4} + \frac{1}{2^5 +5} \ldots \\
&= \frac{1}{3} + \frac{1}{6} + \frac{1}{11} + \frac{1}{20} + \frac{1}{37} + \ldots
\end{align*}

In this case, $s_1 = 1/3$, $s_2 = 1/2$, and $s_3 = 13/22$.
So at least from the first couple partial sums, the partial sums seem to be monotonically increasing.

Now we have to look for clues.
The series $\sum \frac{1}{2^n}$ is a geometric series where $\frac{a}{1-r} = \frac{1}{1 - 1/2} = 2$.

Furhtermore, $\frac{1}{2^n} > \frac{1}{2^n +n}$ so by the comparsion tests, our series here should
converge.
\\~\\


Does $\sum^{\infty}_{n=1} \frac{\sin(n)}{n^2}$ converge?
\\

By a similar route, we saw in section 2.4 that $\sum 1/n^p$ when $p>1$ converges.
And since $\sin (n)$ oscillates between 1 and -1, then we can use the comparison test again to
see that the series converges.
\\~\\


Does $1 - \frac{3}{4} + \frac{4}{6} - \frac{5}{8} + \frac{6}{10} - \frac{7}{12} + \ldots$ converge?
\\

This series seems to go as $\sum (-1)^{n+1} \frac{n+1}{2n}$.
Which made me think of the alternating series.
But we need to see if the sequence $\frac{n+1}{2n}$ actually converges to 0.
Luckily, some algebra can help us right away
$$
\frac{n+1}{2n} = \frac{1}{2} + \frac{1}{2n}
$$
No matter how big $n$ gets, $(a_n) \rightarrow 1/2$.
It seems like we cannot say that it converges.
And since the series alternates between positive and negative values, then the series diverges.
\\~\\

Does $1 + \frac{1}{2} - \frac{1}{3} + \frac{1}{4} + \frac{1}{5} - \frac{1}{6} + \frac{1}{7} + \frac{1}{8} -\frac{1}{9} + \ldots$ converge?
\\

The trick here is to group terms, we always see two positive terms followed by a negative one.
And they always follow the pattern of the harmonic series but as
$$
\frac{1}{n} + \frac{1}{n+1} - \frac{1}{n+2} \geq \frac{1}{n}
$$
The inequality comes because $\frac{1}{n+1} > \frac{1}{n+2}$, so $\frac{1}{n+1} - \frac{1}{n+2}$
is a non-negative number that gets added to $1/n$.
And by using the comparison test, we can tell this series diverges.
\\~\\

Does $1 - \frac{1}{2^2} +\frac{1}{3} -\frac{1}{4^2} + \frac{1}{5} -\frac{1}{6^2} +\frac{1}{7} -\frac{1}{8^2} + \ldots$
converge?
\\

Interestingly, if we separate the terms for this series, it seems like we are adding
$\sum \frac{1}{2n -1}$ and $\sum -\frac{1}{(2n)^2} = \sum -\frac{1}{4}\frac{1}{n^2}$.

The latter series could be argued it converges by using the algebraic limit theorem for series
since we have $\sum 1/n^2$.
However, the former seems like the harmonic series, but only taking odd numbers.
\\~\\



\textbf{2.7.3}

An alternate way to prove the comparison tests for infinite series theorem using the monotone converge theorem
goes as follows.

If $0 \geq a_k \geq b_k$ and $\sum^{\infty}_{k=1} b_k$ converges, then we know that
the sequence of partial sums also converges.
And since the sequence of partial sums converges then it must be bounded.

The limit of the sequence of partial sums of $\sum b_k = \lim s_m$ can also be seen as an upper bound
for the sequence of partial sums of $\sum a_k$, since $0 \geq a_k \geq b_k$ for all $k\in\mathbb{N}$.

Also, since $0 \geq a_k$, then each successive term in the sequence of partial sums must be
equal to or greater than the previous one, so the sequence of partial sums is a monotone sequence
and thus converges as per the MCT.
\\~\\



\textbf{2.7.5}

Prove that the series $\sum^{\infty}_{n=1} 1/n^p$ converges if and only if $p > 1$.
\\

Abbott provides us with a wonderful hint: we can use the geometric series.
In example 2.7.5, we saw that
$$
\sum^{\infty}_{k=0} ar^k = \frac{a}{1-r}
$$
if and only if $|r| < 1$.

So let's say that $a=1$, we then have
$$
\sum^{\infty}_{k=0} r^k = \frac{1}{1-r}
$$

The above converges if and only if $|r|<1$, so it means that $r$ could be written as
$1/n$ where we require that $|n|>1$.
We can then write the geometric series as
$$
\sum^{\infty}_{k=0} \frac{1}{n^k} = \frac{1}{1-1/n} = \frac{n}{n-1}
$$

We could do a comparison test now: we have a series with $1/n^k$, where $k$ is the index,
and the other as $1/n^p$, where $n$ is now the index.
If we used $n$ as an index and $p$ as some fixed number on both, we'd be comparing
$1/p^n$ (geometric series with $|p|>1$) with $1/n^p$.
After a finite ammount of terms, the exponential $p^n$ will be greater than $n^p$,
so $\frac{1}{p^n} \leq \frac{1}{n^p}$.
This last inequality tells us that a plan ol' comparison test will get us nowhere since the geometric
series is a lower bound to a p-series.

The next logica trial would be to try and bound a p-series by something that looks like a
geometric series.
For example,
\begin{align*}
\sum^{\infty}_{n=1} \frac{1}{n^p} &= 1 + \frac{1}{2^p} + \frac{1}{3^p} + \frac{1}{4^p} + \frac{1}{5^p} + \frac{1}{6^p} + \frac{1}{7^p} + \frac{1}{8^p} + \ldots \\
& \leq 1 + \frac{1}{2^p} + \frac{1}{2^p} + \frac{1}{4^p} + \frac{1}{4^p} + \frac{1}{4^p} + \frac{1}{4^p} + \frac{1}{8^p} + \ldots \\
&= 1 + \frac{2}{2^p} + \frac{4}{4^p} + \frac{8}{8^p} + \ldots \\
&= 1 + \frac{1}{2^{p-1}} + \frac{1}{4^{p-1}} + \frac{1}{8^{p-1}} + \ldots \\
&= 1 + \frac{1}{2^{p-1}} + \frac{1}{(2^2)^{p-1}} + \frac{1}{(2^3)^{p-1}} + \ldots \\
&= 1 + \frac{1}{2^{p-1}} + \frac{1}{2^{2(p-1)}} + \frac{1}{2^{3(p-1)}} + \ldots \\
&= \sum_{k=0} \left( \frac{1}{2^{p-1}} \right)^k = \frac{1}{1 - 2^{p-1}} \\
&= \frac{2^{p-1}}{2^{p-1} - 1} 
\end{align*}
\\~\\



\textbf{2.7.7}

Show that if $a_n > 0$ and $\lim na_n = l$ with $l\neq 0$, then
the series $\sum a_n$ diverges.
\\

These problems feel like it may be a place where we may be able to invoke theorem 2.7.3: if the series
$\sum a_n$ converges, then $(a_n) \rightarrow 0$.
But let's think about the problem a bit more first.
\\

If $\lim na_n = l \neq 0$ then it means that $(n a_n) \rightarrow l$.
We also know that we are supposed to be looking for info that tells us whether $\sum a_n$
converges or diverges.
To show that a series converges, we usually evaluate whether their partial sums have a limit,
$\lim s_m$ exists, in this case the partial sums would look like
$s_m = a_1 + 2a_2 + + 3a_3 + \ldots + ma_m$.

Since $a_n > 0, \forall n$ and since $\lim na_n = l \neq 0$, then it means eventually, when $n\geq n$,
the terms will be greater than zero as well.
We could envision then that in order for the $n$s to balance with the $a_n$s in order to arive at a
limit $l$, that the $a_n$s must grow like $\mathcal{O}(1/n)$ because if they grew like $\mathcal{O}(1)$ then the
$n$ would overpower the sequence.
Similarly, if the $a_n$ grew like $\mathcal{O}(1/n^2)$, then $\lim na_n = 0$.

And since $a_n \sim 1/n$, then $\sum a_n$ diverges because $\sum \frac{1}{n}$ diverges.
\\~\\

Assume $a_n > 0$ and $\lim n^2 a_n$ exists.
Show that $\sum a_n$ converges.
\\

Similar train of thought as above but this time, since we must balance out with $n^2$,
then $a_n$ must grow as any sort of p-series with $p>1$, and since all of these converge, then
$\sum a_n$ converges.
\\~\\




\textbf{2.7.9}

\textbf{Ratio Test:} given a series $\sum^{\infty}_{n=1} a_n$ with $a_n \neq 0$,
the ratio test states that if $(a_n)$ satisfies
$$
\lim \left| \frac{a_{n+1}}{a_n} \right| = r < 1
$$

Then the series converges absolutely.
\\

Let $r^{\prime}$ satisfy $r < r^{\prime} < 1$.
Explain why there exists an $N$ such that $n\geq N$
implies $|a_{n+1}| \leq |a_n|r^{\prime}$.
\\

$$
\lim \left| \frac{a_{n+1}}{a_n} \right| = r < 1
$$
can be seen as,
$$
\left| \frac{a_{n+1}}{a_n} - r \right| < \epsilon
$$
or
$$
\frac{a_{n+1}}{a_n} \in (-(r+\epsilon), r+\epsilon)
$$

So if we only want a one-sided Inequality, we can do
$$
\left| \frac{a_{n+1}}{a_n} - r \right| < r+\epsilon = r^{\prime}
$$
Which can be rewritten as
$$
| a_{n+1} | \leq | a_n | r^{\prime}
$$
\\~\\

Why does $|a_N| \sum (r^{\prime})^n$ converge?
\\

From above, we have $| a_{n+1} | \leq | a_n | r^{\prime}$,
so $|a_{N+1}| \leq |a_N|r^{\prime}$, and $|a_{N+2}| \leq |a_{N+1}|r^{\prime}$.
The latter expression implying that $|a_{N+2}| \leq |a_{N+1}|r^{\prime} \leq |a_N|(r^{\prime})^2$.
Hence $|a_{N+n}| \leq |a_N| (r^{\prime})^n$.

Now, remember that for a series to converge we must have
$$
|a_{m+1} + a_{m+2} + \ldots + a_n| \leq |a_{m+1}| + |a_{m+2}| + \ldots + |a_n| < \epsilon
$$

If we try to put what we know into a similar form we have
$$
|a_N| + |a_{N+1}| + |a_{N+2}| + \ldots + |a_{N+n}| \leq
|a_N| + |a_N|(r^{\prime}) + |a_N|(r^{\prime})^2 + \ldots + + |a_N|(r^{\prime})^n 
$$

Which leads us to
$$
\sum |a_n| \leq \sum |a_N| (r^{\prime})^n = |a_N| \sum (r^{\prime})^n
$$

Luckily for us, $\sum (r^{\prime})^n$ is a geometric series which converges to $1/(1- r^\prime)$
since $0 < r^\prime < 1$.
Then, by the comparison test so does $\sum |a_n|$.
\\

Since $\sum |a_n|$ converges, then, by the absolute convergence test, so does $\sum a_n$.
\\~\\



\textbf{2.7.10}

\textbf{infinite Products:} Review exercise \ref{abbott:2.4.10}.

Does
$$
\frac{2}{1}\cdot \frac{3}{2}\cdot \frac{5}{4}\cdot \frac{9}{8}\cdot \frac{17}{16} \ldots
$$
converge?
\\

The reason Abbott tolds us to look at problem \ref{abbott:2.4.10} is because the above can be rewritten as
$$
\left(1 + \frac{1}{1} \right) \left(1 + \frac{1}{2} \right) \left(1 + \frac{1}{4} \right)
\left(1 + \frac{1}{8} \right) \left(1 + \frac{1}{16} \right) \ldots
$$
which itself can be rewritten as
$$
\left(1 + \frac{1}{2^0} \right) \left(1 + \frac{1}{2^1} \right) \left(1 + \frac{1}{2^2} \right)
\left(1 + \frac{1}{2^3} \right) \left(1 + \frac{1}{2^4} \right) \ldots
$$

If we recall the formula $1 + x \leq e^x$ then
\begin{align*}
& \left(1 + \frac{1}{2^0} \right) \left(1 + \frac{1}{2^1} \right) \left(1 + \frac{1}{2^2} \right)
\left(1 + \frac{1}{2^3} \right) \left(1 + \frac{1}{2^4} \right) \ldots  \\
&\leq e^{1/2^0} e^{1/2^1} e^{1/2^2} e^{1/2^3} e^{1/2^4} \ldots \\
&= \exp{\sum_{n=0} \frac{1}{2^n}}
\end{align*}

The latter expression converges to
$$
\exp{\frac{1}{1 - 1/2}} = \exp{2}
$$
So our original product does converge and is bounded by $e^2$.
\\~\\


The infinite product
$\frac{1}{2}\cdot \frac{3}{4}\cdot \frac{5}{6}\cdot \frac{7}{8}\cdot \frac{9}{10}\ldots$
certainly converges. Why?
Does it converge to zero?
\\

No simple pattern seems evident at first to let's try coming up with formulas for it.
The following seem to work
$$
\prod_{n=0} \frac{2n+1}{2n+2}
= \prod_{n=1} \frac{2n - 1}{2n}
= \prod_{n=1} \left( 1 - \frac{1}{2n} \right)
$$

The last one, once again, has the form $1+x$, where $x = - 1/2n$.
So let's try looking at
\begin{align*}
\prod_{n=1} \left( 1 - \frac{1}{2n} \right) &\leq \prod_{n=1} \exp \left( -\frac{1}{2n} \right) \\
\exp \left( -\frac{1}{2} \sum_{n=1} \frac{1}{n} \right)
\end{align*}
The $\sum 1/n$ of course is a harmonic series and it diverges.
However, notice that the divergent term is within a $e^{-x}$ term, so it actually grows smaller
and smaller as n grows.
So the $n=1$ is an upper bound.
Hence, we can see how the product converges and how it convegres to zero -
$\lim_{x\rightarrow\infty} e^{-x} = 0$.
\\~\\


In 1655, John Wallis famously derived the formula
$$
\left(\frac{2\cdot2}{1\cdot3}\right)
\left(\frac{4\cdot4}{3\cdot5}\right)
\left(\frac{6\cdot6}{5\cdot7}\right)
\left(\frac{8\cdot8}{7\cdot9}\right)
\ldots
= \frac{\pi}{2}
$$
Show that the left side converges to something.
\\

While we think about proving the convergence we came up with this way of rewriting the product
\begin{align*}
\prod_{n=1} \frac{2n\cdot2n}{(2n-1)(2n+1)}
&= \prod_{n=1} \frac{(2n)^2}{(2n)^2 -1} \\
&= \prod_{n=1} \left( 1 + \frac{(2n)^2 - (2n)^2 +1}{(2n)^2 -1} \right) \\
&= \prod_{n=1} \left( 1 + \frac{1}{(2n)^2 -1} \right) \\
&\leq \exp \left( \sum_{n=1} \frac{1}{(2n)^2 -1} \right)
\end{align*}
\\~\\


\textbf{2.7.12}

\textbf{Summation by parts:}
Let $(x_n)$ and $(y_n)$ be sequences,
let $s_n = x_1 + x_2 + \ldots + x_n$ and set $s_0 = 0$.
Use the observation that $x_j = s_j - s_{j-1}$
to verify the formula
$$
\sum^{n}_{j=m} x_j y_j =
s_n y_{n+1} - s_{m-1}y_m + \sum^{n}_{j=m} s_j (y_j - y_{j+1})
$$
\\

Let's fish for useful information.
One of the terms in the formula is
\begin{align*}
s_n - s_{m-1} &= 
    (x_1 + \ldots + x_{m-1} + x_m + x_{m+1} \ldots + x_n)
    - (x_1 + \ldots x_{m-1}) \\
&= x_m + x_{m+1} + \ldots + x_n
\end{align*}

Our next idea was to actualy write out the terms of the sum and see what patterns emerged.

First,
$$
x_m y_m = (s_m - s_{m-1})y_m = s_m y_m - s_{m-1} y_m
$$

The next term, $j=m+1$,
$$
x_{m+1} y_{m+1} = (s_{m+1} - s_m) y_{m+1} = 
s_{m+1} y_{m+1} - s_m y_{m+1}
$$

so thus far,
\begin{align*}
\sum^{m+1}_{j=m} x_j y_j &= - s_{m-1} y_m + s_m y_m - s_m y_{m+1} + s_{m+1} y_{m+1} \\
&= s_{m+1} y_{m+1} - s_{m-1} y_m + s_m (y_m - y_{m+1})
\end{align*}

Let's add another term and this time let's say that $n = (m+1)+1$,
$$
x_n y_n = (s_n - s_{m+1}) y_n =
s_n y_n - s_{m+1} y_n
$$
Our sum then becomes
\begin{align*}
\sum^{n=(m+1)+1}_{j=m} x_j y_j &= s_{m+1} y_{m+1} - s_{m-1} y_m + s_m (y_m - y_{m+1}) + s_n y_n - s_{m+1} y_n \\
&= s_n y_n - s_{m-1} y_m + s_m (y_m - y_{m+1}) + s_{m+1} (y_{m+1} - y_n)
\end{align*}

Note, that if we go this way, the last term was $s_{m+1} (y_{m+1} - y_n)$.
So our sum does not contain a $s_n$ term, we are 1 step away from the sort of expression we want.
To get the $s_n$ term we would need something like $s_{n} (y_{n} - y_{n+1}) = s_{n}y_{n} - s_n y_{n+1}$.
And it just so happens that we do already have the $s_n y_n$ bit at the begining of the previous expression,
so all we need to do is to add a $s_n y_{n+1}$ to cancel out the last term we are looking to introduce.
\\

Let's try working backwards now.
If we expand out the series we have
\begin{align*}
\sum^{n}_{j=m} x_j y_j
&= \begin{split} \\
    s_n y_{n+1} - s_{m-1}y_m   + s_m(y_m -y_{m+1})    + s_{m+1}(y_{m+1} -y_{m+2})       \\ + s_{m+2}(y_{m+2} -y_{m+3})      + \ldots + s_n(y_n - y_{n+1}) \\
    \end{split} \\
&= \begin{split} \\
    s_n y_{n+1} - s_{m-1}y_m   + s_m y_m -s_m y_{m+1} + s_{m+1}y_{m+1} - s_{m+1}y_{m+2} \\ + s_{m+2}y_{m+2} -s_{m+2}y_{m+3} + \ldots + s_n y_n - s_n y_{n+1} \\
    \end{split} \\
&= \begin{split} \\
    s_n y_{n+1} + (s_m - s_{m-1})y_m + (s_{m+1} -s_m)y_{m+1} + (s_{m+2} -s_{m+1})y_{m+2} \\ + \ldots + s_n y_n + s_n y_{n+1}
    \end{split} \\
&= \begin{split}
    s_n y_{n+1} + x_m y_m + x_{m+1}y_{m+1} + x_{m+2}y_{m+2} + \ldots + x_n y_n + s_n y_{n+1}
    \end{split}
\end{align*}
\\~\\



\textbf{2.7.13}

\textbf{Abel's Test:}
Abel's test for convergence states that if the series $\sum^{\infty}_{k=1} x_k$ converges,
and if $(y_k)$ is a sequence satisfying
$$
y_1 \geq y_2 \geq y_3 \geq \ldots \geq 0
$$
then the series $\sum_{k=1} x_k y_k$ converges.
\\

In the previous exercise we saw that
$$
\sum^{n}_{k=m} x_k y_k = s_n y_{n+1} - s_{m-1}y_m + \sum^{n}_{k=m} s_k (y_k - y_{k+1})
$$
With $s_0 = 0$.
So if $k=m=1$, then the above simplies to
$$
\sum^{n}_{k=m} x_k y_k = s_n y_{n+1} + \sum^{n}_{k=m} s_k (y_k - y_{k+1})
$$
\\

Now, use the comparison test to argue that $\sum_{k=1} s_k (y_k - y_{k+1})$ converges absolutely,
and show how this leads directly to a proof of Abel's test.
\\


Let's look at the series in parts.
Since $(y_k)$ is monotonically decreasing and non-negative, then $y_k - y_{k+1}$ will always be
equal to or greather than zero.
As per $s_k$, since we know that $(s_k)$ converges, we also know that the partial sums are bounded
by some $M$.
Thus
$$
0 \leq \sum_{k=1} \left| s_k (y_k - y_{k+1}) \right| \leq \sum M y_1 = M y_1 n
$$

So our upper bound there doesn't tell us anything.
Turns out that the trick is to look again at $\sum y_k - y_{k+1}$.
If you write it out you'll see that it is a telescoping series (memorize the way it looks!).
So we can refine our upper bound as such,
$$
0 \leq \sum_{k=1} \left| s_k (y_k - y_{k+1}) \right| \leq  M (y_1 - y_n)
$$

And since the upper bound is now a finite number, then by the comparison test, $\sum x_k y_k$
must converge as well.
\\~\\



\textbf{2.7.14}

\textbf{Dirichlet's Test:} Dirichlet's test for convergence states that if the partial sums
of $\sum x_k$ are bounded (but not necessarily convergent),
and if $(y_k)$ is a sequence satisfying $y_1 \geq y_2 \geq y_3 \geq \ldots \geq 0$
with $\lim y_k = 0$,
then the series $\sum x_k y_k$ converges.
\\

The hypothesis for Dirichlet's test differs from Abel in the fact that we did not assume that
$(y_k) \rightarrow 0$, instead we use the restructuring in terms of partial sums to create a
telescoping series that gave us a finite upper bound.
Dirichlet also doesn't require $\sum x_k$ to converge - which is how the alternating series
test can be seen as a case of Dirichlet's test.



%%%%%%%%%%%%%%%%%%%%%%%%%%%%%%%%%%%%%%%%%%%%%%%%%%%%%%%%%%%%%%%%%%%%%%%%%%%%%%
\subsection{Double Summations and Products of infinite Series}

\subsubsection{Exercises}

\textbf{2.8.1}

Compute $\lim s_{nn}$ for
$$
\begin{bmatrix}
-1     & \frac{1}{2} & \frac{1}{4} & \frac{1}{8} & \frac{1}{16} & \ldots \\[6pt]
0      & -1          & \frac{1}{2} & \frac{1}{4} & \frac{1}{8}  & \ldots \\[6pt]
0      & 0           & -1          & \frac{1}{2} & \frac{1}{4}  & \ldots \\[6pt]
0      & 0           & 0           & -1          & \frac{1}{2}  & \ldots \\[6pt]
0      & 0           & 0           & 0           & -1           & \ldots \\[6pt]
\vdots & \vdots      & \vdots      & \vdots      & \vdots       & \ddots
\end{bmatrix}
$$
This is the grid of real numbers $\{ a_{ij} : i,j \in \mathbb{N} \}$,
where $a_{ij} = 1/2^{j-i}$ if $j>i$, $a_{ij} = -1$ if $j=i$, and $a_{ij} = 0$ if $j<i$.
\\

If we add row by row, then we can see that
$$
\frac{1}{2} + \frac{1}{4} + \frac{1}{8} + \frac{1}{16} + \ldots
$$
is the geometric series with $a=1/2$ (since the geometric series is defined as $\sum_{k=0} ar^k$
and the first term must be $ar^0 = a$) and $r=1/2$.
That plus the starting term of $-1$ means that every row will add up to zero.
So the entire sum is zero is we go this route.
\\

On the other hand, if we add column by column, then we have the series
$$
-1 - \frac{1}{2} - \frac{1}{4} - \frac{1}{8} - \frac{1}{16} - \ldots
$$
Which is another geometric series, but this time with $a=-1$ and $r=1/2$.
So the double sum is now -2 (vs 0 in the previous case).
\\

When we do the sum over squares, we end up doing the same sum as when we sum by columns.
\\~\\



\textbf{2.8.2}

Show that if the iterated series
$$
\sum_{i=1} \sum_{j=1} \left| a_{ij} \right|
$$
converges (meaning that for each fixed $i\in\mathbb{N}$ the series $\sum_{j=1} |a_{ij}|$ converges
to some real number $b_i$ and the series $\sum_{i=1} b_i$ converges as well),
then the iterated series
$$
\sum_{i=1} \sum_{j=1} a_{ij}
$$
converges.
\\

Abbott once again gives us a great starting point.
If we assume that $\sum_{j=1} |a_{ij}|$ converges then $\sum_{j=1} a_{ij}$ must also converge.

If we refer to $\sum_{j=1} a_{ij}$ as $c_i$, then $c_i \sim |s_{i,m+1} + \ldots + s_{i,n}|$
which is less than $b_i \sim |s_{i,m+1}| + \ldots + |s_{i,n}|$.
Then we can assume that $\sum_{i=1} b_i$ converges, then so must $\sum_{i=1} c_i$ since
$|s_{i,m+1} + \ldots + s_{i,n}| \leq |s_{i,m+1}| + \ldots + |s_{i,n}|$.
\\~\\


\textbf{2.8.3}

If we define
$$
t_{mn} = \sum^{m}_{i=1} \sum^{n}_{j=1} |a_{ij}|
$$

and we assume that
$$
t_{mn} = \sum^{\infty}_{i=1} \sum^{\infty}_{j=1} |a_{ij}|
$$
converges.

Then $\lim_{n\rightarrow\infty} t_{nn}$ exist since
$$
t_{mn} = \sum^{m}_{i=1} \sum^{n}_{j=1} |a_{ij}|
\leq
\sum^{\infty}_{i=1} \sum^{\infty}_{j=1} |a_{ij}|
$$
\\

And since $t_{nn}$ converges, then we can use it as a Cauchy sequence.

By the same logic, and borrowing some of the notation from the previous exercise,
since $c_i \leq b_i$, then $|t_{nn} - t_{mm}| = |b_{m+1} + \ldots + b_n|$
and in turn $|s_{nn} - s_{mm}| = |c_{m+1} +\ldots + c_{n}| \leq |b_{m+1} + \ldots + b_n| < \epsilon$.
\\~\\



\textbf{2.8.4}

Let $\epsilon>0$ be arbitrary and argue that there exists an $N_1 \in \mathbb{N}$
such that $m,n \geq N_1$ implies $B - \frac{\epsilon}{2} < t_{mn} \leq B$,
when
$$
t_{mn} = \sum^{m}_{i=1} \sum^{n}_{j=1} |a_{ij}|
$$
and
$$
B = \sup\{ t_{mn} : m,n\in\mathbb{N} \}
$$
\\

The fact that $t_{mn} \leq B$ comes from the fact that $B$ is the supremum.
For the rest of the inequality it is useful to look back to \textbf{lemma 1.3.8} which stated
the following: assume $s\in\mathbb{R}$ is an upper bound for a set $A\subseteq\mathbb{R}$.
Then $s = \sup A$ if and only if, for every choice of $\epsilon >0$, there exists an element
$a\in A$ satisfying $s - \epsilon < a$.
\\

Now show that there exists and $N$ such that
$$
|s_{mn} - S| < \epsilon
$$
for all $m,n > N$.
\\

Here we assume $S = \sup\{ s_{mn} : m,n\in\mathbb{N} \}$.

There is an interesting consequence from the argument for $t_{mn}$ in which if we find
some $\epsilon >0$, then
$$
|t_{mn} - B| = \left| \sum^{m}_{i=1} \sum^{n}_{j=1} |a_{ij}| \right| < \epsilon
$$

From there,
\begin{align*}
|s_{mn} - S| &= \left| \sum^{m}_{i=1} \sum^{n}_{j=1} a_{ij} - \sum^{\infty}_{i=1} \sum^{\infty}_{j=1} a_{ij}  \right| \\
&= \left| \sum^{\infty}_{i=m+1} \sum^{\infty}_{j=n+1} a_{ij} \right| \\
&\leq \left| \sum^{\infty}_{i=m+1} \sum^{\infty}_{j=n+1} |a_{ij}| \right| \\
&= |t_{mn} - B | \\
&< \epsilon
\end{align*}
\\~\\


\textbf{2.8.5}

In the previous examples we have shown how that the absolute convergence theorem applies
to double sums and that "rectangular" partial sums of $t_{mn}$ and $s_{mn}$ do exist.
Along with showing the equivalence between the conventional notation for convergence and
the Cauchy type convergence for double sums.
Here, we aim to show that for double sums that absolutely converge, it doesn't matter if you
sum by rows and then by columns, or vice versa, the limit for the double sums exist and
any rearrangement gives you the same answer.
\\

If we defined the sum of a row as
$$
r_i = \sum^{n}_{j=1} a_{ij}
$$

Show that for all $m\geq N$
$$
\left| (r_1 + r_2 + \ldots + r_m) - S \right| < \epsilon
$$

Another bit to keep in mind from our previous proofs is that $\lim r_i$ exists and we can make use of the algebraic limit theorem to
perform the rest of the double sum and have some confidence that the sum of the limits for each row is the sum of the rows.
Similarly, the order limit theorem will give us some confidence in stating that the sum of the rows - since each row is less than or equal to the simit of the double sums -
will result in a number that is less than or equal to the double sum.

So we can make another argument for convergence by re-arranging the above
\begin{align*}
\left| (r_1 + r_2 + \ldots + r_m) - S \right| &= \left| \sum^{m}_{i=1} r_i - \sum^{m}_{i=1} \sum^{n}_{j=1} a_{ij} \right| \\
&= \left| \sum^{m}_{i=1} \left( r_i - \sum^{n}_{j=1} a_{ij} \right) \right| \\
&\leq \sum^{m}_{i=1} \left|  r_i - \sum^{n}_{j=1} a_{ij} \right|
\end{align*}

Here, in the last step, the term inside of the absolute value is another restatement of the limit of $r_i$,
which since we know it converges can be restated as
$$
| r_i - \sum^{n}_{j=1} a_{ij} | < \epsilon
$$
So we can make it
$$
| r_i - \sum^{n}_{j=1} a_{ij} | < \epsilon / m
$$

And thus our previous set of inequalities ends as
\begin{align*}
\left| (r_1 + r_2 + \ldots + r_m) - S \right| &= \left| \sum^{m}_{i=1} \left( r_i - \sum^{n}_{j=1} a_{ij} \right) \right| \\
&\leq \sum^{m}_{i=1} \left|  r_i - \sum^{n}_{j=1} a_{ij} \right| \\
&\leq m \frac{\epsilon}{m} = \epsilon
\end{align*}
\\

The exact same argument can be made when we add by columns since our convergence argument
for \textbf{2.8.3} still holds since we never specified any limitations on how the partial
sums were to be made.
\\~\\


\textbf{2.8.6}

If we now perform the double summation along diagonals (think about diagonals in a matrix going left to right)
where $i+j$ is equal a constant, for example
$$
d_2 = a_{11}    \quad    d_3 = a_{12} + a_{21}  \quad   d_4 = a_{13} + a_{22} + a_{31}
$$
and in general
$$
d_k = a_{1,k-1} + a_{2,k-2} + \ldots a_{k-1, 1}
$$

Then $\sum^{\infty}_{k=2} d_k$ represents another way of performing the double summation.
\\

Assuming everything up-to-this point is correct, show that $\sum^{\infty}_{k=2} d_k$
converges absolutely.
\\

What we can do here to show that the sum along diagonals converges absolutely is to note that
for $n\geq2$
$$
\sum^{n}_{k=2} |d_k| \leq t_{nn}
$$

Since $t_{nn}$ contains all the terms in $\sum^{n}_{k=2} |d_k|$ and more.
And since the $\sum^{n}_{k=2} |d_k|$ is monotonically increasing and bounded, then it converges.
\\

Now, let's try and imitate our proof for theorem 2.8.1 to show that $\sum_{k=2} d_k$
converges to $\lim_{n\rightarrow\infty} s_{nn} = S$.
\\

This one gets tricky.
Our first attempt was to go this route
\begin{align*}
|\sum_{k=2} d_k - S| &= |\sum_{k=2} d_k - s_{mn} + s_{mn} - S| \\
&\leq |\sum_{k=2} d_k - s_{mn}| + |s_{mn} - S| \\
&< |\sum_{k=2} d_k - s_{mn}| + \epsilon \\
&= |\sum_{k=2} d_k - \sum^{m}_{i} \sum^{n}_{j} a_{ij} | + \epsilon
\end{align*}

Since we know that $\lim s_{mn} = S$, we could arange the above as the argument for a
Cauchy series, i.e., something like $|s_{mn} - s{pq}| < \epsilon$ since the diagonal sums
do contain a decent deal of whats in $s_{mn}$, we just need to look far enough and make
the diagonal as close as possible to $s_{mn}$.
For example, the sum up to $d_{k=3}$ will contain all but 1 elements of $s_{22}$.
Similarly, the sum up to $d_{k=4}$ will contain all but 3 elements of $s_{33}$.

Following that logic we can see how $|\sum_{k=2} d_k - \sum^{m}_{i} \sum^{n}_{j} a_{ij} | $
becomes arbitrarily small as $n\rightarrow\infty$ when $m=n$ since $\lim s_{nn}$ converges.
\\~\\



\textbf{2.8.7}

Assume that $\sum^{\infty}_{i=1} a_i$ converges absolutely to A, and $\sum^{\infty}_{j=1} b_j$
converges absolutely to B.

Show that the iterated iterated sum $\sum_i \sum_j |a_i b_j|$ converges so that we may apply
theorem 2.8.1.
\\

\begin{align*}
\sum_i \sum_j |a_i b_j| &= \sum_i \sum_j |a_i| |a_j| \\
&= \sum_i \left( |a_i| \sum_j |b_j| \right) \\
&= \sum_i |a_i| \sum_j |b_j| \\
&= A \sum_j |b_j| \quad \text{algebraic limit theorem for series } \sum_k ca_k = cA \\
&= AB
\end{align*}
\\~\\

Let $s_{nn} = \sum^{n}_{i=1} \sum^{n}_{j=1} a_i b_j$, and prove that
$\lim_{n\rightarrow\infty} s_{nn} = AB$.
Conclude that
$$
\sum_i \sum_j a_i b_j =
\sum_j \sum_i a_i b_j =
\sum_k d_k
= AB
$$
\\

\begin{align*}
\lim s_{nn} &= \sum_i \sum_j a_i b_j \\
&= \lim \left(\sum_i a_i\right) \left(\sum_j b_j\right) \\
&= \lim s^{a}_{n} s^{b}_{n} \\
&= AB
\end{align*}

We already know that $\sum \sum |a_i b_j|$ converges to AB and we know that
$$
\left(\sum_i a_i\right)\left(\sum_j a_j\right) =
\sum_k d_k
$$
and we know that $\lim_{n\rightarrow\infty} s_{nn} = AB$.
Thus we can call upon theorem 2.8.1 to beleive that
$$
\lim_{n\rightarrow\infty} s_{nn} =
\sum_k d_k =
\sum_i \sum_j a_i b_j =
\sum_j \sum_i a_i b_j =
AB
$$
\section{Basic Topology of R}

%%%%%%%%%%%%%%%%%%%%%%%%%%%%%%%%%%%%%%%%%%%%%%%%%%%%%%%%%%%%%%%%%%%%%%%%
\subsection{Open and Closed Sets}

The empy set $\emptyset$ is considered an open subset of the real line, becuase if it wans't it would imply
that there is something in the emptyset that is "closed"
(negation of "for all $a\in O$ there exist a $V_\epsilon \subseteq O$").
\\

There is a theorem mentioned in topology that goes something along the lines of:
the union of two non-disjoint intervals is an interval.
The idea behind it being that in order for something to be an interval, it must must not have gaps
between its endpoints.
For example, the union of $[0,1] \cup [2,3]$ is not an internal since it doesn't include $[1,2]$.
\\

Open balls or $\epsilon$-neighborhoods make sense right away until you ought to give a proper argument.
The way to read the definitions for these is to think that you always want a "ball" around some
point, and you want to see if balls of any radius will be able to fit within another interval.
For example, in Abbott example 3.2.2(ii), we take $\epsilon$ to be some positive value by
looking at the difference between the point $x\in(c,d)$ and its left and right endpoints.
($x-c > 0$ and $d-x > 0$). (It must be a strict inequality because otherwise $|x-a| < 0$ would cause us trouble.)
And that definition of $\epsilon$ gives you a simple way to look for open balls in $(c,d)$.

To see the above, draw out a line and see how no matter what radius you end up using, as long as
it is less than $\epsilon$, all the open balls you draw will fit within $(c,d)$.
\\

\textbf{Theorem 3.2.3}

Note the interesting bit of logic being used here: since $O_{\lambda^\prime}$ is open then
$V_{\epsilon}(a) \subseteq O_{\lambda^\prime}$.
Since $O_{\lambda^\prime}$ is some arbitrary member of $O = \cup_{\lambda \in \Lambda} O_{\lambda}$,
then $O_{\lambda^\prime} \subseteq O$.
Putting these together $V_{\epsilon}(a) \subseteq O_{\lambda^\prime} \subseteq O$.
\\

\textbf{Definition 3.2.4}
By now, we've seen that $\epsilon$-neighboorhods are equivalent to open balls,
$$
V_{\epsilon} (a) \sim B(r=\epsilon, a) \sim \{ \mathbf{x}\in\mathbb{R}^n : |\mathbf{x} - \mathbf{a}| < r(=\epsilon) \}
$$
The neighboorhod part comes from $x\in S^{int} = \{ \mathbf{x}\in S : B(r,\mathbf{x}) \in S \text{ for some } r>0 \}$.
Which is a math way of saying that a neighboorhod consist of all points close to our center who are
still members of the set in question.
\\

Folland gives a conventional view of a closed set by first defining \textbf{boundary points}
of a set as points where every ball centered on them contains points both in $S$ and $S^c$,
These points may then belong to either the set in question or its compliment ($x\in S \cup S^v$).
And the set of all boundary points creates a boundary for $S$
$$
\partial S = \{ \mathbf{x}\in\mathbb{R}^n : B(r,\mathbf{x})\cap S \neq \emptyset \text{ and }
    B(r,\mathbf{x})\cap S^c \neq \emptyset \}
$$

However, instead of following this route, Abbott goes onto to construct closed intervals
with limit points which instead of relying on geometry, relies on our knowledge of series.

The definition of a limit point is literally the defintion of the limit of a sequence in terms of
$\epsilon$-neighboorhods as $\lim (a_n) = x \rightarrow |a_n - x| < \epsilon$.
\\~\\


\textbf{Theorem 3.2.5}

Notice how in the first part of the proof we are being told about an $\epsilon = 1/n$.
This is because when we prove that a limit exist, we must always provide a relationship for an
$\epsilon = \epsilon (n)$ so that $|a_n - x|<\epsilon$.
\\


\textbf{Example 3.2.9}

The first example is $A = \{ \frac{1}{n} : n\in\mathbb{N} \}$, when $\epsilon = 1/n - 1/(n+1)$.
Abbott helps us out here because otherwise we would have to figure out how to prove that
1.) the limit point is zero (but that is not in A), 2.) every other $x\in\mathbb{R}$ is either
$1/n$ or it never intersects with other member of A (think about some $(1/n - \epsilon, 1/n +\epsilon)$).

In our case it helps to see it this way,
\begin{align*}
v_{\epsilon} (1/n) &= \{ x\in\mathbb{R} : \left|x - \frac{1}{n}\right| < \epsilon \} = \left(\frac{1}{n} - \epsilon, \frac{1}{n} + \epsilon \right) \\
&= \left(\frac{1}{n} - \frac{1}{n} + \frac{1}{n+1}, \frac{1}{n} + \frac{1}{n} - \frac{1}{n+1} \right) \\
&= \left(\frac{1}{n+1}, \frac{2}{n} - \frac{1}{n+1} \right) \\
&= \left(\frac{1}{n+1}, \frac{2n + 2 - n}{n(n+1)} \right) \\
&= \left(\frac{1}{n+1}, \frac{n + 2}{n(n+1)} \right) \\
&= \left(\frac{n}{n(n+1)}, \frac{n + 2}{n(n+1)} \right)
\end{align*}

The only number we can put in between that interval, following our requirements, is $\frac{n+1}{n(n+1)} = \frac{1}{n}$.
\\~\\


\textbf{Theorem 3.2.13}
In the first part of the proof, we want to prove that $O^c$ contains all of its limit points.
Containing all of its limits points means that every $\epsilon$-neighboorhod of any limit point would
contain other points in $O^c$.
That is why if $x\in O$ would lead to a contradiction: if a single $x \notin O^c$, then at least one of its
$V_\epsilon (x) \subseteq O$ ($O$ and $O^c$ have no members in commmon, so if an $\epsilon$-neighboorhod is contained
in one then it cannot contain points in the other.)

When reading the converse statement, again, keep in mind that the concept of limit points is to look for
neighborhoods that are fully contained within a set, every $o \in v_\epsilon $ is also $x \ in O$ if
$v_\epsilon \in O$.
So looking at a limit point that is not part of $O^c$, means that $v_{\epsilon} (x) \in O$ by definition.
\\~\\


%%%%%%%%%%%%%%%%%%%%%%%%%%%%%%%%%%%%%%%%%%%%%%%%
\subsubsection{Exercises}

\textbf{3.2.1}

\textbf{(a)} Where in the proof of Theorem 3.2.3 part (ii) does the assumption that the collection of
open sets be finite gets used?
\\

Theorem 3.2.3 states that: the intersection of a finite collection of open sets is open.
The "finite-ness" is used when we look for the smallest $\epsilon$-neighborhood contained in every $O_k$.
If we assume that we are looking at the intersection of an infinite ammount of open sets, then the nested-interval
property doesn't hold and we cannot guarantee that there is an element contained within the inifity of neighborhoods.
See \ref{abbott:1.4.3} to see why.
\\~\\

\textbf{(b)} Give an example of a countable collectionof open sets $\{O_1, O_2, O_3, \ldots\}$
whose intersection $\cup^{\infty}_{n=1} O_n$ is closed, not empty and not all of $\mathbb{R}$.
\\

\textbf{3.2.5}

\textbf{Theorem 3.2.8} A set $F \subseteq \mathbb{R}$ is closed if and only if every
Cauchy sequence contained in $F$ has a limit that is also contained in $F$.
\\

Quick reminder, a Cauchy sequence $(a_n)$ is one which for every $\epsilon >0$
$\exists N\in\mathbb{N}$ such that whenever $m,n \geq N$ $|a_n - a_m| < \epsilon$.
And we also know that every convergent sequence is also a Cauchy sequence.
\\

For any $m,n \geq N$, $|f_n - f_m| < \epsilon$ for any $\epsilon$.
From this part of the definition of Cauchy sequences we can start forming $\epsilon$-neighboorhods
around any $f_i \in F$, $i \geq N$.
And because Cauchy sequences converge, $(f_n) \rightarrow x$, then our $\epsilon$-beighborhoods
will all contain elements of $F$, and we can change $m,n$ so as to have $f_n \neq x$.
Thus all of the limit poins will be contained in our set $F$.
\\

To do the proof in the other direction, assume that every Cauchy sequence we can find has a limit
contained in $F$.
Because Cauchy sequences converge, then we have $|f_n - x| < \epsilon$ - elements of $F$
that are within $\epsilon$-neighborhoods that intersect other members of $F$ since $|f_n - f_m|<\epsilon$.
\\~\\
\section{Introduction}

%%%%%%%%%%%%%%%%%%%%%%%%%%%%%%%%%%%%%%%%%%%%%%%%%%%%%%%%%%%%%%%%%%%%%%%
\subsection{Gaps in the rationals}

$$
A = \{p\in\mathbb{Q^+}: p^2 < 2 \}
$$

$$
B = \{p\in\mathbb{Q^+}: p^2 > 2 \}
$$

Rudin shows that A contains no largest number -
for every p in A, we can find a q in A such that $p < q$.
It also shows that B coontains no smaller number -
for ever p, we can find a q such that $p > q$.

The trick to obtain equation 4 is to subtract 2 from $\left(\frac{2p+2}{p+2}\right)^2$.

Then, to get the $q^2 < 2$ or $q^2 > 2$ inequalities, convert
$$
q^2 - 2 = \frac{2(p^2 -2)}{(p+2)^2} \rightarrow q^2 = 2 + \frac{2(p^2 -2)}{(p+2)^2}
$$


%%%%%%%%%%%%%%%%%%%%%%%%%%%%%%%%%%%%%%%%%%%%%%%%%%%%%%%%%%%%%%%%%%%%%%%%
\subsection{Consequences of completeness}

There is a very important theorem presented in 1.11 and given as an exercise in Abbott as
\ref{abbott:1.3.3}.

Abbott introuces it to elucidate the point to the interesting case when 
if a is an upper bound for A, and if $a\in A$, then it must be that $a = \sup A$.

The way to read Rudin's proof is as follows:

Suppose S is an ordered set with the least-upper-bound property (set is not-empty, bounded above, and supremum exist).
$B \subset S$ is not empty, and B is bounded below.
Let L be the set of all lower bounds of B, $L = \{ y\in S : y \leq x , x\in B \}$.
Then $\alpha = \sup L $ exists in ($\alpha \in S$) and
$$
\alpha = \inf B \text{ and } \alpha \in S
$$

\begin{itemize}
    \item B is bounded below thus $L \neq \emptyset$.
    \item All $x\in B$ are upper bounds of L.
\end{itemize}

Since S has the least-upper-bound property, and since L is bounded above, then $\alpha = \sup L$ must exist
in S - this is the definition of the least-upper-bound-property that we assumed applies for our "universe"
S.

By our definition of the suppremum,
\begin{itemize}
    \item $\alpha$ is an upper bound of L.
    \item If $\gamma < \alpha$ then $\gamma$ must not be an upper bound of L ($\gamma \in L$ but $\gamma \notin B$).
    \item $\alpha$ is the smallest upper bound, so $\alpha \leq x$ for all $x\in B$.
    \item If $\alpha \leq x$ then it means that it is a lower bound of B and thus $\alpha \in L$.
\end{itemize}

The last two properties we listed mean that if there were some $\beta$ such that $\alpha < \beta$,
then $\beta \in B$ - $\beta$ is an upper bound of L but not in L.

This restatement of the least upper bound (any upper boundes greater than the least upper bounds are greater than or equal to it)
also happens to be a statement for the definition of the greatest lower bound.
Look at it this way, we said that there exist an $\alpha$ that is a lower bound of B ($\alpha \leq x$)
and any $\beta > \alpha$ is not a lower bound.
In some other places that last phrase might have been written as $\alpha \geq y$, where $y$ is
any other lower bound of B (which again is the defintion of all elements of L).



%%%%%%%%%%%%%%%%%%%%%%%%%%%%%%%%%%%%%%%%%%%%%%%%%%%%%
\subsection{Fields}

We have that
$$
y = 0 + y = -x + (x + y)
= -x + (x + z)
$$
If $z = -x$, then
$$
y = -x + (x + y)
= -x + (x + (-x)) = (-x + x) + (-x)
= -x
$$

So $x + y = 0 = x + (-x)$ when we use $x+y = x+z$.

To prove
$$
- (-x) = x
$$

We can start with $x+y = x + (-x) = 0$.
Since to ever x corresponds an element $-x$ such that $x+(-x)=0$, then
$-(x) + -(y) = 0 = -(x) -(-x) = 0 = -x + x$.
\\

We can use similar logic to prove proposition 1.15.

$$
y = 1y = x \left(\frac{1}{x}\right) y = \left(\frac{1}{x}\right) (xy)
= \left(\frac{1}{x}\right) (xz) = z
$$

To prove (b) take $z=1$.
To prove (c) take $z = 1/x$.

If $x\neq 0$, then there exists an $1/x$ such that $1 (1/x) = 1$.
We know that $x(1/x) = 1$, then $1/1 = (1/x)(1/(1/x))$.
\\~\\

In the case of \textbf{proposition 1.16(c)}
Since
$$
(-x)y + xy = 0
$$
and 1.14(c) says that if $x+y = 0$, then $y=-x$, the above expression can be seen as $(-x)y + z = 0$
so $z = -(-x)y = xy$.
In the last step we used 1.14(d).
\\~\\


\textbf{Theorem 1.13}

For part c, where it states that
$$
|zw| = |z| |w|
$$

The intermediate steps are as follows:
$$
|zw|^2 = |(zw)(\bar{z}\bar{w})|^2 = \left[ (a+bi)(c+di) \right] \left[ (a-bi)(c-di) \right] 
$$
$$
= \left[ (ac - bd) + (ad + bc)i \right] \left[ (ac - bd) - (ad + bc)i \right]
$$
$$
= (ac - bd)^2 + (ad + bc)^2 - i(ac - bd)(ad + cb) + i(ac - bd)(ad + cb)
$$
$$
(ac - bd)^2 + (ad + bc)^2 = |z|^2 |w|^2 = \left( |z||w| \right)^2
$$
\\~\\



%%%%%%%%%%%%%%%%%%%%%%%%%%%%%%%%%%%%%%%%%%%%%%%%%%%%%%%%%%%%%%%%%%%%%%%%%%%%
\subsection{constructing $\mathbb{R}$ from $\mathbb{Q}$}

Property no. 2 of a cut says that when $p\in\alpha$ and $q\in\mathbb{Q}$
$$
q < p \rightarrow q\in\alpha
$$
or, equivalently,
$$
q\notin\alpha \rightarrow p < q
$$

The second implication that Rudin states is not so easy to follow.
But a way to make sense of it is by noting that the condition we are examining,
when $p\in\alpha$ and $q\in\mathbb{Q}$
$$
q < p \rightarrow q\in\alpha
$$

means that the cut $\alpha$ is "\textbf{closed downwards}", meaning that for all
$p\in\alpha$ is $q < p$ then $p\in\alpha$ (yes, we just restated it because it helps when you read it out loud).

This also means that if you were an $r$ and you were less than an $s$, but you knew that
you are not in the cut $\alpha$ then the thing greater than you, $s$, must not be in the cut either,
because all custs are closed downwards.

Again, $\alpha$ is closed downawards and contains no greatest element.
\\~\\
\section{Basic Topology}


\end{document}