\section{Introduction}

%%%%%%%%%%%%%%%%%%%%%%%%%%%%%%%%%%%%%%%%%%%%%%%%%%%%%%%%%%%%%%%%%%%%%%%
\subsection{Gaps in the rationals}

$$
A = \{p\in\mathbb{Q^+}: p^2 < 2 \}
$$

$$
B = \{p\in\mathbb{Q^+}: p^2 > 2 \}
$$

Rudin shows that A contains no largest number -
for every p in A, we can find a q in A such that $p < q$.
It also shows that B coontains no smaller number -
for ever p, we can find a q such that $p > q$.

The trick to obtain equation 4 is to subtract 2 from $\left(\frac{2p+2}{p+2}\right)^2$.

Then, to get the $q^2 < 2$ or $q^2 > 2$ inequalities, convert
$$
q^2 - 2 = \frac{2(p^2 -2)}{(p+2)^2} \rightarrow q^2 = 2 + \frac{2(p^2 -2)}{(p+2)^2}
$$


%%%%%%%%%%%%%%%%%%%%%%%%%%%%%%%%%%%%%%%%%%%%%%%%%%%%%%%%%%%%%%%%%%%%%%%%
\subsection{Consequences of completeness}

There is a very important theorem presented in 1.11 and given as an exercise in Abbott as
\ref{abbott:1.3.3}.

Abbott introuces it to elucidate the point to the interesting case when 
if a is an upper bound for A, and if $a\in A$, then it must be that $a = \sup A$.

The way to read Rudin's proof is as follows:

Suppose S is an ordered set with the least-upper-bound property (set is not-empty, bounded above, and supremum exist).
$B \subset S$ is not empty, and B is bounded below.
Let L be the set of all lower bounds of B, $L = \{ y\in S : y \leq x , x\in B \}$.
Then $\alpha = \sup L $ exists in ($\alpha \in S$) and
$$
\alpha = \inf B \text{ and } \alpha \in S
$$

\begin{itemize}
    \item B is bounded below thus $L \neq \emptyset$.
    \item All $x\in B$ are upper bounds of L.
\end{itemize}

Since S has the least-upper-bound property, and since L is bounded above, then $\alpha = \sup L$ must exist
in S - this is the definition of the least-upper-bound-property that we assumed applies for our "universe"
S.

By our definition of the suppremum,
\begin{itemize}
    \item $\alpha$ is an upper bound of L.
    \item If $\gamma < \alpha$ then $\gamma$ must not be an upper bound of L ($\gamma \in L$ but $\gamma \notin B$).
    \item $\alpha$ is the smallest upper bound, so $\alpha \leq x$ for all $x\in B$.
    \item If $\alpha \leq x$ then it means that it is a lower bound of B and thus $\alpha \in L$.
\end{itemize}

The last two properties we listed mean that if there were some $\beta$ such that $\alpha < \beta$,
then $\beta \in B$ - $\beta$ is an upper bound of L but not in L.

This restatement of the least upper bound (any upper boundes greater than the least upper bounds are greater than or equal to it)
also happens to be a statement for the definition of the greatest lower bound.
Look at it this way, we said that there exist an $\alpha$ that is a lower bound of B ($\alpha \leq x$)
and any $\beta > \alpha$ is not a lower bound.
In some other places that last phrase might have been written as $\alpha \geq y$, where $y$ is
any other lower bound of B (which again is the defintion of all elements of L).



%%%%%%%%%%%%%%%%%%%%%%%%%%%%%%%%%%%%%%%%%%%%%%%%%%%%%
\subsection{Fields}

We have that
$$
y = 0 + y = -x + (x + y)
= -x + (x + z)
$$
If $z = -x$, then
$$
y = -x + (x + y)
= -x + (x + (-x)) = (-x + x) + (-x)
= -x
$$

So $x + y = 0 = x + (-x)$ when we use $x+y = x+z$.

To prove
$$
- (-x) = x
$$

We can start with $x+y = x + (-x) = 0$.
Since to ever x corresponds an element $-x$ such that $x+(-x)=0$, then
$-(x) + -(y) = 0 = -(x) -(-x) = 0 = -x + x$.
\\

We can use similar logic to prove proposition 1.15.

$$
y = 1y = x \left(\frac{1}{x}\right) y = \left(\frac{1}{x}\right) (xy)
= \left(\frac{1}{x}\right) (xz) = z
$$

To prove (b) take $z=1$.
To prove (c) take $z = 1/x$.

If $x\neq 0$, then there exists an $1/x$ such that $1 (1/x) = 1$.
We know that $x(1/x) = 1$, then $1/1 = (1/x)(1/(1/x))$.
\\~\\

In the case of \textbf{proposition 1.16(c)}
Since
$$
(-x)y + xy = 0
$$
and 1.14(c) says that if $x+y = 0$, then $y=-x$, the above expression can be seen as $(-x)y + z = 0$
so $z = -(-x)y = xy$.
In the last step we used 1.14(d).
\\~\\


\textbf{Theorem 1.13}

For part c, where it states that
$$
|zw| = |z| |w|
$$

The intermediate steps are as follows:
$$
|zw|^2 = |(zw)(\bar{z}\bar{w})|^2 = \left[ (a+bi)(c+di) \right] \left[ (a-bi)(c-di) \right] 
$$
$$
= \left[ (ac - bd) + (ad + bc)i \right] \left[ (ac - bd) - (ad + bc)i \right]
$$
$$
= (ac - bd)^2 + (ad + bc)^2 - i(ac - bd)(ad + cb) + i(ac - bd)(ad + cb)
$$
$$
(ac - bd)^2 + (ad + bc)^2 = |z|^2 |w|^2 = \left( |z||w| \right)^2
$$
\\~\\



%%%%%%%%%%%%%%%%%%%%%%%%%%%%%%%%%%%%%%%%%%%%%%%%%%%%%%%%%%%%%%%%%%%%%%%%%%%%
\subsection{constructing $\mathbb{R}$ from $\mathbb{Q}$}

Property no. 2 of a cut says that when $p\in\alpha$ and $q\in\mathbb{Q}$
$$
q < p \rightarrow q\in\alpha
$$
or, equivalently,
$$
q\notin\alpha \rightarrow p < q
$$

The second implication that Rudin states is not so easy to follow.
But a way to make sense of it is by noting that the condition we are examining,
when $p\in\alpha$ and $q\in\mathbb{Q}$
$$
q < p \rightarrow q\in\alpha
$$

means that the cut $\alpha$ is "\textbf{closed downwards}", meaning that for all
$p\in\alpha$ is $q < p$ then $p\in\alpha$ (yes, we just restated it because it helps when you read it out loud).

This also means that if you were an $r$ and you were less than an $s$, but you knew that
you are not in the cut $\alpha$ then the thing greater than you, $s$, must not be in the cut either,
because all custs are closed downwards.

Again, $\alpha$ is closed downawards and contains no greatest element.
\\~\\