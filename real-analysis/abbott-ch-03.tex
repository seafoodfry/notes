\section{Basic Topology of R}

The empy set $\emptyset$ is considered an open subset of the real line, becuase if it wans't it would imply
that there is something in the emptyset that is "closed"
(negation of "for all $a\in O$ there exist a $V_\epsilon \subseteq O$").
\\

There is a theorem mentioned in topology that goes something along the lines of:
the union of two non-disjoint intervals is an interval.
The idea behind it being that in order for something to be an interval, it must must not have gaps
between its endpoints.
For example, the union of $[0,1] \cup [2,3]$ is not an internal since it doesn't include $[1,2]$.
\\

Open balls or $\epsilon$-neighborhoods make sense right away until you ought to give a proper argument.
The way to read the definitions for these is to think that you always want a "ball" around some
point, and you want to see if balls of any radius will be able to fit within another interval.
For example, in Abbott example 3.2.2(ii), we take $\epsilon$ to be some positive value by
looking at the difference between the point $x\in(c,d)$ and its left and right endpoints.
($x-c \geq 0$ and $d-x \geq 0$).
And that definition of $\epsilon$ gives you a simple way to look for open balls in $(c,d)$.

To see the above, draw out a line and see how no matter what radius you end up using, as long as
it is less than $\epsilon$, all the open balls you draw will fit within $(c,d)$.