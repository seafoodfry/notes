\section{Basic Topology of R}

%%%%%%%%%%%%%%%%%%%%%%%%%%%%%%%%%%%%%%%%%%%%%%%%%%%%%%%%%%%%%%%%%%%%%%%%
\subsection{Open and Closed Sets}

The empy set $\emptyset$ is considered an open subset of the real line, becuase if it wans't it would imply
that there is something in the emptyset that is "closed"
(negation of "for all $a\in O$ there exist a $V_\epsilon \subseteq O$").
\\

There is a theorem mentioned in topology that goes something along the lines of:
the union of two non-disjoint intervals is an interval.
The idea behind it being that in order for something to be an interval, it must must not have gaps
between its endpoints.
For example, the union of $[0,1] \cup [2,3]$ is not an internal since it doesn't include $[1,2]$.
\\

Open balls or $\epsilon$-neighborhoods make sense right away until you ought to give a proper argument.
The way to read the definitions for these is to think that you always want a "ball" around some
point, and you want to see if balls of any radius will be able to fit within another interval.
For example, in Abbott example 3.2.2(ii), we take $\epsilon$ to be some positive value by
looking at the difference between the point $x\in(c,d)$ and its left and right endpoints.
($x-c > 0$ and $d-x > 0$). (It must be a strict inequality because otherwise $|x-a| < 0$ would cause us trouble.)
And that definition of $\epsilon$ gives you a simple way to look for open balls in $(c,d)$.

To see the above, draw out a line and see how no matter what radius you end up using, as long as
it is less than $\epsilon$, all the open balls you draw will fit within $(c,d)$.
\\

\textbf{Theorem 3.2.3}

Note the interesting bit of logic being used here: since $O_{\lambda^\prime}$ is open then
$V_{\epsilon}(a) \subseteq O_{\lambda^\prime}$.
Since $O_{\lambda^\prime}$ is some arbitrary member of $O = \cup_{\lambda \in \Lambda} O_{\lambda}$,
then $O_{\lambda^\prime} \subseteq O$.
Putting these together $V_{\epsilon}(a) \subseteq O_{\lambda^\prime} \subseteq O$.
\\

\textbf{Definition 3.2.4}
By now, we've seen that $\epsilon$-neighborhoods are equivalent to open balls,
$$
V_{\epsilon} (a) \sim B(r=\epsilon, a) \sim \{ \mathbf{x}\in\mathbb{R}^n : |\mathbf{x} - \mathbf{a}| < r(=\epsilon) \}
$$
The neighborhood part comes from $x\in S^{int} = \{ \mathbf{x}\in S : B(r,\mathbf{x}) \in S \text{ for some } r>0 \}$.
Which is a math way of saying that a neighborhood consist of all points close to our center who are
still members of the set in question.
\\

Folland gives a conventional view of a closed set by first defining \textbf{boundary points}
of a set as points where every ball centered on them contains points both in $S$ and $S^c$,
These points may then belong to either the set in question or its compliment ($x\in S \cup S^v$).
And the set of all boundary points creates a boundary for $S$
$$
\partial S = \{ \mathbf{x}\in\mathbb{R}^n : B(r,\mathbf{x})\cap S \neq \emptyset \text{ and }
    B(r,\mathbf{x})\cap S^c \neq \emptyset \}
$$

However, instead of following this route, Abbott goes onto to construct closed intervals
with limit points which instead of relying on geometry, relies on our knowledge of series.

The definition of a limit point is literally the defintion of the limit of a sequence in terms of
$\epsilon$-neighborhoods as $\lim (a_n) = x \rightarrow |a_n - x| < \epsilon$.
\\~\\


\textbf{Theorem 3.2.5}

Notice how in the first part of the proof we are being told about an $\epsilon = 1/n$.
This is because when we prove that a limit exist, we must always provide a relationship for an
$\epsilon = \epsilon (n)$ so that $|a_n - x|<\epsilon$.
\\


\textbf{Example 3.2.9}

The first example is $A = \{ \frac{1}{n} : n\in\mathbb{N} \}$, when $\epsilon = 1/n - 1/(n+1)$.
Abbott helps us out here because otherwise we would have to figure out how to prove that
1.) the limit point is zero (but that is not in A), 2.) every other $x\in\mathbb{R}$ is either
$1/n$ or it never intersects with other member of A (think about some $(1/n - \epsilon, 1/n +\epsilon)$).

In our case it helps to see it this way,
\begin{align*}
v_{\epsilon} (1/n) &= \{ x\in\mathbb{R} : \left|x - \frac{1}{n}\right| < \epsilon \} = \left(\frac{1}{n} - \epsilon, \frac{1}{n} + \epsilon \right) \\
&= \left(\frac{1}{n} - \frac{1}{n} + \frac{1}{n+1}, \frac{1}{n} + \frac{1}{n} - \frac{1}{n+1} \right) \\
&= \left(\frac{1}{n+1}, \frac{2}{n} - \frac{1}{n+1} \right) \\
&= \left(\frac{1}{n+1}, \frac{2n + 2 - n}{n(n+1)} \right) \\
&= \left(\frac{1}{n+1}, \frac{n + 2}{n(n+1)} \right) \\
&= \left(\frac{n}{n(n+1)}, \frac{n + 2}{n(n+1)} \right)
\end{align*}

The only number we can put in between that interval, following our requirements, is $\frac{n+1}{n(n+1)} = \frac{1}{n}$.
\\~\\


\textbf{Theorem 3.2.13}
In the first part of the proof, we want to prove that $O^c$ contains all of its limit points.
Containing all of its limits points means that every $\epsilon$-neighborhood of any limit point would
contain other points in $O^c$.
That is why if $x\in O$ would lead to a contradiction: if a single $x \notin O^c$, then at least one of its
$V_\epsilon (x) \subseteq O$ ($O$ and $O^c$ have no members in commmon, so if an $\epsilon$-neighborhood is contained
in one then it cannot contain points in the other.)

When reading the converse statement, again, keep in mind that the concept of limit points is to look for
neighborhoods that are fully contained within a set, every $o \in v_\epsilon $ is also $x \ in O$ if
$v_\epsilon \in O$.
So looking at a limit point that is not part of $O^c$, means that $v_{\epsilon} (x) \in O$ by definition.
\\~\\


%%%%%%%%%%%%%%%%%%%%%%%%%%%%%%%%%%%%%%%%%%%%%%%%
\subsubsection{Exercises}

\textbf{3.2.1}

\textbf{(a)} Where in the proof of Theorem 3.2.3 part (ii) does the assumption that the collection of
open sets be finite gets used?
\\

Theorem 3.2.3 states that: the intersection of a finite collection of open sets is open.
The "finite-ness" is used when we look for the smallest $\epsilon$-neighborhood contained in every $O_k$.
If we assume that we are looking at the intersection of an infinite ammount of open sets, then the nested-interval
property doesn't hold and we cannot guarantee that there is an element contained within the inifity of neighborhoods (it may or mat not exist).
See \ref{abbott:1.4.3}.
\\~\\

\textbf{(b)} Give an example of a countable collectionof open sets $\{O_1, O_2, O_3, \ldots\}$
whose intersection $\cup^{\infty}_{n=1} O_n$ is closed, not empty and not all of $\mathbb{R}$.
\\~\\


\textbf{3.2.2}

Let
$$
A = \left\{ (-1)^n + \frac{2}{n} : n\in\mathbb{N} \right\},
$$
$$
B = \left\{ x \in\mathbb{Q} : 0 < x < 1 \right\},
$$
and
$$
C = \left\{ \frac{(-1)^n n}{n+1} : n\in\mathbb{N} \right\}
$$

\textbf{(a)} what are the limit points?
\\

Let's plugin a couple numbers!
In $A$, when $n=1$ we got $-1 + 2 = 1$; when $n=2$, $1 + 1 = 2$; $n=3$, $-1 + 2/3 = -1/3$; $n=4$, $1 + 1/2 = 3/2$;
so $A = \{ 1, 2, -\frac{1}{3}, \frac{3}{2}, -\frac{3}{5}, \ldots \}$.
Holistically, the first term oscillates between $-1$ and $1$, while the second one is monotonically decreasing
starting at 2 and then decreasing to 0 as $n$ grows unbounded.
Since we are adding or subtracting monotonically decreasing terms to -1 and 1, and since the first two terms are 1 and 2,
then all other values will oscillate between the two, getting closer and closer to both of those values.
So our limit points here are $1$ and $-1$. (they are literally the limit of a sequence contained in $A$.)

Now, $B$.
Back in \textbf{example 3.2.9}, we saw that a property of $\mathbb{Q}$ is that its set of limit points is \textbf{ALL}
of $\mathbb{R}$.
So the set of limit points for $B$ is not just 0 and 1 but the entire interval $[0,1]$.

Finally $C$. Again, plugin in some numbers, $n=1$; $-1/2$; $n=2$, $2/3$, etc.
So, we get $C = \{-\frac{1}{2}, \frac{2}{3}, -\frac{3}{4}, \frac{4}{5}, \ldots, \frac{10}{11}, -\frac{11}{12}, \ldots \}$.
We again see that the elements of $C$ approach -1 and 1 as $n$ grows without bounds.
So the limit points are -1 and 1.

Let's do a bit more work for $C$, just because.
We can see that a limit point for $C$ is 1 because the even terms grow as $(\frac{n}{n+1})$.
The difference between consecutive even terms is
$\left|\frac{n}{n+1} - \frac{n+2}{n+3}\right| = \left|\frac{3(n+3) - (n+1(n+2))}{(n+1)(n+3)}\right| = \left|\frac{-2}{(n+1)(n+3)}\right| = \frac{2}{(n+1)(n+3)}$.
We want to use this data to verify that every $\epsilon$-neighborhood of a limit point does indeed intercept our set $C$
at some point other than the limit point we are looking at.
Which means that we want to see $V_{\epsilon} (1) \cap C \neq \emptyset$ and $V_{\epsilon} (1) \cap C \neq 1$.
We can see $V_{\epsilon} (1) \cap C \neq \emptyset$ right away since $\frac{n}{n+1} \neq 1$ ever.
Leaving us with the task of verifying that $\{ x\in\mathbb{R} : |x-1| < \epsilon \} \cap C \neq \emptyset$ for all $\epsilon > 0$.
The previous expression can in turn be simplified to $\{ c_n\in C : |c_n - 1| < \epsilon \}$
\textbf{which just magically became a restatement of the criteria for the convergence of a series!} -
think about it, we want to find values in our neighborhood that intercept with $C$, so instead of searching through all of $\mathbb{R}$
we could search through $C$.
This onw reduces to $\left| \frac{n}{n+1} - 1 \right| = \frac{1}{n+1} < \epsilon$.
Since $ \frac{1}{n+1} > \frac{2}{(n+1)(n+3)}$, the distance between two terms in the series, we should be able to
find elements of $C$ within our $\epsilon$-neighborhoods.

Note that in all of these cases we could readily evaluate the limits to get the limit points because the limit points
are not actually memebers of the sets.
\\


\textbf{(b)} Are the sets open or closed?
\\

$A$ is not closed because it doesn't contain all of its limit points, $-1 \notin A$.
But $A$ is also not open since every $V_\epsilon (a), a\in A$ will have some irrational number that is not part of A,
so there is not an $V_{\epsilon} (a) \subseteq A$ (tere isn't a neighborhood that is completely within $A$).

Similarly, $B$ is not closed, because it doesn't not contain its limit points, and also not open
because every $V_\epsilon (b), b\in B$ is not completely contained within $B$
(continuous $\epsilon$-neighborhoods are not a subset of $B$).

Same story for $C$.
\\


\textbf{(c)} Do the set contain isolated points?
\\


Every point in $A$ except 1 and -1 are isolated points.

In the case of $B$, $B$ is dense in $[0,1]$ - every $b\in B \subseteq [0,1]$ - so $B \setminus [0,1] = \emptyset$, so
$B$ has no isolated points.

Finally, all elements in $C$ are isolated points.
\\

\textbf{(d)} Find the closure of the sets.
\\

Our closures are, $\bar{A} = A \cup \{-1, 1\}$, $\bar{B} = [0,1]$, and $\bar{C} = C \cup \{-1, 1\}$.
\\~\\


\phantomsection
\label{abbott:3.2.3}
\textbf{3.2.3}

Decide whether the following sets are open, closed, or neither.
If a set is not open, find a point in the set for which there is no $\epsilon$-neighborhood contained in the set.
If a set is not closed, find a limit point that is not contained in the set. 
\\

\textbf{(a)} $\mathbb{Q}$
\\

Abbott already shared with us that all the reals are the limit points for $\mathbb{Q}$.
So $\mathbb{Q}$ is not closed.
But we also already noted in the previous exercise that continuous $\epsilon$-neighborhoods are not contained in
$\mathbb{Q}$ - for all elements in $\mathbb{Q}$, it is not the case that there is an $\epsilon$-neighborhood
that is contained within $\mathbb{Q}$, $V_\epsilon (q) \not\subseteq \mathbb{Q}$.

Another way of putting it: $\mathbb{Q}$ is not is not open because irrationals in the $\epsilon$-neighborhoods
are not part of $\mathbb{Q}$. And $\mathbb{Q}$ is not closed because it contains irrational limit points.
\\

\textbf{(b)} $\mathbb{N}$
\\

It is not open by the same logic as above.
$\mathbb{N}$ has no limit points, but it is considered closed.

Think about the two definitions for limit points that we have: there is no sequence contained within $\mathbb{N}$
with a limit and also, it is not the case that every $\epsilon$-neighborhood intercepts $\mathbb{N}$.
\\

\textbf{(c)} $\{ x\in\mathbb{R} : x \neq 0 \}$
\\

$\mathbb{R} \setminus \{0\}$ is open because every element has an $\epsilon$-neighborhood that is contained within
$\mathbb{R} \setminus \{0\}$.
It is not closed because 0 is a limit point but it is not included - for example, $1/n \rightarrow 0$.
\\

\textbf{(d)} $\left\{ \sum \frac{1}{n^2} : n\in\mathbb{N} \right\}$
\\

The limit of the sum is $\pi^2 / 6$, its elements are $\{1, 5/4, \ldots \}$.
So similar sitation as above, it is not closed because it doesn't contain its limit.
It is also not open since it only contains rationals.
\\

\textbf{(e)} $\left\{ \sum \frac{1}{n} : n\in\mathbb{N} \right\}$
\\
Similar logic to the previous answer but this time the set is closed since $\sum 1/n$ doesn't converge,
so there is no limit point.
It is not open for the same reasons  as above.
\\~\\



\phantomsection
\label{abbott:3.2.4}
\textbf{3.2.4}
\\

Let $A$ be nonempty and bounded above so that $s = \sup A$ exists.

\textbf{(a)} Show that $s\in \bar{A}$
\\

Interestingly, this exercise is a great connection to \ref{abbott:1.3.3} where we find out that if a set is closed
then the supremum of the set is a member of the set.

In general, $\bar{A} = A \cup L$, where $L$ is the set of limit points.
So one way to try and prove this is to see whether $s \in L$, or in general $s \in A$.
If you remember \textbf{lemma 1.3.8},
it states that if $s \in \mathbb{R}$ is an upper bound for a set $A \subseteq \mathbb{R}$,
then $s = \sup A$ if and only if, \textbf{for every} choice of $\epsilon > 0$,
there exists an element $a \in A$ satisfying $s - \epsilon < a$.

Now, if you remember the definition of a limit point, it states that a limit point is a limit point
if every $\epsilon$-neighborhood of a limit point intercepts the set $A$ at some point other than itself.
Using lemma 1.38, we can see then that every $V_\epsilon (s)$ will contain some $a$ such that $s - \epsilon < a$.
Hence, \textbf{we can think of the supremum as a limit point}.
\\

\textbf{(b)} Can an open set contain its supremum?
\\

This is a great exercise because it gives us the opportunity to go from hypothesis to proof.
Based on what we know about supremums, it would be reasonable to doubt that an open set will contain its supremum.
So let's try and prove that directly.

If $A$ is an open set, then it means that every $a \in A$ has a $V_\epsilon (a)$ such that $V_{\epsilon} (a) \subseteq A$.
That is, there is at least one $\epsilon$-neighborhood for every element of $A$ that is equal to or a subset of $A$.
Which means that at least one $\epsilon$-neighborhood contains to elements external to $A$.
This means that if we wanted to show that $s \not\in A$ then we would need to show that every $V_\epsilon (s)$
contains elements that are outside of $A$.
To see this, remember that $s$ is the least upper bound, so any $s+\epsilon$ will be an upper bound but it will be
outside the range of $A$.
If $s+\epsilon \in A$, then it would mean that $s \neq \sup A$.
\\~\\




\textbf{3.2.5}
\\

\textbf{Theorem 3.2.8} A set $F \subseteq \mathbb{R}$ is closed if and only if every
Cauchy sequence contained in $F$ has a limit that is also contained in $F$.
\\

Quick reminder, a Cauchy sequence $(a_n)$ is one which for every $\epsilon >0$
$\exists N\in\mathbb{N}$ such that whenever $m,n \geq N$ $|a_n - a_m| < \epsilon$.
And we also know that every convergent sequence is also a Cauchy sequence.
\\

Let's assume that every Cauchy sequence we know is contained in $F$ and their limits are too contained in $F$.
Since Cauchy sequences are convergent sequences, then it means that all the limit points we can find
will be contained in $F$, thus meeting the defintion of a closed set.
\\

The above argument was the proof for the backwards direction, now let's go forward.
Assume that $F$ is closed.
Since $F$ is closed, then it must contain all of its limit points.
Since every limit point, $l$, is the limit of some sequence contained in $F$, then it means that we have
$\lim (f_n) = l$.
And since every convergent sequence is a cauchy seuqnece, then we can state our theorem.
\\~\\



\textbf{Exercise 3.2.12}
\\

Let $A$ be an uncountable set and let $B$ be the set of real numbers that divides $A$ into two uncountable sets;
that is, $s \in B$ if both $\{ x : x \in A, x < s \}$ and $\{ x : x \in A, x > s\}$ are uncountable.
Show that $B$ is nonempty and open.
\\

We are asked for a couple things right from the begining.
First, $A \not\sim \mathbb{N}$ and $B \subseteq \mathbb{R}$.
Furthermore, $B \neq \emptyset$ and is open.
Remainder, $B$ being open means that for every $s\in B$, $B \subseteq \mathbb{R}$ (we got that)
and $V_\epsilon (s) \subseteq B$ for some $\epsilon >0$.
So for $B$ to be open it must be a subset of the reals and every single one of its members must have
an $\epsilon$-neighborhood completely contained within $B$.
Yet another way of saying this is that $\forall s \in B$, $(s-\epsilon, s+\epsilon) \in B$, for some $\epsilon >0$.
(Finding a single $s\in B$ whose $V_{\epsilon} (s)$ was not contained in $B$ would also serve our purposes in someway.)
\\

Before we go and dive into what we think are solutions, there is another very important thing to mention.
Notice how the problem satets "let $B$ be the set of real numbers that divides $A$ into two uncountable sets".
In a way this means that $B$ is sort of like a function - we don't chose it, instead we feed it some uncountable $A$
and out comes a $B$ that divides the $A$ into two uncountable sets.
\\

One argument that can eb made is as follows: let's call $C_1 = \{ x : x \in A, x < s \}$
and $C_2 = \{ x : x \in A,  s < x \}$.
To show that $B \neq \emptyset$, consider $x \in C_1$ and $y\in C_2$. Since $C_1$ and $C_2$
are uncountable, then there are nonempty and thus $(x+2)/2 \in B$.

Now to show that $B$ is open consider $(-\infty, s)$ and $(s, \infty)$ when $s\in B$.
We know there must exist an $x \in (-\infty , s)$ and a $y \in (s, \infty)$.
So we can form an open interval $(x,y)$.
Since $A$, $C_1$, and $C_2$ are uncountable, we can have an inifity of such intervals.
If we look at the midpoints between any $x$ and any $y$, then we will be able to obtain some $s \in B$.
From these we can form our open interval as $(s-\epsilon , s+\epsilon) \in B$.
To make the last connection revisit example 3.2.2 (ii).
\\

There are also some great answers in the internet that merit discussion.
Let's take a look at
\href{https://math.stackexchange.com/questions/3042397/show-a-set-is-nonempty-and-open}{Show a set is nonempty and open}.
There is a particular argument made there that pretty much goes as follows: if $B$ is not open, then
$\forall \epsilon >0$, there would exists an $x_\epsilon \in (x-\epsilon , x+\epsilon)$ such that
either $(-\infty , x_\epsilon) \cap A$ or $(x_\epsilon , \infty) \cap A$ is countable.

The thing that makes this worth mentioning is that it serves as remmainder that is not necessarily the case that if
a set is not open then it is close.
As some of the previous exercsies showed us, sets can be neither open nor close when they contain "holes".

As for the rest of the argument made in the above matheschange link, along with
\href{https://math.stackexchange.com/questions/1800147/a-is-uncountable-and-b-divides-a-in-two-uncountable-sets-show-that-b-is}{$A$
is uncountable and $B$ divides $A$ in two uncountable sets. Show that $B$ is nonempty and open}
and
\href{https://math.stackexchange.com/questions/4071927/show-that-an-uncountable-set-can-be-separated-into-two-disjoint-uncountable-subs}{
    Show that an uncountable set can be separated into two disjoint uncountable subsets}
relie on the following principles.

Consider, $B_1 = \{ x\in\mathbb{R} : (-\infty,x) \cup A \text{ is uncountable} \}$.
The way to read this is that $B_1$ is the set of $x$s such that an interval $(-\infty,x)$ overlaps with $A$ resulting in an uncountable set of numbers within their overlap.
For example, if $A = [0,1]$, then $x$ could be 1, but it could also be 2, 3, etc., $(-\infty,1), (-\infty, 2), (-\infty,3)$ all result
in an uncoutable set when we look at their interception with $A$.
Also, because it is the case that for any $y$ that is $y > x$ for any $x \in B_1$, then $B_1$ is \textbf{upward closed} ($(-\infty,x)$ is the upper set).

Following the same line of arguments, $B_2 = \{ x\in\mathbb{R} : (x,\infty) \cup A \text{ is uncountable} \}$
corresponds to a \textbf{downward closed set} where for any $x\in B_2$ if $x > y$ for some $y\in\mathbb{R}$, then
$y \in B_2$.

The rest of the argument relies on something we proved back in \ref{abbott:1.5.3}, which is that the union
of countable sets is a countable set.
Which should point us to a proof by contradiction since we have been talking about uncountable sets.
The thing to note is that since $B_1$ is upward close, then for any $x\in B_1$, $x+\epsilon \in B_1$,
so we only need to make an argument for $x-\epsilon \in B_1$.
And the proof by contradiciton goes as follows: if it is not the case that $x-\epsilon \in B_1$ for some $\epsilon > 0$,
then $(-\infty, x-1/n) \cup A$ must be countable for all $n\in\mathbb{N}$
(if not open, then there is a whole, and the whole is because we have some countable amount of elements between $x-1/n$ and $x$).

This in turn means that
$$
\bigcup_{n\in\mathbb{N}} \left(-\infty,x-\frac{1}{n}\right) \cup A = (-\infty,x) \cup A
$$
is countable, and we land in a contradiciton.

Last things to mention are that $B_1 = (-\infty, b_1)$, where $b_1$ is either $\inf B_1$ or $-\infty$.
Similarly, $B_2$ is $(b_2, \infty)$, with $b_2$ either $\sup B_2$ or $\infty$.
Therefore $B_1 \cup B_2 = \mathbb{R}$, $b_1 > b_2$, and $B = B_1 \cap B_2 \neq \emptyset$.
\\~\\




\textbf{3.2.13}
\\

Prove that the only sets that are both open and closed are $\mathbb{R}$ and $\emptyset$.
\\

The way to do these sorts of proves is to look for a contradiction, but before we tak that path,
let's verify our conditions.
Starting out simple, $\emptyset$. The $\emptyset$ is both open and close by matter of convention, notation,
etc.

Now $\mathbb{R}$. $\mathbb{R}$ is open because every element of the reals has an $\epsilon$-neighborhood
that contains only reals.
The reals are also closed because any limit point that it can have will be a real itself, so
all limit points of $\mathbb{R}$ are members of $\mathbb{R}$.

Now, let's fish for contraditions.
One way is to use what we found in \ref{abbott:3.2.4}, that is: open sets cannot contain their supremum (or infinimum).
So let's say there is some $A$ that is both open and closed but it is not $\mathbb{R}$ or $\emptyset$.
If this is the case, then $A$ must be unbounded (it has no supremum or infinimum since it is open).
And if $A$ has continuous [$\epsilon$-neighborhoods] elements, then it will be open.
But since $A$ is closed, then it must also contain all of its limit points,
So $A$ is a set that is unbounded, contains all of its limit points, and has continuous $\epsilon$-neighborhoods.
The only set having that property are the reals.

Another route of argument is to note that if $A$ is open and closed, then $A^c$ must also be open and closed.
Furthermore, if we want $A^c \neq \emptyset$, and again we run into trouble because $A^c$ must also be unbounded
but contain all of its limit points while not being equal to the reals.
And also worth mentioning that the limit points don't have $\epsilon$-neighborhoods contained within $A$ or $A^c$ for that matter.
\\~\\



\textbf{3.2.14}
\\

A dual notion to the closure of a set is the interior of a set.
The interior of $E$ is deonted $E^o$ and defined as
$$
E^o = \{ x\in E : \exists V_\epsilon (x) \subseteq E \}
$$

Results about closures and interiors posses a useful symmetry.
\\

(a) Show that $E$ is closed if and only if $\overline{E} = E$.
Show that $E$ is open if and only if $E^o = E$.
\\

We know that $\overline{E} = E \cup L$, where $L$ is the set of limit points of $E$.
If we have $\overline{E} = E \cup L = E$, then $L \subseteq E$ ($E$ must contain all of its limit points).
Thus we agreement with out previous defintion of closed sets.

Similarly, if $E^o = E$, then it means that every $x\in E$ has an $\epsilon$-neighborhood that is contained
in $E$, again matching our definition.
\\


(b) Show that $\overline{E}^c = (E^c)^o$, and similarly that $(E^o)^c = \overline{E^c}$.
\\

First, let's shake it and see what comes out
$$
\overline{E}^c = (\overline{E})^c = (E \cup L)^c = E^c \cap L^c
$$
If $E$ is closed, then $E=\overline{E}$, and $E^c = (E^c)^o$.
So right away we can simplify $\overline{E}^c = (\overline{E})^c = E^c = (E^c)^o$.

Now, if $E$ is open, then $E = E^o$, and $E^c = \overline{(E^c)}$.
Doing the same sort of manipulation as above would actually give us the second equality we are looking to prove.
So great, but let's try something else so we can prove this second case.
If we look at $\overline{E}^c = (\overline{E})^c = (E \cup L)^c$, then we are looking into the universe of elements
that do not belog to $E$ and are not limit points of $E$.
Since these elements are do not intercept $E$, then they are contained within $E^c$.
However, being contained within $E^c$ does not mean that there is an $V_\epsilon (x) \subseteq E^c$
for every $x\in E^c$.
This is only the case if $x\in (E^c)^o$, otherwise, there will be areas of $E$ where an $\epsilon$-neighboorhod
will contain points outside of $E^c$.
\\~\\



\textbf{3.2.15}
\\

A set $A$ is called an $F_\sigma$ set if it can be written as the countable union of closed sets.
A set $B$ is called a $G_\delta$ set if it can be written as the countable intersection of open sets.
\\

This exercise is lovely, first, because we saw in \textbf{theorem 3.2.14 (i)} that the union of a
finite collection of closed sets is closed.
And then in \textbf{theorem 3.2.3 (ii)} that the intersection of a finite collection of open sets is open.

Since a countable amount of sets can be either finite or inifinite, it must mean that we can form
sets $F_\sigma$ and $G_\delta$ by looking at a countably inifite amount of sets in each case.
Otherwise we would end with contradicitons.
\\

Some great pages to read are \href{https://en.wikipedia.org/wiki/F%CF%83_set}{wiki page on $F_\sigma$ sets}
and \href{https://en.wikipedia.org/wiki/G%CE%B4_set}{wiki page on $G_\delta$ sets}.
\\


(a) Show that a closed interval $[a,b]$ is a $G_\delta$ set.
\\

The interesting bit here is that we have to figure out how to use the intersection of a countable amount
of open sets to form a closed set.
The countable amount must be infitine, as otherwise a finitie intersection of open sets results in an open set.

With this countably infinite intersection of open sets, we must be able to create a set that contains all of its
limit points.

Approaching this problem "geometrically" helps a tad: open sets contain neighborhood that are completely contained within
themselves but do not contain any of their limits.
Requiring at least one $\epsilon$-neighborhood for each member of an open set to be contained within itself
gives us a sense of "continuity" - a lack of "holes" - while a lack of limit points means that we can get close
to them and as the intersection of open sets looks for an inifite amount of open sets, for the limit points should
show uo themselves as the limit of an inifite number of open sets.

In short, something like
$$
\bigcap_{n\in\mathbb{N}} \left( a-\frac{1}{n}, b+\frac{1}{n} \right)
$$
Could just be the trick we need since each individual set is open but as $n\rightarrow\infty$,
then $1/n \rightarrow 0$, so the endpoints of the open intervals get closer and closer to $a$ and $b$
and they always contain $a$ and $b$.
\\~\\


(b) Show that the half-open interval $(a,b]$ is both a $G_\delta$ and an $F_\sigma$ set.
\\

Similar logic as above, we could have the $G_\delta$ set
$$
\bigcap_{n\in\mathbb{N}} \left( a, b+\frac{1}{n} \right)
$$
or the $F_\sigma$ set
$$
\bigcup_{n\in\mathbb{N}} \left[ a+\frac{1}{n}, b \right]
$$
For the case of the $F_\sigma$ set, note how we look at $a + \frac{1}{n}$, so in this case we are always
using an endpoint to the right of $a$, so the lower bound never contains $a$ but it approaches it as $n\rightarrow\infty$.
\\~\\


(c) Show that $\mathbb{Q}$ is an $F_\sigma$ set, and the set of irrationals $\mathbb{I}$ forms a $G_\delta$ set.
(We will see in section 3.5 that $\mathbb{Q}$ is not a $G_\delta$ set, nor $\mathbb{I}$ is an $F_\sigma$ set.)
\\

Since $\mathbb{Q}$ is countable, then $\cup_{q\in\mathbb{Q}} \{q\} = \mathbb{Q}$.

To see that the irrationals form a $G_\delta$ set, let's take the compliment of the above,
$$
\mathbb{I} = \mathbb{Q}^c = \left( \cup_{q\in\mathbb{Q}} \{q\}  \right) = \cap_{q\in\mathbb{Q}} \{q\}^c
$$

To finish the argument quickly, note that since $\{q\}$ is closed, then $\{q\}^c$ must be open.

Now, to carry the conversation into a future topic, see
\begin{enumerate}
    \item \href{https://math.stackexchange.com/questions/1459067/are-singletons-always-closed}{are singletons always closed?}
    \item \href{https://math.stackexchange.com/questions/17649/are-singleton-sets-in-mathbbr-both-closed-and-open}{Are Singleton sets in R both closed and open?}
\end{enumerate}

The general argument for why a singleton is closed in the topology of $\mathbb{R}$ is quite interesting.
To show that $\{x\}$ is closed, you want to argue that $\mathbb{R}\setminus\{x\}$, its complement, is open.
Again, in an open set, every member has an $\epsilon$-neighborhood contained within their set.
So we want to see that for any $a\in\mathbb{R}\setminus\{x\}$, $V_\epsilon (a) \subseteq \mathbb{R}\setminus\{x\}$.

If we look at the $V_\epsilon (a)$ where $\epsilon = |a - x|$, then we are looking at the open interval
$(a-\epsilon, a+\epsilon)$.
Since $\epsilon$ is the distance from $a$ to $x$, then $a \pm \epsilon$ puts us right on-top of $x$,
but since an $\epsilon$-neighborhood is an open interval, then $x \not\in (a-\epsilon, a+\epsilon)$,
meaning that $V_\epsilon (a) \subseteq \mathbb{R}\setminus\{x\}$.
Hence $\mathbb{R}\setminus\{x\}$ is open.

The logic to the above argument is similar to Example 3.2.2 (ii).
\\~\\







%%%%%%%%%%%%%%%%%%%%%%%%%%%%%%%%%%%%%%%%%%%%%%%%%%%%%%%%%%%%%%%%%%%%%%%%%%%%%%%%%%%%%%
\subsection{Compact Sets}


%%%%%%%%%%%%%%%%%%%%%%%%%%%%%%%%%%%%%%%%%%%%%
\subsubsection{Exercises}


\textbf{3.3.1}
\\

Show that if $K$ is compact and nonempty then $\sup K$ and $\inf K$ both exist and are elements of $K$.
\\

Being compact means that $K$ is closed and bounded, and a bounded set of real numbers will have a supremum and infinitum
(axiom of completeness).

Furthermore, since the supremum and infinimum are limit points and are thus contained within $K$.
Lemma 1.3.8 says that for every $\epsilon > 0$, there exists an element $a\in K$ such that $s - \epsilon < a$.
So we can get arbitrarily close to the supremum $s$ and form a sequence such that $|a - s| < \epsilon$.
\\~\\



\phantomsection
\label{abbott:3.3.2}
\textbf{3.3.2}
\\

Decide which of the following sets are compact.
For those that are not compact, give an example of a sequence contained in the given set that does not posses
a subsequence converging to a limit in the set.

Note that this problem is similar to \ref{abbott:3.2.3}.
\\

(a) $\mathbb{N}$
\\

The set of natural numbers is not compact.
The natural numbers themselves are a seuqnece that does not converge.
A simple series to show this, $n, n\in\mathbb{N}$.
\\

(b) $\mathbb{Q} \cap [0,1]$
\\

Remember that the reals are the limit points of the rational numbers.
However, the intersection of the rationals and the closed interval $[0,1]$
is the set of rationals in the interval $[0,1]$ (including 0 and 1).
Since the intersection does leave us with the endpoints then we do have some of the limit points
but not all of them (since that would require the interval to have all of the reals between 0 and 1 as well).

A simple series to make our point, $(x_n) \rightarrow 1/\sqrt{2}$.
\\

(c) The cantor set
\\

This one is a bit more surprising, one could think that maybe the cantor set is not compact
becuase maybe the reals are the limit points for it, as the case for the rational numbers, but this is not the case!

First, the cantor set is bounded since $C \subset [0,1]$.
Furhtermore, the cantor set, $C$, is made up of an infitine intersection of closed sets (Abbott gives us that hint).
And as per theorem 3.2.14 (ii), the intersection of an arbitrary collection of closed sets is open.
Hence, the cantor set is bounded and close and thus compact.

But before we move on, there is another lovely argument to be made,
The cantor set is defined as
$$
C = [0,1] \setminus 
\left[
    \left(\frac{1}{3}, \frac{2}{3}\right) \cup
    \left(\frac{1}{9}, \frac{2}{9}\right) \cup
    \left(\frac{7}{9}, \frac{8}{9}\right) \cup
    \ldots
\right]
$$
Its complement is
$$
C^c = (-\infty, 0) \cup (1, \infty) \cup
\left[
    \left(\frac{1}{3}, \frac{2}{3}\right) \cup
    \left(\frac{1}{9}, \frac{2}{9}\right) \cup
    \left(\frac{7}{9}, \frac{8}{9}\right) \cup
    \ldots
\right]
$$
This time we get to use theorem 3.23 (i) which says that the union of of an arbitrary collection of oepn sets is open.
And since $C^c$ is open, then $C$ must be closed.
\\

(d) $\left\{ 1 + \frac{1}{2^2} + \frac{1}{3^2} + \ldots + \frac{1}{n^2} : n\in\mathbb{N} \right\}$
\\

This is one of our favourite p-series.
Since $p>1$, then it converges, and this one in particular converges to $\pi^2 / 6$.
Since the series converges, then its bounded.
Since its bounded, then by the Bolzano-Weierstrass theorem there is a subsequence that converges.
The only thing we need if for the limit to be within the set.

One could make the argument that based on the criteria for convergence, if we find an $n$ that is sufficiently large
then the limit point "is in".
However this argument falls short. The sum approaches its limit, but the sum of rationals is not exactly equal
to the irrational number. 
\\

(e) $\left\{ 1, \frac{1}{2}, \frac{2}{3}, \frac{3}{4}, \frac{4}{5}, \ldots \right\}$
\\

This sequence converges to 1, and since 1 is in the set then it is closed.
It is also bounded.
Hence it is compact.
\\~\\



\textbf{3.3.3}
\\

Prove the converse of theorem 3.3.4 by showing that if a set $K \subseteq \mathbb{R}$ is closed and bounded,
then it is compact.
\\

For $K$ to be compact, every sequence in $K$ must have a subsequence that converges to a limit that
is also in $K$.

We know that $K$ is bounded, so there exists an $M > 0$such that $|a| < M$ for all $a\in K$.
Because every $|a| < M$, then every sequence in $K$ is bounded.
Then, the Bolzano-Weierstrass theorem then tells us from any of such sequences that we can form, there will be a
subsequence that will converge to some limit.

Furthermore, because a closed set contains all of its limit points, then every limit from any of the
convergent subsequences that can be made will be contained within $K$.
This last statement is the defintion of compactness.
\\~\\




\textbf{3.2.4}
\\

Assume $K$ is compact and $F$ is closed.
Decide if the following sets are compact, closed, both, or neither.
\\

(a) $K \cap F$
\\

As per theorem 3.2.14 (i), the union of a finite collection collection of closed sets is closed.
So this union is closed.
Since $K$ is bounded, then the intersection of the two sets will also be bounded.
Thus the intersection is compact.
\\

(b) $\overline{F^c \cup K^c}$
\\

$F^c$ is open, and $K^c$ is open and unbounded ($[0,1]^c = (-\infty,0) \cup (1, \infty)$).
The union of an arbitrary collection of open sets is open.
Here comes an interesting bit, the complement of a bounded set is an unbounded set but the complement of an
unbounded set is not necessarily a bounded set (i.e., $(0,\infty)^c = (-\infty,0]$).
So $F^c \cup K^c$ is open and could be bounded or unbounded.

The closure then is the union of $F^c \cup K^c$ and its set of limit points.
So the set is closed but not necessarily bounded.
\\

(c) $K\setminus F = \{ x \in K : x\not\in F \}$
\\

Since $K \setminus F = K \cap F^c$, we know right away that the resulting set is bounded.
However, we have the intersection of a closed set and an open set so we need more information to give a defintie answer.
Some examples, if $K = F = [0,1]$, then $K \setminus F$ is open. - we are taking away the limit points.
If $K = [0,1]$ and $F^c = (2,3)$, then $K \setminus F$ is empty, so it is open and closed.
\\

(d) $\overline{K \cap F^c}$
\\

Borrowing from our previous answer, we have the intersection of a closed and bounded set with an open set.
So the resulting set is bounded.
We got this far last time, but this time we are looking at the closure of the same set, and since the closure of
a set always results in a closed set, then we end up with something that is both bounded and closed, thus
compact.
\\~\\



\textbf{3.3.7}
\\

As some more evidence of the surprising nature of Cantor set, follow these steps to show that the sum
$C + C = \{ x+y : x,y\in C \}$
is equal to the closed interval $[0,2]$.
(Keep in mind that $C$ has zero length and contains no intervals.)

Because $C \subseteq [0,1]$, $C + C \subseteq [0,2]$, so we only need to prove the reverse inclusion
$[0,2] \subseteq C + C$.
Thus, given $s \in  [0,2]$, we must find two elements $x,y \in C$ satisfying $x + y = s$.
\\

(a) Show that there exists $x_1, y_1 \in C_1$ for which $x_1 + y_1 = s$.
Show in general that, for an arbitrary $n \in \mathbb{N}$, we can always find $x_n, y_n \in C_n$
for which $x_n + y_n = s$.
\\

Finding an $x,y \in C_0$ such that $x + y = s$ is the trivial case.
Now, to find an $x_1, y_1 \in C_1$.

$C_1 = [0,1] \setminus \left(\frac{1}{3}, \frac{2}{3}\right) = \left[0, \frac{1}{3}\right] \cup \left[\frac{2}{3}, 1\right]$.
And with $C_1 + C_1 = \left( \left[0, \frac{1}{3}\right] \cup \left[\frac{2}{3}, 1\right] \right) + \left( \left[0, \frac{1}{3}\right] \cup \left[\frac{2}{3}, 1\right] \right)$
we can cover the entire range of $[0,2]$ including the middle third we took out because any $x_1 \in \left[0, \frac{1}{3}\right]$
added to any $y_1 \in \left[0, \frac{1}{3}\right]$ will cover the range of $\left[0, \frac{2}{3}\right]$.

Similarly, any $x_1 \in \left[0, \frac{1}{3}\right]$ added to any $y_1 \in \left[\frac{2}{3}, 1\right]$
will cover the range of $\left[\frac{2}{3}, 1\frac{1}{3}\right]$.

Lastly, any $x_1 \in \left[\frac{1}{3}, \frac{2}{3}\right]$ added to any $y_1 \in \left[\frac{2}{3}, 1\right]$
will cover the range of $\left[1\frac{1}{3}, 2\right]$.

so, from our possible combinations we end up with
$x_1 + y_1 \in \left[0, \frac{2}{3}\right] \cup \left[\frac{2}{3}, 1\frac{1}{3}\right] \cup \left[1\frac{1}{3}, 2\right] = [0,2]$.
symbolically, $C_1 + C_1 = [0,2]$.
This forms the base case for induction.
Now we need to formulate the inductive step.
\\

To help us take the inductive step, there is an interesting pattern to notice.
For example, $C_1 = \left[0, \frac{1}{3}\right] \cup \left[\frac{2}{3}, 1\right]$
and $C_2 = \left( \left[0, \frac{1}{9}\right] \cup \left[\frac{2}{9}, \frac{1}{3}\right] \right) \cup \left( \left[\frac{2}{3}, \frac{7}{9}\right] \cup \left[\frac{8}{9}, 1\right] \right)$.
If you compute $3\cdot C_2$ you'll get
$3C_2 = \left( \left[0, \frac{1}{3}\right] \cup \left[\frac{2}{3}, 1\right] \right) \cup \left( \left[2, 2\frac{1}{3}\right] \cup \left[2\frac{2}{3}, 3\right] \right)$.
Which we can rewrite as $3C_2 = C_1 \cup \left( 2 + C_1 \right)$.
So if we symbolically repeat the steps we performed in the base step, we get
\begin{align*}
3C_2 + 3C_2 &= \left( C_1 \cup \left( 2 + C_1 \right) \right) + \left(C_1 \cup \left( 2 + C_1 \right) \right) \\
    &= \left( C_1 + C_1 \right) \cup \left( C_1 + (2 + C_1) \right) \cup \left(4 + C_1 + C_1 \right) \\
    &= [0,2] \cup [2,4] \cup [4,6] \\
    &= [0,6]
\end{align*}
Note in particular the middle term that resulted in the interval $[2.4]$.
If we were doing a simple $(x+y) + (x+y)$ then we just get $2x + 2y$.
But in the case of $(x \cup y) + (x \cup y)$ it equals $(x+x) \cup (x+y) \cup (x+y) \cup (y+y) = (x+x) \cup (x+y) \cup (y+y)$.
We made the previous reasoning by thinking about $a + (x \cup y)$ being equal to $a+x \cup a+y$ and generalizing from there.

To complete the inductive step, generalize the above by changing $c_1$ to $C_n$ and $C_2$ to $C_{n+1}$.
\\

(b) Keeping in mind that the sequences $(x_n)$ and $(y_n)$ do not necessarily converge,
show how they can nevertheless be used to produce the desired $x,y \in C$ satisfying $x + y = s$.
\\

$(x_n)$ and $(y_n)$ may not necessarily converge since the $x$s and $y$s are being picked from
non-continuous intervals.

Back in problem \ref{abbott:3.3.2}, we showed that the cantor set is compact,
So every sequence in $C$ has a subsequence that converges to a limit that is also in $C$.
So we can chose some $(x_{n_k})$ and some $(y_{n_k})$ such that they converge to $x$ and $y$
respectively. Since $x$ and $y$ are contained in $C \subseteq C$, then $x + y = s \in [0,2]$.

A similar line or argument can be made by noting that since $C$ is bounded, then $(x_n)$ and $(y_n)$ are bounded,
and the Bolzano-Weierstrass theorem tells us that we can find a subsequence that does converge.
From there, the same reasoning can be reused.
\\~\\



\textbf{3.3.8}
\\

Let $K$ and $L$ be nonempty compact sets, and define
$$
d = \inf\{ |x-y| : x\in K, y\in L \}
$$
This turns out to be a reasonable definition for the distance between $K$ and $L$.
\\

(a) If $K$ and $L$ are disjoint, show $d>0$ and that $d = |x_0 - y_0|$ for some $x_0 \in K$ and $y_0 \in L$.
\\

We'll combine the answer to this with the answer to the next question.
\\

(b) Show that it's possible to have $d = 0$ if we assume only that the disjoint sets $K$ and $L$ are closed.
\\

These two questions are super interesting.
Conventional experience would have us think that as long as two sets are disjoint, then $d > 0$
since the two sets do not share any elements so $x\neq y$.
However the second question gives us a hint: when the two sets are closed then somehow someway $d = 0$.
Which makes us think that being able to have a zero "distance" is somehow possible due to the difference between
a pair of sets that are closed and bounded and other pair that is just closed.

Because we are comparing a pair that is closed and bounded with a pair of sets that is just closed,
we may be thinking of the same boring old set of sets in which to be closed you must be bounded.
However, if we think back to limit points, we can have a set that is closed if the set doesn't have any limit points
(it doesn't have to contain its limit points if these don't exist).

For example, if $K = {n : n\in\mathbb{N}}$ and $L = \{ n + \frac{1}{n} : n\in\mathbb{N} \}$, then neither set has
limit points.
And if there are no limit points then they can be considered as closed.

However, if $K$ and $L$ are compact, then any sequence contained in them has a subsequence that converges
to some limit that is also contained within them.
From there, you can define $d = \lim_{n\rightarrow \infty} |x_n - y_n| = |x_0 - y_0|$.
\\




\textbf{3.3.9}
\\

Follow these steps to prove the final implication of theorem 3.3.8.

Assume $K$ satisfies i) $K$ is compact and ii) $K$ is closed and bounded, and let
$\{ O_\lambda : \lambda \in \Lambda \}$ be an open cover for $K$.
For contradiciton, let's assume that no finite subcover exists.
Let $I_0$ be a closed interval containing $K$.
\\

(a) Show that there exists a nested sequence of closed intervals $I_0 \supseteq I_1 \supseteq I_2 \supseteq \ldots$
with the property that, for each $n$, $I_n \cap K$ cannot be finitely covered and $\lim |I_n| = 0$.
\\

One way to do this is to follow the sort of thing we saw in Example 3.3.7, since we saw that by dividing an
interval we could obtain subsets that had no open cover.
So we can define $|I_n| = |I_0| / 2^{n-1}$ (we do $n-1$ so that $I_{n=1} = I_0$).

Again, following the sort of reasoning we saw in example 3.3.7, we can see that these $I_n \cap K$ cannot be finitely
covered and that $\lim I_n = \lim |I_0|/2^{n-1} = 0$.
\\

(b) Argue that there exists an $x\in K$ such that $x \in I_n$ for all $n$.
\\

The nested interval property tells us that there msut exist some $x \in I_n$ for all $n$.
So we need to show that that $x$ is also in $K$.
This additonal detail comes from looking at our previous answer: since $K$ is compact, then $I_n \cap K$
will also be compact, and now we can use the nested compact set property which tells us that
there is an $x \in I_n \cap K$ for all $n$, so $x\in K$ and $x\in I_n$.
\\

(c) Because $x\in K$, there must exists an open set $O_{\lambda_0}$ from the original collection that
contains $x$ as an element.
Explain how this leads to the desired contradiciton.
\\

We began this whole journey assuming that there was an open cover for $K$, so as there must be an
$x \in I_n \cap K$, so too must there be an $x \in O_{\lambda_0}$.
Now, since $\lim |I_n| = 0$, then we should be able to find some $I_{n_0} \subseteq O_{\lambda_0}$,
that is an $I_{n_0}$ that fits in $O_{\lambda_0}$.
Which finally leads us to a contradiction because we assumed that $I_n \cap K$ could not be finitely covered.
\\~\\




\textbf{3.3.10}
\\

Here is an alternate proof to the one from given in Exercise \textbf{3.3.9}
for the final implication in the Heine-Borel theorem.

Consider the special case where $K$ is a closed interval.
Let $\{ O_\lambda : \lambda \in \Lambda \}$ be an open cover for $[a,b]$
and define $S$ to be the set of all $x \in [a,b]$ such that
$[a,x]$ has a finite subcover from $\{ O_\lambda : \lambda \in \Lambda \}$.
\\


We start with knowledge that $\{O_\lambda\}$ is an open cover for $[a,b]$.
Meaning that $[a,b] \subseteq \bigcup_{\lambda \in \Lambda} Q_\lambda$.
Remember that the union of an arbitrary number of open sets is itself an open set, so an open cover is itself open.
Furthermore, every $O_\lambda$ is open, meaning that for every $a \in O_\lambda$, there exists
and $V_\epsilon (a) = \{ x\in\mathbb{R} : |x-a| < \epsilon \} \subseteq O_\lambda$.

Thus, since $x \in [a,b] \subseteq \bigcup_{\lambda} Q_\lambda$, then there exists
$V_\epsilon (x) \subseteq \bigcup_{\lambda} Q_\lambda$ for all $x \in S$.
\\


(a) Argue that $S$ is nonempty and bounded, and thus $s = \sup S$ exist.
\\
$S$ is nonempty since $x=b$ has the entire cover and $x=a$ must have some cover $\{ O_\lambda \}$,
as long as $a \ in \{O_\lambda \}$ for some $\lambda \in \Lambda$.

Furthermore, $S$ is also bounded since our cover if for $[a,b]$ and $x \leq b$ for all $x \in S$.
\\


(b) Now show $s = b$, which implies $[a,b]$ has a finite subcover.
\\

Making an allusion to the definiton of the supremum: let's say that $s < b$, then because $[a,b]$ does
have an open cover, then these should be some $s < y$ such that $[a,y]$ also has a cover, so $\sup S$ wouldn'T
equal this value that is less than $y$, our first constradiction.

And if we say that $s > b$, then we are demanding the existence of another neighboorhod that will cover
the difference between $s$ and $b$, thus leading to another contradiciton.

Since our cover includes the endpoints of our interval, we can mimick of construction of a finite subcover from
back in example 3.3.7, one big open set that goes $(a-\epsilon , b+\epsilon)$.
\\


(c) Finally, prove the theorem for an arbitrary closed and bounded set $K$.
\\

The process we just argued is generalizable to any $a$ and $b$ for simple intervals.
But if there were gaps, and we were looking at sets, then we would need to look at the infinimum and supremum
of the set to build a finite open subcover.

And if the set did not contain any limit points, that's where the last equivalence with the other definitions of
compactness comes into play: we need the set to be bounded, otherwise we wouldn't be able to construct a finite subcover.
Think back to Exercise \ref{abbott:3.2.3} and how $\mathbb{N}$ are a closed interval.
Our construct only works if the supremum and infinimum exist.
\\~\\








%%%%%%%%%%%%%%%%%%%%%%%%%%%%%%%%%%%%%%%%%%%%%%%%%%%%%%%%%%%%%%%%%%%%%%%%%%%%%%%%%%%%%%%%%%%
\subsection{Perfect Sets and Connected Sets}


%%%%%%%%%%%%%%%%%%%%%%%%%%%%%%%%%%%%%%%%%%%%%%%%%%%%%%%%%%%%
\subsubsection{Exercises}