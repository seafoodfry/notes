\section{Basic Topology of R}

The empy set $\emptyset$ is considered an open subset of the real line, becuase if it wans't it would imply
that there is something in the emptyset that is "closed"
(negation of "for all $a\in O$ there exist a $V_\epsilon \subseteq O$").
\\

There is a theorem mentioned in topology that goes something along the lines of:
the union of two non-disjoint intervals is an interval.
The idea behind it being that in order for something to be an interval, it must must not have gaps
between its endpoints.
For example, the union of $[0,1] \cup [2,3]$ is not an internal since it doesn't include $[1,2]$.
\\

Open balls or $\epsilon$-neighborhoods make sense right away until you ought to give a proper argument.
The way to read the definitions for these is to think that you always want a "ball" around some
point, and you want to see if balls of any radius will be able to fit within another interval.
For example, in Abbott example 3.2.2(ii), we take $\epsilon$ to be some positive value by
looking at the difference between the point $x\in(c,d)$ and its left and right endpoints.
($x-c \geq 0$ and $d-x \geq 0$).
And that definition of $\epsilon$ gives you a simple way to look for open balls in $(c,d)$.

To see the above, draw out a line and see how no matter what radius you end up using, as long as
it is less than $\epsilon$, all the open balls you draw will fit within $(c,d)$.
\\

\textbf{Theorem 3.2.3}

Note the interesting bit of logic being used here: since $O_{\lambda^\prime}$ is open then
$V_{\epsilon}(a) \subseteq O_{\lambda^\prime}$.
Since $O_{\lambda^\prime}$ is some arbitrary member of $O = \cup_{\lambda \in \Lambda} O_{\lambda}$,
then $O_{\lambda^\prime} \subseteq O$.
Putting these together $V_{\epsilon}(a) \subseteq O_{\lambda^\prime} \subseteq O$.
\\

\textbf{Definition 3.2.4}
By now, we've seen that $\epsilon$-neighboorhods are equivalent to open balls,
$$
V_{\epsilon} (a) \sim B(r=\epsilon, a) \sim \{ \mathbf{x}\in\mathbb{R}^n : |\mathbf{x} - \mathbf{a}| < r(=\epsilon) \}
$$
The neighboorhod part comes from $x\in S^{int} = \{ \mathbf{x}\in S : B(r,\mathbf{x}) \in S \text{ for some } r>0 \}$.
Which is a math way of saying that a neighboorhod consist of all points close to our center who are
still members of the set in question.
\\

Folland gives a conventional view of a closed set by first defining \textbf{boundary points}
of a set as points where every ball centered on them contains points both in $S$ and $S^c$,
These points may then belong to either the set in question or its compliment ($x\in S \cup S^v$).
And the set of all boundary points creates a boundary for $S$
$$
\partial S = \{ \mathbf{x}\in\mathbb{R}^n : B(r,\mathbf{x})\cap S \neq \emptyset \text{ and }
    B(r,\mathbf{x})\cap S^c \neq \emptyset \}
$$

However, instead of following this route, Abbott goes onto to construct closed intervals
with limit points which instead of relying on geometry, rely on our knowledge of series.