
%%%%%%%%%%%%%%%%%%%%%%%%%%%%%%%%%%%%%%%%%%%%%%%%%%%%%%%%%%%%%%%%%%%%%%%
\section{Rearrangements of infinite series}


%%%%%%%%%%%%%%%%%%%%%%%%%%%%%%%%%%%%%%%%%%%%%%%%%%%%%%%%%%%%%%%%%%%%%%%
\subsection{Limit of a sequence}


%%%%%%%%%%%%%%%%%%%%%%%%%%%%%%%%%%%

\subsubsection{Exercises}

\textbf{2.2.2}

$$
\lim \frac{2n+1}{5n+4} = \frac{2}{5}
$$

We want to show that,
$$
\left| \frac{2n+1}{5n+4} - \frac{2}{5} \right| < \epsilon
$$
For any $\epsilon > 0 \in \mathbb{R}$ as long as $n \geq N \in \mathbb{N}$.
To find $N$,

$$
\left| \frac{2n+1}{5n+4} - \frac{2}{5} \right| =
\left| \frac{5(2n+1) - 2(5n+4)}{5(5n+4)}  \right| =
\left| \frac{-3}{5(5n+4)} \right| =
\frac{3}{25n+20} < \epsilon
$$

We can do the last simplification because $N$ is some positive number.
Simplifying the above we get,
$$
3 < \epsilon (25N + 20)
$$
or
$$
\frac{3}{25\epsilon} - \frac{20}{25} = \frac{3}{25\epsilon} - \frac{4}{5} < N
$$

Note that we could obtain a simpler expression if we note that
$$
\frac{3}{5(5n+4)} < \frac{1}{n} < \epsilon
$$

So $N > 1/\epsilon$, which follows the same pattern as our previous answer.
The first answer gives us a more accurate convergence rate.
\\

$$
\lim \frac{2n^2}{n^3 + 3} = 0
$$

Following similar logic, we are looking at
$$
\left| \frac{2n^2}{n^3 + 3} \right| < \epsilon
$$

Plain algebra doesn't really help us isolate $n$, so again look for patterns.
$$
\frac{2n^2}{n^3 + 3} < \frac{2n^2}{n^3} = \frac{2}{n} < \epsilon
$$
\\

Now for,
$$
\lim \frac{\sin (n^2)}{n^{1/3}} = 0
$$

$\sin$ will never be greater than one, so we can proceed as follows.
$$
\frac{\sin (n^2)}{n^{1/3}} < \frac{1}{n^{1/3}} < \epsilon
$$

So $N > 1 / \epsilon^3$.
\\~\\



\textbf{2.2.6}

\textbf{Uniqueness of limites:} if the limit of a sequence that converges were not unique, then we
should be able to find any other number that meets our requirements - trying
to prove that there are multiple limits for a generic sequence is hard, same as trying to then
generalize this result to any sequence.
Intead our approach will be as follows.

If the sequence $a_n$ converges to $a$, then let's assume that there it also converges to $b$.
If we see that $a = b$, then we have our proof, else, we have a great publication.
So we know that

$$
| a_n - a | < \epsilon
$$
and
$$
| a_n - b | < \epsilon
$$

For some arbitrary epsilon.
Given the definition of converge, the above requirements must hold for some $N \in \mathbb{N}$, but
that $N$ does not need to be the same for the two expressions above.
So let's follow along using some $N = \max (N_1, N_2)$, where $N_1$ and $N_2$ are the numbers that would
make the above expressions hold for any $\epsilon$.

We then have,

$$
| a_N - a | < \epsilon
$$
and
$$
| a_N - b | < \epsilon
$$

Now, we are assuming that $a \neq b$, meaning that $| a - b | > 0$.
If we use the triangle inequality,
$$
| a - b | = | a - a_N + a_N - b | = | (a - a_N) + (a_N - b) |
$$
$$
\leq | a - a_N | + | a_N - b |
$$

Since $|x-y| = |y-x|$, we then have
$$
| a - b | \leq | a - a_N | + | a_N - b | = | a_N - a | + | a_N - b |
$$
$$
< \epsilon + \epsilon = \epsilon'
$$

Interestingly enough, we already saw a theorem of this sort back in theorem 1.2.6, which stated that
two real numbers $a$ and $b$ are equal iff $\forall \epsilon > 0$ it follows that $|a-b|<\epsilon$.
This is exactly what we are trying to get at.
But for the sake of argument, we could also see that we reach a contradiction if we continue.
Let's assume $\epsilon = 1/10$, then,

$$
|a-b| < 2\epsilon = \frac{2}{10}|a-b|
$$
And a number cannot be less than (a fraction of) itself.
\\~\\





%%%%%%%%%%%%%%%%%%%%%%%%%%%%%%%%%%%%%%%%%%%%%%%%%%%%%%%%%%%%%%%%%%%%%%%
\subsection{The algebraic and order limit theorems}

why does $|b_n - b| < |b|/2$  implies $|b_n| > |b|/2$ ?

We have been seeing that for a lot of convergence proofs we end up using the triangle inequality.
And this is a good tool to try out here!
Let's try this out firsrt,

$$
|b_n| = | b_n - b + b | \leq |b_n - b| + |b| < \frac{|b|}{2} + |b| = \frac{3|b|}{2}
$$

So we get a sense that we have to multiply $|b|$ by a number greater than 1 in order to go above
$|b_n|$.
But its a weak argument.
Let's try it the other way now!

$$
|b| = |b - b_n + b_n| \leq |b - b_n| + |b_n| = |b_n -b| + |b_n| < \frac{|b|}{2} + |b_n|
$$
Which can be simplified to
$$
\frac{|b|}{2} < |b_n|
$$

Which is the bit we did need.
\\~\\

In the \textbf{order limit theorem} proof we also see the statement: $|a_N - a| < |a|$ implies
that $a_N < 0$.
One way to see this is by noting that
$$
|a_N - a| < |a| \rightarrow -|a| < a_N - a < |a|
$$

If $a<0$, then $-|a| = a < 0$ and $|a| = -a > 0$, so
$$
-|a| < a_N - a < |a| \rightarrow a < a_N - a < -a \rightarrow 2a < a_N < 0
$$

The above tells us that in this case we would land with $a_N < 0$.
\\~\\


\subsubsection{Exercises}

\textbf{2.3.1}

If $x_n \rightarrow 0$, then we know that $|x_n| < \epsilon$ when $n \geq N$.
Now let's look at $|\sqrt{x_n}| = \sqrt{x_n}$.

A useful thing to note here is that if $x_n \geq 0$, then the order limit theorem says
that $x\geq 0$.

Coming back to $|\sqrt x_n| = \sqrt{x_n}$ we can also see that $|x_n| = x_n < \epsilon$ in this case,
so $ = \sqrt{x_n} =\sqrt |x_n|$. Note that we can now make the
value under the square root as small as we want
$$
|\sqrt x_n | < \sqrt |x_n| < \epsilon
$$
\\~\\

If $x_n \rightarrow x$, then
$$
| \sqrt x_n - \sqrt x | = | \sqrt x_n - \sqrt x | \left| \frac{\sqrt x_n + \sqrt x}{\sqrt x_n + \sqrt x} \right|
$$
$$
= \frac{|x_n - x|}{\sqrt x_n + \sqrt x}
$$

Since $\sqrt x_n + \sqrt x \geq \sqrt x$, then we could further simplify the above expression to
$$
\frac{|x_n - x|}{\sqrt{x_n} + \sqrt{x}} \leq \frac{|x_n - x|}{\sqrt{x}}
$$

Because a smaller denominator, results in a larger overall quantity.
We now know that the numerator matches the expression that we know convrges, so we can now chose
a $\epsilon' = \epsilon \sqrt{x}$, so that

$$
\frac{|x_n - x|}{\sqrt{x}} < \frac{\epsilon \sqrt{x}}{\sqrt{x}} = \epsilon'
$$
\\~\\


\textbf{2.3.2}

We know that $x_n \rightarrow 2$, so if we look at $ \frac{2x_n -1}{3} $, we can apply
the same operations to $|x_n \rightarrow 2|$,
$$
\left| \frac{2x_n -1}{3} - \frac{2(2) -1}{3} \right| = \left| \frac{2x_n -1}{3} - \frac{3}{3} \right|
$$
$$
= \frac{1}{3} |(2x_n - 1) - 3| = \frac{1}{3} | 2x_n - 4 | = \frac{2}{3} |x_n - 2| < \frac{2}{3} \epsilon
$$
\\~\\

Let's now look at $1/x_n$.
$$
\left| \frac{1}{x_n} - \frac{1}{2} \right| = \left| \frac{2 - x_n}{2x_n} \right| = 
\frac{| x_n - 2 |}{|2x_n|} = \frac{| x_n - 2 |}{2|x_n|}
$$

If we look at the denominator, we'll see a $|x_n|$ which we know is bounded, $|x_n| < M$, because
$x_n$ is a convergent series.
However, an upper bound in the denominator, gives us a lower bound on the overall expression, which
is not what we want.
We instead need to look for an inequality of the form $x_n \geq \delta > 0$.

Let's look at,
$$
|2| = |2 - x_n + n_x| \leq |x_n - 2| + |x_n| < \epsilon + |x_n|
$$
Which means $|x_n| > |2| - \epsilon$.
Similarly,
$$
|x_n| = |x_n - 2 + 2| \leq |x_n - 2| + |2| < \epsilon + |2|
$$
Which means that $|x_n| < \epsilon + |2|$.
Putting these two expressions together we get $|2| - \epsilon < |x_n| < |2| + \epsilon$.
The left side of it gives us a lower bound which can be greater than 0 if we pick a good value for $\epsilon$.

Back to our original problem,
$$
\frac{| x_n - 2 |}{2|x_n|} < \frac{| x_n - 2 |}{2(|2| - \epsilon)} = \frac{| x_n - 2 |}{2}
$$
If we chose $\epsilon = 1$.
\\~\\



\textbf{2.3.3}

\textbf{Squeeze theorem:} if $x_n \leq y_n \leq z_n$ for all $n\in\mathbb{N}$, and if
$\lim x_n = \lim z_n = l$, then $\lim y_n = l$ as well.

Let's assume that $\lim x_n = x$, $\lim y_n = y$, and $\lim z_n = z$ for now.
By the order limit theorem we have know that $x \leq y \leq z$.
Since $x = z = l$, then $y=l$.
\\~\\


\textbf{2.3.5}

If $x_n$ converges, then it means that $|z_n - l| < \epsilon$ $\forall \epsilon >0 \in \mathbb{R}$
when $n \geq N \in \mathbb{N}$.
So when $n \geq N$, all terms in $x_n$ must be "$\epsilon$ close to whatever happens to be $l$."

This implies that all terms after $n>N$ must be "close to $l$"
$$
| x_n - l | < \epsilon
$$
and
$$
| y_n - l | < \epsilon
$$

Thus, if for some reason $x_n$ and $y_n$ diverged or converged to different values, then the above conditions
would not be met, and thus $z_n$ would not converge.
\\~\\



\textbf{2.3.6}

Consider
$$
b_n = n - \sqrt{n^2 + 2n}
$$

Find $\lim b_n$ given $1/n \rightarrow 0$; the fact that when $x \geq 0$ if $x_n \rightarrow x$, then
$\sqrt x_n \rightarrow \sqrt x$; and the algebraic limit theorem.

$$
n - \sqrt{n^2 + 2n}
= n - \sqrt{n^2 +2n} \left( \frac{n + \sqrt{n^2 + 2n}}{n + \sqrt{n^2 + 2n}} \right) 
= \frac{n^2 - n^2 - 2n }{n + \sqrt{n^2 + 2n}}
$$
$$
= \frac{-2n}{n + \sqrt{n^2 + 2n}} 
= \frac{-2}{1 + \sqrt{1 + 2/n}}
$$

Now we can use the info given clearly,
$$
\lim b_n = \lim \left( \frac{-2}{1 + \sqrt{1 + 2/n}} \right)
= \frac{-2}{1 + \sqrt{1 + \lim{2/n} }}
= \frac{-2}{1+ \sqrt{1}}
= -1
$$
\\~\\


\textbf{2.3.11}

\textbf{Cesaro mean:} if $x_n \rightarrow x$, then does
$$
y_n = \frac{x_1 + x_2 + \ldots + x_n}{n} \rightarrow x
$$
?

If we think about $\epsilon$-neighborhoods, then we can envision that at some point all members of the
series hover extremely close around a given value, so $n$ elements of the same value divided by $n$
would be "the value".
In a more math way,

$$
| y_n - x | 
= \left| \frac{x_1 + x_2 + \ldots + x_n}{n} - x \right|
= \left| \frac{|x_1 -x| + |x_2-x| + \ldots + |x_n-x|}{n} \right|
$$

Since $x_n$ converges, we know that for any $n \geq N$, $|x_n - x| < \epsilon$.
So
$$
\left| \frac{|x_1 -x| + |x_2-x| + \ldots + |x_n-x|}{n} \right|
< \left| \frac{|x_1 -x| + |x_2-x| + \ldots + \epsilon + \ldots + \epsilon + \ldots}{n} \right|
$$

If we define $M = \max{|x_1 -x|, |x_2-x|, \ldots , |x_{n-1} - x|}$, then
$$
\left| \frac{|x_1 -x| + |x_2-x| + \ldots + |x_n-x|}{n} \right|
< \left| \frac{M + M + \ldots + \epsilon + \ldots + \epsilon + \ldots}{n} \right|
$$

If we chose an $n$ that's far out, we the above expression grows as $M/n$, which will tend to
zero ($\lim \frac{1}{n} = 0$).
\\~\\



\textbf{2.3.13}

\textbf{Iterated limits:} given a doubly indexed array $a_{mn}$ where $m, n \in \mathbb{N}$,
what should $\lim_{m,n \rightarrow \infty} a_{mn}$ represent?

Let $a_{mn} = m / (m+n)$ and compute the iterated limits
$$
\lim_{m\rightarrow\infty} \left( \lim_{n\rightarrow\infty} a_{mn} \right)
$$
and
$$
\lim_{n\rightarrow\infty} \left( \lim_{m\rightarrow\infty} a_{mn} \right)
$$

Define $\lim_{m,n\rightarrow\infty} a_{mn} = a$ to mean that for all $\epsilon > 0$
there exists an $N \in \mathbb{N}$ such that if both $m,n \geq N$, then $|a_{mn} - a|<\epsilon$.
\\~\\



%%%%%%%%%%%%%%%%%%%%%%%%%%%%%%%%%%%%
\subsection{The monotone convergence theorem and a first look at infinite series}

\subsubsection{Exercises}

\textbf{2.4.1} For thiese sorts of problems it helps to always plugin a couple numbers to see the patterns.
The sequence we are working with is
$$
x_{n+1} = \frac{1}{4 - x_n}
$$
where $x_1 = 3$.

Let's start by finding the first couple terms in the sequence,
$$
x_1 = 3
$$
$$
x_2 = \frac{1}{4 - 3} = \frac{1}{1} = 1
$$
$$
x_3 = \frac{1}{4 - 1} = \frac{1}{3} = 1/3
$$
$$
x_4 = \frac{1}{4 - 1/3} = \frac{1}{\frac{12}{3} - \frac{1}{3}} = \frac{1}{11/3} = 3/11
$$
$$
x_5 = \frac{1}{4 - 3/11} = \frac{1}{\frac{44}{11} - \frac{3}{11}} = \frac{1}{41/11} = 11/41
$$
$$
x_6 = \frac{1}{4 - 11/41} = \frac{1}{\frac{164}{41} - \frac{11}{41}} = \frac{1}{153/41} = 41/153
$$

We could even obtain the rest of the numbers in this sequence, given our initial conditions with
\begin{lstlisting}[language=Python]
def new_term(numerator, denominator):
    new_denominator = 4*denominator - numerator
    return denominator, new_denominator
\end{lstlisting}

So it seems like we have a case of the $x_n > x_{n+1}$, or a decreasing function.
Induction is demanded here and now since we already have the base case of $x_1 > x_2$.
For the rest of the numbers, let's go step by step...

If we assume that $x_n \geq x_{n+1}$, then $4 - x_n \leq 4 - x_{n+1}$, since we are subtracting a bigger number
on the left side.
However, smaller denominators make for greater rational numbers, so $\frac{1}{4-x_n} \geq \frac{1}{4 - x_{n+1}}$.
This expression holds up becuause of our initial conditions (what if $x_1 = 4.5$?).

Intuitvely, we can also follow the pattern we saw.
After $x_2$, the next couple terms in the sequence were numbers smaller than 1.
If the new terms get smaller, then their subscessors will get close to $1/4$, or so we think!

But the monotone convergence theorem can help us out here.
Since the sequence is decreasing and 3 can serve as our bound, then this sequence should converge.
\\

\textbf{This is cool so put attention...}
based on everything we have seen thus far $\lim x_n = \lim x_{n+1}$.
So if we take the limit of each side of the recurisve equation
to explicitly compute $\lim x_n$, then we get,

$$
\lim x_n = x = \lim x_{n+1}
$$
So
$$
\lim x_n = x
$$
and
$$
\lim x_n = \frac{1}{4- \lim x_n} = \frac{1}{4-x}
$$
Which leads us to
$$
x = \frac{1}{4-x} \rightarrow 4x - x^2 = 1
$$
Or $x^2 - 4x + 1 = 0$.
We can readily use the quadratic equation now to find our solution,
$$
x = \frac{ -(-4) \pm \sqrt{ 16 - 4 } }{2} = \frac{4 \pm \sqrt{12}}{2} = 2 \pm \sqrt{3}
$$
If $x = 2 + \sqrt{3}$ then $x>3$ and this contradicts everything that we have seen thus far about the sequence
being decreasing.
So $x = 2 - \sqrt{3}$ must be it (and this matches the answer we get from usign the code above)!
\\~\\


\textbf{2.4.2}
This time let's work with the recurisevely defined sequence
$$
y_{n+1} = 3 - y_n
$$
When $y_1 = 1$.

Plugin in a couple numbers,
$$
y_1 = 1
$$
$$
y_2 = 3 - y_1 = 3-1 = 2
$$
$$
y_3 = 3 - y_2 = 3 - 2 = 1
$$
$$
y_4 = 3 - y_3 = 3 - 1 = 2
$$
$$
y_5 = 3 - y_4 = 3 - 2 = 1
$$
$$
y_6 = 3 - y_5 = 3 - 1 = 2
$$
$$
y_7 = 3 - y_6 = 3 - 2 = 1
$$
So this time the series is not monotone but alternating.
Hence, we have no reason to believe that his converges, if anything it looks like it diverges (without doing any more work).
Because of this, it would not be sensible to follow the same procedure as in \textbf{2.4.1}
and apply the limit on both sides of the recurisve definition of this sequence, because
the limits do not exist.
\\

Now, if instead we were working with
$$
y_{n+1} = 3 - 1/y_n
$$
and $y_1 = 1$, then
$$
y_1 = 1
$$
$$
y_2 = 3 - 1/1 = 2
$$
$$
y_3 = 3 - 1/2 = 5/2
$$
$$
y_4 = 3 - 2/5 = \frac{15 - 2}{5} = 13/5
$$
$$
y_5 = 3 - 5/13 = \frac{39 - 5}{13} = 34/13
$$
$$
y_6 = 3 - 13/34 = \frac{102 - 13}{34} = 89/34 \approx 2.6176471
$$

We can see that $y_1 \leq y_2$, so maybe $y_n \leq y_{n+1}$?

As in problem \textbf{2.4.1}, we have the hypothesis that $y_n \leq y_{n+1}$,
That implies that $1/y_n \geq 1/y_{n+1}$, since 1 divided by a large denominator results in a small number.
Which in turns implies that $3 - 1/y_n \leq 3 - 1/y_{n+1}$, since 3 minus $1/y_{n+1}$ is 3 minus a smaller number.
Thus $y_{n+1} \leq y_{n+2}$.
We see that the series is monotonically increasing, but is it bounded (and thus convergent)?

What we do know is that $y_n > 0$, then $y_{n+1} = 3 - 1/y_n \leq 3$.
So 3 could be an upper bound as long as we are sure thant $y_n$ stays positive.

The way for $y_n < 0$ to happen would be for $3 - 1/y_n < 0$ or $3 < 1/y_n$, or $y_n < 1/3$.
Since we know that the sequence is increasing, as long as start with a number above $1/3$ we should be okay
and we should be bounded by 3.

Now we actually have some justification for attempting to calculate the limits of our
recursive formula.
But the algebra is not simple, so we wont.
\\~\\



\textbf{2.4.3}
does
$$
\sqrt{2}, \sqrt{2 + \sqrt{2}}, \sqrt{2 + \sqrt{2 + \sqrt{2}}}, \ldots
$$
converge?

Let's try following some of the similar steps as in the previous problems.

We can define the sequence as $x_1 = \sqrt{2}$ and
$$
x_{n+1} = \sqrt{2 + x_n}
$$

We can see that $x_1 = \sqrt{2} \leq x_2 = \sqrt{2 + \sqrt{2}}$, so it seems like the sequence would
be increasing as $n$ increases.
If $x_{n+1} \geq x_{n}$, then
$$
2 + x_{n+1} \geq 2 + x_n \rightarrow \sqrt{2 + x_{n+1}} \geq \sqrt{2 + x_n}
\rightarrow x_{n+2} \geq x_{n+1}
$$

(Remember problem 2.3.1? It comes in handy a lot!)
So the sequence does seem to be an increasing one.
Now let's see if we can argue that it is bounded.

Right now we know that $0 < x_1 = \sqrt{2} < 2$.
So what if we said that $x_n < 2$?
Well we could aso say that $2 + x_n < 4$, or that $\sqrt{2 + x_n} = x_{n+1} < 2$.
Which makes us believe that 2 is a bound for this sequence.

Given these two findings, we can assume that $\lim x_n$ exists and that $\lim x_n = \lim x_{n+1}$.
So if we apply the limit to both sides of our re-defined sequence we get
$$
\lim x_{n+1} = x = \lim \sqrt{2 + x_n} = \sqrt{2 + \lim x_n} = \sqrt{2 + x}
$$
$$
\rightarrow x = \sqrt{2 + x}
$$
A bit of algebra leads us to
$$
x = \sqrt{2 + x} \rightarrow x^2  = 2 + x \rightarrow
x^2 - x - 2 = 0
$$
And a little bit of the quadratic equation would land us in
$$
\frac{-(-1) \pm \sqrt{1 - 4 (-2)}}{2} = \frac{1 \pm \sqrt{1 + 8}}{2}
= \frac{1 \pm 3}{2}
$$
$x$ is either 2 or -1.
For our case, $x=2$ must be the one.
\\~\\

Now, what about the sequence
$$
\sqrt 2, \sqrt{2 \sqrt 2}, \sqrt{2 \sqrt{2 \sqrt 2}}, \ldots
$$

Let's go through this as we have been practicing...
$x_1 = \sqrt 2$, and
$$
x_{n+1} = \sqrt{2x_n}
$$

Since $\sqrt 2 > 1$, $x1 \leq x_2$, so again we are increasing.
Now let's say $x_{n+1} \geq x_n$, in which case
$$
x_{n+1} \geq x_n \rightarrow \sqrt{2x_{n+1}} \geq \sqrt{2x_n}
\rightarrow x_{n+2} \geq x_{n+1}
$$

Similarly, $x_1 = \sqrt 2 < 2$, so if $x_n < 2$, then $\sqrt{2x_n} =x_{n+1} < \sqrt{4} = 2$,
so again $x_n$ is bounded by 2.

Now that we got our excuse to apply the limits, we have
$$
x = \sqrt{2x} \rightarrow x^2 = 2x
$$
or $x = \pm 2$, and again we chose the positive value.
\\~\\



\phantomsection
\label{abbott:2.4.4}

\textbf{2.4.4}

\textbf{Show that the monotone convergence theorem can also be used to prove the archimedean property
without making use of the axiom of completeness.}

The archiemedean property states that: given any number $x\in\mathbb{R}$, there exists an
$n\in\mathbb{N}$ satisfying $n>x$.
And that given any real number $y>0$, there exists an $n$ satisfying $1/n < y$.

The trick to seeing this is to find two sequences that are convergent, because if they are convergent,
then they must have an upper bound, and thus the archiemedean property flows.

One way is to find some sequence we know and see how it the archimedean principle makes itself visible.
For example, we know that $\lim 1/n = 0$, so $|1/n - 0| = 1/n < \epsilon$, and if we set
$\epsilon$ to some real number $y$, then we have second statement of the archiemedean principle.

Another way, is to follow the exact same line of reasoning used in the book when proving it.
In the book we went with a proof by contradaction - we assumed that the natural numbers were bounded.
Since the natural numbers are monotonically increasing and we are assuming that they are bounded,
then the sequence of natural numbers must converge to some number.
Which is hard to argue.

Mathematically, we are saying that $\lim n = N$, so $\lim n+1 = N+1$, meaning that $N = N+1$.

There is a similar argument made here
\href{https://math.stackexchange.com/questions/90127/tfae-completeness-axiom-and-monotone-convergence-theorem}{TFAE: Completeness Axiom and Monotone Convergence Theorem}.
\\~\\


\textbf{Use the monotone convergence theorem to supply a proof for the nested interval property.}

The nested interval property tells us that for each $n\in\mathbb{N}$, if we are given a close interval
$I_n = [a_n, b_n] = \{ x\in\mathbb{R}: a_n \leq x \leq b_n \}$,
and if $I_{n+1} \subseteq I_n$, then
$\bigcap^{\infty}_{n=1} I_n \neq \emptyset$.

For this part, we can also follow the logic of the book.
The sequence of $a_n$ is increasing and bounded by any $b_n$, so it must converge to some value.
Similar argument can be made for the sequence $b_n$ but this one being bounded from below and decreasing.
Since $\lim a_n = a$ and $\lim b_n = b$ and $a_n \leq b_n$ for all $n$, then the order limit theorem
tells us that $a \leq b$.
From here can follow the argument in the book by looking at $a_n \leq a$ for a particular $I_n$.
In this same instance, $a<b_n$, so $a\in I_n$ for all $n$.
\\~\\



\textbf{2.4.5}
Let $x_1 = 2$ and
$$
x_{n+1} = \frac{1}{2}\left( x_n + \frac{2}{x_n} \right)
$$

To show that $\forall n$, $x^{2}_{n} \geq 2$, we can use induction.
$x^{2}_{1} = 4 \geq 2$.
To see the rest of the argument,
$$
x^{2}_{n+1} = \frac{1}{4}x^{2}_{n} + \frac{1}{x^{2}_{n}} + 1
$$
Hence, if $x^{2}_{n} \geq 2$, then $x^{2}_{n+1} \geq 2$.
\\

Now that we have an argument made for $x^{2}_{n} \geq 2$ for all $n$, let's see if we can prove that
$x_n - x_{n+1} \geq 0$.

One way of seeing this is by exploring whether the sequence is decreasing.
We know that $x_1 = 2 > x_2 = 3/2$.
Now let's poke around and see what we can find about $x_n$ in general.

$$
\frac{1}{2}\left(x_n + \frac{2}{x_n}\right) = \frac{1}{2}\left(\frac{x^{2}_{n} +2}{x^{2}_{n}}\right) 
\leq \frac{1}{2} (x^{2}_{n} + 2) \geq 1
$$
So its a tad odd to carry on with an induction argument given the expression we just made.

So let's try another way,
$$
x_n - x_{n+1} = x_n - \frac{1}{2}x_n - \frac{1}{x_n} = \frac{1}{2}x_n - \frac{1}{x_n}
$$

If $x_n - x_{n+1} \geq 0$, then
$$
\frac{1}{2}x_n - \frac{1}{x_n} \geq 0
\rightarrow \frac{1}{2}x_n \geq \frac{1}{x_n}
\rightarrow x^{2}_{n} \geq 2
$$
Which matches what we had above.

If instead we had assumed that $x_n - x_{n+1} < 0$, then we would have ended up with $x^{2}_{n} < 2$.
\\

Since we have showned that the sequence is decreasing and bounded, then we know that it converges, and we
can apply the limit to both sides of our definition.
This will land us with $\lim x_n = \pm 2$, but only $+2$ is a valid limit for us.
\\~\\



\textbf{2.4.6}

\textbf{Explain why the geometric mean is always less than the arithmetic mean:} why is
$\sqrt{xy} \leq (x+y)/2$, for any two positive real numbers $x$ and $y$.

To see why, let's square both sides
$$
\sqrt{xy} \leq \frac{x+y}{2} \rightarrow xy \leq \frac{1}{4} (x^2 + y^2 + 2xy)
\rightarrow 4xy \leq x^2 + y^2 + 2xy
$$
The last expression can be re-arranged to
$$
0 \leq x^2 + y^2 - 2xy = (x - y)^2
$$

Which helps us see that only in the case in which $x=y$, will the geometric and the arithmetic
means be equal, otherwise, the arithmetic mean will always be greater.
\\

If we have $0 \leq x_1 \leq y_1$ and define
$$
x_{n+1} = \sqrt{x_n y_n}
$$
and
$$
y_{n+1} = \frac{x_n + y_n}{2}
$$

and we want to see whether $\lim x_n$ and $\lim y_n$ esit, then the simplest thing is to investigate
whether the sequences are monotone and bounded.

We already saw that the geometric mean is always less than the arithmetic mean, so $x_n \leq y_n$
for all $n\in\mathbb{N}$.
Thus we can carry our initial condition as a general case and assume that $x_n \leq y_n$ throughout.

If we now look at $x_{n+1}$ we see that if it was that if $x_{n+1} = \sqrt{x_n x_n}$.
However, $y_n$ is greater than or equal to $x_n$ so $x_{n+1} \geq x_n$.

As for $y_n$, it helps if we rewrite it as $y_{n+1} = \frac{1}{2}x_n + \frac{1}{2}y_n$.
Again, $y_{n+1} = y_n$ if $x_n = y_n$, but since $x_n$ is equal to or smaller, then $y_{n+1} \leq y_{n}$.

So we see that $x_n$ is increasing, while $y_n$ is decreasing.
We also see that if $x_n = y_n$ we have some sort of equilibrium where $x_{n+1} = x_n$ and
$y_{n+1} = y_n$.
Which shows us that $\lim x_n = \lim y_n$ when that is the case.
\\~\\



\textbf{2.4.7}

\textbf{Limit superior} let $(a_n)$ be a bounded sequence.

Prove that the sequence defined by $y_n = \sup\{a_k : k\geq n\}$ converges.

Let's see if the monotone convergence theorem can help us out here too.
The sequence $y_n$ essentially looks at all the elements of the sequence $a_n$ that come after
a given $k$.
So for a given $y_n$ we have a corresponding least-upper bound from the subsequence $\{a_k : k\geq n\}$.
$y_1$ will then be the supremum of the entire sequence $a_n$.
$y_2$ will be the supremum if $a_n$ did not have its first element $a_1$, and so on and so forth.
Since $y_1$ is the supremum for the entire sequence, all other spremums from sub-sequences will be less
than or equal to it.
Similar argument can be made for $y_2$, any of the following supremums will be smaller than or equal to it.
So we can already see that the sequence is decreasing and bounded by $y_1 = \sup a_n = \alpha$.
\\

The \textbf{limit superior} of $a_n$ is defined as follows:
$$
\lim \sup a_n = \lim y_n
$$
Where $y_n$ is the sequence we defined above, and we also made an argument for its existance.

Now what about $\lim \inf a_n$?
Given our definition of $y_n$, we could define $z_n = \inf\{a_k : k\geq n\}$.
In such an instance, $z_1$ would be the infinimum of the entire sequence.
Similarly, $z_2$ would be the infinimum of $a_n$ without its first element, so it could be the same
(if the sequence is decreasing), it could be of greater value (if the sequence is increasing),
or it could be equal (if the first two values are similar).

Since our definition starts with the entire sequence and then gets rid of the elements at the begining,
we will not be finding any infinimum value that is smaller than any of the $z_n$ that have been computed before.
Thus we can see that $z_n$ is increasing and bounded from below by $z_1 = = \inf a_n = \beta$.

However, $a_n$ is itself bounded, so this definition, although it results in an increasing sequence
and the infinimum is a lower bound, it also converges and thus $\lim \inf a_n$ should exist.
\\

From experience, definition, whatever you want to call it, $\inf a_n \leq \sup a_n$.
And it would make sense to think that $\lim \inf a_n \leq \lim \sup a_n$.
And thus the oreder limit theorem makes this make sense.
\\

Now, if if $\lim \inf a_n = \lim \sup a_n$, thenthe squeeze theorem can get us $\lim a_n$ as it would
be the same value as the limit superior and the limit inferior.
And if we know that $\lim a_n$ exists, then Intuitvely we can imagine that after some point, all elements
of $a_n$ never leave some $\epsilon$-neighborhood (all values get arbitrarily close to the value they converge to).
In such an intansce, the supremum and infinimum would be equal to one another and the order limit theorem
would again come in handy to see that all three limits must equal each other.
\\~\\



\textbf{2.4.8}

Find an explicit formula for the sequence of partial sums and determine if the following series converge:

$$
\sum^{\infty}_{n=1} \frac{1}{2^n}
$$

Writing out the terms, we get
$$
\sum^{\infty}_{n=1} \frac{1}{2^n} = 1/2 + 1/4 + 1/8 + \ldots
$$

This is a geometric series where the common term is $1/2$.
The explicit formula for geometric series is
$$
\sum^{\infty}_{n=0} r^n = \frac{1}{1-r}
$$
But since we are starting at $n=1$, we need to subtract the first term.
Thus
$$
\sum^{\infty}_{n=1} \frac{1}{2^n}
= -1 + \sum^{\infty}_{n=1} r^n
= -1 + \frac{1}{1-\frac{1}{2}} = 1
$$
\\

In the case of
$$
\sum^{\infty}_{n=1} \frac{1}{n(n+1)}
$$
This is the wikipedia example of a telescoping series.

For this series,
$$
\sum^{\infty}_{n=1} \frac{1}{n(n+1)} = \lim_{N\rightarrow\infty} \left(1 - \frac{1}{N+1}\right) = 1
$$

And finally,
$$
\sum^{\infty}_{n=1} \log\left(\frac{n+1}{n}\right)
$$

Is a telescoping series of sorts since $\log \frac{n+1}{n} = \log (n+1) - \log n$.
So we have
$$
\sum^{\infty}_{n=1} \log\left(\frac{n+1}{n}\right)
= (\log 2 - \log 1) + (log 3 - \log 2) + (\log 4 - \log 3) + (\log 5 - \log 4) + (\log 6 - \log 5) + \ldots
$$
$$
= -\log 1 + \lim \log N
$$
Which doesn't seem like it converges, since $\log{x}$ is an increasing function.
\\~\\



\textbf{2.4.9}

Show that if $\sum^{\infty}_{n=0} 2^n b_{2^n}$ diverges, then so does $\sum^{\infty}_{n=1} b_n$.

Abbott is trully a great teacher.
The beauty of this proof starts with the fact that we are proving the contrapositive of the
cauchy condensation test.
The following argument is completely taken from
\href{https://math.stackexchange.com/questions/1736699/proof-of-cauchy-condensation-test-using-contrapositive}{Proof of Cauchy Condensation Test using contrapositive}.
(I thought about it, and thought about it, and kept on thinking about it and never got anywhere.)

Since the series $b_n$ is decreasing, we were able to define
$$
s_{2^{k+1}-1} =
b_1 + (b_2 + b_3) + (b_4 + b_5 + b_6 + b_7) + \ldots + (b_{2^k} + \ldots + b_{2^{k+1}-1})
$$
$$
\leq b_1 + (b_2 + b_2) + (b_4 + b_4 + b_4 + b_4) + \ldots + (b_{2^k} + \ldots + b_{2^{k}})
$$
$$
= b_1 + 2(b_2) + 2^2(b_4) + \ldots + 2^k (b_{2^k})
= t_k
$$

One of the tricks used above is that $b_n \geq b_{n+1}$.
If we went the other route, if replaced some of the terms in our sequence with the smallest terms
within their bracketed groups then we could obtain a sequence that is smaller.

$$
s_{2^{k+1}-1} =
b_1 + (b_2 + b_3) + (b_4 + b_5 + b_6 + b_7) + \ldots + (b_{2^k} + \ldots + b_{2^{k+1}-1})
$$
$$
\geq b_1 + (b_3 + b_3) + (b_7 + b_7 + b_7 + b_7) + \ldots + (b_{2^{k+1}-1} + \ldots + b_{2^{k+1}-1})
$$

And we really recommend that you write out the terms and see the patterns but take a look at the end.
$s_{2^{k+1}-1}$ ends in $(b_{2^k} + \ldots + b_{2^{k+1}-1})$, it goes all the way to the index term in $s$
and the bracketing starts at the previous $2^{k}$.

Then, we can see how
$$
s_{2^k} = b_1 + b_2 + b_3 + \ldots
= b_1 + (b_2 + b_3) + \ldots + (b_{2^{k-1}+1} + \ldots + b_{2^k})
$$

We should also note that the factor infront of the last bracketing is the $2^n$ that happens to be closest.
Hence,
$$
s_{2^k}
= b_1 + (b_2 + b_3) + \ldots + (b_{2^{k-1}+1} + \ldots + b_{2^k})
$$
$$
\geq b_1 + 2 (b_3) + 2^2 (b_7) + \ldots + 2^{k-1} (b_{2^{k}})
$$

and since sequences are infinite, we could change the offset as such
$$
s_{2^k}
= b_1 + b_2 + (b_3 + b_4) + (b_5 + b_6 + b_7 + b_8) + \ldots + (b_{2^{k-1}+1} + \ldots + b_{2^k})
$$
$$
\geq b_1 + b_2 + 2 (b_4) + 2^2 (b_8) + \ldots + 2^{k-1} (b_{2^{k}})
$$
$$
= b_1 + t_{k-1}
$$

And so we get our proof.
\\~\\

However, a peek at wikipedia reveals another very great trick.
We know that
$$
t_k = b_1 + 2b_2 + 4b_4 + \ldots
$$

This could be bracketed also as
$$
t_k = (b_1 + b_2) + (b_2 + b_4 + b_4 + b_4) + (b_4 + b_8 + \ldots) + \ldots
$$
And since the series is decreasing,
$$
t_k = (b_1 + b_2) + (b_2 + b_4 + b_4 + b_4) + (b_4 + b_8 + \ldots) + \ldots
$$
$$
\leq (b_1 + b_1) + (b_2 + b_2 + b_3 + b_3) + (b_4 + b_4 + \ldots) + \ldots
$$
$$
= 2\sum^{\infty}_{n=1} b_n
$$

Hence the cauchy condensation test can be expanded to
$$
\sum^{\infty}_{n=1} b_n \leq \sum^{\infty}_{n=0} 2^n b_n \leq 2\sum^{\infty}_{n=1} b_n
$$
\\~\\



\textbf{2.4.10}

The infinite product
$$
\prod^{\infty}_{n=1} b_n = b_1 b_2 b_3 \ldots
$$
can be understood in terms of its sequence of partial products (like infinite series and partial sums)
$$
p_m = \prod^{m}_{n=1} b_n = b_1 b_2 \ldots b_m
$$

We will focus here on the special class of infinite products that looks as such
$$
\prod^{\infty}_{n=1} (1 + a_n) = (1+a_1)(1+a_2)(1 +a_3)\ldots
$$
Where $a_n \geq 0$.
\\

If $a_n = 1/n$ we get this interesting looking partial product
$$
p_m = \left(1+1\right) \left(1+\frac{1}{2}\right) \left(1+\frac{1}{3}\right) \left(1+\frac{1}{4}\right) \left(1+\frac{1}{2}\right) \ldots \left(1+\frac{1}{m}\right)  
$$
$$
= \left(\frac{2}{1}\right) \left(\frac{3}{2}\right) \left(\frac{4}{3}\right) \left(\frac{5}{4}\right) \ldots \left(\frac{m+1}{m}\right) 
$$
$$
= \frac{m+1}{1} = m+1
$$
However, since $m \rightarrow \infty$ in the infinite product, we could think that this case diverges.
\\

Now, when $a_n = 1/n^2$, we get
$$
p_m =
\left(1+ \frac{1}{1} \right) \left(1+ \frac{1}{2^2} \right) \left(1+ \frac{1}{3^2} \right) \left(1+ \frac{1}{4^2} \right) \left(1+ \frac{1}{5^2} \right) \ldots \left(1+ \frac{1}{m^2} \right)
$$
$$
= \left(\frac{2}{1} \right) \left(\frac{2^2 +1}{2^2} \right) \left(\frac{3^2 +1}{3^2} \right) \left(\frac{4^2 +1}{4^2} \right) \left(\frac{5^2 +1}{5^2} \right) \ldots \left(\frac{m^2 +1}{m^2} \right)
$$
Here the first term is 2 but as $m$ grows, the terms begin to get closer and closer to 1.
\\


Can we show that the sequence of partial products converges if and only if
$\sum^{\infty}_{n=1} a_n$ converges?

Abbott gave us the trick to this, if we use the identity $1+x \leq e^x$, then we can rewrite the
partial product as follows
$$
p_m = (1+a_1) (1+a_2) + \ldots (1+a_m)
\leq e^{a_1} e^{a_1} \ldots e^{a_m}
= e^{s_m}
$$

Hence, if $\sum^{\infty}_{n=1} a_n \rightarrow a$, then $e^{s_m}$ is a number, and thus we get our
proof - the series is increasing and bounded.

To prove the other route (if the infiinite product converges then the series converges),
it sufices to show that $p_m \geq s_m$. (Expand the product and you will see terms correspodning to
the partial sum plus other additional terms.)
\\~\\



%%%%%%%%%%%%%%%%%%%%%%%%%%%%%%%%%%%%%%%%%%%%%%%%%%%%%%%%%%%%%%%%%%%%%%%%%%%%%%%%%%%%%%%%%
\subsection{subsequences and the Bolzano-Weierstrass theorem}


\subsubsection{Exercises}

\textbf{2.5.3}

Assume $a_1 + a_2 + 1_3 + \ldots$ converges to a limit L
(i.e., the sequence of partial sums $(s_n) \rightarrow L$).
Show that any \textbf{regrouping} of the terms
$$
\left(a_1 + a_2 + \ldots + a_{n_1}\right) +
\left(a_{n_1 +1} + a_{n_1 +2} + \ldots + a_{n_2}\right) +
\left(a_{n_2 +1} + a_{n_2 +2} + \ldots + a_{n_3}\right) +
\ldots
$$
leads to a series that converges to L.
\\

A sequence of partial partial sums $s_n \rightarrow L$.
A regrouping of terms ends up creating a subsequence of $s_n$.
And since $s_n$ converges, the any of its subsequences converge as well.
\\~\\


\phantomsection
\label{abbott:2.5.4}

\textbf{2.5.4}

Assume the nested property is true and use it to provide a proof for the axiom of completeness.
\\

Some brief reminders:
\begin{itemize}
    \item Axiom of completeness: every non-empty set of real numbers that is bounded above has a least upper bound.
    \item nexted property: assume we are given a close interval $I_n$ and that $I_{n+1} \subseteq I_n$, then $\cap I_n \neq \emptyset$.
\end{itemize}

When proving the nested interval property, we laid out all $I_n = [a_n, b_n]$ on the real number line.
Since all intervals were nested, then we had an order for the set of $A = \{a_n : \in\mathbb{N}\}$
and $B = \{b_n : n\in\mathbb{N}\}$, where $a_1$ and $b_1$ were at the values fartest apart.
Then we saw that all $b\in B$ served as upper bounds, and thus a suppremum $x$ must exist such that
$a_n \leq x \leq b_n$.
And since this inequality held for all intervals (as they were all nested and equal to or smaller than their predecessor)
then we had a proof that there was an element that was part of all the intervals.
\\

When proving the Bolzano-Weierstrass theorem, we also built nested intervals with the same properties
as the ones we saw in the nested interval theorem.
So if we assume that the nested interval theorem is true, then right away we can see that
from the bisecting of our Bolzano-Weierstrass prove that there must be at least one element that is
part of all the intervals.

If we start with the nested interval property then we also get an upper bound for our sequence ($b_n$).

The only thing left to get the axiom of completeness from the use of the nested interval property on the
Bolzano-Weierstrass theorem is to see that as we continue forming subintervals, each of of length
$M(1/2)^{k-1}$, then as $k\rightarrow\infty$, the length will tend to zero (example 2.5.3).
And since the length of the intervals tends to zero then the element $x$ that we found above, can indeed
be the least upper bound, since we can make the lenght of the interval as small as we want ($\epsilon$)
such that $x-\epsilon$ is no longer an upper bound, $x-\epsilon \leq a_n$.
\\

The reason we are to assume that $1/2^{k-1}$ is a convergent sequence is because the lenght of
the subinterval $I_k$ is supposed to converge to 0, and using the normal $\epsilon$ convention we want
to prove that something like $1/n < \epsilon$ is true for any $\epsilon > 0$.
$\epsilon$ is a real number and so we are implicitly making use of the Archimedean property, which
we originally proved using the Axion of Completeness, which we are trying to prove here.
\\~\\


\textbf{2.5.6}

Show that $\lim b^{1/n}$ exists for all $b\geq 0$ and find the value of the limit.
\\

The series $b^{1/n}$ is decreasing, since $\frac{1}{n} > \frac{1}{b+1}$.
Since it is monotonically decreasing and abounded by $b$, then it must converge.
Go again, if $\lim b^{1/n} = l$, then $0 \leq l \leq b$.

Since $b^{1/n} \rightarrow l$, then so must $b^{1/(n+1)} \rightarrow$.
If we take the limit of both and equate then, we get
$$
b^{1/n} = b^{1/(n+1)} \rightarrow b = b^{\frac{n}{n+1}} = b^{1 + 1/n} = bb^{1/n}
$$
Hence
$$
b^{1/n} = 1
$$

If $b=0$, $\lim b^{1/n} = 0$. Recall that one problem where we proved that $\lim a_n = 0$,
then $\lim \sqrt{a_n} = 0$.
\\~\\



\phantomsection
\label{abbott:2.5.7}

\textbf{2.5.7}

When $|b| < 1$, then $b^n$ is decreasing and abounded by 1, hence convergent.
We follow similar steps as in the previous problem, then $b^n = b^{n+1} = l$.
And the only numbers that stay the same no matter how many powers you raise them to are
0 and 1. But since $|b| < 1$, the $|b^n| < 1$, so $\lim b^n = 0$.
\\~\\



\textbf{2.5.8}

Another way we could prove the Bolzano-Weierstrass theorem is by finding monotone subsequences
in a bounded sequence.
That way the monotone convergence theorem can be used to prove convergence.

If we can find a series of peak terms within our sequence, then we can build a monotone subsequence
from the sequence of peak terms.
Note that the peak points would define a decreasing subsequence.

However, if there is a finite number (or zero) peak points, then instead of peak points, we can look
for "nadir points" and instead define a monotonically increasing subsequence.
\\~\\



\textbf{2.5.9}

\textbf{Direct proof of Bolzano-Weierstrass theorem using the Axiom of Completeness}

Let $(a_n)$ be a bounded sequence, and define the set
$$
S = \{ x\in\mathbb{R} : x < a_n \text{ for inifinetely many terms in } a_n \}
$$

Show that there exists a subsequence $(a_{n_k})$ converging to $s = \sup S$.
\\

Since $(a_n)$ is bounded, then S will also be bounded.
Thus there must exist $s = \sup S$.
If we chose a subsequence $(a_{n_k})$ that is monotonically increasing, whose values go up to the bound
of $(a_n)$, the by the monotone convergence theorem, this subsequence will converge.
This subsequence should be such that $|a_{n_k} - s| < \epsilon$ for any $\epsilon > 0$
and any $k > N$.
Which is possible because the parent sequence converges and we are restricting ourselves to a mnonotone
increasing sequence.
\\~\\



%%%%%%%%%%%%%%%%%%%%%%%%%%%%%%%%%%%%%%%%%%%%%%%%%%%%%%%%%%%%%%%%%%%%%%
\subsection{The Cauchy Criterion}

In \textbf{lemma 2.6.3, cauchy sequences are bounded}, there is a bit of logic that was used back in
\textbf{theorem 2.3.2 every convergent sequence is bounded},
and also when de discussed the triangle inequality.

When we say that $|x_n - x| < \epsilon$, this can also be seen as $\epsilon < x_n - x < \epsilon$.
Thus, $|x_n - x_m| < 1$, could be seen as $-1 < x_n - x_m < 1$ or $-(x_m + 1) < x_n < x_m + 1$.


\subsubsection{Exercises}

\textbf{2.6.1}

\textbf{Every convergent sequence is a Cauchy sequence}

If we have a sequence $(x_n)$ such that $\lim x_n = x$, then
$$
|x_n - x| < \epsilon
$$
Whenever $n \geq N \in \mathbb{N}$.

Thus, if we look at a cauchy sequence, with $n, m \geq N$,
$$
|x_n - x_m| = |x_n - x + x - x_m| \leq |x_n - x| + |x - x_m| = |x_n - x| + |x_m - x| < \epsilon_1 + \epsilon_2
$$

We could have gone the other way around and started with,
$$
|x_n - x| \leq |x_n - x_m| + |x_m - x| < |x_n - x_m| + \epsilon
$$

And from there we would be back to the first use of the triangle inequality.

\textbf{Note:} it could be easy to make an argument about making $x_m = x$ and then throwing in another
epsilon. However, we would be fooling ourselves with the fake simplicity (why didn't we use that same
argument instead of using the triangle inequality above?).
To see why the fake simplicity is not correct, see the rest of contents in Abbott section 2.6.
Where Abbott instead went and used the Bolzano-Weierstrass theorem.

Also, it is possible that the limit is not actually a memeber of the sequence, think of $\lim 1/n$.
\\~\\



\textbf{2.6.3}

If $(x_n)$ and $(y_n)$ are cauchy sequences, then one easy way to prove that $(x_n + y_n)$ is
to use the cauchy criterion.
The cauchy criterion states that $(x_n)$ and $(y_n)$ must be convergent,
and the algebraic limit theorem then implies $(x_n + y_n)$ is convergent and hence cauchy.

Remember that the cauchy criterion says that a sequence converges if and only if it is a cauchy sequence.
\\

Now try giving a direct argument that $(x_n + y_n)$ is a cauchy sequence that does not use the
cauchy criterion or the algebraic limit theorem.
\\

We know that
$$
| x_n - x | < \epsilon \Longleftrightarrow |x_n - x_m| < \epsilon
$$
for all $\epsilon > 0$ when $n\geq N_1$.
We also know that
$$
| y_n - x | < \epsilon \Longleftrightarrow |y_n - y_m| < \epsilon
$$
for all $\epsilon > 0$ when $n\geq N_2$.

To make a direct argument, we need to prove that
$$
|(x_n + y_n) - (x_m + y_m)| < \epsilon
$$
for all $\epsilon >0$ when $n \geq \max{N_1, N_2}$.

To do so, note that
$$
|(x_n + y_n) - (x_m + y_m)| 
= |(x_n - x_m) + (y_n - y_m)|
$$
$$
\leq |x_n - x_m| + |y_n - y_m|
< \epsilon + \epsilon
$$
\\

Give a direct argument for $(x_n y_n)$ now.
\\

We now want to show that
$$
|(x_n y_n) - (x_m y_m)| < \epsilon
$$

Let's use a the same trick Abbott employed back when we were proving the algebraic
limit theorem.

\begin{align*}
    |(x_n y_n) - (x_m y_m)| &= |x_n y_n - x_m y_n + x_m y_n - x_m y_n|      \\
    &\leq |x_n y_n - x_m y_n| + |x_m y_n - x_m y_m|     \\
    &= |y_n| |x_n - x_m| + |x_m| |y_n - y_m|    \\
    &< M_2 \frac{\epsilon}{M_2} + M_1 \frac{\epsilon}{M_1} \\
\end{align*}

In the last step we used the fact that convergent sequences are bounded and defined
$M_1$ as the upper bound for $x_n$ and $M_2$ as the upper bound for $y_n$.
\\~\\



\textbf{2.6.7}

Exercises \ref{abbott:2.4.4} (use MCT to prove AoC) and
\ref{abbott:2.5.4} (use NIP, assume AP, use BW methodology to prove AoC) establish the equivalence of the axiom of
completeness and the monotone convergence theorem.
They also show the nested interval property is equivalent to these two when the archimedean property
is pressumed to be true.
\\

Before we carry on, we want to note Rudin's statement of the \textbf{Archimedean property (and the density
of the rational numbers within the reals)}:
If $x,y\in\mathbb{R}$ and $x>0$, then there is a $n>0 \in \mathbb{N}$ such that
$$
nx > y
$$
Abbott states this as $n > x$ for any $x\in\mathbb{R}$.
\\

If $x,y\in\mathbb{R}$ and $x<y$, then there is a $p\in\mathbb{Q}$ such that
$$
x < p < y
$$

Abbot states this as $1/n < y$ for any $y>0 \in\mathbb{R}$ and $n\in\mathbb{N}$.
\\



\textbf{Assume the Bolzano-Weierstrass theorem to be true and use it to construct a proof of the
monotone convergence theorem} without making any appeal to the archimedean property.
This shows that BW, AoC, and MCT are equivalent.
\\

By the BW theorem, we know that a bounded sequence has a convergent subsequence.
The subsequence generated by BW meets the condition that the elements we select
must be so $n_k > n_{k-1} > \ldots > n_1$ so that $a_{n_k}\in I_k$.
Which means that we are not going around picking elements as we want, we pick elements in the order that
they show up in the original sequence.

Ultimately, we know that if BW applies then $(a_n)$ is bounded and that we can build a subsequence
$(a_{n_k}) \rightarrow x$.

For the MCT, the sequence must also be bounded and all elements must be monotonically increasing or decreasing.
Let's pick the case in which the elements are monotonically increasing and our sequence is bounded above.

So
$$
|a_{n_k} - x| < \epsilon
$$
When $k \geq N$, and $a_{n_{k+1}} \geq a_{n_k}$.

Thus,
$$
- (x+\epsilon) < a_{n_k} < x+\epsilon
$$
or
$$
- (x+\epsilon) < a_{n_k} \leq a_{n_{k+1}} < x+\epsilon
$$

In this last expression, the added term works because the sequence is monotonically increasing,
but we know that for any $k\geq N$, $|a_{n_k} - x| < \epsilon$ must hold.
This last expression can also be rewritten as
$$
|a_{n_{k+1}} - x| < \epsilon
$$
And thus our prove.
\\~\\



\textbf{Use the Cauchy criterion to prove BW}, and find the point in the argument where the
Archimedean property is implicitly required.
This establishes the final link in the equivalence of the five characterizations of completeness.
\\

The argument for the BW theorem that Abbott shows us is rather generic.
One thing we can do is to note when it makes use of the NIP.
Originally this was to find what the limit for the subsequence could be.
If we instead assumed the Cauchy criterion, then we'd know that the subsequence is bounded
and that after some point, the elements of the sequence get close to each other.

Then the implicit use of thw Archimedean property comes from the assumption that $M(1/2)^{k-1}$
converges.
($M(1/2)^{k-1}$ is a rational number that we want to make smaller tthan some arbitrary real $\epsilon$.)
\\~\\

How do we know it is impossible to prove the AoC starting from the Archimedean property?
\\

The Archimedean property tells us that we can always find an $n\in\mathbb{N}$
that is greater than some real number.
However, it tells us about upper bounds, it doesn't tell us about the least-upper bounds
(or greatest upper bounds).
So the Archimedean property leaves us with possible holes in our number line.

Also, the first part of the Archimedean property, $n>x$, is true if $x\in\mathbb{Q}$.
So it doesn't give us any clues about us needing the reals.
\\~\\



%%%%%%%%%%%%%%%%%%%%%%%%%%%%%%%%%%%%%%%%%%%%%%%%%%%%%%%%%%%%%%%%%%%%%%%%%%%%%%%%%%%%%
\subsection{Properties of infiinite Series}

\textbf{Cauchy Criterion for Series}

Since $n > m \geq N$,
$$
s_m = a_1 + a_2 + \ldots + a_N + \ldots + a_m
$$
and
$$
s_n = a_1 + a_2 + \ldots + a_N + \ldots + a_m + \ldots + a_n
$$

So
$$
|s_n - s_m| = 
\left| (\cancel{a_1} + \cancel{a_2} + \ldots + \cancel{a_N} + \ldots + \cancel{a_m} + \ldots + a_n) - 
(\cancel{a_1} + \cancel{a_2} + \ldots + \cancel{a_N} + \ldots + \cancel{a_m})\right|
$$
$$
= \left| a_{m+1} + a_{m+2} + \ldots + a_n \right|
$$



\subsubsection{Exercises}


\textbf{2.7.1}

The \textbf{alternating series tests} goes as follows:
let $(a_n)$ be a sequence satisfying
\begin{enumerate}
    \item $a_1 \geq a_2 \geq \ldots a_n \geq a_{n+1} \geq \ldots$
    \item $(a_n) \rightarrow 0$
\end{enumerate}

Then the alternating series $\sum^{\infty}_{n=1} (-1)^{n+1} a_n$ converges.

Proving the alternating series tests ammounts to showing that the sequence of partial sums
$$
s_n = a_1 - a_2 + a_3 - a_4 + \ldots \pm a_n
$$
converges.

\textbf{Different characterizations of completeness lead to different proofs.}
\\

%%%%%%%%%%%%%%%
\textbf{Prove it by showing that $(s_n)$ is a Cauchy sequence.}
\\

We have
$$
s_n = a_1 - a_2 + a_3 - a_4 + \ldots \pm a_n
$$
So if we chose an $n > m \geq N$
$$
| s_n - s_m | = |(a_1 - a_2 + a_3 - a_4 + \ldots \pm a_n) - (a_1 - a_2 + a_3 - a_4 + \ldots \pm a_m)|
$$
$$
= | \pm a_{m+1} \mp a_{m+2} \pm \ldots \pm a_n |
$$

In the special case in which $n = m + 1$, we get
$$
| s_n - s_m | = | a_n | < \epsilon
$$

The appended inequality is possible because $(a_n) \rightarrow 0$, so we can meet any $\epsilon$
as long as we move $N$ farther out.

From there, we can see how we could chose the terms so that
$$
| s_n - s_m | = |a_{m+1} - a_{m+2}| + \ldots + |a_{n-1} - a_n|
$$
\\~\\


%%%%%%%%%%%%%
\textbf{Prove it using the NIP}
\\

Let's play around with the sample alternating sequence
$$
10 -9 +8 -7 +6 -5 +4 -3 +2 -1 +0
$$

\begin{center}
\begin{tabular}{ |c|c| }
    \hline
    $s_1$ & 10       \\ 
    \hline
    $s_2$ & 10-9 = 1 \\  
    \hline
    $s_3$ & 1+8 = 9  \\
    \hline
    $s_4$ & 9-7 = 2  \\
    \hline
    $s_5$ & 2+6 = 8  \\
    \hline
    $s_6$ & 8-5 = 3  \\
    \hline
    $s_7$ & 3+4 = 7  \\
    \hline
    $s_8$ & 7-3 = 4  \\
    \hline
    $s_9$ & 4+2 = 6  \\
    \hline
    $s_{10}$ & 6-1 = 5 \\
    \hline
\end{tabular}
\end{center}

Using this as example, we can see how we could build intervals $I_n$ such that $I_{n+1} \subseteq I_n$
by chosing $I_n = [s_{2n}, s_{2n+1}]$.
That way we can guarantee that there will be an $x\in\cap I_n$.
Also, since the length of $I_n$ will become smaller and smaller, we can then have
$|s_{2n} - x| < \epsilon$ or $|s_{2n+1} - x| < \epsilon$.
\\~\\


\textbf{Consider the sequences $(s_{2n})$ and $(s_{2n+1})$, and show that the MCT leads to a third proof.}
\\

Another good way to llok at the alternating series is as follows:
$$
s_n = (a_1 - a_2) + (a_3 - a_4) + \ldots \pm a_n
$$

Each group consists of terms that are great than or equal to zero, give our preconditions.
Each grouping resulting in the sum of some non-negative number that get's smaller and smaller.

We can also read it as:
$$
s_n = a_1 - (a_2 - a_3) - (a_4 - a_5) - \ldots
$$

In this case, $s_1$ is an upper bound and we continuously subtract smaller and smaller numbers.

Comparing these two findings, we can see that the sequence is monotonically increasing and that its
upper bound is $s_1$, so MCT says it must converge.
\\

Now, let's turn to $s_{2n}$, which can equal $s_2, s_4, s_6$ etc.,
and $s_{2n+1}$, which can equal $s_3, s_5, s_7$, etc.

If you write out a couple terms you will see how Abbott was trying to help us see what we describe above
and it offers us a slightly different argument.

$s_{2n+1}$, when we count from $n=0$, and if we assume that $a_n \geq 0$ for all n,
then we have a bounded and monotonically decreasing sequence - what we mentioned above.
We can also see that $s_{2n+1} = s_{2n} + a_{2n+1}$, or $s_{2n} = s_{2n+1} - a_{2n+1}$.
So we can do this,
$$
|s_{2n} - l| \leq |s_{2n} - s_{2n+1}| + |s_{2n+1} - l|
= |-a_{2n+1}| + |s_{2n+1} - l|
$$
The first term on the right we can make it smaller by choosing greater by going farther to the right,
the second term is just a restatement that $s_{2n+1}$ converges.
\\~\\




\textbf{2.7.2}

Does $\sum^{\infty}_{n=1} \frac{1}{2^n + n}$ converge?
\\

This series look a tad like
\begin{align*}
\sum_{n=1} \frac{1}{2^n + n} &= \frac{1}{2+1} + \frac{1}{2^2 +2} + \frac{1}{2^3 + 3} + \frac{1}{2^4 +4} + \frac{1}{2^5 +5} \ldots \\
&= \frac{1}{3} + \frac{1}{6} + \frac{1}{11} + \frac{1}{20} + \frac{1}{37} + \ldots
\end{align*}

In this case, $s_1 = 1/3$, $s_2 = 1/2$, and $s_3 = 13/22$.
So at least from the first couple partial sums, the partial sums seem to be monotonically increasing.

Now we have to look for clues.
The series $\sum \frac{1}{2^n}$ is a geometric series where $\frac{a}{1-r} = \frac{1}{1 - 1/2} = 2$.

Furhtermore, $\frac{1}{2^n} > \frac{1}{2^n +n}$ so by the comparsion tests, our series here should
converge.
\\~\\


Does $\sum^{\infty}_{n=1} \frac{\sin(n)}{n^2}$ converge?
\\

By a similar route, we saw in section 2.4 that $\sum 1/n^p$ when $p>1$ converges.
And since $\sin (n)$ oscillates between 1 and -1, then we can use the comparison test again to
see that the series converges.
\\~\\


Does $1 - \frac{3}{4} + \frac{4}{6} - \frac{5}{8} + \frac{6}{10} - \frac{7}{12} + \ldots$ converge?
\\

This series seems to go as $\sum (-1)^{n+1} \frac{n+1}{2n}$.
Which made me think of the alternating series.
But we need to see if the sequence $\frac{n+1}{2n}$ actually converges to 0.
Luckily, some algebra can help us right away
$$
\frac{n+1}{2n} = \frac{1}{2} + \frac{1}{2n}
$$
No matter how big $n$ gets, $(a_n) \rightarrow 1/2$.
It seems like we cannot say that it converges.
And since the series alternates between positive and negative values, then the series diverges.
\\~\\

Does $1 + \frac{1}{2} - \frac{1}{3} + \frac{1}{4} + \frac{1}{5} - \frac{1}{6} + \frac{1}{7} + \frac{1}{8} -\frac{1}{9} + \ldots$ converge?
\\

The trick here is to group terms, we always see two positive terms followed by a negative one.
And they always follow the pattern of the harmonic series but as
$$
\frac{1}{n} + \frac{1}{n+1} - \frac{1}{n+2} \geq \frac{1}{n}
$$
The inequality comes because $\frac{1}{n+1} > \frac{1}{n+2}$, so $\frac{1}{n+1} - \frac{1}{n+2}$
is a non-negative number that gets added to $1/n$.
And by using the comparison test, we can tell this series diverges.
\\~\\

Does $1 - \frac{1}{2^2} +\frac{1}{3} -\frac{1}{4^2} + \frac{1}{5} -\frac{1}{6^2} +\frac{1}{7} -\frac{1}{8^2} + \ldots$
converge?
\\

Interestingly, if we separate the terms for this series, it seems like we are adding
$\sum \frac{1}{2n -1}$ and $\sum -\frac{1}{(2n)^2} = \sum -\frac{1}{4}\frac{1}{n^2}$.

The latter series could be argued it converges by using the algebraic limit theorem for series
since we have $\sum 1/n^2$.
However, the former seems like the harmonic series, but only taking odd numbers.
\\~\\



\textbf{2.7.3}

An alternate way to prove the comparison tests for infinite series theorem using the monotone converge theorem
goes as follows.

If $0 \geq a_k \geq b_k$ and $\sum^{\infty}_{k=1} b_k$ converges, then we know that
the sequence of partial sums also converges.
And since the sequence of partial sums converges then it must be bounded.

The limit of the sequence of partial sums of $\sum b_k = \lim s_m$ can also be seen as an upper bound
for the sequence of partial sums of $\sum a_k$, since $0 \geq a_k \geq b_k$ for all $k\in\mathbb{N}$.

Also, since $0 \geq a_k$, then each successive term in the sequence of partial sums must be
equal to or greater than the previous one, so the sequence of partial sums is a monotone sequence
and thus converges as per the MCT.
\\~\\