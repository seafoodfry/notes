%%%%%%%%%%%%%%%%%%%%%%%%%%%%%%%%%%%%%%%%%%%%%%%%%%%%%%%%%%%%%%%%%%%%%%%%%%%
\section{Whittaker: Complex Numbers}

One of the problems posed in chapter 1 goes as follows:

Determine the nth roots of unity by aid of the argand diagram and show that the number of primitive roots
(roots the powers of each give all the roots) is the number of integers (including unity) less than n and prime to it.

To start, the "Nth roots of unity" mean the solutions to the equation
$$
x^n = 1
$$

Using Euler's formula, $e^{i\theta} = \cos\theta + i\sin\theta$,
$$
\left( e^{i\theta} \right)^n = e^{in\theta} =
\left( \cos\theta + i\sin\theta \right)^n = 
\cos n\theta + i\sin n\theta = 1
$$

From there we can see that multiples of $2\pi$ will be solutions, that means
we can express argument (or phase) as $\theta = 2i\pi k /n$, which turns our solution into
$$
e^{2i\pi k /n} = \cos \left(\frac{2\pi k}{n}\right) + i\sin \left(\frac{2\pi k}{n}\right)
$$
for $k = 0, \dots, n-1$.
Note that we have $n$ roots, not $n+1$ as $k$ ranges from 0 to $n-1$ because when $k=n$ we get the same root as when $k=1$.
But we have $n$ roots, that are equally spaced and correspond to $\frac{2\pi k}{n}$.

The next interesting bit of the problem is about the \textbf{primitive roots} which are
"roots the power of each which give all the roots".
One could also express this as a complex number $z_{k}$ such that $(z_k)^k$ gives the root $k$.
This is the \textbf{generator of a cyclic group} see
\href{https://en.wikipedia.org/wiki/Cyclic_group}{wikipedia: cyclic group}.

Now, to see why the number of primitive roots is the number of integers (including unity) less than $n$ and prime to it
we need to go back to our argument of why we $k$ ranged from 0 to $n-1$ instead of $n$ and then see that any number
coprime (or relatively prime) to $n$ (any number that shares a factor with $n$ other than 1) will result in a "degenerate root".

For example, if $n=6$, then $k = \{ 0, 1, 2, 3, 4, 5 \}$.
From that set, the integers that are coprime to 6 are 1 and 5.
You can also see why this is so by explicitly listing the multiples of $2\pi k / n$ for this example

\begin{align*}
2\pi k / n &= \{ 0, 2\pi/6, 2(2)\pi/6, 2(3)\pi/6, 2(4)\pi/6, 2(5)\pi/6 \} \\
&= \{ 0, \pi/3, 2\pi/3, \pi, 4\pi/3, 5\pi/3 \}
\end{align*}