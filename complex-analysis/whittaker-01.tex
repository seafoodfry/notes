%%%%%%%%%%%%%%%%%%%%%%%%%%%%%%%%%%%%%%%%%%%%%%%%%%%%%%%%%%%%%%%%%%%%%%%%%%%
\section{Whittaker: Complex Numbers}

One of the problems posed in chapter 1 goes as follows:

Determine the nth roots of unity by aid of the argand diagram and show that the number of primitive roots
(roots the powers of each give all the roots) is the number of integers (including unity) less than n and prime to it.

To start, the "Nth roots of unity" mean the solutions to the equation
$$
x^n = 1
$$

Using Euler's formula, $e^{i\theta} = \cos\theta + i\sin\theta$,
$$
\left( e^{i\theta} \right)^n = e^{in\theta} =
\left( \cos\theta + i\sin\theta \right)^n = 
\cos n\theta + i\sin n\theta = 1
$$

From there we can see that multiples of $2\pi$ will be solutions, that means
we can express argument (or phase) as $\theta = 2i\pi k /n$, which turns our solution into
$$
e^{2i\pi k /n} = \cos \left(\frac{2\pi k}{n}\right) + i\sin \left(\frac{2\pi k}{n}\right)
$$
for $k = 0, \dots, n-1$.
Note that we have $n$ roots, not $n+1$ as $k$ ranges from 0 to $n-1$ because when $k=n$ we get the same root as when $k=1$.
But we have $n$ roots, that are equally spaced and correspond to $\frac{2\pi k}{n}$.

The next interesting bit of the problem is about the \textbf{primitive roots} which are
"roots the power of each which give all the roots".
This, along with the rest of the problem, is outlined in
\href{https://en.wikipedia.org/wiki/Root_of_unity}{wikipedia: root of unity}:

Given a primitive nth root of unity $\omega$, the other nth roots are powers of $\omega$.
This means that the group of the nth roots of unity is a \textbf{cyclic group} and the primitive root is a
generator of the cyclic group, \href{https://en.wikipedia.org/wiki/Cyclic_group}{wikipedia: cyclic group}.

Now, to see why the number of primitive roots is the number of integers (including unity) less than $n$ and prime to it
we need to go back to our argument of why we $k$ ranged from 0 to $n-1$ instead of $n$ and then see that any number
that is not coprime (or relatively prime) to $n$ (any number that shares a factor with $n$ other than 1)
will result in a root that could have been obtained in another way.

For example, if $n=6$, then $k = \{ 0, 1, 2, 3, 4, 5 \}$.
From that set, the integers that are coprime to 6 are 1 and 5.
The actual roots are
\begin{itemize}
    \item $e^{0} = 1$
    \item $e^{2i\pi / 6} = e^{i\pi / 3}$
    \item $e^{2(2)i\pi / 6} = e^{2i\pi / 3}$
    \item $e^{2(3)i\pi / 6} = e^{i\pi} = -1$
    \item $e^{2(4)i\pi / 6} = e^{4i\pi / 3}$
    \item $e^{2(5)i\pi / 6} = e^{5i\pi / 3}$
\end{itemize}

The first primitve root is $z_1 = e^{2i\pi / 6} = e^{i\pi / 3}$,
for which
\begin{itemize}
    \item $z_{1}^{1} = e^{i\pi / 3}$
    \item $z_{1}^{2} = e^{2i\pi / 3}$
    \item $z_{1}^{3} = e^{i\pi} = -1$
    \item $z_{1}^{4} = e^{4i\pi / 3}$
    \item $z_{1}^{5} = e^{5i\pi / 3}$
    \item $z_{1}^{6} = e^{2i\pi } = 1$
\end{itemize}

We see that the powers $z_{1}^{1}, z_{1}^{2}, z_{1}^{3}, z_{1}^{4}, z_{1}^{5}, z_{1}^{6}$ cycle through all the roots.

Similarly then for $k = 5$, the other primitive root, $z_{5} = e^{5i\pi / 3}$, and so
\begin{itemize}
    \item $z_{5}^{1} = e^{5i\pi / 3}$
    \item $z_{5}^{2} = e^{10i\pi / 3} = e^{(6+4)i\pi / 3} = 1 \cdot e^{4i\pi / 3}$
    \item $z_{5}^{3} = e^{15i\pi / 3} = e^{(6*2+3)i\pi / 3} = e^{3i\pi / 3} = -1$
    \item $z_{5}^{4} = e^{20i\pi / 3} = e^{(6*3+2)i\pi / 3} = e^{2i\pi / 3}$
    \item $z_{5}^{5} = e^{25i\pi / 3} = e^{(6*4+1)i\pi / 3} = e^{i\pi / 3}$
    \item $z_{5}^{6} = e^{30i\pi / 3} = e^{(6*5)i\pi / 3} = 1$
\end{itemize}

Again we see that the powers of $z_{5}^{1}, z_{5}^{2}, z_{5}^{3}, z_{5}^{4}, z_{5}^{5}, z_{5}^{6}$ cycle through all the roots
exactly once.

And to explicitly see why a primitive root is such, let's try computing the roots for $k=2$, here $z_2 = e^{2(2)i\pi/6} = e^{2i\pi/3}$
\begin{itemize}
    \item $z_{2}^{1} = e^{2i\pi / 3}$
    \item $z_{2}^{2} = e^{4i\pi / 3}$
    \item $z_{2}^{3} = e^{6i\pi / 3} = 1$
    \item $z_{2}^{4} = e^{(6+2)i\pi / 3} = e^{2i\pi / 3}$
\end{itemize}

So after $x_{2}^{3}$ the roots repeat and we don't even see them all.

This is concept of primitive roots is so interesting that there is even a function associated to it:
\href{https://en.wikipedia.org/wiki/Euler%27s_totient_function}{wikipedia: Euler's totient function}
counts the positive integers up to a given integer n that are relatively prime to n.
Which means that it also gives the number of primitive roots of unity.