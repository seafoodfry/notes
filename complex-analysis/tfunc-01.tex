\section{Inifinite Series, Products, and Integrals}

\subsection{Uniform Convergence}

\textbf{Note:} when we speak of uniform convergence, the interval can be closed or open.
Titchmarsh just uses $(a,b)$ to cover the general case.
\\

The more general case for the first test of uniform Convergence we see is stated as follows:

The series $\sum u_n(x)$ is uniformly convergent
($\forall \epsilon>0$, we can find $n_0 \geq N$ depending on $\epsilon$ but not on $x$,
such that $|s(x) - s|<\epsilon$, for every $n\geq n_0$ for every value in $(a,b)$)
if $|u_n(x)| \leq v_n(x)$, and $\sum v_n(x)$ is uniformly convergent.

If we try to make an argument by contradiciton and assume that $\sum u_n(x)$ is not
uniformly convergent, then the series could still converge but it could be the case that
as $x$ approaches some point on the interval $(a,b)$, $n_0$ may become infinetely large.
Additionally, the series could just be a divergent series.
Either way it means we are not able to find an $n_0$ such that $|s(x) - s| < \epsilon$
for any $n \geq n_0$, for any $\epsilon >0$ and for any $x\in (a,b)$.
This means that
$$
|u_{n+1}(x) + u_{n+2}(x) + \ldots |
$$
keeps on changing as $n$ or $x$ change.

Since,
$$
|u_{n+1}(x) + u_{n+2}(x) + \ldots | \leq |u_{n+1}(x)| + |u_{n+2}(x)| + \ldots
$$
Then any $v_n(x)$ such that $v_{n+1}(x) \geq |u_{n+1}(x)|$ would also grow indefinetely and thus
lead to a contradiciton.
\\~\\

\textbf{Examples}

A proof to see why $\sum_{n=0} x^n$ is uniformly convergent in $c\in [a,b]$
when $-1 < a < b < 1$ can be seen by comparing the exercise 2.5.3 from Abbott.
\\

The trigonometric series $\sum_{n=1} \frac{\cos nx}{n^2}$ is convergent anywhere because
$-1 \leq \cos x \leq 1$ so $\left| \frac{\cos nx}{n^2} \right|$ behaves like a convergent p-series.
\\

Similarly the Dirichlet series $\sum_{n=1} n^{-s} = \sum_{n=1} \frac{1}{n^s}$ is uniformly convergent
in $x\in [a,b]$ if $1 < a < b$ because its absolute value is eequal to a convergent p series.
The sum is refered to as the Riemann zeta function $\zeta (s)$.