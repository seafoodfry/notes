\section{Complex Numbers}

%%%%%%%%%%%%%%%%%%%%%%%%%%%%%%%%%%%%%%%%%%%%%%%%%%%%%%%%%%%%%%%%%%%%%%%%%%%
\subsection{Basic Algebraic Properties}

A handy thing to keep written down
$$
z^{-1} = \left(\frac{a}{z^2 + b^2}, \frac{-b}{a^2 + y^2}\right)
$$

The way to think about it is as if you wanted to find $c$ and $d$ such that
$$
\frac{1}{a + bi} = c + di
$$

The trick here is as follows
$$
\frac{1}{a + bi} \left( \frac{a - bi}{a - bi} \right)= \frac{a - bi}{a^2 + b^2}
$$
\\



Also,
$$
|z|^2 = z\bar{z} = (a + ib)(a - ib) = a^2 + b^2
$$

The generalization of $|z|^2 = (\Re(z))^2 + (\Im(z))^2$ does hold!
\\


Note that the product of two complex numbers is very diffeent from the scalar or vector products
done in vector spaces over the reals.
This notion of a \textbf{billinear form} is what is often used to distinguish between different
algebras.

Also note that $z_1 < z_2$ has no meaning, so the order field properties we are used to from
real numbers don't apply as such.
However $|z_1| < |z_2|$ does make sense.
\\

The distance between two points $(x_1, y_1)$ and $(x_2, y_2)$ is $|z_1 - z_2|$.
\\

The complex numbers lying on a circle with center $z_0$ and radius $R$ satisfy the equation
$$
|z - z_0| = R
$$

A wonderful example of this last interpretation is
$$
|z - 3i| + |z + 3i| = |z - 3i| + |z - (-3i)| = 12
$$
This equation represents the set of all points whose distance from the two set points, $F_1(0,3)$
and $F_2(0,-3)$, is 12.
This turns out to be the ellipse with foci $F_1(0,3)$ and $f_2(0,-3)$.
Kline has some great exercises to get you acquainted with Ellipses, parabolas, and hyperbolas.


%%%%%%%%%%%%%%%%%%%%%%%%%%%
\subsubsection{Exercises}

\textbf{2.2}

Some interesting properties

$$
z + \bar{z} = (a+ib) + (a-ib) = 2a = 2\Re (z)
$$

Similarly,
$$
z - \bar{z} = (a+ib) - (a-ib) = 2ib = 2\Im (z)
$$

Following the same mechanics,
$$
\Re (iz) = \Re (i(a+ib)) = \Re (ai-b) = -\Im (z)
$$
And
$$
\Im (iz) = \Im (ai -b) = \Re (z)
$$


%%%%%%%%%%%%%%%%%%%%%%%%%%%%%%%%%%%%%%%%%%%%%%%%%%%%%%%%%%%%%%%%%%%%%%%
\subsection{Triangle Inequality}

There is a briliant example in this section, go read it!

The heart of the example is in noticing that the triangle inequality gives us an upper and a lower
bound for the sum of two numbers.
The upper bound comes from
$$
|z_1 + z_2| \leq |z_1| + |z_2|
$$
and the lower bound from
$$
|z_1 - z_2| \geq \left| |z_1| - |z_2| \right|
$$

\subsubsection{polynomials}

If $n$ is a positive integer, and if $a_0, a_1, a_2, \ldots$ are complex constants,
where $a_n \neq 0$,
$$
P(z) = a_0 + a_1 z + a_2 z^ + \ldots a_n z_6
$$
if a polynomial of degree n.
For some positive number $R$, the reciprocal $1/P(z)$ satisfies the inequality
$$
\left| \frac{1}{P(z)} \right| < \frac{2}{|a_n| R^n}
$$
whenever $|z| > R$.
Geometrically, this tells us that the modulus of the reciprocal $1/P(z)$
is bounded from above when z is exterior to the circle $|z| = R$.

The tricky bit of the argument for the abvove is given when the author's say
"Now that a sufficiently large positive number $R$ can be found such that each of the
quotients on the right in inequality (9) is less than the number $|a_n |/(2n)$ when
$|z| > R$..."
What ci-dessus means in practice is that when we have
$$
|w| \leq \frac{|a_0|}{|z|^n} + \frac{|a_1|}{|z|^{n-1}} + \ldots + \frac{|a_{n-1}|}{|z|}
$$
We are actually dividing $|w||z|^n$ by $R^n$ since $R^n \sim |z|^n$, and so Churchill
is saying that we can find some $z$ such that when we raise it to a power $n$ and divide $|w||z|^n$
by it, that we will have all $n$ terms on the right side of the inequality be less than the $n$-th
term divided by $n$ (go far out ebough and there will be a $z$ value that will make this truth).

The rest of the argument in the book is algebra.


%%%%%%%%%%%%%%%%%%%%%%%%%%%%%%%%%%%%%%
\subsubsection{Exercises}

%%%%%%%%%%%%%%%%%%%%
\textbf{Ex 4}
\\
Verify $\sqrt{2}|z| \geq |\Re(z)| + |\Im(z)|$.
Suggestion: reduce the above inequality to $(|x| - |y|)^2 \geq 0$.
\\

This exercise is similar to one in Folland's advanced calculus, exercise 2 in section 1.1 Euclidean Spaces and Vectors
(we do provide a solution for that example in the calculus doc).

Working backwards,
$$
(\sqrt{2} |z|)^2 = 2|z|^2 = 2x^2 + 2y^2
$$

Following the tip of the book, a thing to try is something like
$$
(|\Re(z)| + |\Im(z)|)^2 = |x|^2 + |y|^2 + 2|x||y| = x^2 + y^2 + 2|x||y|
$$

Then
\begin{align*}
(\sqrt{2} |z|)^2  -  (|\Re(z)| + |\Im(z)|)^2 &= x^2 + y^2 - 2|x||y| \\
&= |x|^2 + |y|^2 - 2|x||y| \\
&= (|x| - |y|)^2 \geq 0
\end{align*}
\\~\\


%%%%%%%%%%%%%%%%%%%
\textbf{Ex 6}
\\
Using the fact that $|z_1 - z_2|$ is the distance between the points $z_1$ and $z_2$,
give a geometric argument that $|z - 1| = |z + i|$ represents the line through the origin
whose slope is $-1$.
\\

This one is a very interesting case.
$|z-1|$ would correspond to the distance from a point $z$ to the point $1$ (the x-axis),
$|z+i|$ would correspond to the distance from a point $z$ to the point $-i$ (the y-axis).
Since both are a unit from the origin, in their respective axis.
The expression above then equates these two distances giving us a straightline that passes through
the origin with a $-1$ slope.
\\~\\


%%%%%%%%%%%%%%%
\textbf{Ex 7}
\\
Show that for $R$ sufficiently large, the polynomial $P(z)$ satisfies
$$
|P_n(z)| \leq 2 |a_n||z|^n
$$
whenever $|z| > R$.
Suggestion: observe that there is a positive number $R$ such that the modulus
of each quotient in $|w| \leq |a_0|/|z|^n + |a_1|/|z|^{n-1} + \ldots + |a_{n-1}|/|z|$
is less than $|a_n|/n$ when $|z| > R$.
\\

In the argument in which the original inuequality was made, there was a step where we use the following
$|a_n + w| \geq \left||a_n| - |w|\right|$ to come up with a lower bound.
If we instead looked for an upper bound, we could look at
$$
|a_n + w| \leq |a_n| + |w| < |a_n| + \frac{1}{2}|a_n| < 2|a_n| 
$$

The rest of the argument flows when we plug that into expression (10),
$$
\left| P_n(z) \right| = |a_n + w| |z|^n < 2|a_n||z|^n
$$
\\~\\


%%%%%%%%%%%%%
\textbf{Ex 8}
\\
Let $z_1 = x_1 + iy_1$ and $z_2 = x_2 + iy_2$.
Use simple algebra to show that
$$
|z_1 z_2| = |(x_1 + iy_1) (x_2 + iy_2)| = \sqrt{ (x_{1}^{2} + y_{1}^{2}) (x_{2}^{2} + y_{2}^{2}) }
$$
then point out how the identity $|z_1 z_2| = |z_1| |z_2|$ follows.
\\

The trick to the first part is to make use of
$|z| = \sqrt{ (\Re(z))^2 + (\Im(z))^2 }$ 

First of,
$$
(x_1 + iy_1) (x_2 + iy_2) = (x_1 x_2 - y_1 y_2) + i (x_1 y_2 + x_2 y_1)
$$

From there, we can see that
\begin{align*}
\Re(z_1 z_2)^2 &= (x_1 x_2 - y_1 y_2) (x_1 x_2 - y_1 y_2) \\
&= (x_1 x_2)^2 + (y_1 y_2)^2 - 2(x_1 x_2)(y_1 y_2)
\end{align*}
and
\begin{align*}
\Im(z_1 z_2)^2 &= (x_1 y_2 + x_2 y_1) (x_1 y_2 + x_2 y_1) \\
&= (x_1 y_2)^2 + (x_2 y_1)^2 + 2(x_1 x_2)(y_1 y_2)
\end{align*}

It then follows that
\begin{align*}
|z_1 z_2| &= \sqrt{ (x_1 x_2)^2 + (y_1 y_2)^2 - 2(x_1 x_2)(y_1 y_2)   +   (x_1 y_2)^2 + (x_2 y_1)^2 + 2(x_1 x_2)(y_1 y_2) } \\
&= \sqrt{ (x_1 x_2)^2 + (y_1 y_2)^2   +  (x_1 y_2)^2 + (x_2 y_1)^2  } \\
&= \sqrt{ (x_{1}^{2} + y_{1}^{2}) (x_{2}^{2} + y_{2}^{2}) }
\end{align*}

Since $|z| = \sqrt{x^2 + y^2}$, we can see how the above reordering is equivalent to
\begin{align*}
|z_1 z_2| &= \sqrt{ (x_{1}^{2} + y_{1}^{2}) (x_{2}^{2} + y_{2}^{2}) } \\
&= \sqrt{ x_{1}^{2} + y_{1}^{2} } \sqrt{ x_{2}^{2} + y_{2}^{2} } \\
&= |z_1| |z_2|
\end{align*}
\\~\\


\textbf{Ex 9}
\\
If we use the result from the previous exercise and assume have $z = z_1 = z_2$, we have
$$
|z^2| = |z||z| = |z|^2
$$
We could use this as the base case for an induction argument ($n=2$).

Then for our hypothesis, we assume that $|z^m| = |z|^m$, when $n=m$, so it must also
hold for $n=m+1$,
$$
|z^{m+1}| = |z^m z| = |z^\prime z| = |z^\prime| |z| = |z^m| |z| = |z|^m |z| = |z|^{m+1}
$$

%%%%%%%%%%%%%%%%%%%%%%%%%%%%%%%%%%%%%%%%%%%%%%%%%%%%%%%%%%%%%%%%%%%%%%%%%%%
\subsection{6 Complex Conjugates}

\subsubsection{Exercises}

\textbf{Ex 4}
\\
Show that,
$$
\overline{z^2} = \bar{z}^2
$$

Note that
\begin{align*}
\overline{z^2} &= \overline{z z} \\
    &= \bar{z} \bar{z} \\
    &= \bar{z}^2
\end{align*}


%%%%%%%%%%%%%%%%%%%%%%%%%%
\textbf{Ex 9}
\\
By factoring $z^4 -4z^2 + 3$ into two quadratic factors and using the inequality $|z_1 + z_2| \geq \left| |z_1| - |z_2| \right|$,
show that if $z$ lies on the circle $|z| = 2$, then
$$
\left| \frac{1}{z^4 - 4z^2 + 3} \right| \leq \frac{1}{3}
$$

To start with the factoring of our polynomial into two quadratic factors.
The three constants we have are $A=1$, $B=-4$, and $C=3$.
So we need some combination that when multiplied will equal $AC = 3$ and when summed $B=-4$.
If we choose 1 and 3 we can get that combination, and since $A=1$, then dividing both of those numbers by 1 results in no change.
Thus we have,
$$
\left| z^4 - 4z^2 + 3 \right| = \left| \left(z-1\right) \left(z-3\right) \right| =
\left| \left(z-1\right) \right| \left| \left(z-3\right) \right|
$$

If we now use the inequality that was recommended to us, we get
\begin{align*}
\left| z^4 - 4z^2 + 3 \right| &= \left| \left(z^2 - 1\right) \right| \left| \left(z^2 - 3\right) \right| \\
&\geq \left| |z|^2 - |-1| \right| \left| |z|^2 - |-3| \right| \\
&= \left| |z|^2 - 1 \right| \left| |z|^2 -3 \right| \\
&= \left| 4-1 \right| \left| 4-3 \right| \\
&= 3
\end{align*}

The final result comes from inverting the two quantities.
\\~\\


%%%%%%%%%%%%%%%%%%%%%%%%%%
\textbf{Ex 10}
\\
Show that,
\begin{enumerate}
    \item $z$ is real if and only if $z = \bar{z}$
    \item $z$ is either real or pure imaginary if and only if $\bar{z}^2 = z^2$
\end{enumerate}

To see the why for the first one, note that if $z = x + iy = x - iy = \bar{z}$,
then $(x, y) = (x, -y)$.
Saying that $x = x$ leads us to a tautology, so no issue there, but $y = -y$ leads us to say
that $z$ must be real.
\\

Using a similar argument,
$$
\bar{z}^2 = \overline{z^2} = \overline{x^2 - y^2 + 2ixy} = x^2 - y^2 + 2ixy
$$
The last step has $xy = -xy$, so this must be zero.
And this happens when $z$ is either real or pure imaginary.
\\~\\


%%%%%%%%%%%%%%%%%%%%%%%%%%
\textbf{Ex 13}
\\
Show that the equation $|z - z_0| = R$ of a circle, centered at $z_0$ with radisu $R$
can be written as
$$
|z|^2 - 2\Re(zz_0) + |z_0|^2 = R
$$

To see why, note that
\begin{align*}
|z - z_0| &= (z - z_0)^* (z - z_0) \\
    &= (\bar{z} - \bar{z_0}) (z - z_0) \\
    &= \bar{z}z - \bar{z}z_0 - z\bar{z_0} + \bar{z_0}z_0 \\
    &= |z|^2 - (z\bar{z_0} + \bar{z}z_0) + |z_0|^2 \\
    &= |z|^2 - 2\Re(z\bar{z_0}) + |z_0|^2
\end{align*}


\textbf{Ex 14}
\\
Show that the hyperbola $x^2 - y^2 = 1$ can be rewritten as $z^2 - \bar{z}^2 = 2$.
\\

Well, $z^2 = (z + iy)(z + iy) = x^2 + 2ixy - y^2$.
And we already saw that $\bar{z}^2 = \overline{z^2}$, so $\bar{z}^2 = x^2 - 2ixy - y^2$.
From there the answer follows.

Just kidding, here are the rest of the steps.
But keep in mind that,
\begin{enumerate}
    \item $\left( z + \overline{z} \right) = z^2 + \overline{z}^2 + z\overline{z} + \overline{z}z$, and
    \item $\left( z - \overline{z} \right) = z^2 + \overline{z}^2 - z\overline{z} - \overline{z}z$
\end{enumerate}

So if, $x^2 - y^2 = 1$, then,

\begin{align*}
x^2 - y^2 &= \Re^2 (z) - \Im^2 (z) = \left(\frac{z+\overline{z}}{2}\right)^2 - \left(\frac{z-\overline{z}}{2i}\right)^2 \\
&= \frac{1}{4}\left(z+\overline{z}\right)^2 + \frac{1}{4}\left(z-\overline{z}\right)^2 \\
&= \frac{1}{2} \left(z^2 + \overline{z}^2\right)
\end{align*}

Which is equivalent to saying that $2\left(x^2 - y^2\right) = z^2 + \overline{z}^2 = 2$, hence our answer.
\\~\\



%%%%%%%%%%%%%%%%%%%%%%%%%%%%%%%%%
\textbf{Ex 15}
\\

Follow the steps below to give an algebraic derivation of the triangle inequality
$$
\left| z_1 + z_2 \right| \leq |z_1| + |z_2|
$$

\begin{enumerate}[(a)]
\item Show that
$$
\left| z_1 + z_2 \right|^2 =
\left(z_1 + z_2\right)\left(\overline{z_1} + \overline{z_2}\right) =
z_1 \overline{z_1} + \left(z_1 \overline{z_2} + \overline{z_1 \overline{z_2}}\right) + z_2 \overline{z_2}
$$

This first step is just plain algebra, though keep in mind that $z\overline{z} = |z|^2$ and that
$\overline{z_1 \overline{z_2}} = \overline{z_1} z_2$.

\item Point out why $z_1 \overline{z_2} + \overline{z_1 \overline{z_2}} = 2 \Re\left(z_1 \overline{z_2}\right) \leq 2|z_1| |\overline{z_2}|$

Now we are using the identity $z+\overline{z} = 2\Re(z)$, where $z = z_1 \overline{z_2}$
along with the inequality $\Re(z) \leq \left|\Re(z)\right| \leq |z|$.

\item Use the results in parts (a) and (b) to obtain the inequality
$$
\left| z_1 + z_2 \right|^2 \leq \left( |z_1| + |z_2| \right)^2,
$$
and note how the triangle inequality follows.

This last inequality is obtained by boting that
\begin{align*}
\left| z_1 + z_2 \right|^2 &=
    z_1 \overline{z_1} + \left(z_1 \overline{z_2} + \overline{z_1 \overline{z_2}}\right) + z_2 \overline{z_2} \\
&\leq |z_1|^2 + |z_2|^2 + 2|z_1| |\overline{z_2}| \\
&= \left( |z_1| + |z_2| \right)^2
\end{align*}

and since both, left and right, quantities are positive, then we get $\left| z_1 + z_2 \right| \leq |z_1| + |z_2|$.

\end{enumerate}

%%%%%%%%%%%%%%%%%%%%%%%%%%%%%%%%%%%%%%%%%%%%%%%%%%%%%%%%%%%%%%%%%%%%%%%
\subsection{8 Product's and Powers in Exponential Form}

%%%%%%%%%%%%%%%%%%%%%%%%%%%%%%%%%%%%%%%%%%%%%%%%%%%%%%%%%%%%%%%%%%%%%%%
\subsubsection{De Moivre's Theorem}
$$
(a + ib)^n = [r(\cos \theta + i \sin \theta)]^n = r^n (\cos n\theta + i \sin n\theta)
$$
$$
r = |z| = \sqrt{a^2 + b^2}
$$
$$
\theta = \tan^-1 \frac{y}{x}
$$

De Moiver's Theorem 

%%%%%%%%%%%%%%%%%%%%%%%%%%%%%%%%%%%%%%%%%%%%%%%%%%%%%%%%%%%%%%%%%%%%%%%
\subsection{9 Arguments of products and Quotients}

%%%%%%%%%%%%%%%%%%%%%%%%%%%%%%%%%%%%%%%%%%%%%%%%%%%%
\subsubsection{Roots of Complex Numbers}

Nth roots: for any positive integer n, the nth distinct roots of
$(a+ ib)^n = r^n (\cos nx +i\sin nx)$ are
$$
r^{\frac{1}{n}} \left[\cos \frac{x + 2\pi k}{n} + i\sin \frac{x + 2\pi k}{n}\right]
$$
for $k = 0, 1, \dots, n-1$.


%%%%%%%%%%%%%%%%%%%%%%%%%%%%%%%%%%%%%%%%%%%%%%%%%%%%
\subsubsection{Exercises}

\textbf{9.1}
\\

Find the principal argument $\Arg z$ when
\begin{enumerate}
\item $z = \frac{-2}{1 + \sqrt{3}i}$

$$
\Arg (-2) = \pi
$$

$$
\Arg (1 + \sqrt{3}i) = \frac{\pi}{3}
$$
To get the above answer, we drew it out and searched for special triangles.
It just so happens that we can use the 30-60-90 triangle whose $\tan \sqrt{3} = \pi/3$.

If we subtract the two, we get $2\pi/3$ which falls within the valid range for a principal argument, $-\pi < \Theta \leq \pi$.

\item $z = \left(\sqrt{3}-i\right)^6$

If we graph where $\sqrt{3}-i$ is, we will find it in the foruth quadrant, so the principal argument must be negative.
The angle in this case is $\tan 1/\sqrt{3}$, which is now the 30-degrees side of the 30-60-90 triangle and so $\Theta = -\pi/6$.

However, now we have to raise it to the 6-th power.
$\left(e^{i\theta}\right)^6 = e^{-i\pi} = -1$.
Also, $r = |z| = \sqrt{\overline{z}z} = \sqrt{\Re{(z)}^2 + \Im{(z)}^2 } = 2$.
So $z^6 = (2^6)(-1) = -64$.
Since the value is all real and it lies in the negative side of the graph, the principal argument is $\Theta = \pi$.

\end{enumerate}


%%%%%%%%%%%%%%%%%%%%%%%%%%%%%%%%%
\textbf{9.4}
\\

Using the fact that the modulus $|e^{i\theta} - 1|$ is the distance between teh points $e^{i\theta}$ and $1$
(See section 4), give a geometric argument to find a value of $\theta$ in the interval $0 \leq \theta < 2\pi$
that satisifes $|e^{i\theta} - 1| = 2$.
\\

We saw in section 4 that $|z - z_0| = R$ can be interpreted as the set of complex numbers $z$ ($e^{i\theta}$)
centered at $z_0$ (1) and with a radius of $R$ (2).
So what complex numbers, lying on the unit circle, also lie on the circle centered at 1 and whose radisu is 2?
The point that meets this criteria is $\theta = \pi$.
\\~\\


