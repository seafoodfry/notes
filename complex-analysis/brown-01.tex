\section{Complex Numbers}

%%%%%%%%%%%%%%%%%%%%%%%%%%%%%%%%%%%%%%%%%%%%%%%%%%%%%%%%%%%%%%%%%%%%%%%%%%%
\subsection{Basic Algebraic Properties}

A handy thing to keep written down
$$
z^{-1} = \left(\frac{a}{z^2 + b^2}, \frac{-b}{a^2 + y^2}\right)
$$

also,
$$
|z|^2 = |z\bar{z}| = (a + ib)(a - ib) = a^2 + b^2
$$

\subsubsection{Exercises}

\textbf{2.2}

Some interesting properties

$$
z + \bar{z} = (a+ib) + (a-ib) = 2a = 2\Re (z)
$$

Similarly,
$$
z - \bar{z} = (a+ib) - (a-ib) = 2ib = 2\Im (z)
$$

Following the same mechanics,
$$
\Re (iz) = \Re (i(a+ib)) = \Re (ai-b) = -\Im (z)
$$
And
$$
\Im (iz) = \Im (ai -b) = \Re (z)
$$




%%%%%%%%%%%%%%%%%%%%%%%%%%%%%%%%%%%%%%%%%%%%%%%%%%%%%%%%%%%%%%%%%%%%%%%
\subsubsection{De Moivre's Theorem}
$$
(a + ib)^n = [r(\cos \theta + i \sin \theta)]^n = r^n (\cos n\theta + i \sin n\theta)
$$
$$
r = |z| = \sqrt{a^2 + b^2}
$$
$$
\theta = \tan^-1 \frac{y}{x}
$$

De Moiver's Theorem 

%%%%%%%%%%%%%%%%%%%%%%%%%%%%%%%%%%%%%%%%%%%%%%%%%%%%
\subsubsection{Roots of Complex Numbers}

Nth roots: for any positive integer n, the nth distinct roots of
$(a+ ib)^n = r^n (\cos nx +i\sin nx)$ are
$$
r^{\frac{1}{n}} \left[\cos \frac{x + 2\pi k}{n} + i\sin \frac{x + 2\pi k}{n}\right]
$$
for $k = 0, 1, \dots, n-1$.