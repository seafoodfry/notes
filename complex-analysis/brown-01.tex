\section{Complex Numbers}

%%%%%%%%%%%%%%%%%%%%%%%%%%%%%%%%%%%%%%%%%%%%%%%%%%%%%%%%%%%%%%%%%%%%%%%%%%%
\subsection{Basic Algebraic Properties}

A handy thing to keep written down
$$
z^{-1} = \left(\frac{a}{z^2 + b^2}, \frac{-b}{a^2 + y^2}\right)
$$

also,
$$
|z|^2 = |z\bar{z}| = (a + ib)(a - ib) = a^2 + b^2
$$

The generalization of $|z|^2 = (\Re(z))^2 + (\Im(z))^2$ does hold!
\\


Note that the product of two complex numbers is very diffeent from the scalar or vector products
done in vector spaces over the reals.
This notion of a \textbf{billinear form} is what is often used to distinguish between different
algebras.

Also note that $z_1 < z_2$ has no meaning, so the order field properties we are used to from
real numbers don't apply as such.
However $|z_1| < |z_2|$ does make sense.
\\

The distance between two points $(x_1, y_1)$ and $(x_2, y_2)$ is $|z_1 - z_2|$.
\\

The complex numbers lying on a circle with center $z_0$ and radius $R$ satisfy the equation
$$
|z - z_0| = R
$$

A wonderful example of this last interpretation is
$$
|z - 3i| + |z + 3i| = |z - 3i| + |z - (-3i)| = 12
$$
This equation represents the set of all points whose distance from the two set points, $F_1(0,3)$
and $F_2(0,-3)$, is 12.
This turns out to be the ellipse with foci $F_1(0,3)$ and $f_2(0,-3)$.
Kline has some great exercises to get you acquainted with Ellipses, parabolas, and hyperbolas.


%%%%%%%%%%%%%%%%%%%%%%%%%%%
\subsubsection{Exercises}

\textbf{2.2}

Some interesting properties

$$
z + \bar{z} = (a+ib) + (a-ib) = 2a = 2\Re (z)
$$

Similarly,
$$
z - \bar{z} = (a+ib) - (a-ib) = 2ib = 2\Im (z)
$$

Following the same mechanics,
$$
\Re (iz) = \Re (i(a+ib)) = \Re (ai-b) = -\Im (z)
$$
And
$$
\Im (iz) = \Im (ai -b) = \Re (z)
$$


%%%%%%%%%%%%%%%%%%%%%%%%%%%%%%%%%%%%%%%%%%%%%%%%%%%%%%%%%%%%%%%%%%%%%%%
\subsection{Triangle Inequality}

There is a briliant example in this section, go read it!

The heart of the example is in noticing that the triangle inequality gives us an upper and a lower
bound for the sum of two numbers.
The upper bound comes from
$$
|z_1 + z_2| \leq |z_1| + |z_2|
$$
and the lower bound from
$$
|z_1 - z_2| \geq \left| |z_1| - |z_2| \right|
$$

\subsubsection{polynomials}

If $n$ is a positive integer, and if $a_0, a_1, a_2, \ldots$ are complex constants,
where $a_n \neq 0$,
$$
P(z) = a_0 + a_1 z + a_2 z^ + \ldots a_n z_6
$$
if a polynomial of degree n.
For some positive number $R$, the reciprocal $1/P(z)$ satisfies the inequality
$$
\left| \frac{1}{P(z)} \right| < \frac{2}{|a_n| R^n}
$$
whenever $|z| > R$.
Geometrically, this tells us that the modulus of the reciprocal $1/P(z)$
is bounded from above when z is exterior to the circle $|z| = R$.

The tricky bit of the argument for the abvove is given when the author's say
"Now that a sufficiently large positive number $R$ can be found such that each of the
quotients on the right in inequality (9) is less than the number $|a_n |/(2n)$ when
$|z| > R$..."
What ci-dessus means in practice is that when we have
$$
|w| \leq \frac{|a_0|}{|z|^n} + \frac{|a_1|}{|z|^{n-1}} + \ldots + \frac{|a_{n-1}|}{|z|}
$$
We are actually dividing $|w||z|^n$ by $R^n$ since $R^n \sim |z|^n$, and so Churchill
is saying that we can find some $z$ such that when we raise it to a power $n$ and divide $|w||z|^n$
by it, that we will have all $n$ terms on the right side of the inequality be less than the $n$-th
term divided by $n$ (go far out ebough and there will be a $z$ value that will make this truth).

The rest of the argument in the book is algebra.

\subsubsection{Exercises}

\textbf{Ex 4}
\\
Verify $\sqrt{2}|z| \geq |\Re(z)| + |\Im(z)|$.
Suggestion: reduce the above inequality to $(|x| - |y|)^2 \geq 0$.
\\~\\


\textbf{Ex 6}
\\
Using the fact that $|z_1 - z_2|$ is the distance between the points $z_1$ and $z_2$,
give a geometric argument that $|z - 1| = |z + i|$ represents the line through the origin
whose slope is $-1$.
\\

\textbf{Ex 7}
\\
Show that for $R$ sufficiently large, the polynomial $P(z)$ satisfies
$$
|P(z)| \leq 2 |a_n||z|^n
$$
whenever $|z| > R$.
Suggestion: observe that there is a positive number $R$ such that the modulus
of each quotient in $|w| \leq |a_0|/|z|^n + |a_1|/|z|^{n-1} + \ldots + |a_{n-1}|/|z|$
is less than $|a_n|/n$ when $|z| > R$.
\\~\\

\textbf{Ex 8}
\\
Let $z_1 = x_1 + iy_1$ and $z_2 = x_2 + iy_2$.
Use simple algebra to show that
$$
|z_1 z_2| = |(x_1 + iy_1) (x_2 + iy_2)| = \sqrt{ (x_{1}^{2} + y_{1}^{2}) (x_{2}^{2} + y_{2}^{2}) }
$$
then point out how the identity $|z_1 z_2| = |z_1| |z_2|$ follows.
\\

The trick to the first part is to make use of
$|z| = \sqrt{ (\Re(z))^2 + (\Im(z))^2 }$ 

First of,
$$
(x_1 + iy_1) (x_2 + iy_2) = (x_1 x_2 - y_1 y_2) + i (x_1 y_2 + x_2 y_1)
$$

From there, we can see that
\begin{align*}
\Re(z_1 z_2)^2 &= (x_1 x_2 - y_1 y_2) (x_1 x_2 - y_1 y_2) \\
&= (x_1 x_2)^2 + (y_1 y_2)^2 - 2(x_1 x_2)(y_1 y_2)
\end{align*}
and
\begin{align*}
\Im(z_1 z_2)^2 &= (x_1 y_2 + x_2 y_1) (x_1 y_2 + x_2 y_1) \\
&= (x_1 y_2)^2 + (x_2 y_1)^2 + 2(x_1 x_2)(y_1 y_2)
\end{align*}

It then follows that
\begin{align*}
|z_1 z_2| &= \sqrt{ (x_1 x_2)^2 + (y_1 y_2)^2 - 2(x_1 x_2)(y_1 y_2)   +   (x_1 y_2)^2 + (x_2 y_1)^2 + 2(x_1 x_2)(y_1 y_2) } \\
&= \sqrt{ (x_1 x_2)^2 + (y_1 y_2)^2   +  (x_1 y_2)^2 + (x_2 y_1)^2  } \\
&= \sqrt{ (x_{1}^{2} + y_{1}^{2}) (x_{2}^{2} + y_{2}^{2}) }
\end{align*}

Since $|z| = \sqrt{x^2 + y^2}$, we can see how the above reordering is equivalent to
\begin{align*}
|z_1 z_2| &= \sqrt{ (x_{1}^{2} + y_{1}^{2}) (x_{2}^{2} + y_{2}^{2}) } \\
&= \sqrt{ x_{1}^{2} + y_{1}^{2} } \sqrt{ x_{2}^{2} + y_{2}^{2} } \\
&= |z_1| |z_2|
\end{align*}
\\~\\


\textbf{Ex 9}
\\
If we use the result from the previous exercise and assume have $z = z_1 = z_2$, we have
$$
|z^2| = |z||z| = |z|^2
$$
We could use this as the base case for an induction argument ($n=2$).

Then for our hypothesis, we assume that $|z^m| = |z|^m$, when $n=m$, so it must also
hold for $n=m+1$,
$$
|z^{m+1}| = |z^m z| = |z^\prime z| = |z^\prime| |z| = |z^m| |z| = |z|^m |z| = |z|^{m+1}
$$


%%%%%%%%%%%%%%%%%%%%%%%%%%%%%%%%%%%%%%%%%%%%%%%%%%%%%%%%%%%%%%%%%%%%%%%
\subsection{De Moivre's Theorem}
$$
(a + ib)^n = [r(\cos \theta + i \sin \theta)]^n = r^n (\cos n\theta + i \sin n\theta)
$$
$$
r = |z| = \sqrt{a^2 + b^2}
$$
$$
\theta = \tan^-1 \frac{y}{x}
$$

De Moiver's Theorem 

%%%%%%%%%%%%%%%%%%%%%%%%%%%%%%%%%%%%%%%%%%%%%%%%%%%%
\subsection{Roots of Complex Numbers}

Nth roots: for any positive integer n, the nth distinct roots of
$(a+ ib)^n = r^n (\cos nx +i\sin nx)$ are
$$
r^{\frac{1}{n}} \left[\cos \frac{x + 2\pi k}{n} + i\sin \frac{x + 2\pi k}{n}\right]
$$
for $k = 0, 1, \dots, n-1$.